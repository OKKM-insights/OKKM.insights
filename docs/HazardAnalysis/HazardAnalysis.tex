\documentclass{article}
% comment for testing workflow v33
\usepackage{booktabs}
\usepackage{tabularx}
\usepackage{hyperref}
\usepackage{multirow}
\usepackage{float}
\usepackage{longtable}
\usepackage{url}

\hypersetup{
    colorlinks=true,       % false: boxed links; true: colored links
    linkcolor=red,          % color of internal links (change box color with linkbordercolor)
    citecolor=green,        % color of links to bibliography
    filecolor=magenta,      % color of file links
    urlcolor=cyan           % color of external links
}

\title{Hazard Analysis\\\progname}

\author{\authname}

\date{}

%% Comments

\usepackage{color}

\newif\ifcomments\commentstrue %displays comments
%\newif\ifcomments\commentsfalse %so that comments do not display

\ifcomments
\newcommand{\authornote}[3]{\textcolor{#1}{[#3 ---#2]}}
\newcommand{\todo}[1]{\textcolor{red}{[TODO: #1]}}
\else
\newcommand{\authornote}[3]{}
\newcommand{\todo}[1]{}
\fi

\newcommand{\wss}[1]{\authornote{blue}{SS}{#1}} 
\newcommand{\plt}[1]{\authornote{magenta}{TPLT}{#1}} %For explanation of the template
\newcommand{\an}[1]{\authornote{cyan}{Author}{#1}}

%% Common Parts

\newcommand{\progname}{Software Engineering} % PUT YOUR PROGRAM NAME HERE
\newcommand{\authname}{Team \#11, OKKM Insights
\\ Mathew Petronilho
\\ Oleg Glotov
\\ Kyle McMaster
\\ Kartik Chaudhari} % AUTHOR NAMES                  

\usepackage{hyperref}
    \hypersetup{colorlinks=true, linkcolor=blue, citecolor=blue, filecolor=blue,
                urlcolor=blue, unicode=false}
    \urlstyle{same}
                                


\begin{document}

\maketitle
\thispagestyle{empty}

~\newpage

\pagenumbering{roman}

\begin{table}[hp]
\caption{Revision History} \label{TblRevisionHistory}
\begin{tabularx}{\textwidth}{llX}
\toprule
\textbf{Date} & \textbf{Developer(s)} & \textbf{Change}\\
\midrule
October 25 & Kyle, Kartik, Mathew, Oleg & Rev 0\\
\bottomrule
\end{tabularx}
\end{table}

~\newpage

\tableofcontents

~\newpage

\pagenumbering{arabic}


\section{Introduction}
This document presents the hazard analysis for OrbitWatch, an online application that uses an AI-powered crowdsourcing model to build labeled satellite image datasets.
A hazard in the context of our application is any condition or event that could lead to undesirable outcomes, such as system failures, data corruption, data leakage, performance degradation, or security vulnerabilities.
This document aims to detect, analyze, and mitigate potential hazards that are applicable to our application.

\section{Scope and Purpose of Hazard Analysis}

\subsection{Scope}
The Hazard Analysis for the OKKM Insights platform is confined to identifying and evaluating potential hazards within our internal system architecture, specifically focusing on the API backend, database, and AI model components. This analysis combines all risks associated with the application's functionality, performance, and security that could adversely affect the system's operation.
\subsection{Purpose}
The primary purpose of this Hazard Analysis is to systematically identify, assess, and mitigate risks that could lead to significant losses for OKKM Insights. The potential losses associated with these hazards include:
\begin{enumerate}
    \item Monetary Losses: Prolonged platform unavailability can result in direct financial losses due to decreased service sales. Additionally, functionality disruptions may result in extra costs related to incident response and system recovery efforts.
    \item Client Attrition: Downtime or unreliable performance can decrease client confidence, leading to the loss of existing clients and deterring potential new clients from engaging with our services.
    \item User Attrition: A subpar user experience, characterized by inefficiencies in task labeling or system responsiveness, can drive users away from the platform. High user turnover diminishes the quality and volume of labeled data, which is essential for training accurate AI models.
\end{enumerate}
By acknowledging these potential losses, the Hazard Analysis underscores the importance of proactive risk management in maintaining the platform's reliability, security, and user satisfaction.

\section{System Boundaries and Components}

\subsection{Front End (FE)}
\textbf{Potential Hazards:}
\begin{itemize}
    \item \textbf{User Input Errors:} The frontend may not properly validate inputs (e.g., incorrectly labeled data or unauthorized access attempts).
    \item \textbf{UI Design Flaws:} Poor usability could cause users to make mistakes, such as submitting incorrect labels or missing important instructions.
    \item \textbf{Cross-Site Scripting (XSS):} JavaScript code that improperly checks user inputs could allow malicious scripts to be executed in the user's browser.
    \item \textbf{Session Hijacking:} Improper management of session tokens and cookies could allow attackers to impersonate users.
\end{itemize}
\textbf{Mitigations:}
\begin{itemize}
    \item Implement thorough input validation and provide clear feedback to users. 
    Input validation will include restricting input types, using regular expressions to determine that text inputs are of the correct form, and checking that submitted images have at least one label or choice selected. 
    We will deter bad image labels by linking accuracy of a labeler with their rewards, and giving less weight to the labels of an inaccurate labeler when aggregating labeled images.
    \item Ensure that critical actions like labeling are supported by warnings or confirmations to reduce errors.
    \item Use Content Security Policy (CSP) headers and HTML escaping techniques to prevent XSS attacks.
    \item Secure cookies with \texttt{HttpOnly} and \texttt{Secure} flags and employ strong session management practices.
\end{itemize}

\subsection{Back End (BE)}
\subsubsection{Login Service}
\textbf{Potential Hazards:}
\begin{itemize}
    \item \textbf{Authentication Failure:} Unauthorized access could allow malicious actors to label data incorrectly or access sensitive customer information.
    \item \textbf{Session Hijacking:} If session management is not handled correctly, attackers could hijack valid user sessions.
\end{itemize}
\textbf{Mitigations:}
\begin{itemize}
    \item Implement strong authentication mechanisms (e.g., two-factor authentication) and monitor for suspicious login activity.
    \item Secure cookies and session tokens, implementing timeout features.
\end{itemize}

\subsubsection{Money Service}
\textbf{Potential Hazards:}
\begin{itemize}
    \item \textbf{Payment Failures:} Incorrect payments could cause disputes between the labelers and the platform, leading to operational delays.
    \item \textbf{Financial Data Breaches:} Weak encryption or improper payment processor integration could expose sensitive financial data.
\end{itemize}
\textbf{Mitigations:}
\begin{itemize}
    \item Secure payment processing with encryption and multi-step verification for payout transactions.
    \item Implement industry-standard encryption for payment details and integrate with PCI-DSS-compliant payment gateways.
\end{itemize}

\subsubsection{Task Delegation Service}
\textbf{Potential Hazards:}
\begin{itemize}
    \item \textbf{Task Misallocation:} If tasks are assigned to unqualified labelers, the quality of data labeling could decrease, leading to flawed training data.
\end{itemize}
\textbf{Mitigations:}
\begin{itemize}
    \item Employ task delegation algorithms that consider labelers’ accuracy and expertise.
\end{itemize}

\subsection{Database (DB)}
\textbf{Potential Hazards:}
\begin{itemize}
    \item \textbf{Data Corruption or Loss:} Corrupted or missing data, especially labeled data, can introduce significant inaccuracies in model training and analysis.
    \item \textbf{SQL Injection:} If queries are not parameterized, attackers may manipulate query inputs, resulting in unauthorized access or data manipulation.
\end{itemize}
\textbf{Mitigations:}
\begin{itemize}
    \item Implement backups and data redundancy mechanisms to prevent loss or corruption.
    \item Use parameterized queries or prepared statements to prevent SQL injection.
    \item Ensure proper data validation before storing data.
\end{itemize}

\subsection{Machine Learning Task Allocation Model}
\textbf{Potential Hazards:}
\begin{itemize}
    \item \textbf{Model Misconfiguration:} The allocation model may incorrectly assign tasks if it misinterprets labeler performance data, leading to inefficiency and low-quality labeled datasets.
    \item \textbf{Bias in Task Allocation:} If the model favors certain users, this could introduce biases in the labeling process, affecting data accuracy.
    \item \textbf{Adversarial Attacks:} Machine learning models might be vulnerable to adversarial attacks that can skew the task allocations.
\end{itemize}
\textbf{Mitigations:}
\begin{itemize}
    \item Continuously monitor and test the task allocation model’s performance and adjust for fairness and accuracy.
    \item Implement adversarial training or model-hardening techniques to defend against attacks.
\end{itemize}

\subsection{Libraries (e.g., BoTorch, TensorFlow, PyTorch)}
\textbf{Potential Hazards:}
\begin{itemize}
    \item \textbf{Library Bugs or Vulnerabilities:} External libraries may have bugs or security vulnerabilities that can affect system performance or introduce security risks.
    \item \textbf{Version Conflicts:} Dependency issues may arise when integrating BoTorch with other machine learning libraries.
    \item \textbf{Optimization Failures:} Incorrect hyperparameter optimization might result in poor allocation of labeling tasks.
    \item \textbf{Adversarial Model Attacks:} TensorFlow or PyTorch models may be susceptible to adversarial attacks, leading to incorrect classifications.
\end{itemize}
\textbf{Mitigations:}
\begin{itemize}
    \item Regularly update libraries and conduct security audits to ensure all dependencies are secure.
    \item Use virtual environments or Docker to manage dependencies and avoid conflicts.
    \item Rigorously test optimization outcomes using test datasets.
    \item Implement adversarial defenses, such as adversarial training and data augmentation.
\end{itemize}

\subsection{Python (General Back-End Processing)}
\textbf{Potential Hazards:}
\begin{itemize}
    \item \textbf{Performance Bottlenecks:} Python can introduce performance issues, especially in CPU-bound processes.
    \item \textbf{Memory Leaks:} Improper memory management can cause resource exhaustion, leading to crashes.
    \item \textbf{Security Vulnerabilities:} Since Python is dynamically typed, unexpected inputs could lead to runtime errors, potentially compromising data integrity.
\end{itemize}
\textbf{Mitigations:}
\begin{itemize}
    \item Use Cython or external compiled languages for performance-critical tasks.
    \item Regularly run memory profiling tools to ensure no memory leaks.
    \item Implement input validation and use static analysis tools like \texttt{bandit} to enforce security best practices.
\end{itemize}

\subsection{Docker (Containerization)}
\textbf{Potential Hazards:}
\begin{itemize}
    \item \textbf{Container Breakout:} Misconfigurations or unpatched vulnerabilities in Docker containers could allow attackers to escape the container and gain access to the host system.
    \item \textbf{Resource Exhaustion:} Poor resource management inside containers could lead to resource exhaustion, affecting system performance.
\end{itemize}
\textbf{Mitigations:}
\begin{itemize}
    \item Use security best practices such as running containers with the least privilege, using \texttt{Seccomp} profiles, and isolating sensitive workloads.
    \item Set resource limits on containers to prevent exhaustion of host system resources.
\end{itemize}

\section{Critical Assumptions}

In developing the OKKM Insights platform, several critical assumptions have to be made about the system's functionality and reliability. These assumptions are necessary to focus the development process and mitigation efforts on areas with the highest potential hazards.

\subsection*{Third-Party Libraries and Maintenance}
\begin{itemize}
  \item \textbf{Assumption:} All utilized third-party libraries will remain freely available and continue to be actively maintained by their respective developers.
  \item \textbf{Rationale:} The platform relies on these libraries for core functionalities such as image processing, web development, and AI model training. Assuming ongoing maintenance and availability allows the team to focus on building platform-specific features without diverting resources to manage library-related issues.
\end{itemize}

\subsection*{Consistency of Public APIs}
\begin{itemize}
  \item \textbf{Assumption:} Public APIs integrated into the platform will maintain consistency in their logic, structure, and functionality, ensuring uninterrupted access to required data.
  \item \textbf{Rationale:} Reliable access to satellite imagery and other data sources through stable APIs is crucial for the platform's operations. Consistent APIs ensure that data retrieval processes remain functional without constant adjustments.
\end{itemize}

\subsection*{User Participation and Engagement}
\begin{itemize}
  \item \textbf{Assumption:} A sufficient number of users will engage with the platform to provide timely and accurate image labeling.
  \item \textbf{Rationale:} The success of the crowd-sourcing model depends on active user participation. Assuming consistent user engagement ensures that datasets can be built and updated efficiently.
\end{itemize}

\subsection*{Scalability of Cloud Infrastructure}
\begin{itemize}
  \item \textbf{Assumption:} The chosen cloud infrastructure will scale efficiently to handle varying loads without performance hits.
  \item \textbf{Rationale:} The platform's ability to handle large volumes of data and user interactions relies on scalable cloud services. Assuming scalability ensures that the system can grow with increasing demand.
\end{itemize}

\subsection*{Financial Stability for Compensation}
\begin{itemize}
  \item \textbf{Assumption:} Adequate financial resources will be available to compensate users for their labeling efforts.
  \item \textbf{Rationale:} Reliable compensation is essential for sustaining user participation and promoting high-quality data labeling.
\end{itemize}

\newpage
\section{Failure Mode and Effect Analysis}

\setlength\LTleft{-4.3cm}
\scriptsize
\begin{longtable}{|l|l|l|l|l|l|l|l|}
\hline
\textbf{\begin{tabular}[c]{@{}l@{}}Design \\ Function\end{tabular}} &
  \textbf{\begin{tabular}[c]{@{}l@{}}Failure \\ Modes\end{tabular}} &
  \textbf{Effects of Failure} &
  \textbf{Causes of Failure} &
  \textbf{Detection} &
  \textbf{Recommended Action} &
  \textbf{SR} &
  \textbf{Ref} \\ \hline
\endfirsthead
%
\endhead
%
\multirow{2}{*}{\begin{tabular}[c]{@{}l@{}}Account \\ Creation\end{tabular}} &
  \begin{tabular}[c]{@{}l@{}}User already \\ exists\end{tabular} &
  \begin{tabular}[c]{@{}l@{}}User can not \\ create an\\ account\end{tabular} &
  Email is duplicated &
  \begin{tabular}[c]{@{}l@{}}Compare the email \\ entered with the user \\ database records\\ to see if the email is in use\end{tabular} &
  \begin{tabular}[c]{@{}l@{}}Notify the user that the email \\ is associated with another\\  account\\ \\ Prompt them to give another \\ email or sign in with the one \\ they entered\end{tabular} &
   SE2 &
  H1-1 \\ \cline{2-8} 
 &
  \begin{tabular}[c]{@{}l@{}}Invalid input \\ syntax and \\ length\end{tabular} &
  \begin{tabular}[c]{@{}l@{}}User can not\\ create an\\ account\end{tabular} &
  \begin{tabular}[c]{@{}l@{}}1) Email is not valid\\ 2) Password is not \\ strong enough\end{tabular} &
  \begin{tabular}[c]{@{}l@{}}1) Use regular expressions \\ to  detect if the string \\ pattern  is valid\\ 2) Use regular expressions \\ to detect if all password \\ requirements are met\end{tabular} &
  \begin{tabular}[c]{@{}l@{}}1) Notify the user that they must\\ enter a valid email and give an \\ example of a valid email\\ 2) Tell the user what password\\ requirements they have and \\ have not satisfied\end{tabular} &
   SE2,3 &
  H1-2 \\ \hline
\multirow{2}{*}{Log In} &
  \begin{tabular}[c]{@{}l@{}}Incorrect \\ credentials \\ entered\end{tabular} &
  \begin{tabular}[c]{@{}l@{}}User can not\\ access application\end{tabular} &
  \begin{tabular}[c]{@{}l@{}}1) No account with\\ the entered email\\ exists\\ 2) Password does\\ not match records\end{tabular} &
  \begin{tabular}[c]{@{}l@{}}1) Compare email entered \\ with database records to \\ see if account exists\\ 2) Compare password \\ entered with what is stored \\ in the database for\\ the entered email\end{tabular} &
  \begin{tabular}[c]{@{}l@{}}1) Tell user account does not\\ exist and prompt them to \\ make one\\ 2) Tell user password is incorrect\\ and prompt password recovery\end{tabular} &
   SE0,1&
  H2-1 \\ \cline{2-8} 
 &
  \begin{tabular}[c]{@{}l@{}}Excessive \\ permissions \\ given\end{tabular} &
  \begin{tabular}[c]{@{}l@{}}Users can perform\\ unauthorized\\ actions\end{tabular} &
  \begin{tabular}[c]{@{}l@{}}Application paths\\ are unprotected\end{tabular} &
  \begin{tabular}[c]{@{}l@{}}Check user login token \\ each time a new page of \\ the website is accessed\end{tabular} &
  \begin{tabular}[c]{@{}l@{}}Tell the user to log in\\ \\ Deny access if they try to \\ access a \\ page they should not\end{tabular} &
  SE0,1 &
  H2-2 \\ \hline
\multirow{4}{*}{\begin{tabular}[c]{@{}l@{}}Labeling \\ Satallite \\ Images\end{tabular}} &
  \begin{tabular}[c]{@{}l@{}}Internet \\ connection \\ is lost\end{tabular} &
  \begin{tabular}[c]{@{}l@{}}Users can not \\ submit labeled\\ images or \\ navigate the website\end{tabular} &
  \begin{tabular}[c]{@{}l@{}}Internet connection\\ is weak or power\\ is lost\end{tabular} &
  \begin{tabular}[c]{@{}l@{}}Device shows no internet\\ connection\end{tabular} &
  \begin{tabular}[c]{@{}l@{}}Any labeled photos or created \\ projects that have already been\\ submitted have been saved\\ \\ Progress is resumed when \\ connection is re-established\end{tabular} &
   SE4&
  H3-1 \\ \cline{2-8} 
 &
  \begin{tabular}[c]{@{}l@{}}Application is \\ closed\end{tabular} &
  Same as H3-1 &
  \begin{tabular}[c]{@{}l@{}}Power outage or\\ misclick\end{tabular} &
  \begin{tabular}[c]{@{}l@{}}Application is no longer \\ running on the users \\ device\end{tabular} &
  \begin{tabular}[c]{@{}l@{}}Any labeled photos or created\\ projects that have already been\\ submitted have been saved\\ \\ Progress is resumed on log in\end{tabular} &
   SE4&
  H3-2 \\ \cline{2-8} 
 &
  \begin{tabular}[c]{@{}l@{}}Unlabelled data \\ is submitted\end{tabular} &
  \begin{tabular}[c]{@{}l@{}}Bad data is \\ added to the \\ dataset\end{tabular} &
  Misclick &
  \begin{tabular}[c]{@{}l@{}}On submission, \\ application checks that \\ there are as many\\ labels as requested \\ by the job\end{tabular} &
  \begin{tabular}[c]{@{}l@{}}Reject a submission if no \\ labeling was done\end{tabular} &
  SE5 & 
  H3-3 \\ \cline{2-8} 
 &
  \begin{tabular}[c]{@{}l@{}}Mass labeling \\ done too quickly\end{tabular} &
  \begin{tabular}[c]{@{}l@{}}Bad data is \\ added and \\ reward system\\ is abused\end{tabular} &
  \begin{tabular}[c]{@{}l@{}}Bots have been\\ deployed to make\\ quick labels\end{tabular} &
  \begin{tabular}[c]{@{}l@{}}User is submitting data \\ at an unreasonable \\ speed\end{tabular} &
  \begin{tabular}[c]{@{}l@{}}Implement a submission cool \\ down to prevent \\ bot submissions \\ \\ Reward system is based \\ on accuracy\end{tabular} &
  SE5 &
  H3-4 \\ \hline
\multirow{2}{*}{\begin{tabular}[c]{@{}l@{}}Backend \\ Server \\ and \\ API \\ Requests\end{tabular}} &
  Server crashes &
  \begin{tabular}[c]{@{}l@{}}All services \\ provided\\ by the server \\ are down\end{tabular} &
  \begin{tabular}[c]{@{}l@{}}Software error on\\ server side\end{tabular} &
  Error found in logs &
  \begin{tabular}[c]{@{}l@{}}Monitor errors in logs\\ \\ Notify users that the \\ server is down\end{tabular} &
   SE6&
  H4-1 \\ \cline{2-8} 
 &
  \begin{tabular}[c]{@{}l@{}}API is not \\ responding\end{tabular} &
  \begin{tabular}[c]{@{}l@{}}All services \\ provided\\ by an API \\ do not work\end{tabular} &
  \begin{tabular}[c]{@{}l@{}}API service provider\\ is down or\\ overwhelmed\end{tabular} &
  \begin{tabular}[c]{@{}l@{}}Response from the \\ API has an error code\end{tabular} &
  \begin{tabular}[c]{@{}l@{}}Retry all API requests after a \\ specific amount of time\\ \\ Monitor errors in logs\end{tabular} &
   SE6&
  H4-2 \\ \hline
\multirow{3}{*}{Data Storage} &
  \begin{tabular}[c]{@{}l@{}}User account is\\ compromised\end{tabular} &
  \begin{tabular}[c]{@{}l@{}}User info is\\ exposed and \\ they will be\\ dissatisfied \\ with the\\ application\end{tabular} &
  \begin{tabular}[c]{@{}l@{}}Lack of encryption\\ and protection of\\ sensitive infomation\end{tabular} &
  \begin{tabular}[c]{@{}l@{}}User notifies the team \\ of lost reward balance \\ or lost account access\end{tabular} &
  \begin{tabular}[c]{@{}l@{}}Ensure user passwords are \\ encrpyted when stored\\ \\ Ensure financial transactions \\ are secure\\ \\ Password reset occurs \\ through a trusted \\ source such as email\end{tabular} &
   SE8,9 &
  H5-1 \\ \cline{2-8} 
 &
  \begin{tabular}[c]{@{}l@{}}Duplicate entry \\ occurs\end{tabular} &
  \begin{tabular}[c]{@{}l@{}}Data inconsistency,\\ unneccesary \\ storage\\ usage, and slower\\ query performance\end{tabular} &
  \begin{tabular}[c]{@{}l@{}}Lack of \\ constraints/validation\end{tabular} &
  \begin{tabular}[c]{@{}l@{}}Check the database \\ entries\end{tabular} &
  \begin{tabular}[c]{@{}l@{}}Ensure the database has \\ unique keys\\ \\ Set up a duplicate key \\ procedure on the\\ database\end{tabular} &
  SE7 &
  H5-2 \\ \cline{2-8} 
 &
  \begin{tabular}[c]{@{}l@{}}Database is \\ compromised\end{tabular} &
  \begin{tabular}[c]{@{}l@{}}Data inconsistency,\\ malicous entries, \\ and data leaks\end{tabular} &
  SQL injection &
  Check database entries &
  \begin{tabular}[c]{@{}l@{}}Use parameterized queries\\ \\ Avoid dynamic SQL strings\end{tabular} &
  SE10 &
  H5-3 \\ \hline
\end{longtable}

\normalsize
\newpage
\section{Safety and Security Requirements}

A comprehensive list of requirements can be found in the \href{https://github.com/OKKM-insights/OKKM.insights/blob/main/docs/SRS/SRS.pdf}{SRS}.

\subsubsection*{NFR-SE4}
\begin{itemize}
  \item \textbf{Description:} The application shall prohibit new user actions when user is disconnected from the system.
  \item \textbf{Rationale:} The system should avoid batch updates upon reconnect as to prevent malicious users creating a false package of updates.
  \item \textbf{Fit Criterion:} Upon user disconnect and reconnect, the system will not accept any updates from the user not submitted before the disconnect occured.
\end{itemize}

\subsubsection*{NFR-SE6}
\begin{itemize}
  \item \textbf{Description:} The system shall, during system failure, notify users trying to make a request of the system failure.
  \item \textbf{Rationale:} The system should provide good warning messages to users to prevent additional frustration.
  \item \textbf{Fit Criterion:} When services are unavailable, the system shall provide an error message to the user making the request of the unavailable service.
\end{itemize}

\subsubsection*{NFR-SE7}
\begin{itemize}
  \item \textbf{Description:} The system shall reject duplicate entries in database.
  \item \textbf{Rationale:} Each entry to the database should be a unique data piece. If two entries are identical, they must differ by at least the creation time.
  \item \textbf{Fit Criterion:} No two entries in the database shall exist together for longer than 24 hours from their introduction into the database.
\end{itemize}



\section{Roadmap}

\textbf{Primary Requirements}
The first priority for the system is to ensure integrity in the labels collected. This means we will first prioritize requirements
related to the user authentication process (
\textbf{NFR-SE0, NFR-SE1}), data integrity (
  \textbf{NFR-SE5, NFR-SE6, NFR-SE7}), and user privacy(
    \textbf{NFR-SE8, NFR-SE9}).

\textbf{Secondary Requirements}
Additionally, the system must also be prepared for malicious actions from users. This is out of scope for completion during the capstone course, but must be implemented
before exposing the system to a large pool of users. These requirements are related to malicious inputs (
  \textbf{NFR-SE10}) and potentially harmful inputs (
    \textbf{NFR-SE2, NFR-SE3}). In the event of a system failure, we would also like to ensure that error messages are helpful to users (
      \textbf{NFR-SE4}).

\newpage{}

\section*{Appendix --- Reflection}

The purpose of reflection questions is to give you a chance to assess your own
learning and that of your group as a whole, and to find ways to improve in the
future. Reflection is an important part of the learning process.  Reflection is
also an essential component of a successful software development process.  

Reflections are most interesting and useful when they're honest, even if the
stories they tell are imperfect. You will be marked based on your depth of
thought and analysis, and not based on the content of the reflections
themselves. Thus, for full marks we encourage you to answer openly and honestly
and to avoid simply writing ``what you think the evaluator wants to hear.''

Please answer the following questions.  Some questions can be answered on the
team level, but where appropriate, each team member should write their own
response:


\begin{enumerate}
    \item What went well while writing this deliverable?
    As a team, we have a lot more experience working with eachother and assigning tasks based on expertise. This showed during this stage of the project. We were able to
    assign the tasks that could be completed independently so that our team meeting was focused on brainstorming the parts that would be better to work on together. Truthfully, we 
    were unable to work much on this deliverable until midway through last week (with the exception of Mathew, who provided the team with the first iteration of the harzard table). If we 
    didn't have the experience and trust in each other, it would have been much more stressful to ensure completion of this document.
    
    During this deliverable, we had the opportunity to research topics related to the app’s safety and the threats it may be subjected to. Previously, when writing about hazards, 
    we were only aware of some basic risks that we had heard about since childhood. However, investigating each topic related to our SRS made us think in-depth about other potential hazards. 
    We would say the aspect of research based on existing SRS went really well, as we learned a lot of new things. Although this deliverable was relatively shorter than the previous ones, we developed the habit of starting early. In the back of my mind, we knew what to 
    do and what was coming early on when this assignment was released. Another positive aspect was starting this assignment earlier and establishing a flow of ideas.

    \item What pain points did you experience during this deliverable, and how
    did you resolve them?
    As discussed above, there was a real challenge for the team to allocate time towards this deliverable. The team had personal commitments over reading week
    and 2/3 midterms in the first days back from reading week. We resolved them through a carefully planned team meeting, which was designed to gain alignment on any team decisions
    early in the week, so that the team could complete their assigned work when they had time to spare.

    Another pain point was identifying specific issues related to our project. The libraries we are going to use aren’t definitive yet, and since they may change in the future, 
    we had to anticipate potential hazards concerning future libraries we might use. We did write specific risks related to a few of them, 
    but generalizing these risks was challenging, especially predicting the problems we might encounter, which is a necessary skill to develop.
    

    \item Which of your listed risks had your team thought of before this
    deliverable, and which did you think of while doing this deliverable? For
    the latter ones (ones you thought of while doing the Hazard Analysis), how
    did they come about?
    We were aware of risks to the data integrity before this deliverable. We have been thinking to methods to ensure that users are rewarded for being consistently accurate. Kartik also 
    made us aware of SQL injection and API vunerabilities, which he had learned about through his internship. Another thing that was raised was that deprecated libraries can cause significant 
    issues and how important it is to use stable versions during the development phase.\\
    One risk that we discovered during this deliverable was around failures for the system to operate as intended, mostly related to API/ internet failures. Mathew first raised this issue, and
    as a team we discussed how the system should respond. We decided that if a client disconnects during a labeling session, we should reject labels sent to us after the disconnection and before
    we can reestablish connection with the user. They may have legitimate labels from this window if the system sends a buffer of images, but there is a risk that a malicious user could prepare 
    a package of automatically generated fake labels upon reconnection to increase their earnings. This bot activity would be hard to detect and mitigate, so we have decided to be conservative and block all labels
    made when not actively connected.

    \item Other than the risk of physical harm (some projects may not have any
    appreciable risks of this form), list at least 2 other types of risk in
    software products. Why are they important to consider?
    A big risk is privacy. Users trust us to keep their personal information secure. In the past, the only way a malicious agent could collect so much personal data at one time 
    would be to break into a public records building. Now, there are thousands (maybe millions) of possible places they could get it. And they only need to find the one weakest link to break in. 
    It is the responsibility of every software designer to build with security at top of mind. This will ensure that the users are respected and protected.
    Also, there is a risk of behaviour modification from the use of software. This is harder to plan for, but after the introduction of software such as social media sites, it is unboubtably present.
    Misinformation, social isolation, and low-self image have all been in some way attributed to use of social media. The products we build should enhance the lives of the users they are built for
    and designing a site which exacerbates these symptoms is irresponsible.

\end{enumerate}
\newpage
\section{References}
\sloppy

\begin{itemize}
    \item OWASP. Cross-Site Scripting (XSS). Available at: \url{https://owasp.org/www-community/attacks/xss/}.
    \item The SSL Store. The Ultimate Guide to Session Hijacking (aka Cookie Hijacking). Available at: \url{https://www.thesslstore.com/blog/the-ultimate-guide-to-session-hijacking-aka-cookie-hijacking/}.
    \item Mozilla Developer Network. Content Security Policy (CSP). Available at: \url{https://developer.mozilla.org/en-US/docs/Web/HTTP/CSP}.
    \item Neontri. PCI DSS Compliance. Available at: \url{https://neontri.com/blog/pci-dss-compliance/}.
    \item Imperva. Session Hijacking. Available at: \url{https://www.imperva.com/learn/application-security/session-hijacking/}.
    \item Label Your Data. Bias in Machine Learning. Available at: \url{https://labelyourdata.com/articles/bias-in-machine-learning}.
    \item Reversing Labs. How to Harden ML Models Against Adversarial Attacks. Available at: \url{https://www.reversinglabs.com/blog/how-to-harden-ml-models-against-adversarial-attacks}.
    \item Hyperopt GitHub Issues. Hyperopt Version Conflicts. Available at: \url{https://github.com/hyperopt/hyperopt/issues/642}.
    \item HiddenLayer. What's in the Box? Understanding Adversarial Attacks on ML Models. Available at: \url{https://hiddenlayer.com/research/whats-in-the-box/}.
    \item Medium. Python Security Best Practices for Writing Secure Code. Available at: \url{https://medium.com/@VAISHAK_CP/python-security-best-practices-for-writing-secure-code-a6a9130e3748}.
    \item Docker Documentation. Seccomp Security Profiles. Available at: \url{https://docs.docker.com/engine/security/seccomp/}.
\end{itemize}
\sloppy

\end{document}
