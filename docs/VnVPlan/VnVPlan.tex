\documentclass[12pt, titlepage]{article}

\usepackage{booktabs}
\usepackage{tabularx}
\usepackage{hyperref}
\usepackage{enumitem}
\usepackage{amssymb}

\newlist{todolist}{itemize}{2}
\setlist[todolist]{label=$\square$}

\hypersetup{
    colorlinks,
    citecolor=blue,
    filecolor=black,
    linkcolor=red,
    urlcolor=blue
}
\usepackage[round]{natbib}

\input{../Comments}
%% Common Parts

\newcommand{\progname}{Software Engineering} % PUT YOUR PROGRAM NAME HERE
\newcommand{\authname}{Team \#11, OKKM Insights
\\ Mathew Petronilho
\\ Oleg Glotov
\\ Kyle McMaster
\\ Kartik Chaudhari} % AUTHOR NAMES                  

\usepackage{hyperref}
    \hypersetup{colorlinks=true, linkcolor=blue, citecolor=blue, filecolor=blue,
                urlcolor=blue, unicode=false}
    \urlstyle{same}
                                


\begin{document}

\title{System Verification and Validation Plan for \progname{}} 
\author{\authname}
\date{\today}
	
\maketitle

\pagenumbering{roman}

\section*{Revision History}

\begin{tabularx}{\textwidth}{p{3cm}p{2cm}X}
\toprule {\bf Date} & {\bf Version} & {\bf Notes}\\
\midrule
Date 1 & 1.0 & Notes\\
Date 2 & 1.1 & Notes\\
\bottomrule
\end{tabularx}

~\\
\wss{The intention of the VnV plan is to increase confidence in the software.
However, this does not mean listing every verification and validation technique
that has ever been devised.  The VnV plan should also be a \textbf{feasible}
plan. Execution of the plan should be possible with the time and team available.
If the full plan cannot be completed during the time available, it can either be
modified to ``fake it'', or a better solution is to add a section describing
what work has been completed and what work is still planned for the future.}

\wss{The VnV plan is typically started after the requirements stage, but before
the design stage.  This means that the sections related to unit testing cannot
initially be completed.  The sections will be filled in after the design stage
is complete.  the final version of the VnV plan should have all sections filled
in.}

\newpage

\tableofcontents

\listoftables
\wss{Remove this section if it isn't needed}

\listoffigures
\wss{Remove this section if it isn't needed}

\newpage

\section{Symbols, Abbreviations, and Acronyms}

\renewcommand{\arraystretch}{1.2}
\begin{tabular}{l l} 
  \toprule		
  \textbf{symbol} & \textbf{description}\\
  \midrule 
  T & Test\\
  \bottomrule
\end{tabular}\\

\wss{symbols, abbreviations, or acronyms --- you can simply reference the SRS
  \citep{SRS} tables, if appropriate}

\wss{Remove this section if it isn't needed}

\newpage

\pagenumbering{arabic}

This document ... \wss{provide an introductory blurb and roadmap of the
  Verification and Validation plan}

\section{General Information}

\subsection{Summary}

\wss{Say what software is being tested.  Give its name and a brief overview of
  its general functions.}

\subsection{Objectives}

\wss{State what is intended to be accomplished.  The objective will be around
  the qualities that are most important for your project.  You might have
  something like: ``build confidence in the software correctness,''
  ``demonstrate adequate usability.'' etc.  You won't list all of the qualities,
  just those that are most important.}

\wss{You should also list the objectives that are out of scope.  You don't have 
the resources to do everything, so what will you be leaving out.  For instance, 
if you are not going to verify the quality of usability, state this.  It is also 
worthwhile to justify why the objectives are left out.}

\wss{The objectives are important because they highlight that you are aware of 
limitations in your resources for verification and validation.  You can't do everything, 
so what are you going to prioritize?  As an example, if your system depends on an 
external library, you can explicitly state that you will assume that external library 
has already been verified by its implementation team.}

\subsection{Challenge Level and Extras}

\wss{State the challenge level (advanced, general, basic) for your project.
Your challenge level should exactly match what is included in your problem
statement.  This should be the challenge level agreed on between you and the
course instructor.  You can use a pull request to update your challenge level
(in TeamComposition.csv or Repos.csv) if your plan changes as a result of the
VnV planning exercise.}

\wss{Summarize the extras (if any) that were tackled by this project.  Extras
can include usability testing, code walkthroughs, user documentation, formal
proof, GenderMag personas, Design Thinking, etc.  Extras should have already
been approved by the course instructor as included in your problem statement.
You can use a pull request to update your extras (in TeamComposition.csv or
Repos.csv) if your plan changes as a result of the VnV planning exercise.}

\subsection{Relevant Documentation}

\wss{Reference relevant documentation.  This will definitely include your SRS
  and your other project documents (design documents, like MG, MIS, etc).  You
  can include these even before they are written, since by the time the project
  is done, they will be written.  You can create BibTeX entries for your
  documents and within those entries include a hyperlink to the documents.}

\citet{SRS}

\wss{Don't just list the other documents.  You should explain why they are relevant and 
how they relate to your VnV efforts.}

\section{Plan}

This section describes the team, verification plans for the system design and documentation, use of automated testing tools, and the validation plan of the software following implementation.

\subsection{Verification and Validation Team}
When not listed as lead, members of core team will support through team discussion and implementation of feedback.\\
\begin{center}
\begin{tabular}{|c|c|l|}
  \hline
  \textbf{Name} & \textbf{Role} & \textbf{Responsibilities}\\\hline
  Mathew Petronilho & Core team & \textbf{SRS} \\
  & & Lead review of SRS\\
  & & \textbf{Implementation}\\
  & & Support review of implementation\\
  & & Review Implementation for coding \\
  & & standards, comment quality\\
  \hline
  Oleg Glotov & Core team & \textbf{VnV Plan} \\
  & & Support review of VnV Plan\\
  & & Review VnV Plan for formatting and \\
  & & grammar errors\\
  & & \textbf{Implementation}\\
  & & Lead review of Implementation\\
  \hline
  Kartik Chaudhari & Core team & \textbf{Design} \\
  & & Support review of Design\\
  & & Review diagrams and documents for correct \\
  & & notation, formatting and grammar\\
  & & \textbf{VnV Plan}\\
  & & Lead review of VnV Plan\\
  & & \textbf{Implementation}\\
  & & Support review of implementation\\
  \hline
  Kyle McMaster & Core team & \textbf{SRS} \\
  & & Check SRS for formatting and grammar errors\\
  & & Support review of SRS\\
  & & \textbf{Design}\\
  & & Lead review of design\\
  & & Ensure team is following standard design principles\\
  \hline
  Dr. Swati Mishra & Project Supervisor & Validate all team docs in structured review\\\hline
  Capstone Team 10 & Primary Reviewers & Validate all team docs in \\
  & & async review through Git issues\\
  \hline
\end{tabular}
\end{center}

% \wss{Your teammates.  Maybe your supervisor.
%   You should do more than list names.  You should say what each person's role is
%   for the project's verification.  A table is a good way to summarize this information.}

\subsection{SRS Verification Plan}

The SRS will be verified through several channels. First, the document will be reviewed by another capstone team. This will help the team identify issues
which are obscured to the team, due to the additional time they have spent thinking about the project. We expect this feedback to generally consist of 
omitted definitions or unstated assumptions. Since the team is more familiar with the project, it is likely that we have some information which is obvious 
to the team, but is necessary to define for others. The team will collect feedback from our primary reviewers in the issue tracker.\\
The SRS will also be reviewed by our supervisor in a structured review meeting. The team will guide the review with the following checklist:\\\\
\textbf{Constraints \& Assumptions}\\
The following sections are informed by the constraints \& assumptions. Therefore, it is critical that the these are verified first.
\begin{todolist}
\item All constraints are correct and necessary
\item All constraints are unambigious
\item All constraints which should be present are present
\item All constraints are verifiable
\item All assumptions are correct and necessary
\item All assumptions are unambigious
\item All assumptions which should be present are present
\item All assumptions are verifiable
\end{todolist}
\textbf{Data Model}\\
The data model affects how the system will be decomposed in the future design. This in turn affects the requirements, so it should be verified next.
\begin{todolist}
  \item Data model is correct
  \item Data model is complete
  \item Each element of data dictionary is correctly described. There are no extra or missing attributes.
  \item Elements of data dictionary are unambigious
\end{todolist}
\textbf{Functional Requirements}\\
This section describes the functionality of the system. This will be useful for understanding the context of the NFRs.
\begin{todolist}
\item All requirements are correct and necessary
\item All requirements are unambigious
\item All requirements which should be present are present
\item All requirements are verifiable
\item All requirements are feasible
\item All requirements are traceable
\end{todolist}
\textbf{Non-Functional Requirements}\\
The NFRs have now been properly introduced by the other sections and should now be assessed.
\begin{todolist}
  \item All requirements are correct and necessary
  \item All requirements are unambigious
  \item All requirements which should be present are present
  \item All requirements are verifiable
  \item All requirements are feasible
  \item All requirements are traceable
  \end{todolist}
\textbf{General}\\
Each of the sections reviewed should also be monitored for the following criteria.
\begin{todolist}
  \item Document contains properly formatted title, table of contents, references and all necessary sections
  \item Tables and figures are correctly formatted
  \item No grammar errors are present
  \item Each required section is present
  \item All requirements are feasible
  \end{todolist}


% \wss{List any approaches you intend to use for SRS verification.  This may
%   include ad hoc feedback from reviewers, like your classmates (like your
%   primary reviewer), or you may plan for something more rigorous/systematic.}

% \wss{If you have a supervisor for the project, you shouldn't just say they will
% read over the SRS.  You should explain your structured approach to the review.
% Will you have a meeting?  What will you present?  What questions will you ask?
% Will you give them instructions for a task-based inspection?  Will you use your
% issue tracker?}

% \wss{Maybe create an SRS checklist?}

\subsection{Design Verification Plan}
Upon completion of the System Design, the team will verify the design. This verification will involve reviews from the primary reviewers, in the same manner as described above.
This again will provide fresh prospective on the design, and help identify any omitted or underexplained information. The verification will continue with a systematic review with The
project supervisor using the following checklist:\\\\

\textbf{Class Decomposition}\\
\begin{todolist}
  \item All functional requirements are covered by a class/subsystem
  \item All non-functional requirements are covered by a class/subsystem
  \item All classes are correctly sized. That is, each class should be concerned with one purpose
  \item When appropriate, interfaces and abstract classes are used
  \item All class attributes are complete and necessary
  \item When appropriate, design patterns are used to improve design clarity
  \item Function permissions are appropriate for all classes
  \item Classes cannot be simplified without degrading the understanability of the system
  \end{todolist}
  \textbf{General}\\
\begin{todolist}
  \item The types of all attributes are listed
  \item The argument and return types of all functions are listed
  \item Correct notation is used to describe class relationships
  \item Class diagram is legible
  \end{todolist}
% \wss{Plans for design verification}

% \wss{The review will include reviews by your classmates}

% \wss{Create a checklists?}

\subsection{Verification and Validation Plan Verification Plan}
This document must be verified and validated, to ensure the validation and verification of other artifacts is correct. Like other documents, this will include review
from our primary reviewers, as well as a review with the project supervisor. The review with the project supervisor will be guided by the folllowing checklist:\\

\textbf{Testing Plan}\\
\begin{todolist}
  \item Testing plan includes a checklist for each artifact
  \item Checklist items are unambigious
  \item Checklist contains all recessary items
  \item Checklist does not contain unnecessary items
  \item Plan includes review from multiple parties
  \item Plan includes method for collecting feedback on artifacts
  \item Plan is verifiable
  \end{todolist}
\textbf{Tests}\\
\begin{todolist}
  \item All functional requirements are covered by one or more tests\\
  \item All non-functional requirements are covered by one or more tests\\
  \item All `units' are sufficiently covered by one or more tests\\
  \item Tests are unambigious
  \item Tests are verifiable
  \item Tests are repeatable
  \end{todolist}

% \wss{The verification and validation plan is an artifact that should also be
% verified.  Techniques for this include review and mutation testing.}

% \wss{The review will include reviews by your classmates}

% \wss{Create a checklists?}

\subsection{Implementation Verification Plan}

As part of the implementation verification plan, the team will conduct a series of static tests. One of which, will be a code walkthrough with members of the team.
Before submitting Rev 0 and Rev 1, the members of the team will schedule time to walk through other members of the team through the code they have written. This will be 
an opportunity for the team to look for code quality and adherence to standards, correctness, and alignment to the system design. During this walkthrough, we expect that 
the lead developer of the code being displayed will explain the overall flow and control structures. In doing so, we expect they will find the errors in their own code.\\\\
The team will also use a suite of tests, which is described in section 4.\\\\
The team will used static analyzers to verify adherence to coding standards to perform type checking. Please refer to section 7.2.2 and 7.2.3 in the Development Plan found \href{https://github.com/OKKM-insights/OKKM.insights/blob/main/docs/DevelopmentPlan/DevelopmentPlan.pdf}{here}.

% \wss{You should at least point to the tests listed in this document and the unit
%   testing plan.}

% \wss{In this section you would also give any details of any plans for static
%   verification of the implementation.  Potential techniques include code
%   walkthroughs, code inspection, static analyzers, etc.}

% \wss{The final class presentation in CAS 741 could be used as a code
% walkthrough.  There is also a possibility of using the final presentation (in
% CAS741) for a partial usability survey.}

\subsection{Automated Testing and Verification Tools}

Please refer to section 7.2.2 and 7.2.3 in the Development Plan found \href{https://github.com/OKKM-insights/OKKM.insights/blob/main/docs/DevelopmentPlan/DevelopmentPlan.pdf}{here}.

% \wss{What tools are you using for automated testing.  Likely a unit testing
%   framework and maybe a profiling tool, like ValGrind.  Other possible tools
%   include a static analyzer, make, continuous integration tools, test coverage
%   tools, etc.  Explain your plans for summarizing code coverage metrics.
%   Linters are another important class of tools.  For the programming language
%   you select, you should look at the available linters.  There may also be tools
%   that verify that coding standards have been respected, like flake9 for
%   Python.}

% \wss{If you have already done this in the development plan, you can point to
% that document.}

% \wss{The details of this section will likely evolve as you get closer to the
%   implementation.}

\subsection{Software Validation Plan}
  As discussed in section 6.2 of the Problem Statement and Goals, found \href{https://github.com/OKKM-insights/OKKM.insights/blob/main/docs/ProblemStatementAndGoals/ProblemStatement.pdf}{here}, 
  we will validate our software through user testing. More specifically, we will conduct testing of the user interface with peers, who will fill in a survey containing both qualitative and quantitative questions.
  This survery will be used to track progress on the usability of the software by comparing results collecting during different iterations of the user interface. See the appendix for a sample of the user survey.

% \wss{If there is any external data that can be used for validation, you should
%   point to it here.  If there are no plans for validation, you should state that
%   here.}

% \wss{You might want to use review sessions with the stakeholder to check that
% the requirements document captures the right requirements.  Maybe task based
% inspection?}

% \wss{For those capstone teams with an external supervisor, the Rev 0 demo should 
% be used as an opportunity to validate the requirements.  You should plan on 
% demonstrating your project to your supervisor shortly after the scheduled Rev 0 demo.  
% The feedback from your supervisor will be very useful for improving your project.}

% \wss{For teams without an external supervisor, user testing can serve the same purpose 
% as a Rev 0 demo for the supervisor.}

% \wss{This section might reference back to the SRS verification section.}

\section{System Tests}

\wss{There should be text between all headings, even if it is just a roadmap of
the contents of the subsections.}

\subsection{Tests for Functional Requirements}

\wss{Subsets of the tests may be in related, so this section is divided into
  different areas.  If there are no identifiable subsets for the tests, this
  level of document structure can be removed.}

\wss{Include a blurb here to explain why the subsections below
  cover the requirements.  References to the SRS would be good here.}

\subsubsection{Area of Testing1}

\wss{It would be nice to have a blurb here to explain why the subsections below
  cover the requirements.  References to the SRS would be good here.  If a section
  covers tests for input constraints, you should reference the data constraints
  table in the SRS.}
		
\paragraph{Title for Test}

\begin{enumerate}

\item{test-id1\\}

Control: Manual versus Automatic
					
Initial State: 
					
Input: 
					
Output: \wss{The expected result for the given inputs.  Output is not how you
are going to return the results of the test.  The output is the expected
result.}

Test Case Derivation: \wss{Justify the expected value given in the Output field}
					
How test will be performed: 
					
\item{test-id2\\}

Control: Manual versus Automatic
					
Initial State: 
					
Input: 
					
Output: \wss{The expected result for the given inputs}

Test Case Derivation: \wss{Justify the expected value given in the Output field}

How test will be performed: 

\end{enumerate}

\subsubsection{Area of Testing2}

...

\subsection{Tests for Nonfunctional Requirements}

\wss{The nonfunctional requirements for accuracy will likely just reference the
  appropriate functional tests from above.  The test cases should mention
  reporting the relative error for these tests.  Not all projects will
  necessarily have nonfunctional requirements related to accuracy.}

\wss{For some nonfunctional tests, you won't be setting a target threshold for
passing the test, but rather describing the experiment you will do to measure
the quality for different inputs.  For instance, you could measure speed versus
the problem size.  The output of the test isn't pass/fail, but rather a summary
table or graph.}

\wss{Tests related to usability could include conducting a usability test and
  survey.  The survey will be in the Appendix.}

\wss{Static tests, review, inspections, and walkthroughs, will not follow the
format for the tests given below.}

\wss{If you introduce static tests in your plan, you need to provide details.
How will they be done?  In cases like code (or document) walkthroughs, who will
be involved? Be specific.}

\subsubsection{Area of Testing1}
		
\paragraph{Title for Test}

\begin{enumerate}

\item{test-id1\\}

Type: Functional, Dynamic, Manual, Static etc.
					
Initial State: 
					
Input/Condition: 
					
Output/Result: 
					
How test will be performed: 
					
\item{test-id2\\}

Type: Functional, Dynamic, Manual, Static etc.
					
Initial State: 
					
Input: 
					
Output: 
					
How test will be performed: 

\end{enumerate}

\subsubsection{Area of Testing2}

...

\subsection{Traceability Between Test Cases and Requirements}

\wss{Provide a table that shows which test cases are supporting which
  requirements.}

\section{Unit Test Description}

\wss{This section should not be filled in until after the MIS (detailed design
  document) has been completed.}

\wss{Reference your MIS (detailed design document) and explain your overall
philosophy for test case selection.}  

\wss{To save space and time, it may be an option to provide less detail in this section.  
For the unit tests you can potentially layout your testing strategy here.  That is, you 
can explain how tests will be selected for each module.  For instance, your test building 
approach could be test cases for each access program, including one test for normal behaviour 
and as many tests as needed for edge cases.  Rather than create the details of the input 
and output here, you could point to the unit testing code.  For this to work, you code 
needs to be well-documented, with meaningful names for all of the tests.}

\subsection{Unit Testing Scope}

\wss{What modules are outside of the scope.  If there are modules that are
  developed by someone else, then you would say here if you aren't planning on
  verifying them.  There may also be modules that are part of your software, but
  have a lower priority for verification than others.  If this is the case,
  explain your rationale for the ranking of module importance.}

\subsection{Tests for Functional Requirements}

\wss{Most of the verification will be through automated unit testing.  If
  appropriate specific modules can be verified by a non-testing based
  technique.  That can also be documented in this section.}

\subsubsection{Module 1}

\wss{Include a blurb here to explain why the subsections below cover the module.
  References to the MIS would be good.  You will want tests from a black box
  perspective and from a white box perspective.  Explain to the reader how the
  tests were selected.}

\begin{enumerate}

\item{test-id1\\}

Type: \wss{Functional, Dynamic, Manual, Automatic, Static etc. Most will
  be automatic}
					
Initial State: 
					
Input: 
					
Output: \wss{The expected result for the given inputs}

Test Case Derivation: \wss{Justify the expected value given in the Output field}

How test will be performed: 
					
\item{test-id2\\}

Type: \wss{Functional, Dynamic, Manual, Automatic, Static etc. Most will
  be automatic}
					
Initial State: 
					
Input: 
					
Output: \wss{The expected result for the given inputs}

Test Case Derivation: \wss{Justify the expected value given in the Output field}

How test will be performed: 

\item{...\\}
    
\end{enumerate}

\subsubsection{Module 2}

...

\subsection{Tests for Nonfunctional Requirements}

\wss{If there is a module that needs to be independently assessed for
  performance, those test cases can go here.  In some projects, planning for
  nonfunctional tests of units will not be that relevant.}

\wss{These tests may involve collecting performance data from previously
  mentioned functional tests.}

\subsubsection{Module ?}
		
\begin{enumerate}

\item{test-id1\\}

Type: \wss{Functional, Dynamic, Manual, Automatic, Static etc. Most will
  be automatic}
					
Initial State: 
					
Input/Condition: 
					
Output/Result: 
					
How test will be performed: 
					
\item{test-id2\\}

Type: Functional, Dynamic, Manual, Static etc.
					
Initial State: 
					
Input: 
					
Output: 
					
How test will be performed: 

\end{enumerate}

\subsubsection{Module ?}

...

\subsection{Traceability Between Test Cases and Modules}

\wss{Provide evidence that all of the modules have been considered.}
				
\bibliographystyle{plainnat}

\bibliography{../../refs/References}

\newpage

\section{Appendix}

This is where you can place additional information.

\subsection{Symbolic Parameters}

The definition of the test cases will call for SYMBOLIC\_CONSTANTS.
Their values are defined in this section for easy maintenance.

\subsection{Usability Survey Questions?}

\wss{This is a section that would be appropriate for some projects.}

\newpage{}
\section*{Appendix --- Reflection}

\wss{This section is not required for CAS 741}

The information in this section will be used to evaluate the team members on the
graduate attribute of Lifelong Learning.

\input{../Reflection.tex}

\begin{enumerate}
  \item What went well while writing this deliverable? 
  \item What pain points did you experience during this deliverable, and how
    did you resolve them?
  \item What knowledge and skills will the team collectively need to acquire to
  successfully complete the verification and validation of your project?
  Examples of possible knowledge and skills include dynamic testing knowledge,
  static testing knowledge, specific tool usage, Valgrind etc.  You should look to
  identify at least one item for each team member.
  \item For each of the knowledge areas and skills identified in the previous
  question, what are at least two approaches to acquiring the knowledge or
  mastering the skill?  Of the identified approaches, which will each team
  member pursue, and why did they make this choice?
\end{enumerate}

\end{document}