\documentclass[12pt, titlepage]{article}

\usepackage{booktabs}
\usepackage{tabularx}
\usepackage{longtable}
\usepackage{hyperref}
\hypersetup{
    colorlinks,
    citecolor=blue,
    filecolor=black,
    linkcolor=red,
    urlcolor=blue
}
\usepackage[round]{natbib}

\input{../Comments}
%% Common Parts

\newcommand{\progname}{Software Engineering} % PUT YOUR PROGRAM NAME HERE
\newcommand{\authname}{Team \#11, OKKM Insights
\\ Mathew Petronilho
\\ Oleg Glotov
\\ Kyle McMaster
\\ Kartik Chaudhari} % AUTHOR NAMES                  

\usepackage{hyperref}
    \hypersetup{colorlinks=true, linkcolor=blue, citecolor=blue, filecolor=blue,
                urlcolor=blue, unicode=false}
    \urlstyle{same}
                                


\begin{document}

\title{System Verification and Validation Plan for \progname{}} 
\author{\authname}
\date{\today}
	
\maketitle

\pagenumbering{roman}

\section*{Revision History}

\begin{tabularx}{\textwidth}{p{3cm}p{2cm}X}
\toprule {\bf Date} & {\bf Version} & {\bf Notes}\\
\midrule
Date 1 & 1.0 & Notes\\
Date 2 & 1.1 & Notes\\
\bottomrule
\end{tabularx}

~\\
\wss{The intention of the VnV plan is to increase confidence in the software.
However, this does not mean listing every verification and validation technique
that has ever been devised.  The VnV plan should also be a \textbf{feasible}
plan. Execution of the plan should be possible with the time and team available.
If the full plan cannot be completed during the time available, it can either be
modified to ``fake it'', or a better solution is to add a section describing
what work has been completed and what work is still planned for the future.}

\wss{The VnV plan is typically started after the requirements stage, but before
the design stage.  This means that the sections related to unit testing cannot
initially be completed.  The sections will be filled in after the design stage
is complete.  the final version of the VnV plan should have all sections filled
in.}

\newpage

\tableofcontents

\listoftables
\wss{Remove this section if it isn't needed}

\listoffigures
\wss{Remove this section if it isn't needed}

\newpage

\section{Symbols, Abbreviations, and Acronyms}

Please refer to Section 4 of Software Requirements Specification found \href{https://github.com/OKKM-insights/OKKM.insights/blob/main/docs/SRS/SRS.pdf}{here}.







\newpage

\pagenumbering{arabic}

This document is intended to provide a description of the specific validation and verification activites that will be completed throughout the development of the GeoWeb System.
The purpose of these activities is to ensure the system requirements agree with stakeholder needs, and to certify the implementation of the system satisifes the stakeholder requirements. This document will
start with a review of the system, a description of the testing plan, and a list of tests to be completed.


\section{General Information}

\subsection{Summary}

\wss{Say what software is being tested.  Give its name and a brief overview of
  its general functions.}

\subsection{Objectives}

\wss{State what is intended to be accomplished.  The objective will be around
  the qualities that are most important for your project.  You might have
  something like: ``build confidence in the software correctness,''
  ``demonstrate adequate usability.'' etc.  You won't list all of the qualities,
  just those that are most important.}

\wss{You should also list the objectives that are out of scope.  You don't have 
the resources to do everything, so what will you be leaving out.  For instance, 
if you are not going to verify the quality of usability, state this.  It is also 
worthwhile to justify why the objectives are left out.}

\wss{The objectives are important because they highlight that you are aware of 
limitations in your resources for verification and validation.  You can't do everything, 
so what are you going to prioritize?  As an example, if your system depends on an 
external library, you can explicitly state that you will assume that external library 
has already been verified by its implementation team.}

\subsection{Challenge Level and Extras}

\wss{State the challenge level (advanced, general, basic) for your project.
Your challenge level should exactly match what is included in your problem
statement.  This should be the challenge level agreed on between you and the
course instructor.  You can use a pull request to update your challenge level
(in TeamComposition.csv or Repos.csv) if your plan changes as a result of the
VnV planning exercise.}

\wss{Summarize the extras (if any) that were tackled by this project.  Extras
can include usability testing, code walkthroughs, user documentation, formal
proof, GenderMag personas, Design Thinking, etc.  Extras should have already
been approved by the course instructor as included in your problem statement.
You can use a pull request to update your extras (in TeamComposition.csv or
Repos.csv) if your plan changes as a result of the VnV planning exercise.}

\subsection{Relevant Documentation}

\wss{Reference relevant documentation.  This will definitely include your SRS
  and your other project documents (design documents, like MG, MIS, etc).  You
  can include these even before they are written, since by the time the project
  is done, they will be written.  You can create BibTeX entries for your
  documents and within those entries include a hyperlink to the documents.}

\citet{SRS}

\wss{Don't just list the other documents.  You should explain why they are relevant and 
how they relate to your VnV efforts.}

\section{Plan}

\wss{Introduce this section.  You can provide a roadmap of the sections to
  come.}

\subsection{Verification and Validation Team}

\wss{Your teammates.  Maybe your supervisor.
  You should do more than list names.  You should say what each person's role is
  for the project's verification.  A table is a good way to summarize this information.}

\subsection{SRS Verification Plan}

\wss{List any approaches you intend to use for SRS verification.  This may
  include ad hoc feedback from reviewers, like your classmates (like your
  primary reviewer), or you may plan for something more rigorous/systematic.}

\wss{If you have a supervisor for the project, you shouldn't just say they will
read over the SRS.  You should explain your structured approach to the review.
Will you have a meeting?  What will you present?  What questions will you ask?
Will you give them instructions for a task-based inspection?  Will you use your
issue tracker?}

\wss{Maybe create an SRS checklist?}

\subsection{Design Verification Plan}

\wss{Plans for design verification}

\wss{The review will include reviews by your classmates}

\wss{Create a checklists?}

\subsection{Verification and Validation Plan Verification Plan}

\wss{The verification and validation plan is an artifact that should also be
verified.  Techniques for this include review and mutation testing.}

\wss{The review will include reviews by your classmates}

\wss{Create a checklists?}

\subsection{Implementation Verification Plan}

\wss{You should at least point to the tests listed in this document and the unit
  testing plan.}

\wss{In this section you would also give any details of any plans for static
  verification of the implementation.  Potential techniques include code
  walkthroughs, code inspection, static analyzers, etc.}

\wss{The final class presentation in CAS 741 could be used as a code
walkthrough.  There is also a possibility of using the final presentation (in
CAS741) for a partial usability survey.}

\subsection{Automated Testing and Verification Tools}

\wss{What tools are you using for automated testing.  Likely a unit testing
  framework and maybe a profiling tool, like ValGrind.  Other possible tools
  include a static analyzer, make, continuous integration tools, test coverage
  tools, etc.  Explain your plans for summarizing code coverage metrics.
  Linters are another important class of tools.  For the programming language
  you select, you should look at the available linters.  There may also be tools
  that verify that coding standards have been respected, like flake9 for
  Python.}

\wss{If you have already done this in the development plan, you can point to
that document.}

\wss{The details of this section will likely evolve as you get closer to the
  implementation.}

\subsection{Software Validation Plan}

\wss{If there is any external data that can be used for validation, you should
  point to it here.  If there are no plans for validation, you should state that
  here.}

\wss{You might want to use review sessions with the stakeholder to check that
the requirements document captures the right requirements.  Maybe task based
inspection?}

\wss{For those capstone teams with an external supervisor, the Rev 0 demo should 
be used as an opportunity to validate the requirements.  You should plan on 
demonstrating your project to your supervisor shortly after the scheduled Rev 0 demo.  
The feedback from your supervisor will be very useful for improving your project.}

\wss{For teams without an external supervisor, user testing can serve the same purpose 
as a Rev 0 demo for the supervisor.}

\wss{This section might reference back to the SRS verification section.}

\section{System Tests}

\wss{There should be text between all headings, even if it is just a roadmap of
the contents of the subsections.}

\subsection{Tests for Functional Requirements}

\wss{Subsets of the tests may be in related, so this section is divided into
  different areas.  If there are no identifiable subsets for the tests, this
  level of document structure can be removed.}

\wss{Include a blurb here to explain why the subsections below
  cover the requirements.  References to the SRS would be good here.}

\subsubsection{Area of Testing1}

\wss{It would be nice to have a blurb here to explain why the subsections below
  cover the requirements.  References to the SRS would be good here.  If a section
  covers tests for input constraints, you should reference the data constraints
  table in the SRS.}
		
\paragraph{Title for Test}

\begin{enumerate}

\item{test-id1\\}

Control: Manual versus Automatic
					
Initial State: 
					
Input: 
					
Output: \wss{The expected result for the given inputs.  Output is not how you
are going to return the results of the test.  The output is the expected
result.}

Test Case Derivation: \wss{Justify the expected value given in the Output field}
					
How test will be performed: 
					
\item{test-id2\\}

Control: Manual versus Automatic
					
Initial State: 
					
Input: 
					
Output: \wss{The expected result for the given inputs}

Test Case Derivation: \wss{Justify the expected value given in the Output field}

How test will be performed: 

\end{enumerate}

\subsubsection{Area of Testing2}

...

\subsection{Tests for Nonfunctional Requirements}

\wss{The nonfunctional requirements for accuracy will likely just reference the
  appropriate functional tests from above.  The test cases should mention
  reporting the relative error for these tests.  Not all projects will
  necessarily have nonfunctional requirements related to accuracy.}

\wss{For some nonfunctional tests, you won't be setting a target threshold for
passing the test, but rather describing the experiment you will do to measure
the quality for different inputs.  For instance, you could measure speed versus
the problem size.  The output of the test isn't pass/fail, but rather a summary
table or graph.}

\wss{Tests related to usability could include conducting a usability test and
  survey.  The survey will be in the Appendix.}

\wss{Static tests, review, inspections, and walkthroughs, will not follow the
format for the tests given below.}

\wss{If you introduce static tests in your plan, you need to provide details.
How will they be done?  In cases like code (or document) walkthroughs, who will
be involved? Be specific.}

\subsubsection{Operational and Environmental}
		
\paragraph{Release Tests}

\begin{enumerate}

\item{Road Map Consistency: T-OE0\\}

Related Reqs: NFR-OE7

Type: Manual, Static
					
Initial State: Application has a release road map that is publicly accessible.
					
Input/Condition: Team member conducts a review.
					
Output/Result: At least \hyperref[MIN_ON_TIME_MILESTONE]{MIN\_ON\_TIME\_MILESTONE}\% of the listed milestones have been met on time.
					
How test will be performed: The team member looks over the road map and cross references the completion date of milestones to the dates listed in the road map.

\item{Beta Testing: T-OE1\\}

Related Reqs: NFR-OE8

Type: Dynamic, Exploratory
					
Initial State: Beta version of application is deployed and accessible
					
Input/Condition: At least \hyperref[BETA_TESTERS]{BETA\_TESTERS} beta testers are provided access to use the application.
					
Output/Result: Feedback on any bugs, navigation issues, or aesthetic problems is provided. Less than \hyperref[MAX_BUGS_FOUND]{MAX\_BUGS\_FOUND} bugs are found.
					
How test will be performed: Testers will be recruited and identified. They will be from fields of interest that include scientists, labelers, and domain experts. Then, the beta testing environment will be set up and the url will be distributed to the testers along with any other set up resources. Specific tasks are provided for testers to complete that focus on the annotation tools, sign up process and project creation. Feedback will be collected through direct comments from the tester.

\item{Regression Testing: T-OE2\\}

Related Reqs: NFR-OE9

Type: Dynamic, Automated
					
Initial State: Application is deployed.
					
Input/Condition: Run regression test suite, consisting of unit tests.
					
Output/Result: All regression tests are passed.
					
How test will be performed: An automated script with regression tests will run when updates are made to the production build.

\end{enumerate}

\subsubsection{Maintainability and Support}
		
\paragraph{Maintenance Tests}

\begin{enumerate}

\item{Ease of Change: T-MS0\\}

Related Reqs: NFR-MR0

Type: Manual, Static
					
Initial State: Application's source repository contains complete documentation.
					
Input/Condition: Competent software developer who has not previously worked on the app reviews documentation and attempts to perform tasks.
					
Output/Result: The developer can easily make a minor update to a specified part of the application.
					
How test will be performed: Give the developer time to read through the documentation. Give them a maintenance task, such as updating the size of the title font to 20px. Observe them and document how long it takes them and if they encountered any troubles.
\end{enumerate}

\subsubsection{Security}
		
\paragraph{Access Tests}

\begin{enumerate}

\item{Logged Out Permissions: T-SE0\\}

Related Reqs: NFR-SE0

Type: Manual, Dynamic, White-box
					
Initial State: Application is deployed.
					
Input/Condition: Tester who is not signed in tries to access application paths for project creation and image labeling (Ex. /projects or /label).
					
Output/Result: The tester is denied access to these paths and is told to sign in.
					
How test will be performed: On the deployed application, the tester will visit all possible paths as a logged out user.

\item{Labeler Permissions: T-SE1\\}
Related Reqs: NFR-SE1

Type: Manual, Dynamic, White-box
					
Initial State: Application is deployed.
					
Input/Condition: Tester who is signed in as a labeler tries to access application paths for project creation.
					
Output/Result: The tester is denied access to these paths. However, the tester has access to paths related to image labeling.
					
How test will be performed: On the deployed application, the tester will visit all possible paths as a labeler.

\item{Invalid Email Format: T-SE2\\}
Related Reqs: NFR-SE2, NFR-SE5

Type: Automatic, Dynamic
					
Initial State: Front-end registration page is created and integrated with the database.
					
Input/Condition: Email with invalid format, such as an empty string or a string missing '@', is entered.
					
Output/Result: Application rejects email and tells the user that the email format is wrong.
					
How test will be performed: A unit test will be performed where the input is entered into the email section of the registration form.

\item{Duplicate Email: T-SE3\\}
Related Reqs: NFR-SE2, NFR-SE5

Type: Automatic, Dynamic
					
Initial State: Front-end registration page is created and integrated with the database.
					
Input/Condition: Email that is already in database is entered.
					
Output/Result: Application rejects email and tells the user that the email is in use.
					
How test will be performed: A unit test will be performed where the input is entered into the email section of the registration form.

\item{Invalid Password Format: T-SE4\\}
Related Reqs: NFR-SE3, NFR-SE5

Type: Automatic, Dynamic
					
Initial State: Front-end registration page is created and integrated with the database.
					
Input/Condition: Password with invalid format, such as an empty string or a string with no numbers, is entered.
					
Output/Result: Application rejects password and tells the user what requirements they have not met.
					
How test will be performed: A unit test will be performed where the input is entered into the password section of the registration form.

\item{System Error: T-SE5\\}
Related Reqs: NFR-SE4, NFR-SE6

Type: Manual, Dynamic
					
Initial State: Application is deployed.
					
Input/Condition: Purposely invoke a system failure, and attempt to perform an action such as a label submission.
					
Output/Result: Application provides an error message on the user interface. The database has not changed in anyway.
					
How test will be performed: Go on to the application, start a labeling task, purposely disconnect from the internet, and try to submit a labeled image.

\end{enumerate}

\paragraph{Integrity Tests}

\begin{enumerate}

\item{Duplicate Entries: T-SE6\\}

Related Reqs: NFR-SE7

Type: Manual, Dynamic
					
Initial State: Database is deployed.
					
Input/Condition: Duplicate database entry is inserted into the database.
					
Output/Result: Database has only one of the inputted entry and the duplicate has been removed.
					
How test will be performed: Attempt to insert the same entry twice into the database through the database UI.

\end{enumerate}

\paragraph{Privacy Tests}

\begin{enumerate}

\item{Encrypted User Data: T-SE7\\}

Related Reqs: NFR-SE8

Type: Manual, Dynamic
					
Initial State: Application is deployed.
					
Input/Condition: Tester registers an account.
					
Output/Result: All sensitive user data that is stored in the database is encrypted.
					
How test will be performed: Tester will create a new account, then check the corresponding user entry in the database and see if the sensitive information is encrypted.

\item{Encrypted Payments: T-SE8\\}

Related Reqs: NFR-SE9

Type: Manual, Dynamic
					
Initial State: Application is deployed.
					
Input/Condition: Tester enters sample payment details to pay for a labeling project that has been created.
					
Output/Result: These details are encrypted and can not be read through packet analyzers. The amount in the request can not be modified by an adversary.
					
How test will be performed: Tester will enter sample payment details, and submit their payment. Using a packet analyzer (such as Wireshark), packets from this request will be looked at to ensure all information is encrypted.

\end{enumerate}

\paragraph{Immunity Tests}

\begin{enumerate}

\item{SQL Injection: T-SE9\\}

Related Reqs: NFR-SE10

Type: Manual, Dynamic
					
Initial State: Application is deployed.
					
Input/Condition: A malicious SQL statement is entered into a text field.
					
Output/Result: The system raises an error telling the user that it is invalid.
					
How test will be performed: Tester will enter a SQL statement such as "\{valid email\}'--" into an input such as the email input. This example has the potential to bypass a password check by commenting out the rest of the SQL query. The tester will check that when this statement is entered, the system gives feedback that it is invalid.

\end{enumerate}

\subsubsection{Cultural}
		
\paragraph{Language Tests}

\begin{enumerate}

\item{Support of Different Languages: T-CU0\\}

Related Reqs: NFR-CU0

Type: Manual, Dynamic
					
Initial State: Application is deployed.
					
Input/Condition: Tester selects a language from a list of available languages.
					
Output/Result: All text on the website is translated and displayed in the selected language.
					
How test will be performed: Tester will check that the language selection list is accessible, and that the most popular languages are included. When a language is selected, the tester will check that the translation has been applied and there is no untranslated or gibberish text. This can be checked for each language.
\end{enumerate}

\subsubsection{Compliance}

\paragraph{Financial Tests}

\begin{enumerate}

\item{Compliant Payment Process: T-CO0\\}

Related Reqs: NFR-SE9

Type: Manual, Static
					
Initial State: Application is deployed.
					
Input/Condition: Qualified Security Assessor (QSA) assesses the application.
					
Output/Result: They determine that it meets the PCI-DSS standard.
					
How test will be performed: A QSA will be found and contacted to perform an assessment. The QSA will be shown all parts of the application that deal will financial transactions and will be able to make a determination on if it meets the standard.
\end{enumerate}
		
\paragraph{Legal Tests}

\begin{enumerate}

\item{System Availability: T-CO1\\}

Related Reqs: NFR-CO0

Type: Manual, Dynamic
					
Initial State: Application is deployed.
					
Input/Condition: Tester changes country they are accessing the application from using a tool such as a VPN.
					
Output/Result: Application is blocked in countries facing economic sanctions by the Government of Canada.
					
How test will be performed: A list of the countries facing economic sanctions by Canada will be compiled. Then, the tester will simulate that they are accessing the application from these countries, and ensure it is unreachable.

\item{Taxes: T-CO2\\}

Related Reqs: NFR-CO1

Type: Manual, Dynamic
					
Initial State: Application is deployed.
					
Input/Condition: Tester redeems a cash balance.
					
Output/Result: If the cash balance exceeds a threshold, a tax form will be issued.
					
How test will be performed: Tester creates a test account with an account balance over the threshold. When they withdraw, they check that a tax form has been emailed to the email associated with the account. The tax form should reflect the withdrawal balance.

\item{Project Availability: T-CO3\\}

Related Reqs: NFR-CO2

Type: Manual, Dynamic
					
Initial State: Application is deployed.
					
Input/Condition: Tester changes country they are accessing the application from using a tool such as a VPN.
					
Output/Result: Specific project is not shown.
					
How test will be performed: A project will be specified to only be distributed in a specific country. Then, the tester will simulate that they are accessing the application from other countries, and ensure the project does not show up.

\end{enumerate}

\subsubsection{User Documentation and Training}

\begin{enumerate}

\item{Helpfulness of User Aids: T-UDT0\\}

Related Reqs: NFR-UD0, NFR-UD1, NFR-UD2, NFR-TR0

Type: Manual, Dynamic
					
Initial State: Application is deployed with all help features. Tutorials and user documentation have been created.
					
Input/Condition: Users attempt to complete a basic labeling task using only the platform’s built-in help resources (help system, quick start guide, tutorials and contextual tooltips).
					
Output/Result: At least \hyperref[MIN_USER_HELP_SATISFACTION]{MIN\_USER\_HELP\_SATISFACTION}\% of users who used a help feature found that feature helpful. With the assistance of the help tools, the user was able to perform the task within \hyperref[MAX_TASK_TIME]{MAX\_TASK\_TIME} minutes.
					
How test will be performed: The purpose of the platform will be explained to the users and the built-in help features will be shown. Then, the labeling task will be given to them. Time to complete task is observed and the help tools they use are recorded. Participants will then fill out a usability survey, which can be viewed in the appendix.

\item{Usefulness of Sandbox: T-UDT1\\}

Related Reqs: NFR-TR1

Type: Automatic, Dynamic
					
Initial State: Application is deployed with all help features. Tutorials and user documentation has been created.
					
Input/Condition: A new user has accessed the platform.
					
Output/Result: At least \hyperref[MIN_PRACTICE_USAGE]{MIN\_PRACTICE\_USAGE}\% of new users utilize the practice environment, with self-assessment scores indicating an average improvement of \hyperref[IMPROVE_IN_ACC]{IMPROVE\_IN\_ACC}\% in labeling accuracy over their first three attempts.
					
How test will be performed: Practice environment utilization will be tracked by the application. Improvement in accuracy will also be tracked. If the metrics meet or succeed our thresholds, then we can conclude the sandbox is useful.
\end{enumerate}

...

\subsection{Traceability Between Test Cases and Requirements}

\wss{Provide a table that shows which test cases are supporting which
  requirements.}

\section{Unit Test Description}

\wss{This section should not be filled in until after the MIS (detailed design
  document) has been completed.}

\wss{Reference your MIS (detailed design document) and explain your overall
philosophy for test case selection.}  

\wss{To save space and time, it may be an option to provide less detail in this section.  
For the unit tests you can potentially layout your testing strategy here.  That is, you 
can explain how tests will be selected for each module.  For instance, your test building 
approach could be test cases for each access program, including one test for normal behaviour 
and as many tests as needed for edge cases.  Rather than create the details of the input 
and output here, you could point to the unit testing code.  For this to work, you code 
needs to be well-documented, with meaningful names for all of the tests.}

\subsection{Unit Testing Scope}

\wss{What modules are outside of the scope.  If there are modules that are
  developed by someone else, then you would say here if you aren't planning on
  verifying them.  There may also be modules that are part of your software, but
  have a lower priority for verification than others.  If this is the case,
  explain your rationale for the ranking of module importance.}

\subsection{Tests for Functional Requirements}

\wss{Most of the verification will be through automated unit testing.  If
  appropriate specific modules can be verified by a non-testing based
  technique.  That can also be documented in this section.}

\subsubsection{Module 1}

\wss{Include a blurb here to explain why the subsections below cover the module.
  References to the MIS would be good.  You will want tests from a black box
  perspective and from a white box perspective.  Explain to the reader how the
  tests were selected.}

\begin{enumerate}

\item{test-id1\\}

Type: \wss{Functional, Dynamic, Manual, Automatic, Static etc. Most will
  be automatic}
					
Initial State: 
					
Input: 
					
Output: \wss{The expected result for the given inputs}

Test Case Derivation: \wss{Justify the expected value given in the Output field}

How test will be performed: 
					
\item{test-id2\\}

Type: \wss{Functional, Dynamic, Manual, Automatic, Static etc. Most will
  be automatic}
					
Initial State: 
					
Input: 
					
Output: \wss{The expected result for the given inputs}

Test Case Derivation: \wss{Justify the expected value given in the Output field}

How test will be performed: 

\item{...\\}
    
\end{enumerate}

\subsubsection{Module 2}

...

\subsection{Tests for Nonfunctional Requirements}

\wss{If there is a module that needs to be independently assessed for
  performance, those test cases can go here.  In some projects, planning for
  nonfunctional tests of units will not be that relevant.}

\wss{These tests may involve collecting performance data from previously
  mentioned functional tests.}

\subsubsection{Module ?}
		
\begin{enumerate}

\item{test-id1\\}

Type: \wss{Functional, Dynamic, Manual, Automatic, Static etc. Most will
  be automatic}
					
Initial State: 
					
Input/Condition: 
					
Output/Result: 
					
How test will be performed: 
					
\item{test-id2\\}

Type: Functional, Dynamic, Manual, Static etc.
					
Initial State: 
					
Input: 
					
Output: 
					
How test will be performed: 

\end{enumerate}

\subsubsection{Module ?}

...

\subsection{Traceability Between Test Cases and Modules}

\wss{Provide evidence that all of the modules have been considered.}
				
\bibliographystyle{plainnat}

\bibliography{../../refs/References}

\newpage

\section{Appendix}

This is where you can place additional information.

\subsection{Symbolic Parameters}

The definition of the test cases will call for SYMBOLIC\_CONSTANTS.
Their values are defined in this section for easy maintenance.

\begin{longtable}{|l|l|l|p{4cm}|}
  \hline
  \textbf{Parameter} & \textbf{Value} & \textbf{Unit} & \textbf{Description} \\ \hline
  \endfirsthead

  \hline
  \textbf{Parameter} & \textbf{Value} & \textbf{Unit} & \textbf{Description} \\ \hline
  \endhead

  \hline
  \endfoot

  \hline
  MIN\_ON\_TIME\_MILESTONE & 80 & \% & Minimum percent of milestones that have been met on time \label{MIN_ON_TIME_MILESTONE} \\ \hline
  BETA\_TESTERS & 50 & People & Number of beta testers \label{BETA_TESTERS} \\ \hline
  MAX\_BUGS\_FOUND & 10 & Bugs & Number of software bugs found \label{MAX_BUGS_FOUND} \\ \hline
  MIN\_USER\_HELP\_SATISFACTION & 80 & \% & Minimum percent of users satisfied with help feature \label{MIN_USER_HELP_SATISFACTION} \\ \hline
  MAX\_TASK\_TIME & 15 & Minutes & Maximum time it takes a user to complete a task \label{MAX_TASK_TIME} \\ \hline
  MIN\_PRACTICE\_USAGE & 80 & \% & Minimum percent of new users who have used the practice sandbox \label{MIN_PRACTICE_USAGE} \\ \hline
  IMPROVE\_IN\_ACC & 20 & \% & Improvement in accuracy of a user after practicing \label{IMPROVE_IN_ACC} \\ \hline
\end{longtable}

\subsection{Usability Survey Questions}
\begin{enumerate}
  \item Did you use the help system to aid in completing your task? Yes/No
  \item If you answered yes, please rate how useful it was in helping you accomplish your task: 1 (Not Useful) - 5 (Very Useful)
  \item Did you use the quick start guide to aid in completing your task? Yes/No
  \item If you answered yes, please rate the clarity and helpfulness of it: 1 (Not Helpful) - 5 (Very Helpful)
  \item Did you notice the tool-tips or pop-ups providing contextual help as you worked? Yes/No
  \item If you answered yes, please rate the clarity and usefulness of the in-app help indicators: 1 (Not Helpful) - 5 (Very Helpful)
  
\end{enumerate}

\newpage{}
\section*{Appendix --- Reflection}

\wss{This section is not required for CAS 741}

The information in this section will be used to evaluate the team members on the
graduate attribute of Lifelong Learning.

\input{../Reflection.tex}

\begin{enumerate}
  \item What went well while writing this deliverable? 
  \item What pain points did you experience during this deliverable, and how
    did you resolve them?
  \item What knowledge and skills will the team collectively need to acquire to
  successfully complete the verification and validation of your project?
  Examples of possible knowledge and skills include dynamic testing knowledge,
  static testing knowledge, specific tool usage, Valgrind etc.  You should look to
  identify at least one item for each team member.
  \item For each of the knowledge areas and skills identified in the previous
  question, what are at least two approaches to acquiring the knowledge or
  mastering the skill?  Of the identified approaches, which will each team
  member pursue, and why did they make this choice?
\end{enumerate}

\end{document}