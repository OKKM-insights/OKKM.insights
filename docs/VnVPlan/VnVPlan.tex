\documentclass[12pt, titlepage]{article}

\usepackage{amsmath} 
\usepackage{booktabs}
\usepackage{tabularx}
\usepackage{hyperref}
\hypersetup{
    colorlinks,
    citecolor=blue,
    filecolor=black,
    linkcolor=red,
    urlcolor=blue
}
\usepackage[round]{natbib}

%% Comments

\usepackage{color}

\newif\ifcomments\commentstrue %displays comments
%\newif\ifcomments\commentsfalse %so that comments do not display

\ifcomments
\newcommand{\authornote}[3]{\textcolor{#1}{[#3 ---#2]}}
\newcommand{\todo}[1]{\textcolor{red}{[TODO: #1]}}
\else
\newcommand{\authornote}[3]{}
\newcommand{\todo}[1]{}
\fi

\newcommand{\wss}[1]{\authornote{blue}{SS}{#1}} 
\newcommand{\plt}[1]{\authornote{magenta}{TPLT}{#1}} %For explanation of the template
\newcommand{\an}[1]{\authornote{cyan}{Author}{#1}}

%% Common Parts

\newcommand{\progname}{Software Engineering} % PUT YOUR PROGRAM NAME HERE
\newcommand{\authname}{Team \#11, OKKM Insights
\\ Mathew Petronilho
\\ Oleg Glotov
\\ Kyle McMaster
\\ Kartik Chaudhari} % AUTHOR NAMES                  

\usepackage{hyperref}
    \hypersetup{colorlinks=true, linkcolor=blue, citecolor=blue, filecolor=blue,
                urlcolor=blue, unicode=false}
    \urlstyle{same}
                                


\begin{document}

\title{System Verification and Validation Plan for \progname{}} 
\author{\authname}
\date{\today}
	
\maketitle

\pagenumbering{roman}

\section*{Revision History}

\begin{tabularx}{\textwidth}{p{3cm}p{2cm}X}
\toprule {\bf Date} & {\bf Version} & {\bf Notes}\\
\midrule
Date 1 & 1.0 & Notes\\
Date 2 & 1.1 & Notes\\
\bottomrule
\end{tabularx}

~\\
\wss{The intention of the VnV plan is to increase confidence in the software.
However, this does not mean listing every verification and validation technique
that has ever been devised.  The VnV plan should also be a \textbf{feasible}
plan. Execution of the plan should be possible with the time and team available.
If the full plan cannot be completed during the time available, it can either be
modified to ``fake it'', or a better solution is to add a section describing
what work has been completed and what work is still planned for the future.}

\wss{The VnV plan is typically started after the requirements stage, but before
the design stage.  This means that the sections related to unit testing cannot
initially be completed.  The sections will be filled in after the design stage
is complete.  the final version of the VnV plan should have all sections filled
in.}

\newpage

\tableofcontents

\listoftables
\wss{Remove this section if it isn't needed}

\listoffigures
\wss{Remove this section if it isn't needed}

\newpage

\section{Symbols, Abbreviations, and Acronyms}

\renewcommand{\arraystretch}{1.2}
\begin{tabular}{l l} 
  \toprule		
  \textbf{symbol} & \textbf{description}\\
  \midrule 
  T & Test\\
  \bottomrule
\end{tabular}\\

\wss{symbols, abbreviations, or acronyms --- you can simply reference the SRS
  \citep{SRS} tables, if appropriate}

\wss{Remove this section if it isn't needed}

\newpage

\pagenumbering{arabic}

This document ... \wss{provide an introductory blurb and roadmap of the
  Verification and Validation plan}

\section{General Information}

\subsection{Summary}

\wss{Say what software is being tested.  Give its name and a brief overview of
  its general functions.}

\subsection{Objectives}

\wss{State what is intended to be accomplished.  The objective will be around
  the qualities that are most important for your project.  You might have
  something like: ``build confidence in the software correctness,''
  ``demonstrate adequate usability.'' etc.  You won't list all of the qualities,
  just those that are most important.}

\wss{You should also list the objectives that are out of scope.  You don't have 
the resources to do everything, so what will you be leaving out.  For instance, 
if you are not going to verify the quality of usability, state this.  It is also 
worthwhile to justify why the objectives are left out.}

\wss{The objectives are important because they highlight that you are aware of 
limitations in your resources for verification and validation.  You can't do everything, 
so what are you going to prioritize?  As an example, if your system depends on an 
external library, you can explicitly state that you will assume that external library 
has already been verified by its implementation team.}

\subsection{Challenge Level and Extras}

\wss{State the challenge level (advanced, general, basic) for your project.
Your challenge level should exactly match what is included in your problem
statement.  This should be the challenge level agreed on between you and the
course instructor.  You can use a pull request to update your challenge level
(in TeamComposition.csv or Repos.csv) if your plan changes as a result of the
VnV planning exercise.}

\wss{Summarize the extras (if any) that were tackled by this project.  Extras
can include usability testing, code walkthroughs, user documentation, formal
proof, GenderMag personas, Design Thinking, etc.  Extras should have already
been approved by the course instructor as included in your problem statement.
You can use a pull request to update your extras (in TeamComposition.csv or
Repos.csv) if your plan changes as a result of the VnV planning exercise.}

\subsection{Relevant Documentation}

\wss{Reference relevant documentation.  This will definitely include your SRS
  and your other project documents (design documents, like MG, MIS, etc).  You
  can include these even before they are written, since by the time the project
  is done, they will be written.  You can create BibTeX entries for your
  documents and within those entries include a hyperlink to the documents.}

\citet{SRS}

\wss{Don't just list the other documents.  You should explain why they are relevant and 
how they relate to your VnV efforts.}

\section{Plan}

\wss{Introduce this section.  You can provide a roadmap of the sections to
  come.}

\subsection{Verification and Validation Team}

\wss{Your teammates.  Maybe your supervisor.
  You should do more than list names.  You should say what each person's role is
  for the project's verification.  A table is a good way to summarize this information.}

\subsection{SRS Verification Plan}

\wss{List any approaches you intend to use for SRS verification.  This may
  include ad hoc feedback from reviewers, like your classmates (like your
  primary reviewer), or you may plan for something more rigorous/systematic.}

\wss{If you have a supervisor for the project, you shouldn't just say they will
read over the SRS.  You should explain your structured approach to the review.
Will you have a meeting?  What will you present?  What questions will you ask?
Will you give them instructions for a task-based inspection?  Will you use your
issue tracker?}

\wss{Maybe create an SRS checklist?}

\subsection{Design Verification Plan}

\wss{Plans for design verification}

\wss{The review will include reviews by your classmates}

\wss{Create a checklists?}

\subsection{Verification and Validation Plan Verification Plan}

\wss{The verification and validation plan is an artifact that should also be
verified.  Techniques for this include review and mutation testing.}

\wss{The review will include reviews by your classmates}

\wss{Create a checklists?}

\subsection{Implementation Verification Plan}

\wss{You should at least point to the tests listed in this document and the unit
  testing plan.}

\wss{In this section you would also give any details of any plans for static
  verification of the implementation.  Potential techniques include code
  walkthroughs, code inspection, static analyzers, etc.}

\wss{The final class presentation in CAS 741 could be used as a code
walkthrough.  There is also a possibility of using the final presentation (in
CAS741) for a partial usability survey.}

\subsection{Automated Testing and Verification Tools}

\wss{What tools are you using for automated testing.  Likely a unit testing
  framework and maybe a profiling tool, like ValGrind.  Other possible tools
  include a static analyzer, make, continuous integration tools, test coverage
  tools, etc.  Explain your plans for summarizing code coverage metrics.
  Linters are another important class of tools.  For the programming language
  you select, you should look at the available linters.  There may also be tools
  that verify that coding standards have been respected, like flake9 for
  Python.}

\wss{If you have already done this in the development plan, you can point to
that document.}

\wss{The details of this section will likely evolve as you get closer to the
  implementation.}

\subsection{Software Validation Plan}

\wss{If there is any external data that can be used for validation, you should
  point to it here.  If there are no plans for validation, you should state that
  here.}

\wss{You might want to use review sessions with the stakeholder to check that
the requirements document captures the right requirements.  Maybe task based
inspection?}

\wss{For those capstone teams with an external supervisor, the Rev 0 demo should 
be used as an opportunity to validate the requirements.  You should plan on 
demonstrating your project to your supervisor shortly after the scheduled Rev 0 demo.  
The feedback from your supervisor will be very useful for improving your project.}

\wss{For teams without an external supervisor, user testing can serve the same purpose 
as a Rev 0 demo for the supervisor.}

\wss{This section might reference back to the SRS verification section.}

\section{System Tests}

\wss{There should be text between all headings, even if it is just a roadmap of
the contents of the subsections.}

\subsection{Tests for Functional Requirements}

\wss{Subsets of the tests may be in related, so this section is divided into
  different areas.  If there are no identifiable subsets for the tests, this
  level of document structure can be removed.}

\wss{Include a blurb here to explain why the subsections below
  cover the requirements.  References to the SRS would be good here.}

\subsubsection{Area of Testing1}

\wss{It would be nice to have a blurb here to explain why the subsections below
  cover the requirements.  References to the SRS would be good here.  If a section
  covers tests for input constraints, you should reference the data constraints
  table in the SRS.}
		
\paragraph{Title for Test}

\begin{enumerate}

\item{test-id1\\}

Control: Manual versus Automatic
					
Initial State: 
					
Input: 
					
Output: \wss{The expected result for the given inputs.  Output is not how you
are going to return the results of the test.  The output is the expected
result.}

Test Case Derivation: \wss{Justify the expected value given in the Output field}
					
How test will be performed: 
					
\item{test-id2\\}

Control: Manual versus Automatic
					
Initial State: 
					
Input: 
					
Output: \wss{The expected result for the given inputs}

Test Case Derivation: \wss{Justify the expected value given in the Output field}

How test will be performed: 

\end{enumerate}

\subsubsection{Area of Testing2}

...

\subsection{Tests for Nonfunctional Requirements}

\wss{The nonfunctional requirements for accuracy will likely just reference the
  appropriate functional tests from above.  The test cases should mention
  reporting the relative error for these tests.  Not all projects will
  necessarily have nonfunctional requirements related to accuracy.}

\wss{For some nonfunctional tests, you won't be setting a target threshold for
passing the test, but rather describing the experiment you will do to measure
the quality for different inputs.  For instance, you could measure speed versus
the problem size.  The output of the test isn't pass/fail, but rather a summary
table or graph.}

\wss{Tests related to usability could include conducting a usability test and
  survey.  The survey will be in the Appendix.}

\wss{Static tests, review, inspections, and walkthroughs, will not follow the
format for the tests given below.}

\wss{If you introduce static tests in your plan, you need to provide details.
How will they be done?  In cases like code (or document) walkthroughs, who will
be involved? Be specific.}







% NFRs 10---------------------------------------------------------
\subsubsection{Look and Feel Tests}
% 10.1
\paragraph{Appearance Tests}
\begin{enumerate}
\item{Responsive Layout Validation: T-LF0\\}

Related Reqs: NFR-LF0

Type: Manual, Usability

Initial State: Users access the application on devices with screen resolutions ranging from 1024$\times$768 pixels to 1920$\times$1080 pixels.

Input/Condition: The application is displayed on various screen sizes within the specified range to verify adaptability and layout consistency.

How test will be performed:
\begin{enumerate}
    \item Select a representative set of screen resolutions within the range 1024$\times$768 to 1920$\times$1080 pixels (e.g., 1024$\times$768, 1366$\times$768, 1440$\times$900, 1600$\times$900, 1920$\times$1080).
    \item Use physical devices or browser developer tools to simulate each selected screen resolution.
    \item Verify that all visual elements (buttons, text, images, menus) are fully visible and do not exceed screen boundaries.
    \item Ensure that the layout remains uncluttered, with appropriate spacing and alignment across different screen sizes.
\end{enumerate}

\item{Interactive Elements Feedback Validation: T-LF1\\}

Related Reqs: NFR-LF1

Type: Manual, Usability

Initial State: Interactive elements (e.g., buttons, links) are present within the application interface.

Input/Condition: Users interact with various interactive elements to verify that visual feedback is provided appropriately.

How test will be performed:
\begin{enumerate}
    \item Identify all interactive elements within the application, including buttons, links, and menu items.
    \item For each interactive element, perform the following interactions: hover, click, and focus (if applicable).
    \item Observe and document the visual feedback provided for each interaction, such as color changes, animations, or shadows.
    \item Ensure that every interactive element provides at least one form of visual feedback as specified in the fit criterion.
\end{enumerate}

\end{enumerate}


% 10.2
\paragraph{Style Tests}
\begin{enumerate}
\item{Unified Visual Design Validation: T-LF2\\}

Related Reqs: NFR-LF2

Type: Manual, Usability

Initial State: The application interface displays all components (buttons, menus, text fields, images, etc.) with the unified visual design specifications.

Input/Condition: Users or testers navigate through the application to assess the consistency of font type, sizing, color, and background tones across all components.

How test will be performed:
\begin{enumerate}
    \item Review the application's style guide or design specifications to understand the expected visual standards.
    \item Navigate through each section of the application, observing the font type, size, color, and background tones used in various components.
    \item Compare each observed element against the style guide to ensure consistency.
    \item Utilize browser developer tools or design inspection tools to verify CSS properties related to fonts and colors.
\end{enumerate}

\end{enumerate}











% NFRs 11---------------------------------------------------------
\subsubsection{Usability and Humanity Requirements}
% 11.1
\paragraph{Ease of Use Tests}
\begin{enumerate}

\item{Navigation Menu Usability Test: T-UH0\\}

Related Reqs: NFR-UH0

Type: Manual, Usability

Initial State: Users have access to the platform's main interface with a well-organized navigation menu.

Input/Condition: Users are asked to locate and access various main features (e.g., labeling tasks, settings, help, account details) using the navigation menu.

How test will be performed:
\begin{enumerate}
    \item Prepare a set of tasks where users must navigate to specific main features (e.g., "Find the labeling tasks section", "Access account details").
    \item Instruct users to perform each task using the navigation menu, counting the number of clicks required to reach the target feature.
    \item Record whether each task is completed within 3 clicks.
    \item Calculate the percentage of tasks where users were able to navigate to the target feature within 3 clicks.
    \item Ensure that at least \hyperref[MIN_USER_USABILITY_METRIC]{MIN\_USER\_USABILITY\_METRIC}\% of the tasks meet the success criterion.
    \item Analyze any tasks that exceed the 3-click limit to identify navigation issues.
\end{enumerate}

\item{Consistency of UI Elements Validation: T-UH1\\}

Related Reqs: NFR-UH1

Type: Manual, Usability

Initial State: The application interface displays all UI elements (buttons, forms, etc.) following the platform’s visual style guidelines.

Input/Condition: Users navigate through different pages of the application, interacting with various UI elements to assess consistency.

How test will be performed:
\begin{enumerate}
    \item Review the application's style guide or design specifications to identify the expected consistent styles for UI elements.
    \item Identify all recurring interactive elements (e.g., buttons, forms, menus) across different pages of the application.
    \item Instruct users to perform the tasks, observing whether they recognize and correctly use the UI elements based on their consistency.
    \item Calculate the percentage of users who correctly recognized and used the recurring interface elements across different pages.
    \item Ensure that at least \hyperref[MIN_USER_USABILITY_METRIC]{MIN\_USER\_USABILITY\_METRIC}\% of users meet the recognition and correct usage criterion.
\end{enumerate}

\end{enumerate}

% 11.2
\paragraph{Personalization and Internationalization Tests}
\begin{enumerate}

\item{Task Preference Matching Validation: T-UH2\\}

Related Reqs: NFR-UH2

Type: Manual, Usability

Initial State: Users have access to their profile settings where they can set preferences for specific types of labeling tasks (e.g., agricultural, urban).

Input/Condition: Users configure their task preferences in their profile settings and begin receiving assigned tasks over a defined period.

How test will be performed: 
\begin{enumerate}
    \item Select a representative sample of users and have them set their task preferences in their profiles.
    \item Over a set period (e.g., two weeks), track the labeling tasks assigned to these users.
    \item Analyze the assigned tasks to determine the percentage that matches the users' specified preferences.
    \item Verify that at least \hyperref[MIN_PREF_MATCH]{MIN\_PREF\_MATCH}\% of the assigned tasks align with the users' preferences.
\end{enumerate}

\item{Localized Format Validation: T-UH3\\}

Related Reqs: NFR-UH3

Type: Manual, Dynamic

Initial State: Users have access to their profile settings where they can set or have their location automatically detected for date, time, and currency preferences.

Input/Condition: Users either allow the application to detect their location automatically or manually set their preferred locale settings for date, time, and currency.

How test will be performed:
\begin{enumerate}
    \item Select a diverse set of users from different geographical locations with varying locale settings.
    \item For each user, ensure that their location is either correctly detected by the application or manually set their preferred locale settings in their profile.
    \item Navigate to sections of the application where date, time, and currency are displayed (e.g., task deadlines, earnings reports).
    \item Verify that the date formats (e.g., DD/MM/YYYY vs. MM/DD/YYYY), time formats (e.g., 24-hour vs. 12-hour), and currency symbols and formats (e.g., \$, €, ¥) align with the user's locale or manual settings.
    \item Repeat the process after changing the user's locale settings to different regions to ensure the formats update accordingly.
    \item Document any discrepancies where the formats do not match the expected locale settings.
    \item Aggregate the results to ensure that all tested instances correctly display localized formats based on user settings.
\end{enumerate}

\end{enumerate}

% 11.3
\paragraph{Learning Tests}
\begin{enumerate}

\item{Progress Tracker Effectiveness Test: T-UH4\\}

Related Reqs: NFR-UH4

Type: Manual, Usability

Initial State: The application includes a fully functional progress tracker that displays completed and pending tutorials and training tasks.

Input/Condition: Users complete a series of tutorials and training tasks within the platform.

How test will be performed:
\begin{enumerate}
    \item Select a representative sample of users who will undergo the tutorials and training tasks.
    \item Instruct users to navigate to the progress tracker before, during, and after completing the tutorials.
    \item Ensure that the progress tracker accurately displays the status of each learning module, indicating which are completed and which are pending.
    \item After users have completed the training, administer a post-training survey asking them to rate the helpfulness of the progress tracker.
    \item Analyze the survey responses to determine the percentage of users who find the progress tracker helpful.
    \item Verify that at least \hyperref[MIN_TRAINING_TRACKER_USEFULNESS]{MIN\_TRAINING\_TRACKER\_USEFULNESS}\% of the surveyed users rate the progress tracker as helpful.
    \item Collect additional qualitative feedback to identify any issues or areas for improvement in the progress tracking feature.
\end{enumerate}

\item{Simulated Labeling Task Validation: T-UH5\\}

Related Reqs: NFR-UH5

Type: Manual, Usability

Initial State: The platform provides a dedicated section for simulated labeling tasks that do not affect actual datasets.

Input/Condition: New users are prompted to complete at least one simulated labeling task before accessing real labeling tasks.

How test will be performed:
\begin{enumerate}
    \item Recruit a representative sample of new users and onboard them to the platform.
    \item Instruct users to complete a simulated labeling task provided in the practice section.
    \item Track the completion rate of simulated tasks to determine if at least \hyperref[MIN_PRACTICE_TASK_COMPLETION]{MIN\_PRACTICE\_TASK\_COMPLETION}\% of users complete a practice task before accessing real data.
    \item After task completion, administer a satisfaction survey to users to gather feedback on their experience with the simulated tasks.
    \item Analyze survey results to ensure that at least \hyperref[MIN_PRACTICE_TASK_SATISFACTION]{MIN\_PRACTICE\_TASK\_SATISFACTION}\% of users report satisfaction with the practice tasks.
    \item Document any issues or feedback received to identify areas for improvement in the simulated labeling experience.
\end{enumerate}


\end{enumerate}

% 11.4
\paragraph{Understandability and Politeness Tests}
\begin{enumerate}

\item{Contextual Help Pop-Up Effectiveness: T-UH6\\}

Related Reqs: NFR-UH6

Type: Manual, Usability

Initial State: The platform includes contextual help pop-ups integrated throughout various features and task interfaces.

Input/Condition: Users interact with different features and tasks, triggering contextual help pop-ups as needed.

How test will be performed:
\begin{enumerate}
    \item Instruct users to perform a set of typical tasks within the platform, ensuring they engage with areas where contextual help pop-ups are available.
    \item Monitor and record instances where contextual help pop-ups appear during user interactions.
    \item Analyze the survey responses to determine if at least \hyperref[MIN_POPUP_USEFULNESS]{MIN\_POPUP\_USEFULNESS}\% of users find the contextual help pop-ups clear and helpful.
    \item Collect qualitative feedback to identify any areas where help pop-ups could be improved or where additional help pop-ups may be needed.
\end{enumerate}

\item{Error Message Effectiveness Test: T-UH7\\}

Related Reqs: NFR-UH7

Type: Manual, Usability

Initial State: The platform is fully functional and capable of generating error messages in response to various user actions and system failures.

Input/Condition: Users perform actions that intentionally trigger different types of errors (e.g., invalid input, network issues, unauthorized access).

How test will be performed:
\begin{enumerate}
    \item Identify and document common error scenarios within the platform.
    \item Instruct users to perform tasks that will trigger each identified error scenario.
    \item Observe and record the error messages displayed, ensuring they include clear explanations of what went wrong.
    \item Verify that each error message provides actionable steps for resolution.
    \item After encountering errors, administer a survey asking users to rate their frustration level on a predefined scale and assess whether the error messages were helpful.
    \item Analyze the survey results to ensure that at least \hyperref[ERRORS_WITH_INSTRUCTIONS]{ERRORS\_WITH\_INSTRUCTIONS}\% of error messages include actionable instructions and that the reported frustration rate is below \hyperref[FRUSTRATION_RATE]{FRUSTRATION\_RATE}\%.
\end{enumerate}


\end{enumerate}

% 11.5
\paragraph{Accessibility Requirements}
\begin{enumerate}

\item{Accessibility Options Validation: T-UH8\\}

Related Reqs: NFR-UH8

Type: Manual, Usability

Initial State: Users have access to the platform's settings or preferences section where they can adjust text size and select different color themes.

Input/Condition: Users adjust the text size up to 200\% and choose among at least three different color themes (e.g., light, dark, high contrast) to improve readability based on their preferences.

How test will be performed:
\begin{enumerate}
    \item Verify that the settings or preferences section includes options for adjusting text size and selecting color themes.
    \item Select each available color theme (e.g., light, dark, high contrast) and observe the changes in the application's appearance.
    \item Adjust the text size incrementally up to 200\%, ensuring that all text remains readable and that no content is cut off or overlaps.
    \item Navigate through various sections and features of the application with each color theme and text size setting to ensure that functionality is not compromised.
    \item Test the adjustments on different devices and screen resolutions to confirm consistency and responsiveness.
    \item Document any instances where text size adjustments cause loss of content or functionality, or where color themes do not apply uniformly across the platform.
    \item Compile the results to verify that all color themes are available and that text size can be increased up to 200\% without any loss of content or functionality.
\end{enumerate}

\item{Keyboard Shortcuts Accessibility Validation: T-UH9\\}

Related Reqs: NFR-UH9

Type: Manual, Usability

Initial State: Users have access to the platform with all core actions available through both mouse and keyboard interactions.

Input/Condition: Users are instructed to perform essential actions using only keyboard shortcuts to assess accessibility and efficiency.

How test will be performed:
\begin{enumerate}
    \item Identify all core actions within the platform (e.g., navigating between tasks, submitting labels) and ensure each has an assigned keyboard shortcut.
    \item Create a comprehensive list of these keyboard shortcuts for reference during testing.
    \item Recruit a representative sample of users, including power users and those with limited mobility, for usability testing.
    \item Instruct users to perform a series of tasks using only keyboard navigation, utilizing the provided shortcuts.
    \item Track the completion rate of each action to determine if users can access and perform at least \hyperref[MIN_ACTIONS_WITH_KEYBOARD]{MIN\_ACTIONS\_WITH\_KEYBOARD}\% of the core actions via keyboard.
    \item Observe and document any difficulties or barriers users encounter while using keyboard shortcuts.
\end{enumerate}
\end{enumerate}





% NFRs 12---------------------------------------------------------
\subsubsection{Performance Requirements}
% 12.1
\paragraph{Speed and Latency Tests}
\begin{enumerate}

\item{New User Account Processing Time Validation: T-PR0\\}

Related Reqs: NFR-PR0, NFR-PR1 

Type: Manual, Performance

Initial State: The system is ready to accept new user account creation requests.

Input/Condition: A set of new user account creation requests are submitted, and their processing times are tracked.

How test will be performed:
\begin{enumerate}
    \item Define a representative sample size of new user account creation requests to be submitted (e.g., 100 requests).
    \item Submit each account creation request to the system, ensuring that the requests are made under typical operating conditions.
    \item Calculate the processing time for each request by determining the difference between the submission and completion timestamps.
    \item Analyze the processing times to calculate the percentage of requests processed within 15 minutes (0.25 hours).
    \item Verify that at least \hyperref[MIN_ACCOUNT_CREATION_SUCCESS]{MIN\_ACCOUNT\_CREATION\_SUCCESS}\% of the requests are processed within the 15-minute threshold.
    \item Additionally, confirm that all requests are processed within the 48-hour limit.
    \item Document any instances where processing times exceed the specified limits and investigate potential causes.
\end{enumerate}

\item{Service Request Completion Time Validation: T-PR1\\}

Related Reqs: NFR-PR2

Type: Manual, Performance

Initial State: The system is ready to accept service requests from Customers, with negotiated time limits defined for each request.

Input/Condition: A set of service requests are submitted with specific negotiated time limits, and their completion times are tracked.

How test will be performed:
\begin{enumerate}
    \item Define a representative sample size of service requests to be submitted (e.g., 100 requests), each with a predefined negotiated time limit, $t_{serviceRequestTimeLimit}$.
    \item Submit each service request to the system under typical operating conditions, ensuring that the negotiated time limits vary to reflect real-world scenarios.
    \item Calculate the completion time for each request by determining the difference between the submission and completion timestamps.
    \item Analyze the completion times to calculate the percentage of requests processed within the negotiated time limit, $t_{serviceRequestTime} < t_{serviceRequestTimeLimit}$.
    \item Verify that at least \hyperref[MIN_REPORT_RETURN]{MIN\_REPORT\_RETURN}\% of the requests are completed within the negotiated time limit.
    \item Additionally, confirm that all requests are completed within the extended time limit of $t_{serviceRequestTimeLimit} + 48$ hours.
\end{enumerate}

\item{Next Image Serving Time Validation: T-PR2\\}

Related Reqs: NFR-PR3

Type: Automated, Performance

Initial State: Labelers are logged into the platform with a queue of images available for labeling.

Input/Condition: Labelers request the next image to label while images are available in the queue.

How test will be performed:
\begin{enumerate}
    \item Ensure the labeling queue has a sufficient number of images available for testing.
    \item Set up automated performance testing tools or scripts to simulate multiple labelers requesting the next image simultaneously.
    \item For each image request, calculate the time difference between the request and image display for each instance.
    \item Verify that for all instances where a next image is available, the serving time does not exceed \hyperref[MAX_IMAGE_DISPLAY_TIME]{MAX\_IMAGE\_DISPLAY\_TIME} seconds.
    \item Identify and document any instances where the serving time exceeds the \hyperref[MAX_IMAGE_DISPLAY_TIME]{MAX\_IMAGE\_DISPLAY\_TIME}-second threshold.
    \item Analyze the collected data to ensure that the system meets the fit criterion.
\end{enumerate}

\item{Payout Processing Time Validation: T-PR3\\}

Related Reqs: NFR-PR4

Type: Manual, Performance

Initial State: Labelers have earned payouts and have submitted payout requests through the system.

Input/Condition: Payout requests are made by labelers, and the system processes these requests to deliver payments.

How test will be performed:
\begin{enumerate}
    \item Define a representative sample size of payout requests to be processed (e.g., 100 requests).
    \item Ensure that the system is configured to handle payout requests under typical operating conditions.
    \item Submit each payout request through the system, recording the timestamp when each request is made.
    \item Calculate the payout delay for each request.
    \item Analyze the payout delays to verify that all requests are processed within 7 business days.
    \item Identify and document any instances where payout delays exceed 7 business days to investigate potential bottlenecks or issues in the payout process.
\end{enumerate}

\end{enumerate}

% 12.3
\paragraph{Precision or Accuracy Tests}
\begin{enumerate}

\item{Label Accuracy Validation: T-PR4\\}

Related Reqs: NFR-PR5

Type: Automated, Quality Assurance

Initial State: The system has access to a dataset of objects \( O \) with known true classifications \( LTrue \).

Input/Condition: The system processes and labels each object in \( O \), producing guessed classifications \( LGuess \).

How test will be performed:
\begin{enumerate}
    \item Compile a representative sample of objects \( O \) with their true classifications \( LTrue(o) \).
    \item Input each object \( o \in O \) into the system to obtain the system-generated labels \( LGuess(o) \).
    \item Compare each \( LGuess(o) \) with the corresponding \( LTrue(o) \) to determine correctness.
    \item Calculate the label accuracy by dividing the number of correctly labeled objects by the total number of objects:
    \[
    \text{Accuracy} = \frac{|\{ o \in O \mid LGuess(o) = LTrue(o) \}|}{|O|}
    \]
    \item Verify that the calculated accuracy is at least \hyperref[MIN_LABEL_ACCURACY]{MIN\_LABEL\_ACCURACY}\%:
    \[
    \text{Accuracy} \geq \hyperref[MIN_LABEL_ACCURACY]{MIN\_LABEL\_ACCURACY}\%
    \]
    \item Generate a report summarizing the accuracy results, including any discrepancies and potential areas for improvement.
    \item If the accuracy meets or exceeds \hyperref[MIN_LABEL_ACCURACY]{MIN\_LABEL\_ACCURACY}\%, the system satisfies the fit criterion. If not, investigate and address factors contributing to lower accuracy.
\end{enumerate}


\end{enumerate}

% 12.4
\paragraph{Robustness or Fault-Tolerance Tests}
\begin{enumerate}

\item{System Uptime Validation: T-PR5\\}

Related Reqs: NFR-PR6

Type: Automated, Performance

Initial State: The system is fully operational and connected to uptime monitoring tools.

Input/Condition: The system operates continuously under normal conditions over a defined monitoring period.

How test will be performed:
\begin{enumerate}
    \item Implement and configure uptime monitoring tools (e.g., Pingdom, Nagios, or AWS CloudWatch) to continuously track the system's availability.
    \item Define the monitoring period (e.g., one calendar month) to collect sufficient uptime and downtime data.
    \item Ensure that all critical system components and services are included in the monitoring setup.
    \item Collect and log uptime and downtime events throughout the monitoring period.
    \item Calculate the uptime percentage using the formula:
    \[
    \text{Uptime Percentage} = \frac{t_{\text{uptime}}}{t_{\text{uptime}} + t_{\text{downtime}}} \times 100\%
    \]
    \item Verify that the calculated uptime percentage exceeds \hyperref[MIN_UPTIME]{MIN\_UPTIME}\%:
    \[
    \text{Uptime Percentage} > \hyperref[MIN_UPTIME]{MIN\_UPTIME}\%
    \]
    \item Identify and document any instances of downtime, including their duration and underlying causes.
\end{enumerate}

\end{enumerate}

% 12.5
\paragraph{Capacity Tests}
\begin{enumerate}

\item{Labeler Capacity Validation: T-PR6\\}

Related Reqs: NFR-PR7

Type: Automated, Performance

Initial State: The system has a baseline number of active labelers and pending service requests.

Input/Condition: Simulate varying numbers of labelers to ensure all service request deadlines are met as per NFR-PR2.

How test will be performed:
\begin{enumerate}
    \item Use load testing tools (e.g., JMeter, Locust) to simulate different numbers of active labelers.
    \item For each load level, generate a set of service requests that need to be processed within the negotiated deadlines.
    \item Monitor the percentage of service requests completed within the required timeframes.
    \item Identify the maximum number of labelers the system can support while maintaining at least \hyperref[MIN_REPORT_RETURN]{MIN\_REPORT\_RETURN}\% of requests meeting the deadlines.
\end{enumerate}



\item{Large Image File Handling Validation: T-PR7\\}

Related Reqs: NFR-PR8

Type: Automated, Performance

Initial State: The system is configured with sufficient storage capacity and is ready to accept image file uploads.

Input/Condition: Image files up to 50 GB in size are uploaded and processed by the system.

How test will be performed:
\begin{enumerate}
    \item Prepare a set of image files, each up to 50 GB in size.
    \item Use automated scripts to upload each large image file to the system.
    \item Monitor the system for any crashes or failures during the upload and storage process.
    \item Verify that each image file is successfully stored without errors.
    \item Initiate processing tasks on the stored image files.
    \item Ensure that processing completes successfully without system crashes or failures.
    \item Compile the results to confirm that the system can handle image files up to 50 GB without crashing or failing.
\end{enumerate}

\end{enumerate}

% 12.6
\paragraph{Scalability or Extensibility Tests}
\begin{enumerate}

\item{Scalability Validation: T-PR9\\}

Related Reqs: NFR-PR9

Type: Automated, Performance

Initial State: The system is configured for scaling operations with monitoring tools in place.

Input/Condition: Simulate the required labeler capacity as specified in NFR-PR7 to test the system's ability to scale accordingly.

How test will be performed:
\begin{enumerate}
    \item Utilize load testing tools (e.g., JMeter, Locust) to simulate the number of labelers required to meet the capacity defined in NFR-PR7.
    \item Gradually increase the number of simulated labelers while monitoring system performance metrics such as response time, CPU usage, memory consumption, and network throughput.
    \item Ensure that the system dynamically scales resources (e.g., adding more servers or allocating more memory) to handle the increased load without performance degradation.
    \item Verify that all service request deadlines defined in NFR-PR7 are consistently met under the simulated load.
    \item Identify and document any performance bottlenecks or scaling issues encountered during the test.
    \item Compile the results to confirm that the system can scale to meet the required capacity as specified in NFR-PR7.
\end{enumerate}

\end{enumerate}





% NFRs 13---------------------------------------------------------
\subsubsection{Operational and Environmental}
% 13.2
\paragraph{Wider Environmental Tests}
\begin{enumerate}

\item{Energy Efficiency Validation: T-OE0\\}

Related Reqs: NFR-OE0

Type: Automated, Performance

Initial State: The system is running on cloud infrastructure with existing server management configurations.

Input/Condition: Implement energy-efficient practices in cloud usage and server management, then measure energy consumption before and after optimization.

How test will be performed:
\begin{enumerate}
    \item Establish baseline energy consumption by monitoring current cloud infrastructure and server operations using energy monitoring tools.
    \item Identify and implement energy-efficient practices such as optimizing server utilization, enabling power-saving modes, and selecting energy-efficient instance types.
    \item Collect energy consumption data for a defined period (e.g., one month) after implementing the optimizations.
    \item Compare the post-optimization energy consumption data against the baseline measurements.
    \item Calculate the percentage reduction in energy usage to verify that it meets or exceeds the \hyperref[ENERGY_REDUCTION_TARGET]{ENERGY\_REDUCTION\_TARGET}\% target.
    \item Identify any deviations or unexpected increases in energy consumption and investigate potential causes.
\end{enumerate}

\end{enumerate}

% 13.3
\paragraph{Adjacent System Interfacing Tests}
\begin{enumerate}

\item{API and Data Format Integration Validation: T-OE1\\}

Related Reqs: NFR-OE1

Type: Automated, Integration

Initial State: The system is configured with API access credentials for at least two major satellite data providers.

Input/Condition: The system attempts to automatically acquire and integrate satellite images from the specified providers using their standardized APIs and data formats.

How test will be performed:
\begin{enumerate}
    \item Identify two major satellite data providers and obtain their API documentation and access credentials.
    \item Configure the system to connect to each provider's API, ensuring support for their standardized data formats.
    \item Develop and execute automated scripts to initiate data acquisition from each provider.
    \item Monitor the system for successful ingestion of satellite images without manual intervention.
    \item Verify that the acquired data is correctly formatted, stored, and integrated into the platform's datasets.
\end{enumerate}


\item{Payment Processor Integration Validation: T-OE2\\}

Related Reqs: NFR-OE2

Type: Automated, Integration

Initial State: The system is configured with API access credentials for reliable and secure payment processors (e.g., Stripe, PayPal).

Input/Condition: Users and clients perform financial transactions through the integrated payment gateways.

How test will be performed:
\begin{enumerate}
    \item Identify and obtain API documentation and access credentials for at least two major payment processors.
    \item Develop automated scripts to simulate various types of transactions, including user compensations and client payments.
    \item Monitor and log each transaction to verify successful processing without errors or delays.
    \item Calculate the transaction success rate by dividing the number of successful transactions by the total number of attempted transactions.
    \item Ensure that the transaction success rate meets or exceeds \hyperref[MIN_TRANSACTION_SUCCESS_RATE]{MIN\_TRANSACTION\_SUCCESS\_RATE}\%.
\end{enumerate}

\item{Multiple Currency Support Validation: T-OE3\\}

Related Reqs: NFR-OE3

Type: Automated, Integration

Initial State: The system is configured to support multiple currencies, including USD, EUR, GBP, and INR.

Input/Condition: Users perform transactions in each supported currency to verify correct processing.

How test will be performed:
\begin{enumerate}
    \item Configure the system with exchange rates for USD, EUR, GBP, and INR.
    \item Develop automated scripts to simulate transactions in each currency.
    \item Execute the scripts and verify:
    \begin{itemize}
        \item Accurate currency conversion.
        \item Correct display of currency symbols and values.
        \item Proper recording of transactions with the correct currency.
    \end{itemize}
    \item Ensure all transactions are processed without errors.
\end{enumerate}

\item{Machine Learning Framework Compatibility Validation: T-OE4\\}

Related Reqs: NFR-OE4

Type: Automated, Integration

Initial State: The system is configured with machine learning frameworks such as TensorFlow, PyTorch, and scikit-learn installed.

Input/Condition: Users train and deploy models using each framework to verify compatibility.

How test will be performed:
\begin{enumerate}
    \item Install TensorFlow, PyTorch, and scikit-learn on the system.
    \item Develop or use existing sample models compatible with each framework.
    \item Train each sample model using the respective framework.
    \item Deploy the trained models to the production environment.
    \item Verify that training and deployment processes complete without errors.
    \item Ensure that the deployed models function correctly within the platform.
\end{enumerate}

\item{Data Pipeline Efficiency Validation: T-OE5\\}

Related Reqs: NFR-OE5

Type: Automated, Performance

Initial State: The system has established data pipelines for transferring labeled datasets between the platform and ML models.

Input/Condition: Large labeled datasets (e.g., 10,000 images) are transferred through the data pipelines.

How test will be performed:
\begin{enumerate}
    \item Prepare a large labeled dataset consisting of 10,000 images.
    \item Initiate the data transfer from the platform to the ML models using the established pipelines.
    \item Monitor the transfer process to measure the time taken to complete the transfer.
    \item Verify that the data transfer completes within the specified timeframe.
    \item Execute batch processing of the large dataset through the pipelines and ensure no errors occur during the process.
    \item Repeat the transfer and processing under different load conditions to assess pipeline robustness.
\end{enumerate}


\end{enumerate}

% 13.4
\paragraph{Productization Tests}
\begin{enumerate}

\item{Web Browser Accessibility Validation: T-OE6\\}

Related Reqs: NFR-OE6

Type: Manual, Usability

Initial State: Users have access to the platform's web URL.

Input/Condition: Users attempt to access and use the platform via various supported web browsers without installing any software.

How test will be performed:
\begin{enumerate}
    \item Identify supported web browsers (e.g., Chrome, Firefox, Safari, Edge).
    \item Using each browser, navigate to the platform's web URL.
    \item Verify that the platform loads correctly without prompting for any software installations.
    \item Perform key tasks (e.g., login, label images, access settings) to ensure full functionality is available through the browser.
    \item Confirm that all users can access and use the platform solely through their web browsers without additional installations.
\end{enumerate}

\end{enumerate}


\subsubsection{Area of Testing1}
		
\paragraph{Title for Test}

\begin{enumerate}

\item{test-id1\\}

Type: Functional, Dynamic, Manual, Static etc.
					
Initial State: 
					
Input/Condition: 
					
Output/Result: 
					
How test will be performed: 
					
\item{test-id2\\}

Type: Functional, Dynamic, Manual, Static etc.
					
Initial State: 
					
Input: 
					
Output: 
					
How test will be performed: 

\end{enumerate}

\subsubsection{Area of Testing2}

...

\subsection{Traceability Between Test Cases and Requirements}

\wss{Provide a table that shows which test cases are supporting which
  requirements.}

\section{Unit Test Description}

\wss{This section should not be filled in until after the MIS (detailed design
  document) has been completed.}

\wss{Reference your MIS (detailed design document) and explain your overall
philosophy for test case selection.}  

\wss{To save space and time, it may be an option to provide less detail in this section.  
For the unit tests you can potentially layout your testing strategy here.  That is, you 
can explain how tests will be selected for each module.  For instance, your test building 
approach could be test cases for each access program, including one test for normal behaviour 
and as many tests as needed for edge cases.  Rather than create the details of the input 
and output here, you could point to the unit testing code.  For this to work, you code 
needs to be well-documented, with meaningful names for all of the tests.}

\subsection{Unit Testing Scope}

\wss{What modules are outside of the scope.  If there are modules that are
  developed by someone else, then you would say here if you aren't planning on
  verifying them.  There may also be modules that are part of your software, but
  have a lower priority for verification than others.  If this is the case,
  explain your rationale for the ranking of module importance.}

\subsection{Tests for Functional Requirements}

\wss{Most of the verification will be through automated unit testing.  If
  appropriate specific modules can be verified by a non-testing based
  technique.  That can also be documented in this section.}

\subsubsection{Module 1}

\wss{Include a blurb here to explain why the subsections below cover the module.
  References to the MIS would be good.  You will want tests from a black box
  perspective and from a white box perspective.  Explain to the reader how the
  tests were selected.}

\begin{enumerate}

\item{test-id1\\}

Type: \wss{Functional, Dynamic, Manual, Automatic, Static etc. Most will
  be automatic}
					
Initial State: 
					
Input: 
					
Output: \wss{The expected result for the given inputs}

Test Case Derivation: \wss{Justify the expected value given in the Output field}

How test will be performed: 
					
\item{test-id2\\}

Type: \wss{Functional, Dynamic, Manual, Automatic, Static etc. Most will
  be automatic}
					
Initial State: 
					
Input: 
					
Output: \wss{The expected result for the given inputs}

Test Case Derivation: \wss{Justify the expected value given in the Output field}

How test will be performed: 

\item{...\\}
    
\end{enumerate}

\subsubsection{Module 2}

...

\subsection{Tests for Nonfunctional Requirements}

\wss{If there is a module that needs to be independently assessed for
  performance, those test cases can go here.  In some projects, planning for
  nonfunctional tests of units will not be that relevant.}

\wss{These tests may involve collecting performance data from previously
  mentioned functional tests.}

\subsubsection{Module ?}
		
\begin{enumerate}

\item{test-id1\\}

Type: \wss{Functional, Dynamic, Manual, Automatic, Static etc. Most will
  be automatic}
					
Initial State: 
					
Input/Condition: 
					
Output/Result: 
					
How test will be performed: 
					
\item{test-id2\\}

Type: Functional, Dynamic, Manual, Static etc.
					
Initial State: 
					
Input: 
					
Output: 
					
How test will be performed: 

\end{enumerate}

\subsubsection{Module ?}

...

\subsection{Traceability Between Test Cases and Modules}

\wss{Provide evidence that all of the modules have been considered.}
				
\bibliographystyle{plainnat}

\bibliography{../../refs/References}

\newpage

\section{Appendix}

This is where you can place additional information.

\subsection{Symbolic Parameters}

The definition of the test cases will call for SYMBOLIC\_CONSTANTS.
Their values are defined in this section for easy maintenance.


% MIN\_USER\_USABILITY\_METRIC & 90 & \% & Minimum percent of users that satisfy usability requirements \label{MIN_USER_USABILITY_METRIC} \\ \hline
% MIN\_PREF\_MATCH & 80 & \% & Minimum percent of label task preference match \label{MIN_PREF_MATCH} \\ \hline
% MIN\_TRAINING\_TRACKER\_USEFULNESS & 80 & \% & Minimum percent of trackers usefulness \label{MIN_TRAINING_TRACKER_USEFULNESS} \\ \hline
% MIN\_PRACTICE\_TASK\_COMPLETION & 85 & \% & Minimum percent of user to complete practice task \label{MIN_PRACTICE_TASK_COMPLETION} \\ \hline
% MIN\_POPUP\_USEFULNESS & 80 & \% & Minimum percent of users to find popups useful \label{MIN_POPUP_USEFULNESS} \\ \hline
% ERRORS\_WITH\_INSTRUCTIONS & 80 & \% & Percent of errors with follow up instructions \label{ERRORS_WITH_INSTRUCTIONS} \\ \hline
% FRUSTRATION\_RATE & 10 & \% & Frustration rate when encountering errors \label{FRUSTRATION_RATE} \\ \hline
% MIN\_ACTIONS\_WITH\_KEYBOARD & 95 & \% & Minimum percent of actions available with keyboard inputs \label{MIN_ACTIONS_WITH_KEYBOARD} \\ \hline
% MIN\_ACCOUNT\_CREATION\_SUCCESS & 90 & \% & Minimum percent of accounts created successfully within 48 hours \label{MIN_ACCOUNT_CREATION_SUCCESS} \\ \hline
% MIN\_REPORT\_RETURN & 90 & \% & Minimum percent of reports returned to customer within time frame \label{MIN_REPORT_RETURN} \\ \hline
% MAX\_IMAGE\_DISPLAY\_TIME & 10 & \% & Maximum time to display image to labeller \label{MAX_IMAGE_DISPLAY_TIME} \\ \hline
% MIN\_LABEL\_ACCURACY & 75 & \% & Minimum percent of label accuracy \label{MIN_LABEL_ACCURACY} \\ \hline
% MIN\_UPTIME & 80 & \% & Minimum percent of system uptime \label{MIN_UPTIME} \\ \hline
% ENERGY\_REDUCTION\_TARGET & 20 & \% & Minimum percent of achieved energy reduction \label{ENERGY_REDUCTION_TARGET} \\ \hline
% MIN\_TRANSACTION\_SUCCESS\_RATE & 99.5 & \% & Minimum percent of successful transactions \label{MIN_TRANSACTION_SUCCESS_RATE} \\ \hline

\subsection{Usability Survey Questions?}

\wss{This is a section that would be appropriate for some projects.}

\begin{enumerate}
    \item On a scale of 1 to 5, how would you rate the readability of text at this screen size? 
    \item On a scale of 1 to 5, how clear was the visual feedback when interacting with buttons and other interactive elements?
    \item On a scale of 1 to 5, how consistent do you find the visual design across different sections of the application?
    \item On a scale of 1 to 5, how intuitive did you find the navigation menu?
    \item On a scale of 1 to 5, how consistent did you find the style of buttons and forms across different pages of the application?
    \item On a scale of 1 to 5, how satisfied are you with the alignment of your assigned tasks to your preferences?
    \item On a scale of 1 to 5, how helpful do you find the progress tracker in monitoring your training progress?
    \item On a scale of 1 to 5, how satisfied are you with the simulated labeling tasks provided for practice?
    \item On a scale of 1 to 5, how clear and understandable did you find the contextual help pop-ups?
    \item On a scale of 1 to 5, how clear and actionable were the error messages you encountered?
    \item On a scale of 1 to 5, how easy was it to adjust the text size to your preference?
    \item On a scale of 1 to 5, how satisfied are you with the available color themes for improving readability?
    \item On a scale of 1 to 5, how easy was it to access the platform through your web browser? 

    
    % Matthews stuff

\end{enumerate}

\newpage{}
\section*{Appendix --- Reflection}

\wss{This section is not required for CAS 741}

The information in this section will be used to evaluate the team members on the
graduate attribute of Lifelong Learning.

The purpose of reflection questions is to give you a chance to assess your own
learning and that of your group as a whole, and to find ways to improve in the
future. Reflection is an important part of the learning process.  Reflection is
also an essential component of a successful software development process.  

Reflections are most interesting and useful when they're honest, even if the
stories they tell are imperfect. You will be marked based on your depth of
thought and analysis, and not based on the content of the reflections
themselves. Thus, for full marks we encourage you to answer openly and honestly
and to avoid simply writing ``what you think the evaluator wants to hear.''

Please answer the following questions.  Some questions can be answered on the
team level, but where appropriate, each team member should write their own
response:


\begin{enumerate}
  \item What went well while writing this deliverable? 
  \item What pain points did you experience during this deliverable, and how
    did you resolve them?
  \item What knowledge and skills will the team collectively need to acquire to
  successfully complete the verification and validation of your project?
  Examples of possible knowledge and skills include dynamic testing knowledge,
  static testing knowledge, specific tool usage, Valgrind etc.  You should look to
  identify at least one item for each team member.
  \item For each of the knowledge areas and skills identified in the previous
  question, what are at least two approaches to acquiring the knowledge or
  mastering the skill?  Of the identified approaches, which will each team
  member pursue, and why did they make this choice?
\end{enumerate}

\end{document}