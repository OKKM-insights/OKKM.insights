\documentclass[12pt, titlepage]{article}
\usepackage[style=numeric]{biblatex} % We are using biblatex with numeric style sorted alphabetically
\addbibresource{references.bib}

\usepackage{booktabs}
\usepackage{tabularx}
\usepackage{hyperref}
\hypersetup{
    colorlinks,
    citecolor=blue,
    filecolor=black,
    linkcolor=red,
    urlcolor=blue
}

%% Comments

\usepackage{color}

\newif\ifcomments\commentstrue %displays comments
%\newif\ifcomments\commentsfalse %so that comments do not display

\ifcomments
\newcommand{\authornote}[3]{\textcolor{#1}{[#3 ---#2]}}
\newcommand{\todo}[1]{\textcolor{red}{[TODO: #1]}}
\else
\newcommand{\authornote}[3]{}
\newcommand{\todo}[1]{}
\fi

\newcommand{\wss}[1]{\authornote{blue}{SS}{#1}} 
\newcommand{\plt}[1]{\authornote{magenta}{TPLT}{#1}} %For explanation of the template
\newcommand{\an}[1]{\authornote{cyan}{Author}{#1}}

%% Common Parts

\newcommand{\progname}{Software Engineering} % PUT YOUR PROGRAM NAME HERE
\newcommand{\authname}{Team \#11, OKKM Insights
\\ Mathew Petronilho
\\ Oleg Glotov
\\ Kyle McMaster
\\ Kartik Chaudhari} % AUTHOR NAMES                  

\usepackage{hyperref}
    \hypersetup{colorlinks=true, linkcolor=blue, citecolor=blue, filecolor=blue,
                urlcolor=blue, unicode=false}
    \urlstyle{same}
                                



\begin{document}

\title{System Verification and Validation Plan for \progname{}} 
\author{\authname}
\date{\today}
	
\maketitle

\pagenumbering{roman}

\section*{Revision History}

\begin{tabularx}{\textwidth}{p{3cm}p{2cm}X}
\toprule {\bf Date} & {\bf Version} & {\bf Notes}\\
\midrule
Date 1 & 1.0 & Notes\\
Date 2 & 1.1 & Notes\\
\bottomrule
\end{tabularx}

~\\
\wss{The intention of the VnV plan is to increase confidence in the software.
However, this does not mean listing every verification and validation technique
that has ever been devised.  The VnV plan should also be a \textbf{feasible}
plan. Execution of the plan should be possible with the time and team available.
If the full plan cannot be completed during the time available, it can either be
modified to ``fake it'', or a better solution is to add a section describing
what work has been completed and what work is still planned for the future.}

\wss{The VnV plan is typically started after the requirements stage, but before
the design stage.  This means that the sections related to unit testing cannot
initially be completed.  The sections will be filled in after the design stage
is complete.  the final version of the VnV plan should have all sections filled
in.}

\newpage

\tableofcontents

\listoftables
\wss{Remove this section if it isn't needed}

\listoffigures
\wss{Remove this section if it isn't needed}

\newpage

\section{Symbols, Abbreviations, and Acronyms}

\renewcommand{\arraystretch}{1.2}
\begin{tabular}{l l} 
  \toprule		
  \textbf{symbol} & \textbf{description}\\
  \midrule 
  T & Test\\
  \bottomrule
\end{tabular}\\

\wss{symbols, abbreviations, or acronyms --- you can simply reference the SRS
  \citep{SRS} tables, if appropriate}

\wss{Remove this section if it isn't needed}

\newpage

\pagenumbering{arabic}

This document ... \wss{provide an introductory blurb and roadmap of the
  Verification and Validation plan}

  \section{2 General Information}

  \subsection{2.1 Summary} \cite{Problem_Statement}
  
  Orbit Watch is an online platform developed by OKKM Insights to streamline and simplify satellite imagery data analysis. The software addresses the scarcity of high-quality, labeled satellite imagery datasets tailored for specific use cases across various industries such as disaster response, environmental monitoring, urban planning, and defense. Leveraging an AI-powered crowd-sourcing model, Orbit Watch enables users to label commercially available satellite images. These labeled datasets are then utilized to train custom computer vision models, enhancing accuracy and efficiency in image analysis. The platform offers a paid service for identifying objects within satellite images and distributes earnings to users who contribute to the labeling efforts.
  
  \subsection{2.2 Objectives} \cite{Problem_Statement}
  
  The primary objectives of the Verification and Validation (V\&V) plan for Orbit Watch are:
  
  \begin{itemize}
      \item \textbf{Ensure High Data Accuracy}: Build confidence in the software's correctness by verifying that the system achieves high classification accuracy for objects reported in the images. This is crucial for extracting useful information and providing value to stakeholders.
      \item \textbf{Validate System Reliability and Accessibility}: Confirm that the system is reliable with minimal downtime and is accessible remotely for both purchasers and labelers, ensuring flexibility and efficiency in usage.
  \end{itemize}
  
  The following objectives are out of scope for this V\&V plan:
  
  \begin{itemize}
      \item \textbf{Verification of External Components}: We will not verify external libraries or services used for image acquisition, payment processing, or AI algorithms. It is assumed that these components have been adequately verified by their respective development teams.
      \item \textbf{Stretch Goals}: Objectives related to automatic data labeling and multi-source integration are considered stretch goals and will not be included in the current V\&V efforts due to time and resource constraints.
  \end{itemize}
  
  \subsection{2.3 Challenge Level and Extras}\cite{Development_Plan} \cite{Problem_Statement}
  
  \, \, \, We aim to classify this project as an advanced level project. \\

  This project can be classified at the \textbf{advanced} challenge level due to several factors:
  
  \begin{itemize}
      \item \textbf{Limited Domain Knowledge}: The team has minimal experience with satellite imagery and computer vision models, requiring additional research and learning to implement effective solutions.
      \item \textbf{Complex Implementation}: Developing a web application from scratch involves challenges in front-end development, integrating secure payment systems, handling parallel image labeling tasks, and designing consensus algorithms for data accuracy.
      \item \textbf{Integration of Multiple Components}: The project demands seamless integration of crowd-sourcing models, AI algorithms, payment processing, and user interfaces, increasing the complexity of development and testing.
  \end{itemize}
  

\textbf{Extras from the Problem Statement} \cite{Problem_Statement}

\begin{itemize}
    \item \textbf{Usability Testing}: This phase involves selecting a group of users, ideally from the target demographic (e.g., professionals in satellite imagery analysis or end-users in agriculture or rescue services). These users interact with the application's interface, performing tasks relevant to the platform's purpose, like labeling images or reviewing AI-generated annotations. They then provide feedback via a structured questionnaire focusing on usability aspects—such as ease of use, intuitiveness, and clarity. This feedback helps refine the user interface to ensure it meets user expectations and requirements.

    \item \textbf{Demonstration Video}: The video will serve as an onboarding tool, guiding users on how to navigate and utilize the platform effectively. It will highlight key features, demonstrate a labeling task from start to finish, and showcase any unique functionalities, such as model-assisted annotations or accuracy checks. The video can be used in training materials or as a quick tutorial accessible directly from the platform to help new users become familiar with the tool's workflow.

    \item \textbf{Formal Proof of Convergence for Labeled Images}: This proof aims to establish that labeled images converge toward a standard of accuracy and consistency. This involves statistical or mathematical validation showing that repeated labeling by different users (or with the assistance of AI) leads to a stable and reliable dataset over time. Proving convergence is crucial in ensuring that the labeling quality remains high, even as new users interact with the platform. This might include metrics such as inter-annotator agreement or AI confidence scores, contributing to the credibility and reliability of the labeled dataset.
\end{itemize}
  
\subsection{2.4 Relevant Documentation}

\bigskip
Created BibTeX entries for our documents and within those entries we have included a hyperlink to the documents. As listed above. Here is a sample of the references for all of the documents. Future references are also included in this sample. \\

\begin{itemize}
    \item Development Plan - \cite{Development_Plan}
    \item Module Guide - \cite{MG}
    \item Module Interface Specification - \cite{MIS}
    \item Verification and Validation Plan - \cite{VnV}
    \item Problem Statement - \cite{Problem_Statement}
    \item Software Requirements Specification - \cite{SRS}
\end{itemize}

By referencing these documents, we ensure that our V\&V efforts are comprehensive and aligned with the project's goals and requirements. Each document provides specific insights that help us design effective verification and validation activities:

\begin{itemize}
    \item \textbf{Problem Statement} (\cite{Problem_Statement}): Outlines the core challenges and objectives of the Orbit Watch project, providing context and motivation for development. It helps align our V\&V activities with the original goals and ensures that the software addresses the identified needs.

    \item \textbf{Development Plan} (\cite{Development_Plan}): Details the project timeline, milestones, and resource allocation. It assists in scheduling V\&V activities appropriately within the development lifecycle, ensuring they are integrated at optimal points for maximum effectiveness.

    \item \textbf{Software Requirements Specification (SRS)} (\cite{SRS}): Defines all functional and non-functional requirements of Orbit Watch. It serves as the foundation for developing test cases to verify that the software meets all specified requirements and performs as intended.

    \item \textbf{Module Guide (MG)} (\cite{MG}): Provides an overview of the system's modular architecture, including the decomposition into modules and their relationships. This information is vital for planning integration testing and ensuring that the modules interact correctly.

    \item \textbf{Module Interface Specification (MIS)} (\cite{MIS}): Details the interfaces for each module, specifying input and output parameters. This is essential for verifying that modules communicate correctly and adhere to their specified interfaces, which is critical for system reliability.

    \item \textbf{Verification and Validation Plan (V\&V Plan)} (\cite{VnV}): Outlines the strategies, methodologies, and resources allocated for the V\&V activities. It guides our testing efforts and ensures that all aspects of the system are adequately verified and validated according to the project's objectives.
\end{itemize}

By utilizing these documents, our V\&V efforts are thorough and focused on delivering a reliable and accurate platform for satellite imagery analysis.


\section{Plan}

\wss{Introduce this section.  You can provide a roadmap of the sections to
  come.}

\subsection{Verification and Validation Team}

\wss{Your teammates.  Maybe your supervisor.
  You should do more than list names.  You should say what each person's role is
  for the project's verification.  A table is a good way to summarize this information.}

\subsection{SRS Verification Plan}

\wss{List any approaches you intend to use for SRS verification.  This may
  include ad hoc feedback from reviewers, like your classmates (like your
  primary reviewer), or you may plan for something more rigorous/systematic.}

\wss{If you have a supervisor for the project, you shouldn't just say they will
read over the SRS.  You should explain your structured approach to the review.
Will you have a meeting?  What will you present?  What questions will you ask?
Will you give them instructions for a task-based inspection?  Will you use your
issue tracker?}

\wss{Maybe create an SRS checklist?}

\subsection{Design Verification Plan}

\wss{Plans for design verification}

\wss{The review will include reviews by your classmates}

\wss{Create a checklists?}

\subsection{Verification and Validation Plan Verification Plan}

\wss{The verification and validation plan is an artifact that should also be
verified.  Techniques for this include review and mutation testing.}

\wss{The review will include reviews by your classmates}

\wss{Create a checklists?}

\subsection{Implementation Verification Plan}

\wss{You should at least point to the tests listed in this document and the unit
  testing plan.}

\wss{In this section you would also give any details of any plans for static
  verification of the implementation.  Potential techniques include code
  walkthroughs, code inspection, static analyzers, etc.}

\wss{The final class presentation in CAS 741 could be used as a code
walkthrough.  There is also a possibility of using the final presentation (in
CAS741) for a partial usability survey.}

\subsection{Automated Testing and Verification Tools}

\wss{What tools are you using for automated testing.  Likely a unit testing
  framework and maybe a profiling tool, like ValGrind.  Other possible tools
  include a static analyzer, make, continuous integration tools, test coverage
  tools, etc.  Explain your plans for summarizing code coverage metrics.
  Linters are another important class of tools.  For the programming language
  you select, you should look at the available linters.  There may also be tools
  that verify that coding standards have been respected, like flake9 for
  Python.}

\wss{If you have already done this in the development plan, you can point to
that document.}

\wss{The details of this section will likely evolve as you get closer to the
  implementation.}

\subsection{Software Validation Plan}

\wss{If there is any external data that can be used for validation, you should
  point to it here.  If there are no plans for validation, you should state that
  here.}

\wss{You might want to use review sessions with the stakeholder to check that
the requirements document captures the right requirements.  Maybe task based
inspection?}

\wss{For those capstone teams with an external supervisor, the Rev 0 demo should 
be used as an opportunity to validate the requirements.  You should plan on 
demonstrating your project to your supervisor shortly after the scheduled Rev 0 demo.  
The feedback from your supervisor will be very useful for improving your project.}

\wss{For teams without an external supervisor, user testing can serve the same purpose 
as a Rev 0 demo for the supervisor.}

\wss{This section might reference back to the SRS verification section.}

\section{System Tests}

\wss{There should be text between all headings, even if it is just a roadmap of
the contents of the subsections.}

\subsection{Tests for Functional Requirements}

\section*{FR0: Customer Account Creation Test}

\begin{enumerate}
    \item \textbf{Test ID}: FR0\_Test\_01
    \item \textbf{Control}: \textbf{Automated}
    \item \textbf{Initial State}: The system is accessible, and no user account currently exists for the test user.
    \item \textbf{Input}:
    \begin{itemize}
        \item Customer provides valid personal information:
        \begin{itemize}
            \item \textbf{Name}: ``Alice Smith''
            \item \textbf{Email}: \texttt{alice.smith@example.com}
            \item \textbf{Password}: ``StrongPassword!2021''
        \end{itemize}
        \item Customer agrees to the system's privacy policy.
    \end{itemize}
    \item \textbf{Output}:
    \begin{itemize}
        \item \textbf{Expected Result}:
        \begin{itemize}
            \item A new user account is created.
            \item Account information is securely stored in the database.
            \item The customer is redirected to the login page with a success message.
        \end{itemize}
    \end{itemize}
    \item \textbf{Test Case Derivation}:
    \begin{itemize}
        \item Based on the requirement that customers must have an account to access system services. This test verifies that with valid input and acceptance of terms, the account creation process is successfully executed and persists data appropriately.
    \end{itemize}
    \item \textbf{How Test Will Be Performed}:
    \begin{itemize}
        \item \textbf{Automated Test Script Execution}:
        \begin{enumerate}
            \item \textbf{Step 1}: Navigate to the account creation page using a testing tool like Selenium WebDriver.
            \item \textbf{Step 2}: Fill in the registration form fields with the provided valid data.
            \item \textbf{Step 3}: Check the agreement box for the privacy policy.
            \item \textbf{Step 4}: Submit the registration form.
            \item \textbf{Step 5}: Verify that the response indicates successful account creation (e.g., success message displayed).
            \item \textbf{Step 6}: Confirm redirection to the login page.
            \item \textbf{Step 7}: Query the database to verify that a new user record exists with the email \texttt{alice.smith@example.com}.
            \item \textbf{Step 8}: Ensure the password is hashed and not stored in plain text.
        \end{enumerate}
        \item \textbf{Cleanup}:
        \begin{itemize}
            \item After verification, delete the test user account from the database to maintain a clean state for future tests.
        \end{itemize}
    \end{itemize}
\end{enumerate}
\section*{FR1: Customer Authentication Test}

\begin{enumerate}
    \item \textbf{Test ID}: FR1\_Test\_01
    \item \textbf{Control}: \textbf{Automated}
    \item \textbf{Initial State}: An existing user account with the following credentials:
    \begin{itemize}
        \item \textbf{Email}: \texttt{user.test@example.com}
        \item \textbf{Password}: ``TestPass\#123''
    \end{itemize}
    \item \textbf{Input}:
    \begin{itemize}
        \item Customer enters the correct login credentials:
        \begin{itemize}
            \item \textbf{Email}: \texttt{user.test@example.com}
            \item \textbf{Password}: ``TestPass\#123''
        \end{itemize}
    \end{itemize}
    \item \textbf{Output}:
    \begin{itemize}
        \item \textbf{Expected Result}:
        \begin{itemize}
            \item Customer is successfully authenticated.
            \item Access to privileged information (e.g., user dashboard) is granted.
            \item A session token or cookie is established for the user session.
        \end{itemize}
    \end{itemize}
    \item \textbf{Test Case Derivation}:
    \begin{itemize}
        \item Ensures that only authenticated customers can access the system, satisfying security requirements.
    \end{itemize}
    \item \textbf{How Test Will Be Performed}:
    \begin{itemize}
        \item \textbf{Automated Test Script Execution}:
        \begin{enumerate}
            \item \textbf{Step 1}: Navigate to the login page.
            \item \textbf{Step 2}: Input the correct email and password.
            \item \textbf{Step 3}: Submit the login form.
            \item \textbf{Step 4}: Verify redirection to the user dashboard or home page.
            \item \textbf{Step 5}: Check for the presence of privileged information on the page.
            \item \textbf{Step 6}: Verify that a session token or authentication cookie has been set.
            \item \textbf{Negative Test}:
            \begin{enumerate}
                \item \textbf{Step 7}: Attempt to access the dashboard with incorrect credentials to ensure access is denied.
            \end{enumerate}
        \end{enumerate}
        \item \textbf{Session Validation}:
        \begin{enumerate}
            \item \textbf{Step 8}: Use the session token to access a secure API endpoint to confirm authentication persistence.
        \end{enumerate}
    \end{itemize}
\end{enumerate}

\section*{FR2: Customer Account Modification Test}

\begin{enumerate}
    \item \textbf{Test ID}: FR2\_Test\_01
    \item \textbf{Control}: \textbf{Manual}
    \item \textbf{Initial State}: Customer is logged in and has access to their account information page.
    \item \textbf{Input}:
    \begin{itemize}
        \item Customer updates personal information fields:
        \begin{itemize}
            \item \textbf{Address}: ``123 New Street, Cityville''
            \item \textbf{Phone Number}: ``555-1234''
            \item \textbf{Profile Picture}: Uploads a new image file.
        \end{itemize}
    \end{itemize}
    \item \textbf{Output}:
    \begin{itemize}
        \item \textbf{Expected Result}:
        \begin{itemize}
            \item Updated personal information is saved and reflected in the database.
            \item The user receives a confirmation message indicating successful update.
        \end{itemize}
    \end{itemize}
    \item \textbf{Test Case Derivation}:
    \begin{itemize}
        \item Confirms that authenticated customers can modify their information, maintaining data accuracy and relevance.
    \end{itemize}
    \item \textbf{How Test Will Be Performed}:
    \begin{itemize}
        \item \textbf{Manual Steps}:
        \begin{enumerate}
            \item \textbf{Step 1}: Log in to the customer account.
            \item \textbf{Step 2}: Navigate to the account settings or profile page.
            \item \textbf{Step 3}: Change the address and phone number to the new values.
            \item \textbf{Step 4}: Upload a new profile picture.
            \item \textbf{Step 5}: Save the changes.
            \item \textbf{Step 6}: Verify that the changes are displayed correctly on the profile page.
            \item \textbf{Step 7}: Check the database to ensure the new information is updated.
            \item \textbf{Step 8}: Log out and log back in to confirm that the changes persist.
        \end{enumerate}
    \end{itemize}
\end{enumerate}

\section*{FR3: Payment Processing Test}

\begin{enumerate}
    \item \textbf{Test ID}: FR3\_Test\_01
    \item \textbf{Control}: \textbf{Automated}
    \item \textbf{Initial State}: Customer is logged in and ready to make a payment for a service request.
    \item \textbf{Input}:
    \begin{itemize}
        \item Valid payment information:
        \begin{itemize}
            \item \textbf{Credit Card Number}: ``4111 1111 1111 1111'' (Test Visa number)
            \item \textbf{Expiry Date}: ``12/25''
            \item \textbf{CVV}: ``123''
            \item \textbf{Billing Address}: Matches the address on file.
        \end{itemize}
    \end{itemize}
    \item \textbf{Output}:
    \begin{itemize}
        \item \textbf{Expected Result}:
        \begin{itemize}
            \item Payment is processed successfully.
            \item A confirmation receipt is generated and emailed to the customer.
            \item The service request status is updated to ``Paid'' or equivalent.
        \end{itemize}
    \end{itemize}
    \item \textbf{Test Case Derivation}:
    \begin{itemize}
        \item Verifies payment integration, ensuring that authenticated customers can complete transactions securely.
    \end{itemize}
    \item \textbf{How Test Will Be Performed}:
    \begin{itemize}
        \item \textbf{Automated Test Script Execution}:
        \begin{enumerate}
            \item \textbf{Step 1}: Navigate to the payment page for the pending service request.
            \item \textbf{Step 2}: Input the valid payment details.
            \item \textbf{Step 3}: Submit the payment form.
            \item \textbf{Step 4}: Mock the payment gateway response if using a sandbox environment.
            \item \textbf{Step 5}: Verify that the system displays a payment success message.
            \item \textbf{Step 6}: Check that a confirmation receipt is generated and sent to the customer's email.
            \item \textbf{Step 7}: Verify that the service request status is updated appropriately in the database.
        \end{enumerate}
        \item \textbf{Security Verification}:
        \begin{enumerate}
            \item \textbf{Step 8}: Ensure that sensitive payment information is not stored in plain text and complies with PCI DSS standards.
        \end{enumerate}
    \end{itemize}
\end{enumerate}

\section*{FR4: Service Request Submission Test}

\begin{enumerate}
    \item \textbf{Test ID}: FR4\_Test\_01
    \item \textbf{Control}: \textbf{Manual}
    \item \textbf{Initial State}: Customer is logged in and has a confirmed payment.
    \item \textbf{Input}:
    \begin{itemize}
        \item Customer fills out the service request form with necessary details:
        \begin{itemize}
            \item \textbf{Service Type}: ``Image Analysis''
            \item \textbf{Description}: ``Analysis of satellite images for deforestation.''
            \item \textbf{Preferred Completion Date}: ``2023-12-31''
        \end{itemize}
    \end{itemize}
    \item \textbf{Output}:
    \begin{itemize}
        \item \textbf{Expected Result}:
        \begin{itemize}
            \item Service request is accepted and logged in the system.
            \item Customer receives a confirmation message and request ID.
        \end{itemize}
    \end{itemize}
    \item \textbf{Test Case Derivation}:
    \begin{itemize}
        \item Ensures that the system accepts valid requests from authenticated customers.
    \end{itemize}
    \item \textbf{How Test Will Be Performed}:
    \begin{itemize}
        \item \textbf{Manual Steps}:
        \begin{enumerate}
            \item \textbf{Step 1}: Navigate to the new service request page.
            \item \textbf{Step 2}: Fill in the form with the input data.
            \item \textbf{Step 3}: Submit the form.
            \item \textbf{Step 4}: Verify that a confirmation message with a unique request ID is displayed.
            \item \textbf{Step 5}: Check the database to confirm that the service request is logged with correct details.
            \item \textbf{Step 6}: Ensure that the service request appears in the customer's list of active requests.
        \end{enumerate}
    \end{itemize}
\end{enumerate}

\section*{FR5: Service Report Delivery Test}

\begin{enumerate}
    \item \textbf{Test ID}: FR5\_Test\_01
    \item \textbf{Control}: \textbf{Automated}
    \item \textbf{Initial State}: Customer is logged in and has a completed service request.
    \item \textbf{Input}:
    \begin{itemize}
        \item Customer navigates to the ``My Reports'' section after being notified of service completion.
    \end{itemize}
    \item \textbf{Output}:
    \begin{itemize}
        \item \textbf{Expected Result}:
        \begin{itemize}
            \item The service report is available for viewing and download.
            \item Report contents are accurate and correspond to the service request.
        \end{itemize}
    \end{itemize}
    \item \textbf{Test Case Derivation}:
    \begin{itemize}
        \item Confirms that customers receive reports on completed services, fulfilling the system’s purpose.
    \end{itemize}
    \item \textbf{How Test Will Be Performed}:
    \begin{itemize}
        \item \textbf{Automated Test Script Execution}:
        \begin{enumerate}
            \item \textbf{Step 1}: Simulate the completion of a service request in the system (can be mocked or set up in a test environment).
            \item \textbf{Step 2}: Log in as the customer.
            \item \textbf{Step 3}: Navigate to the ``My Reports'' or equivalent section.
            \item \textbf{Step 4}: Verify that the completed service report is listed.
            \item \textbf{Step 5}: Open the report and check that it loads correctly.
            \item \textbf{Step 6}: Validate that the report content matches the expected output based on the service provided.
            \item \textbf{Step 7}: Attempt to download the report and ensure the file is intact and accessible.
        \end{enumerate}
    \end{itemize}
\end{enumerate}

\section*{FR6: Image Upload Test}

\begin{enumerate}
    \item \textbf{Test ID}: FR6\_Test\_01
    \item \textbf{Control}: \textbf{Manual}
    \item \textbf{Initial State}: Customer is logged in with an active service request requiring image uploads.
    \item \textbf{Input}:
    \begin{itemize}
        \item Customer uploads multiple image files:
        \begin{itemize}
            \item \texttt{Image1.jpg}: 2\,MB
            \item \texttt{Image2.png}: 3\,MB
            \item \texttt{Image3.tif}: 5\,MB
        \end{itemize}
    \end{itemize}
    \item \textbf{Output}:
    \begin{itemize}
        \item \textbf{Expected Result}:
        \begin{itemize}
            \item All images are successfully uploaded and stored.
            \item Images are correctly linked to the specific service request.
            \item Customer receives an upload success message.
        \end{itemize}
    \end{itemize}
    \item \textbf{Test Case Derivation}:
    \begin{itemize}
        \item Ensures that customers can upload images for requested services, fulfilling the fit criterion.
    \end{itemize}
    \item \textbf{How Test Will Be Performed}:
    \begin{itemize}
        \item \textbf{Manual Steps}:
        \begin{enumerate}
            \item \textbf{Step 1}: Navigate to the image upload section of the active service request.
            \item \textbf{Step 2}: Select the image files for upload.
            \item \textbf{Step 3}: Initiate the upload process.
            \item \textbf{Step 4}: Monitor progress indicators for each file.
            \item \textbf{Step 5}: Verify that a success message is displayed after upload completion.
            \item \textbf{Step 6}: Check the service request details to ensure images are listed.
            \item \textbf{Step 7}: Confirm that the images are stored in the correct directory or database location.
        \end{enumerate}
    \end{itemize}
\end{enumerate}

\section*{FR7: Satellite Image Request Test}

\begin{enumerate}
    \item \textbf{Test ID}: FR7\_Test\_01
    \item \textbf{Control}: \textbf{Automated}
    \item \textbf{Initial State}: Customer is logged in with an active service request that requires satellite images.
    \item \textbf{Input}:
    \begin{itemize}
        \item Geographic coordinates:
        \begin{itemize}
            \item \textbf{Latitude}: $37.7749^\circ$ N
            \item \textbf{Longitude}: $122.4194^\circ$ W (San Francisco, CA)
            \item \textbf{Date Range}: ``2023-01-01'' to ``2023-01-31''
        \end{itemize}
    \end{itemize}
    \item \textbf{Output}:
    \begin{itemize}
        \item \textbf{Expected Result}:
        \begin{itemize}
            \item The system retrieves and stores satellite images corresponding to the provided coordinates and date range.
            \item Customer is notified of successful retrieval.
        \end{itemize}
    \end{itemize}
    \item \textbf{Test Case Derivation}:
    \begin{itemize}
        \item Confirms the system can source images using specified geographical data, aiding in analysis.
    \end{itemize}
    \item \textbf{How Test Will Be Performed}:
    \begin{itemize}
        \item \textbf{Automated Test Script Execution}:
        \begin{enumerate}
            \item \textbf{Step 1}: Input the geographic coordinates and date range into the request form.
            \item \textbf{Step 2}: Submit the request.
            \item \textbf{Step 3}: Mock the satellite data provider's API response if necessary.
            \item \textbf{Step 4}: Verify that the system processes the input without errors.
            \item \textbf{Step 5}: Check that the images are retrieved and stored in the system.
            \item \textbf{Step 6}: Confirm that the images are linked to the correct service request.
            \item \textbf{Step 7}: Validate that the customer receives a notification or confirmation message.
        \end{enumerate}
    \end{itemize}
\end{enumerate}

\section*{FR8: Service Request Failure Alert Test}

\begin{enumerate}
    \item \textbf{Test ID}: FR8\_Test\_01
    \item \textbf{Control}: \textbf{Manual}
    \item \textbf{Initial State}: Customer has initiated a service request that cannot be fulfilled due to invalid parameters.
    \item \textbf{Input}:
    \begin{itemize}
        \item Service request with unfulfillable criteria:
        \begin{itemize}
            \item \textbf{Service Type}: ``Image Analysis''
            \item \textbf{Geographic Coordinates}: Invalid coordinates (e.g., Latitude: $95^\circ$ N)
        \end{itemize}
    \end{itemize}
    \item \textbf{Output}:
    \begin{itemize}
        \item \textbf{Expected Result}:
        \begin{itemize}
            \item Customer receives an alert indicating that the service request cannot be processed.
            \item An explanation of the failure is provided.
        \end{itemize}
    \end{itemize}
    \item \textbf{Test Case Derivation}:
    \begin{itemize}
        \item Ensures customers are promptly notified when requests cannot be processed, enhancing user experience.
    \end{itemize}
    \item \textbf{How Test Will Be Performed}:
    \begin{itemize}
        \item \textbf{Manual Steps}:
        \begin{enumerate}
            \item \textbf{Step 1}: Attempt to submit the service request with invalid coordinates.
            \item \textbf{Step 2}: Observe the system's response.
            \item \textbf{Step 3}: Verify that an alert or error message is displayed to the customer.
            \item \textbf{Step 4}: Ensure the message clearly explains the reason for failure.
            \item \textbf{Step 5}: Check that no service request is logged in the system for the invalid input.
        \end{enumerate}
    \end{itemize}
\end{enumerate}

\section*{FR9: Labeler Account Creation Test}

\begin{enumerate}
    \item \textbf{Test ID}: FR9\_Test\_01
    \item \textbf{Control}: \textbf{Automated}
    \item \textbf{Initial State}: No labeler account exists for the test user in the system.
    \item \textbf{Input}:
    \begin{itemize}
        \item Labeler provides required account information:
        \begin{itemize}
            \item \textbf{Name}: ``Bob Labeler''
            \item \textbf{Email}: \texttt{bob.labeler@example.com}
            \item \textbf{Password}: ``LabelerPass789!''
            \item \textbf{Expertise Area}: ``Satellite Image Annotation''
        \end{itemize}
    \end{itemize}
    \item \textbf{Output}:
    \begin{itemize}
        \item \textbf{Expected Result}:
        \begin{itemize}
            \item A new labeler account is created and securely stored.
            \item Labeler is prompted to complete any additional onboarding steps.
        \end{itemize}
    \end{itemize}
    \item \textbf{Test Case Derivation}:
    \begin{itemize}
        \item Ensures labelers can create accounts to access the system, which is essential for workflow.
    \end{itemize}
    \item \textbf{How Test Will Be Performed}:
    \begin{itemize}
        \item \textbf{Automated Test Script Execution}:
        \begin{enumerate}
            \item \textbf{Step 1}: Navigate to the labeler registration page.
            \item \textbf{Step 2}: Fill in the registration form with the input data.
            \item \textbf{Step 3}: Submit the form.
            \item \textbf{Step 4}: Verify that a success message is displayed.
            \item \textbf{Step 5}: Check the database to ensure the new labeler account exists with the correct details.
            \item \textbf{Step 6}: Confirm that the password is stored securely (hashed).
        \end{enumerate}
        \item \textbf{Cleanup}:
        \begin{itemize}
            \item After verification, delete the test labeler account from the database to maintain a clean state for future tests.
        \end{itemize}
    \end{itemize}
\end{enumerate}

\section*{FR10: Labeler Authentication Test}

\begin{enumerate}
    \item \textbf{Test ID}: FR10\_Test\_01
    \item \textbf{Control}: \textbf{Automated}
    \item \textbf{Initial State}: Labeler account exists with credentials:
    \begin{itemize}
        \item \textbf{Email}: \texttt{labeler.test@example.com}
        \item \textbf{Password}: ``LabelerSecure!2022''
    \end{itemize}
    \item \textbf{Input}:
    \begin{itemize}
        \item Labeler enters correct login credentials.
    \end{itemize}
    \item \textbf{Output}:
    \begin{itemize}
        \item \textbf{Expected Result}:
        \begin{itemize}
            \item Labeler is authenticated successfully.
            \item Access to the labeler dashboard is granted.
        \end{itemize}
    \end{itemize}
    \item \textbf{Test Case Derivation}:
    \begin{itemize}
        \item Confirms authentication mechanisms for labelers, maintaining system security.
    \end{itemize}
    \item \textbf{How Test Will Be Performed}:
    \begin{itemize}
        \item \textbf{Automated Test Script Execution}:
        \begin{enumerate}
            \item \textbf{Step 1}: Navigate to the labeler login page.
            \item \textbf{Step 2}: Input the correct credentials.
            \item \textbf{Step 3}: Submit the login form.
            \item \textbf{Step 4}: Verify redirection to the labeler dashboard.
            \item \textbf{Step 5}: Check for access to labeler-specific features and data.
            \item \textbf{Negative Test}:
            \begin{enumerate}
                \item \textbf{Step 6}: Attempt login with incorrect credentials to ensure authentication fails appropriately.
            \end{enumerate}
        \end{enumerate}
    \end{itemize}
\end{enumerate}

\section*{FR11: Labeler Account Modification Test}

\begin{enumerate}
    \item \textbf{Test ID}: FR11\_Test\_01
    \item \textbf{Control}: \textbf{Manual}
    \item \textbf{Initial State}: Labeler is logged in and on the account settings page.
    \item \textbf{Input}:
    \begin{itemize}
        \item Update personal information:
        \begin{itemize}
            \item \textbf{Expertise Area}: Add ``Aerial Photography Annotation''
            \item \textbf{Contact Number}: ``555-6789''
        \end{itemize}
    \end{itemize}
    \item \textbf{Output}:
    \begin{itemize}
        \item \textbf{Expected Result}:
        \begin{itemize}
            \item Personal information is updated and stored in the database.
            \item Labeler receives a confirmation of successful update.
        \end{itemize}
    \end{itemize}
    \item \textbf{Test Case Derivation}:
    \begin{itemize}
        \item Ensures labelers can maintain current information, which is crucial for assignment matching.
    \end{itemize}
    \item \textbf{How Test Will Be Performed}:
    \begin{itemize}
        \item \textbf{Manual Steps}:
        \begin{enumerate}
            \item \textbf{Step 1}: Navigate to account settings.
            \item \textbf{Step 2}: Modify the expertise area and contact number.
            \item \textbf{Step 3}: Save the changes.
            \item \textbf{Step 4}: Verify that the updated information is displayed.
            \item \textbf{Step 5}: Check the database for updated records.
            \item \textbf{Step 6}: Log out and log back in to confirm persistence.
        \end{enumerate}
    \end{itemize}
\end{enumerate}

\section*{FR12: Labeler Earnings Transfer Test}

\begin{enumerate}
    \item \textbf{Test ID}: FR12\_Test\_01
    \item \textbf{Control}: \textbf{Automated}
    \item \textbf{Initial State}: Labeler is logged in with available earnings exceeding the minimum transfer threshold.
    \item \textbf{Input}:
    \begin{itemize}
        \item Transfer request to linked banking platform:
        \begin{itemize}
            \item \textbf{Amount}: Total available earnings.
            \item \textbf{Bank Account Details}: Pre-verified and linked.
        \end{itemize}
    \end{itemize}
    \item \textbf{Output}:
    \begin{itemize}
        \item \textbf{Expected Result}:
        \begin{itemize}
            \item Earnings are transferred successfully.
            \item Transaction record is created.
            \item Labeler receives confirmation and updated earnings balance.
        \end{itemize}
    \end{itemize}
    \item \textbf{Test Case Derivation}:
    \begin{itemize}
        \item Ensures labelers are compensated accurately and promptly, critical for system trust.
    \end{itemize}
    \item \textbf{How Test Will Be Performed}:
    \begin{itemize}
        \item \textbf{Automated Test Script Execution}:
        \begin{enumerate}
            \item \textbf{Step 1}: Navigate to the earnings or wallet section.
            \item \textbf{Step 2}: Initiate a transfer request for the total available amount.
            \item \textbf{Step 3}: Mock the banking API response if necessary.
            \item \textbf{Step 4}: Verify that the system processes the transfer without errors.
            \item \textbf{Step 5}: Confirm that the earnings balance is updated to reflect the transfer.
            \item \textbf{Step 6}: Check that a transaction record is logged in the database.
            \item \textbf{Step 7}: Validate that a confirmation message or email is sent to the labeler.
        \end{enumerate}
    \end{itemize}
\end{enumerate}

\section*{FR13: Image Annotation Test}

\begin{enumerate}
    \item \textbf{Test ID}: FR13\_Test\_01
    \item \textbf{Control}: \textbf{Manual}
    \item \textbf{Initial State}: Labeler is logged in with images assigned for annotation.
    \item \textbf{Input}:
    \begin{itemize}
        \item Labeler annotates an image using the provided tools:
        \begin{itemize}
            \item Draws bounding boxes around objects.
            \item Adds classification labels to each object.
            \item Saves the annotation.
        \end{itemize}
    \end{itemize}
    \item \textbf{Output}:
    \begin{itemize}
        \item \textbf{Expected Result}:
        \begin{itemize}
            \item Annotated image is stored in the system.
            \item Annotation data is correctly linked to the image and service request.
            \item Labeler receives confirmation of successful submission.
        \end{itemize}
    \end{itemize}
    \item \textbf{Test Case Derivation}:
    \begin{itemize}
        \item Confirms annotation capabilities, essential for image analysis services.
    \end{itemize}
    \item \textbf{How Test Will Be Performed}:
    \begin{itemize}
        \item \textbf{Manual Steps}:
        \begin{enumerate}
            \item \textbf{Step 1}: Access the image annotation interface.
            \item \textbf{Step 2}: Use annotation tools to mark objects.
            \item \textbf{Step 3}: Assign appropriate labels to each annotation.
            \item \textbf{Step 4}: Save and submit the annotations.
            \item \textbf{Step 5}: Verify that a confirmation is received.
            \item \textbf{Step 6}: Check the database to ensure annotations are stored correctly.
            \item \textbf{Step 7}: Attempt to re-access the annotated image to confirm annotations persist.
        \end{enumerate}
    \end{itemize}
\end{enumerate}

\section*{FR14: Consolidated Annotation Report Test}

\begin{enumerate}
    \item \textbf{Test ID}: FR14\_Test\_01
    \item \textbf{Control}: \textbf{Automated}
    \item \textbf{Initial State}: All required labeler annotations are complete for a service request.
    \item \textbf{Input}:
    \begin{itemize}
        \item System triggers consolidation process for annotations.
    \end{itemize}
    \item \textbf{Output}:
    \begin{itemize}
        \item \textbf{Expected Result}:
        \begin{itemize}
            \item Consolidated report is generated if label accuracy meets the predefined threshold.
            \item Report is stored and made accessible to the customer.
        \end{itemize}
    \end{itemize}
    \item \textbf{Test Case Derivation}:
    \begin{itemize}
        \item Verifies that the system consolidates annotations accurately, ensuring quality results for customers.
    \end{itemize}
    \item \textbf{How Test Will Be Performed}:
    \begin{itemize}
        \item \textbf{Automated Test Script Execution}:
        \begin{enumerate}
            \item \textbf{Step 1}: Simulate completion of all annotations for a service request.
            \item \textbf{Step 2}: Initiate the consolidation process (this may be automatic upon annotation completion).
            \item \textbf{Step 3}: Verify that the system calculates label accuracy and compares it against the threshold.
            \item \textbf{Step 4}: Confirm that if the accuracy meets or exceeds the threshold, the report is generated.
            \item \textbf{Step 5}: Check that the report contains consolidated data from all annotations.
            \item \textbf{Step 6}: Ensure the report is stored and accessible to the customer linked to the service request.
            \item \textbf{Step 7}: Validate that notifications are sent to the customer about the report availability.
        \end{enumerate}
        \item \textbf{Edge Case Testing}:
        \begin{enumerate}
            \item \textbf{Step 8}: Repeat the test with annotations that do not meet the accuracy threshold to verify that the report is not generated and appropriate actions are taken (e.g., re-annotation request).
        \end{enumerate}
    \end{itemize}
\end{enumerate}

\subsection{Tests for Nonfunctional Requirements}

\wss{The nonfunctional requirements for accuracy will likely just reference the
  appropriate functional tests from above.  The test cases should mention
  reporting the relative error for these tests.  Not all projects will
  necessarily have nonfunctional requirements related to accuracy.}

\wss{For some nonfunctional tests, you won't be setting a target threshold for
passing the test, but rather describing the experiment you will do to measure
the quality for different inputs.  For instance, you could measure speed versus
the problem size.  The output of the test isn't pass/fail, but rather a summary
table or graph.}

\wss{Tests related to usability could include conducting a usability test and
  survey.  The survey will be in the Appendix.}

\wss{Static tests, review, inspections, and walkthroughs, will not follow the
format for the tests given below.}

\wss{If you introduce static tests in your plan, you need to provide details.
How will they be done?  In cases like code (or document) walkthroughs, who will
be involved? Be specific.}

\subsubsection{Area of Testing1}
		
\paragraph{Title for Test}

\begin{enumerate}

\item{test-id1\\}

Type: Functional, Dynamic, Manual, Static etc.
					
Initial State: 
					
Input/Condition: 
					
Output/Result: 
					
How test will be performed: 
					
\item{test-id2\\}

Type: Functional, Dynamic, Manual, Static etc.
					
Initial State: 
					
Input: 
					
Output: 
					
How test will be performed: 

\end{enumerate}

\subsubsection{Area of Testing2}

...

\subsection{Traceability Between Test Cases and Requirements}

\wss{Provide a table that shows which test cases are supporting which
  requirements.}

\section{Unit Test Description}

\wss{This section should not be filled in until after the MIS (detailed design
  document) has been completed.}

\wss{Reference your MIS (detailed design document) and explain your overall
philosophy for test case selection.}  

\wss{To save space and time, it may be an option to provide less detail in this section.  
For the unit tests you can potentially layout your testing strategy here.  That is, you 
can explain how tests will be selected for each module.  For instance, your test building 
approach could be test cases for each access program, including one test for normal behaviour 
and as many tests as needed for edge cases.  Rather than create the details of the input 
and output here, you could point to the unit testing code.  For this to work, you code 
needs to be well-documented, with meaningful names for all of the tests.}

\subsection{Unit Testing Scope}

\wss{What modules are outside of the scope.  If there are modules that are
  developed by someone else, then you would say here if you aren't planning on
  verifying them.  There may also be modules that are part of your software, but
  have a lower priority for verification than others.  If this is the case,
  explain your rationale for the ranking of module importance.}

\subsection{Tests for Functional Requirements}

\wss{Most of the verification will be through automated unit testing.  If
  appropriate specific modules can be verified by a non-testing based
  technique.  That can also be documented in this section.}

\subsubsection{Module 1}

\wss{Include a blurb here to explain why the subsections below cover the module.
  References to the MIS would be good.  You will want tests from a black box
  perspective and from a white box perspective.  Explain to the reader how the
  tests were selected.}

\begin{enumerate}

\item{test-id1\\}

Type: \wss{Functional, Dynamic, Manual, Automatic, Static etc. Most will
  be automatic}
					
Initial State: 
					
Input: 
					
Output: \wss{The expected result for the given inputs}

Test Case Derivation: \wss{Justify the expected value given in the Output field}

How test will be performed: 
					
\item{test-id2\\}

Type: \wss{Functional, Dynamic, Manual, Automatic, Static etc. Most will
  be automatic}
					
Initial State: 
					
Input: 
					
Output: \wss{The expected result for the given inputs}

Test Case Derivation: \wss{Justify the expected value given in the Output field}

How test will be performed: 

\item{...\\}
    
\end{enumerate}

\subsubsection{Module 2}

...

\subsection{Tests for Nonfunctional Requirements}

\wss{If there is a module that needs to be independently assessed for
  performance, those test cases can go here.  In some projects, planning for
  nonfunctional tests of units will not be that relevant.}

\wss{These tests may involve collecting performance data from previously
  mentioned functional tests.}

\subsubsection{Module ?}
		
\begin{enumerate}

\item{test-id1\\}

Type: \wss{Functional, Dynamic, Manual, Automatic, Static etc. Most will
  be automatic}
					
Initial State: 
					
Input/Condition: 
					
Output/Result: 
					
How test will be performed: 
					
\item{test-id2\\}

Type: Functional, Dynamic, Manual, Static etc.
					
Initial State: 
					
Input: 
					
Output: 
					
How test will be performed: 

\end{enumerate}

\subsubsection{Module ?}

...

\subsection{Traceability Between Test Cases and Modules}

\wss{Provide evidence that all of the modules have been considered.}
				
\printbibliography


\newpage

\section{Appendix}

This is where you can place additional information.

\subsection{Symbolic Parameters}

The definition of the test cases will call for SYMBOLIC\_CONSTANTS.
Their values are defined in this section for easy maintenance.

\subsection{Usability Survey Questions?}

\wss{This is a section that would be appropriate for some projects.}

\newpage{}
\section*{Appendix --- Reflection}

\wss{This section is not required for CAS 741}

The information in this section will be used to evaluate the team members on the
graduate attribute of Lifelong Learning.

The purpose of reflection questions is to give you a chance to assess your own
learning and that of your group as a whole, and to find ways to improve in the
future. Reflection is an important part of the learning process.  Reflection is
also an essential component of a successful software development process.  

Reflections are most interesting and useful when they're honest, even if the
stories they tell are imperfect. You will be marked based on your depth of
thought and analysis, and not based on the content of the reflections
themselves. Thus, for full marks we encourage you to answer openly and honestly
and to avoid simply writing ``what you think the evaluator wants to hear.''

Please answer the following questions.  Some questions can be answered on the
team level, but where appropriate, each team member should write their own
response:


\begin{enumerate}
  \item What went well while writing this deliverable? 
  \item What pain points did you experience during this deliverable, and how
    did you resolve them?
  \item What knowledge and skills will the team collectively need to acquire to
  successfully complete the verification and validation of your project?
  Examples of possible knowledge and skills include dynamic testing knowledge,
  static testing knowledge, specific tool usage, Valgrind etc.  You should look to
  identify at least one item for each team member.
  \item For each of the knowledge areas and skills identified in the previous
  question, what are at least two approaches to acquiring the knowledge or
  mastering the skill?  Of the identified approaches, which will each team
  member pursue, and why did they make this choice?
\end{enumerate}

\printbibliography % For biblatex

\end{document}