\documentclass[12pt, titlepage]{article}

\usepackage{booktabs}
\usepackage{tabularx}
\usepackage{longtable}
\usepackage{hyperref}
\usepackage{enumitem}
\usepackage{amssymb}

\newlist{todolist}{itemize}{2}
\setlist[todolist]{label=$\square$}

\hypersetup{
    colorlinks,
    citecolor=blue,
    filecolor=black,
    linkcolor=red,
    urlcolor=blue
}
\usepackage[round]{natbib}

\input{../Comments}
%% Common Parts

\newcommand{\progname}{Software Engineering} % PUT YOUR PROGRAM NAME HERE
\newcommand{\authname}{Team \#11, OKKM Insights
\\ Mathew Petronilho
\\ Oleg Glotov
\\ Kyle McMaster
\\ Kartik Chaudhari} % AUTHOR NAMES                  

\usepackage{hyperref}
    \hypersetup{colorlinks=true, linkcolor=blue, citecolor=blue, filecolor=blue,
                urlcolor=blue, unicode=false}
    \urlstyle{same}
                                


\begin{document}

\title{System Verification and Validation Plan for \progname{}} 
\author{\authname}
\date{\today}
	
\maketitle

\pagenumbering{roman}

\section*{Revision History}

\begin{tabularx}{\textwidth}{p{3cm}p{2cm}X}
\toprule {\bf Date} & {\bf Version} & {\bf Notes}\\
\midrule
Date 1 & 1.0 & Notes\\
Date 2 & 1.1 & Notes\\
\bottomrule
\end{tabularx}

~\\
\wss{The intention of the VnV plan is to increase confidence in the software.
However, this does not mean listing every verification and validation technique
that has ever been devised.  The VnV plan should also be a \textbf{feasible}
plan. Execution of the plan should be possible with the time and team available.
If the full plan cannot be completed during the time available, it can either be
modified to ``fake it'', or a better solution is to add a section describing
what work has been completed and what work is still planned for the future.}

\wss{The VnV plan is typically started after the requirements stage, but before
the design stage.  This means that the sections related to unit testing cannot
initially be completed.  The sections will be filled in after the design stage
is complete.  the final version of the VnV plan should have all sections filled
in.}

\newpage

\tableofcontents

\listoftables
\wss{Remove this section if it isn't needed}

\listoffigures
\wss{Remove this section if it isn't needed}

\newpage

\section{Symbols, Abbreviations, and Acronyms}

Please refer to Section 4 of Software Requirements Specification found \href{https://github.com/OKKM-insights/OKKM.insights/blob/main/docs/SRS/SRS.pdf}{here}.







\newpage

\pagenumbering{arabic}

This document is intended to provide a description of the specific validation and verification activites that will be completed throughout the development of the GeoWeb System.
The purpose of these activities is to ensure the system requirements agree with stakeholder needs, and to certify the implementation of the system satisifes the stakeholder requirements. This document will
start with a review of the system, a description of the testing plan, and a list of tests to be completed.


\section{General Information}

\subsection{Summary}

\wss{Say what software is being tested.  Give its name and a brief overview of
  its general functions.}

\subsection{Objectives}

\wss{State what is intended to be accomplished.  The objective will be around
  the qualities that are most important for your project.  You might have
  something like: ``build confidence in the software correctness,''
  ``demonstrate adequate usability.'' etc.  You won't list all of the qualities,
  just those that are most important.}

\wss{You should also list the objectives that are out of scope.  You don't have 
the resources to do everything, so what will you be leaving out.  For instance, 
if you are not going to verify the quality of usability, state this.  It is also 
worthwhile to justify why the objectives are left out.}

\wss{The objectives are important because they highlight that you are aware of 
limitations in your resources for verification and validation.  You can't do everything, 
so what are you going to prioritize?  As an example, if your system depends on an 
external library, you can explicitly state that you will assume that external library 
has already been verified by its implementation team.}

\subsection{Challenge Level and Extras}

\wss{State the challenge level (advanced, general, basic) for your project.
Your challenge level should exactly match what is included in your problem
statement.  This should be the challenge level agreed on between you and the
course instructor.  You can use a pull request to update your challenge level
(in TeamComposition.csv or Repos.csv) if your plan changes as a result of the
VnV planning exercise.}

\wss{Summarize the extras (if any) that were tackled by this project.  Extras
can include usability testing, code walkthroughs, user documentation, formal
proof, GenderMag personas, Design Thinking, etc.  Extras should have already
been approved by the course instructor as included in your problem statement.
You can use a pull request to update your extras (in TeamComposition.csv or
Repos.csv) if your plan changes as a result of the VnV planning exercise.}

\subsection{Relevant Documentation}

\wss{Reference relevant documentation.  This will definitely include your SRS
  and your other project documents (design documents, like MG, MIS, etc).  You
  can include these even before they are written, since by the time the project
  is done, they will be written.  You can create BibTeX entries for your
  documents and within those entries include a hyperlink to the documents.}

\citet{SRS}

\wss{Don't just list the other documents.  You should explain why they are relevant and 
how they relate to your VnV efforts.}

\section{Plan}

This section describes the team, verification plans for the system design and documentation, use of automated testing tools, and the validation plan of the software following implementation.

\subsection{Verification and Validation Team}
When not listed as lead, members of core team will support through team discussion and implementation of feedback.\\
\begin{center}
\begin{tabular}{|c|c|l|}
  \hline
  \textbf{Name} & \textbf{Role} & \textbf{Responsibilities}\\\hline
  Mathew Petronilho & Core team & \textbf{SRS} \\
  & & Lead review of SRS\\
  & & \textbf{Implementation}\\
  & & Support review of implementation\\
  & & Review Implementation for coding \\
  & & standards, comment quality\\
  \hline
  Oleg Glotov & Core team & \textbf{VnV Plan} \\
  & & Support review of VnV Plan\\
  & & Review VnV Plan for formatting and \\
  & & grammar errors\\
  & & \textbf{Implementation}\\
  & & Lead review of Implementation\\
  \hline
  Kartik Chaudhari & Core team & \textbf{Design} \\
  & & Support review of Design\\
  & & Review diagrams and documents for correct \\
  & & notation, formatting and grammar\\
  & & \textbf{VnV Plan}\\
  & & Lead review of VnV Plan\\
  & & \textbf{Implementation}\\
  & & Support review of implementation\\
  \hline
  Kyle McMaster & Core team & \textbf{SRS} \\
  & & Check SRS for formatting and grammar errors\\
  & & Support review of SRS\\
  & & \textbf{Design}\\
  & & Lead review of design\\
  & & Ensure team is following standard design principles\\
  \hline
  Dr. Swati Mishra & Project Supervisor & Validate all team docs in structured review\\\hline
  Capstone Team 10 & Primary Reviewers & Validate all team docs in \\
  & & async review through Git issues\\
  \hline
\end{tabular}
\end{center}

% \wss{Your teammates.  Maybe your supervisor.
%   You should do more than list names.  You should say what each person's role is
%   for the project's verification.  A table is a good way to summarize this information.}

\subsection{SRS Verification Plan}

The SRS will be verified through several channels. First, the document will be reviewed by another capstone team. This will help the team identify issues
which are obscured to the team, due to the additional time they have spent thinking about the project. We expect this feedback to generally consist of 
omitted definitions or unstated assumptions. Since the team is more familiar with the project, it is likely that we have some information which is obvious 
to the team, but is necessary to define for others. The team will collect feedback from our primary reviewers in the issue tracker.\\
The SRS will also be reviewed by our supervisor in a structured review meeting. The team will guide the review with the following checklist:\\\\
\textbf{Constraints \& Assumptions}\\
The following sections are informed by the constraints \& assumptions. Therefore, it is critical that the these are verified first.
\begin{todolist}
\item All constraints are correct and necessary
\item All constraints are unambigious
\item All constraints which should be present are present
\item All constraints are verifiable
\item All assumptions are correct and necessary
\item All assumptions are unambigious
\item All assumptions which should be present are present
\item All assumptions are verifiable
\end{todolist}
\textbf{Data Model}\\
The data model affects how the system will be decomposed in the future design. This in turn affects the requirements, so it should be verified next.
\begin{todolist}
  \item Data model is correct
  \item Data model is complete
  \item Each element of data dictionary is correctly described. There are no extra or missing attributes.
  \item Elements of data dictionary are unambigious
\end{todolist}
\textbf{Functional Requirements}\\
This section describes the functionality of the system. This will be useful for understanding the context of the NFRs.
\begin{todolist}
\item All requirements are correct and necessary
\item All requirements are unambigious
\item All requirements which should be present are present
\item All requirements are verifiable
\item All requirements are feasible
\item All requirements are traceable
\end{todolist}
\textbf{Non-Functional Requirements}\\
The NFRs have now been properly introduced by the other sections and should now be assessed.
\begin{todolist}
  \item All requirements are correct and necessary
  \item All requirements are unambigious
  \item All requirements which should be present are present
  \item All requirements are verifiable
  \item All requirements are feasible
  \item All requirements are traceable
  \end{todolist}
\textbf{General}\\
Each of the sections reviewed should also be monitored for the following criteria.
\begin{todolist}
  \item Document contains properly formatted title, table of contents, references and all necessary sections
  \item Tables and figures are correctly formatted
  \item No grammar errors are present
  \item Each required section is present
  \item All requirements are feasible
  \end{todolist}


% \wss{List any approaches you intend to use for SRS verification.  This may
%   include ad hoc feedback from reviewers, like your classmates (like your
%   primary reviewer), or you may plan for something more rigorous/systematic.}

% \wss{If you have a supervisor for the project, you shouldn't just say they will
% read over the SRS.  You should explain your structured approach to the review.
% Will you have a meeting?  What will you present?  What questions will you ask?
% Will you give them instructions for a task-based inspection?  Will you use your
% issue tracker?}

% \wss{Maybe create an SRS checklist?}

\subsection{Design Verification Plan}
Upon completion of the System Design, the team will verify the design. This verification will involve reviews from the primary reviewers, in the same manner as described above.
This again will provide fresh prospective on the design, and help identify any omitted or underexplained information. The verification will continue with a systematic review with The
project supervisor using the following checklist:\\\\

\textbf{Class Decomposition}\\
\begin{todolist}
  \item All functional requirements are covered by a class/subsystem
  \item All non-functional requirements are covered by a class/subsystem
  \item All classes are correctly sized. That is, each class should be concerned with one purpose
  \item When appropriate, interfaces and abstract classes are used
  \item All class attributes are complete and necessary
  \item When appropriate, design patterns are used to improve design clarity
  \item Function permissions are appropriate for all classes
  \item Classes cannot be simplified without degrading the understanability of the system
  \end{todolist}
  \textbf{General}\\
\begin{todolist}
  \item The types of all attributes are listed
  \item The argument and return types of all functions are listed
  \item Correct notation is used to describe class relationships
  \item Class diagram is legible
  \end{todolist}
% \wss{Plans for design verification}

% \wss{The review will include reviews by your classmates}

% \wss{Create a checklists?}

\subsection{Verification and Validation Plan Verification Plan}
This document must be verified and validated, to ensure the validation and verification of other artifacts is correct. Like other documents, this will include review
from our primary reviewers, as well as a review with the project supervisor. The review with the project supervisor will be guided by the folllowing checklist:\\

\textbf{Testing Plan}\\
\begin{todolist}
  \item Testing plan includes a checklist for each artifact
  \item Checklist items are unambigious
  \item Checklist contains all recessary items
  \item Checklist does not contain unnecessary items
  \item Plan includes review from multiple parties
  \item Plan includes method for collecting feedback on artifacts
  \item Plan is verifiable
  \end{todolist}
\textbf{Tests}\\
\begin{todolist}
  \item All functional requirements are covered by one or more tests\\
  \item All non-functional requirements are covered by one or more tests\\
  \item All `units' are sufficiently covered by one or more tests\\
  \item Tests are unambigious
  \item Tests are verifiable
  \item Tests are repeatable
  \end{todolist}

% \wss{The verification and validation plan is an artifact that should also be
% verified.  Techniques for this include review and mutation testing.}

% \wss{The review will include reviews by your classmates}

% \wss{Create a checklists?}

\subsection{Implementation Verification Plan}

As part of the implementation verification plan, the team will conduct a series of static tests. One of which, will be a code walkthrough with members of the team.
Before submitting Rev 0 and Rev 1, the members of the team will schedule time to walk through other members of the team through the code they have written. This will be 
an opportunity for the team to look for code quality and adherence to standards, correctness, and alignment to the system design. During this walkthrough, we expect that 
the lead developer of the code being displayed will explain the overall flow and control structures. In doing so, we expect they will find the errors in their own code.\\\\
The team will also use a suite of tests, which is described in section 4.\\\\
The team will used static analyzers to verify adherence to coding standards to perform type checking. Please refer to section 7.2.2 and 7.2.3 in the Development Plan found \href{https://github.com/OKKM-insights/OKKM.insights/blob/main/docs/DevelopmentPlan/DevelopmentPlan.pdf}{here}.

% \wss{You should at least point to the tests listed in this document and the unit
%   testing plan.}

% \wss{In this section you would also give any details of any plans for static
%   verification of the implementation.  Potential techniques include code
%   walkthroughs, code inspection, static analyzers, etc.}

% \wss{The final class presentation in CAS 741 could be used as a code
% walkthrough.  There is also a possibility of using the final presentation (in
% CAS741) for a partial usability survey.}

\subsection{Automated Testing and Verification Tools}

Please refer to section 7.2.2 and 7.2.3 in the Development Plan found \href{https://github.com/OKKM-insights/OKKM.insights/blob/main/docs/DevelopmentPlan/DevelopmentPlan.pdf}{here}.

% \wss{What tools are you using for automated testing.  Likely a unit testing
%   framework and maybe a profiling tool, like ValGrind.  Other possible tools
%   include a static analyzer, make, continuous integration tools, test coverage
%   tools, etc.  Explain your plans for summarizing code coverage metrics.
%   Linters are another important class of tools.  For the programming language
%   you select, you should look at the available linters.  There may also be tools
%   that verify that coding standards have been respected, like flake9 for
%   Python.}

% \wss{If you have already done this in the development plan, you can point to
% that document.}

% \wss{The details of this section will likely evolve as you get closer to the
%   implementation.}

\subsection{Software Validation Plan}
  As discussed in section 6.2 of the Problem Statement and Goals, found \href{https://github.com/OKKM-insights/OKKM.insights/blob/main/docs/ProblemStatementAndGoals/ProblemStatement.pdf}{here}, 
  we will validate our software through user testing. More specifically, we will conduct testing of the user interface with peers, who will fill in a survey containing both qualitative and quantitative questions.
  This survery will be used to track progress on the usability of the software by comparing results collecting during different iterations of the user interface. See the appendix for a sample of the user survey.

% \wss{If there is any external data that can be used for validation, you should
%   point to it here.  If there are no plans for validation, you should state that
%   here.}

% \wss{You might want to use review sessions with the stakeholder to check that
% the requirements document captures the right requirements.  Maybe task based
% inspection?}

% \wss{For those capstone teams with an external supervisor, the Rev 0 demo should 
% be used as an opportunity to validate the requirements.  You should plan on 
% demonstrating your project to your supervisor shortly after the scheduled Rev 0 demo.  
% The feedback from your supervisor will be very useful for improving your project.}

% \wss{For teams without an external supervisor, user testing can serve the same purpose 
% as a Rev 0 demo for the supervisor.}

% \wss{This section might reference back to the SRS verification section.}

\section{System Tests}

\wss{There should be text between all headings, even if it is just a roadmap of
the contents of the subsections.}

\subsection{Tests for Functional Requirements}

\section*{FR0: Customer Account Creation Test}

\begin{enumerate}
    \item \textbf{Test ID}: T-FR0
    \item \textbf{Control}: \textbf{Automated}
    \item \textbf{Initial State}: The system is accessible, and no user account currently exists for the test user.
    \item \textbf{Input}:
    \begin{itemize}
        \item Customer provides valid personal information:
        \begin{itemize}
            \item \textbf{Name}: ``Alice Smith''
            \item \textbf{Email}: \texttt{alice.smith@example.com}
            \item \textbf{Password}: ``StrongPassword!2021''
        \end{itemize}
        \item Customer agrees to the system's privacy policy.
    \end{itemize}
    \item \textbf{Output}:
    \begin{itemize}
        \item \textbf{Expected Result}:
        \begin{itemize}
            \item A new user account is created.
            \item Account information is securely stored in the database.
            \item The customer is redirected to the login page with a success message.
        \end{itemize}
    \end{itemize}
    \item \textbf{Test Case Derivation}:
    \begin{itemize}
        \item Based on the requirement that customers must have an account to access system services. This test verifies that with valid input and acceptance of terms, the account creation process is successfully executed and persists data appropriately.
    \end{itemize}
    \item \textbf{How Test Will Be Performed}:
    \begin{itemize}
        \item \textbf{Automated Test Script Execution}:
        \begin{enumerate}
            \item \textbf{Step 1}: Navigate to the account creation page using a testing tool like Selenium WebDriver.
            \item \textbf{Step 2}: Fill in the registration form fields with the provided valid data.
            \item \textbf{Step 3}: Check the agreement box for the privacy policy.
            \item \textbf{Step 4}: Submit the registration form.
            \item \textbf{Step 5}: Verify that the response indicates successful account creation (e.g., success message displayed).
            \item \textbf{Step 6}: Confirm redirection to the login page.
            \item \textbf{Step 7}: Query the database to verify that a new user record exists with the email \texttt{alice.smith@example.com}.
            \item \textbf{Step 8}: Ensure the password is hashed and not stored in plain text.
        \end{enumerate}
        \item \textbf{Cleanup}:
        \begin{itemize}
            \item After verification, delete the test user account from the database to maintain a clean state for future tests.
        \end{itemize}
    \end{itemize}
\end{enumerate}
\section*{FR1: Customer Authentication Test}

\begin{enumerate}
    \item \textbf{Test ID}: T-FR1
    \item \textbf{Control}: \textbf{Automated}
    \item \textbf{Initial State}: An existing user account with the following credentials:
    \begin{itemize}
        \item \textbf{Email}: \texttt{user.test@example.com}
        \item \textbf{Password}: ``TestPass\#123''
    \end{itemize}
    \item \textbf{Input}:
    \begin{itemize}
        \item Customer enters the correct login credentials:
        \begin{itemize}
            \item \textbf{Email}: \texttt{user.test@example.com}
            \item \textbf{Password}: ``TestPass\#123''
        \end{itemize}
    \end{itemize}
    \item \textbf{Output}:
    \begin{itemize}
        \item \textbf{Expected Result}:
        \begin{itemize}
            \item Customer is successfully authenticated.
            \item Access to privileged information (e.g., user dashboard) is granted.
            \item A session token or cookie is established for the user session.
        \end{itemize}
    \end{itemize}
    \item \textbf{Test Case Derivation}:
    \begin{itemize}
        \item Ensures that only authenticated customers can access the system, satisfying security requirements.
    \end{itemize}
    \item \textbf{How Test Will Be Performed}:
    \begin{itemize}
        \item \textbf{Automated Test Script Execution}:
        \begin{enumerate}
            \item \textbf{Step 1}: Navigate to the login page.
            \item \textbf{Step 2}: Input the correct email and password.
            \item \textbf{Step 3}: Submit the login form.
            \item \textbf{Step 4}: Verify redirection to the user dashboard or home page.
            \item \textbf{Step 5}: Check for the presence of privileged information on the page.
            \item \textbf{Step 6}: Verify that a session token or authentication cookie has been set.
            \item \textbf{Negative Test}:
            \begin{enumerate}
                \item \textbf{Step 7}: Attempt to access the dashboard with incorrect credentials to ensure access is denied.
            \end{enumerate}
        \end{enumerate}
        \item \textbf{Session Validation}:
        \begin{enumerate}
            \item \textbf{Step 8}: Use the session token to access a secure API endpoint to confirm authentication persistence.
        \end{enumerate}
    \end{itemize}
\end{enumerate}

\section*{FR2: Customer Account Modification Test}

\begin{enumerate}
    \item \textbf{Test ID}: T-FR2
    \item \textbf{Control}: \textbf{Manual}
    \item \textbf{Initial State}: Customer is logged in and has access to their account information page.
    \item \textbf{Input}:
    \begin{itemize}
        \item Customer updates personal information fields:
        \begin{itemize}
            \item \textbf{Address}: ``123 New Street, Cityville''
            \item \textbf{Phone Number}: ``555-1234''
            \item \textbf{Profile Picture}: Uploads a new image file.
        \end{itemize}
    \end{itemize}
    \item \textbf{Output}:
    \begin{itemize}
        \item \textbf{Expected Result}:
        \begin{itemize}
            \item Updated personal information is saved and reflected in the database.
            \item The user receives a confirmation message indicating successful update.
        \end{itemize}
    \end{itemize}
    \item \textbf{Test Case Derivation}:
    \begin{itemize}
        \item Confirms that authenticated customers can modify their information, maintaining data accuracy and relevance.
    \end{itemize}
    \item \textbf{How Test Will Be Performed}:
    \begin{itemize}
        \item \textbf{Manual Steps}:
        \begin{enumerate}
            \item \textbf{Step 1}: Log in to the customer account.
            \item \textbf{Step 2}: Navigate to the account settings or profile page.
            \item \textbf{Step 3}: Change the address and phone number to the new values.
            \item \textbf{Step 4}: Upload a new profile picture.
            \item \textbf{Step 5}: Save the changes.
            \item \textbf{Step 6}: Verify that the changes are displayed correctly on the profile page.
            \item \textbf{Step 7}: Check the database to ensure the new information is updated.
            \item \textbf{Step 8}: Log out and log back in to confirm that the changes persist.
        \end{enumerate}
    \end{itemize}
\end{enumerate}

\section*{FR3: Payment Processing Test}

\begin{enumerate}
    \item \textbf{Test ID}: T-FR3
    \item \textbf{Control}: \textbf{Automated}
    \item \textbf{Initial State}: Customer is logged in and ready to make a payment for a service request.
    \item \textbf{Input}:
    \begin{itemize}
        \item Valid payment information:
        \begin{itemize}
            \item \textbf{Credit Card Number}: ``4111 1111 1111 1111'' (Test Visa number)
            \item \textbf{Expiry Date}: ``12/25''
            \item \textbf{CVV}: ``123''
            \item \textbf{Billing Address}: Matches the address on file.
        \end{itemize}
    \end{itemize}
    \item \textbf{Output}:
    \begin{itemize}
        \item \textbf{Expected Result}:
        \begin{itemize}
            \item Payment is processed successfully.
            \item A confirmation receipt is generated and emailed to the customer.
            \item The service request status is updated to ``Paid'' or equivalent.
        \end{itemize}
    \end{itemize}
    \item \textbf{Test Case Derivation}:
    \begin{itemize}
        \item Verifies payment integration, ensuring that authenticated customers can complete transactions securely.
    \end{itemize}
    \item \textbf{How Test Will Be Performed}:
    \begin{itemize}
        \item \textbf{Automated Test Script Execution}:
        \begin{enumerate}
            \item \textbf{Step 1}: Navigate to the payment page for the pending service request.
            \item \textbf{Step 2}: Input the valid payment details.
            \item \textbf{Step 3}: Submit the payment form.
            \item \textbf{Step 4}: Mock the payment gateway response if using a sandbox environment.
            \item \textbf{Step 5}: Verify that the system displays a payment success message.
            \item \textbf{Step 6}: Check that a confirmation receipt is generated and sent to the customer's email.
            \item \textbf{Step 7}: Verify that the service request status is updated appropriately in the database.
        \end{enumerate}
        \item \textbf{Security Verification}:
        \begin{enumerate}
            \item \textbf{Step 8}: Ensure that sensitive payment information is not stored in plain text and complies with PCI DSS standards.
        \end{enumerate}
    \end{itemize}
\end{enumerate}

\section*{FR4: Service Request Submission Test}

\begin{enumerate}
    \item \textbf{Test ID}: T-FR4
    \item \textbf{Control}: \textbf{Manual}
    \item \textbf{Initial State}: Customer is logged in and has a confirmed payment.
    \item \textbf{Input}:
    \begin{itemize}
        \item Customer fills out the service request form with necessary details:
        \begin{itemize}
            \item \textbf{Service Type}: ``Image Analysis''
            \item \textbf{Description}: ``Analysis of satellite images for deforestation.''
            \item \textbf{Preferred Completion Date}: ``2023-12-31''
        \end{itemize}
    \end{itemize}
    \item \textbf{Output}:
    \begin{itemize}
        \item \textbf{Expected Result}:
        \begin{itemize}
            \item Service request is accepted and logged in the system.
            \item Customer receives a confirmation message and request ID.
        \end{itemize}
    \end{itemize}
    \item \textbf{Test Case Derivation}:
    \begin{itemize}
        \item Ensures that the system accepts valid requests from authenticated customers.
    \end{itemize}
    \item \textbf{How Test Will Be Performed}:
    \begin{itemize}
        \item \textbf{Manual Steps}:
        \begin{enumerate}
            \item \textbf{Step 1}: Navigate to the new service request page.
            \item \textbf{Step 2}: Fill in the form with the input data.
            \item \textbf{Step 3}: Submit the form.
            \item \textbf{Step 4}: Verify that a confirmation message with a unique request ID is displayed.
            \item \textbf{Step 5}: Check the database to confirm that the service request is logged with correct details.
            \item \textbf{Step 6}: Ensure that the service request appears in the customer's list of active requests.
        \end{enumerate}
    \end{itemize}
\end{enumerate}

\section*{FR5: Service Report Delivery Test}

\begin{enumerate}
    \item \textbf{Test ID}: T-FR5
    \item \textbf{Control}: \textbf{Automated}
    \item \textbf{Initial State}: Customer is logged in and has a completed service request.
    \item \textbf{Input}:
    \begin{itemize}
        \item Customer navigates to the ``My Reports'' section after being notified of service completion.
    \end{itemize}
    \item \textbf{Output}:
    \begin{itemize}
        \item \textbf{Expected Result}:
        \begin{itemize}
            \item The service report is available for viewing and download.
            \item Report contents are accurate and correspond to the service request.
        \end{itemize}
    \end{itemize}
    \item \textbf{Test Case Derivation}:
    \begin{itemize}
        \item Confirms that customers receive reports on completed services, fulfilling the system’s purpose.
    \end{itemize}
    \item \textbf{How Test Will Be Performed}:
    \begin{itemize}
        \item \textbf{Automated Test Script Execution}:
        \begin{enumerate}
            \item \textbf{Step 1}: Simulate the completion of a service request in the system (can be mocked or set up in a test environment).
            \item \textbf{Step 2}: Log in as the customer.
            \item \textbf{Step 3}: Navigate to the ``My Reports'' or equivalent section.
            \item \textbf{Step 4}: Verify that the completed service report is listed.
            \item \textbf{Step 5}: Open the report and check that it loads correctly.
            \item \textbf{Step 6}: Validate that the report content matches the expected output based on the service provided.
            \item \textbf{Step 7}: Attempt to download the report and ensure the file is intact and accessible.
        \end{enumerate}
    \end{itemize}
\end{enumerate}

\section*{FR6: Image Upload Test}

\begin{enumerate}
    \item \textbf{Test ID}: T-FR6
    \item \textbf{Control}: \textbf{Manual}
    \item \textbf{Initial State}: Customer is logged in with an active service request requiring image uploads.
    \item \textbf{Input}:
    \begin{itemize}
        \item Customer uploads multiple image files:
        \begin{itemize}
            \item \texttt{Image1.jpg}: 2\,MB
            \item \texttt{Image2.png}: 3\,MB
            \item \texttt{Image3.tif}: 5\,MB
        \end{itemize}
    \end{itemize}
    \item \textbf{Output}:
    \begin{itemize}
        \item \textbf{Expected Result}:
        \begin{itemize}
            \item All images are successfully uploaded and stored.
            \item Images are correctly linked to the specific service request.
            \item Customer receives an upload success message.
        \end{itemize}
    \end{itemize}
    \item \textbf{Test Case Derivation}:
    \begin{itemize}
        \item Ensures that customers can upload images for requested services, fulfilling the fit criterion.
    \end{itemize}
    \item \textbf{How Test Will Be Performed}:
    \begin{itemize}
        \item \textbf{Manual Steps}:
        \begin{enumerate}
            \item \textbf{Step 1}: Navigate to the image upload section of the active service request.
            \item \textbf{Step 2}: Select the image files for upload.
            \item \textbf{Step 3}: Initiate the upload process.
            \item \textbf{Step 4}: Monitor progress indicators for each file.
            \item \textbf{Step 5}: Verify that a success message is displayed after upload completion.
            \item \textbf{Step 6}: Check the service request details to ensure images are listed.
            \item \textbf{Step 7}: Confirm that the images are stored in the correct directory or database location.
        \end{enumerate}
    \end{itemize}
\end{enumerate}

\section*{FR7: Satellite Image Request Test}

\begin{enumerate}
    \item \textbf{Test ID}: T-FR7
    \item \textbf{Control}: \textbf{Automated}
    \item \textbf{Initial State}: Customer is logged in with an active service request that requires satellite images.
    \item \textbf{Input}:
    \begin{itemize}
        \item Geographic coordinates:
        \begin{itemize}
            \item \textbf{Latitude}: $37.7749^\circ$ N
            \item \textbf{Longitude}: $122.4194^\circ$ W (San Francisco, CA)
            \item \textbf{Date Range}: ``2023-01-01'' to ``2023-01-31''
        \end{itemize}
    \end{itemize}
    \item \textbf{Output}:
    \begin{itemize}
        \item \textbf{Expected Result}:
        \begin{itemize}
            \item The system retrieves and stores satellite images corresponding to the provided coordinates and date range.
            \item Customer is notified of successful retrieval.
        \end{itemize}
    \end{itemize}
    \item \textbf{Test Case Derivation}:
    \begin{itemize}
        \item Confirms the system can source images using specified geographical data, aiding in analysis.
    \end{itemize}
    \item \textbf{How Test Will Be Performed}:
    \begin{itemize}
        \item \textbf{Automated Test Script Execution}:
        \begin{enumerate}
            \item \textbf{Step 1}: Input the geographic coordinates and date range into the request form.
            \item \textbf{Step 2}: Submit the request.
            \item \textbf{Step 3}: Mock the satellite data provider's API response if necessary.
            \item \textbf{Step 4}: Verify that the system processes the input without errors.
            \item \textbf{Step 5}: Check that the images are retrieved and stored in the system.
            \item \textbf{Step 6}: Confirm that the images are linked to the correct service request.
            \item \textbf{Step 7}: Validate that the customer receives a notification or confirmation message.
        \end{enumerate}
    \end{itemize}
\end{enumerate}

\section*{FR8: Service Request Failure Alert Test}

\begin{enumerate}
    \item \textbf{Test ID}: T-FR8
    \item \textbf{Control}: \textbf{Manual}
    \item \textbf{Initial State}: Customer has initiated a service request that cannot be fulfilled due to invalid parameters.
    \item \textbf{Input}:
    \begin{itemize}
        \item Service request with unfulfillable criteria:
        \begin{itemize}
            \item \textbf{Service Type}: ``Image Analysis''
            \item \textbf{Geographic Coordinates}: Invalid coordinates (e.g., Latitude: $95^\circ$ N)
        \end{itemize}
    \end{itemize}
    \item \textbf{Output}:
    \begin{itemize}
        \item \textbf{Expected Result}:
        \begin{itemize}
            \item Customer receives an alert indicating that the service request cannot be processed.
            \item An explanation of the failure is provided.
        \end{itemize}
    \end{itemize}
    \item \textbf{Test Case Derivation}:
    \begin{itemize}
        \item Ensures customers are promptly notified when requests cannot be processed, enhancing user experience.
    \end{itemize}
    \item \textbf{How Test Will Be Performed}:
    \begin{itemize}
        \item \textbf{Manual Steps}:
        \begin{enumerate}
            \item \textbf{Step 1}: Attempt to submit the service request with invalid coordinates.
            \item \textbf{Step 2}: Observe the system's response.
            \item \textbf{Step 3}: Verify that an alert or error message is displayed to the customer.
            \item \textbf{Step 4}: Ensure the message clearly explains the reason for failure.
            \item \textbf{Step 5}: Check that no service request is logged in the system for the invalid input.
        \end{enumerate}
    \end{itemize}
\end{enumerate}

\section*{FR9: Labeler Account Creation Test}

\begin{enumerate}
    \item \textbf{Test ID}: T-FR9
    \item \textbf{Control}: \textbf{Automated}
    \item \textbf{Initial State}: No labeler account exists for the test user in the system.
    \item \textbf{Input}:
    \begin{itemize}
        \item Labeler provides required account information:
        \begin{itemize}
            \item \textbf{Name}: ``Bob Labeler''
            \item \textbf{Email}: \texttt{bob.labeler@example.com}
            \item \textbf{Password}: ``LabelerPass789!''
            \item \textbf{Expertise Area}: ``Satellite Image Annotation''
        \end{itemize}
    \end{itemize}
    \item \textbf{Output}:
    \begin{itemize}
        \item \textbf{Expected Result}:
        \begin{itemize}
            \item A new labeler account is created and securely stored.
            \item Labeler is prompted to complete any additional onboarding steps.
        \end{itemize}
    \end{itemize}
    \item \textbf{Test Case Derivation}:
    \begin{itemize}
        \item Ensures labelers can create accounts to access the system, which is essential for workflow.
    \end{itemize}
    \item \textbf{How Test Will Be Performed}:
    \begin{itemize}
        \item \textbf{Automated Test Script Execution}:
        \begin{enumerate}
            \item \textbf{Step 1}: Navigate to the labeler registration page.
            \item \textbf{Step 2}: Fill in the registration form with the input data.
            \item \textbf{Step 3}: Submit the form.
            \item \textbf{Step 4}: Verify that a success message is displayed.
            \item \textbf{Step 5}: Check the database to ensure the new labeler account exists with the correct details.
            \item \textbf{Step 6}: Confirm that the password is stored securely (hashed).
        \end{enumerate}
        \item \textbf{Cleanup}:
        \begin{itemize}
            \item After verification, delete the test labeler account from the database to maintain a clean state for future tests.
        \end{itemize}
    \end{itemize}
\end{enumerate}

\section*{FR10: Labeler Authentication Test}

\begin{enumerate}
    \item \textbf{Test ID}: T-FR10
    \item \textbf{Control}: \textbf{Automated}
    \item \textbf{Initial State}: Labeler account exists with credentials:
    \begin{itemize}
        \item \textbf{Email}: \texttt{labeler.test@example.com}
        \item \textbf{Password}: ``LabelerSecure!2022''
    \end{itemize}
    \item \textbf{Input}:
    \begin{itemize}
        \item Labeler enters correct login credentials.
    \end{itemize}
    \item \textbf{Output}:
    \begin{itemize}
        \item \textbf{Expected Result}:
        \begin{itemize}
            \item Labeler is authenticated successfully.
            \item Access to the labeler dashboard is granted.
        \end{itemize}
    \end{itemize}
    \item \textbf{Test Case Derivation}:
    \begin{itemize}
        \item Confirms authentication mechanisms for labelers, maintaining system security.
    \end{itemize}
    \item \textbf{How Test Will Be Performed}:
    \begin{itemize}
        \item \textbf{Automated Test Script Execution}:
        \begin{enumerate}
            \item \textbf{Step 1}: Navigate to the labeler login page.
            \item \textbf{Step 2}: Input the correct credentials.
            \item \textbf{Step 3}: Submit the login form.
            \item \textbf{Step 4}: Verify redirection to the labeler dashboard.
            \item \textbf{Step 5}: Check for access to labeler-specific features and data.
            \item \textbf{Negative Test}:
            \begin{enumerate}
                \item \textbf{Step 6}: Attempt login with incorrect credentials to ensure authentication fails appropriately.
            \end{enumerate}
        \end{enumerate}
    \end{itemize}
\end{enumerate}

\section*{FR11: Labeler Account Modification Test}

\begin{enumerate}
    \item \textbf{Test ID}: T-FR11
    \item \textbf{Control}: \textbf{Manual}
    \item \textbf{Initial State}: Labeler is logged in and on the account settings page.
    \item \textbf{Input}:
    \begin{itemize}
        \item Update personal information:
        \begin{itemize}
            \item \textbf{Expertise Area}: Add ``Aerial Photography Annotation''
            \item \textbf{Contact Number}: ``555-6789''
        \end{itemize}
    \end{itemize}
    \item \textbf{Output}:
    \begin{itemize}
        \item \textbf{Expected Result}:
        \begin{itemize}
            \item Personal information is updated and stored in the database.
            \item Labeler receives a confirmation of successful update.
        \end{itemize}
    \end{itemize}
    \item \textbf{Test Case Derivation}:
    \begin{itemize}
        \item Ensures labelers can maintain current information, which is crucial for assignment matching.
    \end{itemize}
    \item \textbf{How Test Will Be Performed}:
    \begin{itemize}
        \item \textbf{Manual Steps}:
        \begin{enumerate}
            \item \textbf{Step 1}: Navigate to account settings.
            \item \textbf{Step 2}: Modify the expertise area and contact number.
            \item \textbf{Step 3}: Save the changes.
            \item \textbf{Step 4}: Verify that the updated information is displayed.
            \item \textbf{Step 5}: Check the database for updated records.
            \item \textbf{Step 6}: Log out and log back in to confirm persistence.
        \end{enumerate}
    \end{itemize}
\end{enumerate}

\section*{FR12: Labeler Earnings Transfer Test}

\begin{enumerate}
    \item \textbf{Test ID}: T-FR12
    \item \textbf{Control}: \textbf{Automated}
    \item \textbf{Initial State}: Labeler is logged in with available earnings exceeding the minimum transfer threshold.
    \item \textbf{Input}:
    \begin{itemize}
        \item Transfer request to linked banking platform:
        \begin{itemize}
            \item \textbf{Amount}: Total available earnings.
            \item \textbf{Bank Account Details}: Pre-verified and linked.
        \end{itemize}
    \end{itemize}
    \item \textbf{Output}:
    \begin{itemize}
        \item \textbf{Expected Result}:
        \begin{itemize}
            \item Earnings are transferred successfully.
            \item Transaction record is created.
            \item Labeler receives confirmation and updated earnings balance.
        \end{itemize}
    \end{itemize}
    \item \textbf{Test Case Derivation}:
    \begin{itemize}
        \item Ensures labelers are compensated accurately and promptly, critical for system trust.
    \end{itemize}
    \item \textbf{How Test Will Be Performed}:
    \begin{itemize}
        \item \textbf{Automated Test Script Execution}:
        \begin{enumerate}
            \item \textbf{Step 1}: Navigate to the earnings or wallet section.
            \item \textbf{Step 2}: Initiate a transfer request for the total available amount.
            \item \textbf{Step 3}: Mock the banking API response if necessary.
            \item \textbf{Step 4}: Verify that the system processes the transfer without errors.
            \item \textbf{Step 5}: Confirm that the earnings balance is updated to reflect the transfer.
            \item \textbf{Step 6}: Check that a transaction record is logged in the database.
            \item \textbf{Step 7}: Validate that a confirmation message or email is sent to the labeler.
        \end{enumerate}
    \end{itemize}
\end{enumerate}

\section*{FR13: Image Annotation Test}

\begin{enumerate}
    \item \textbf{Test ID}: T-FR13
    \item \textbf{Control}: \textbf{Manual}
    \item \textbf{Initial State}: Labeler is logged in with images assigned for annotation.
    \item \textbf{Input}:
    \begin{itemize}
        \item Labeler annotates an image using the provided tools:
        \begin{itemize}
            \item Draws bounding boxes around objects.
            \item Adds classification labels to each object.
            \item Saves the annotation.
        \end{itemize}
    \end{itemize}
    \item \textbf{Output}:
    \begin{itemize}
        \item \textbf{Expected Result}:
        \begin{itemize}
            \item Annotated image is stored in the system.
            \item Annotation data is correctly linked to the image and service request.
            \item Labeler receives confirmation of successful submission.
        \end{itemize}
    \end{itemize}
    \item \textbf{Test Case Derivation}:
    \begin{itemize}
        \item Confirms annotation capabilities, essential for image analysis services.
    \end{itemize}
    \item \textbf{How Test Will Be Performed}:
    \begin{itemize}
        \item \textbf{Manual Steps}:
        \begin{enumerate}
            \item \textbf{Step 1}: Access the image annotation interface.
            \item \textbf{Step 2}: Use annotation tools to mark objects.
            \item \textbf{Step 3}: Assign appropriate labels to each annotation.
            \item \textbf{Step 4}: Save and submit the annotations.
            \item \textbf{Step 5}: Verify that a confirmation is received.
            \item \textbf{Step 6}: Check the database to ensure annotations are stored correctly.
            \item \textbf{Step 7}: Attempt to re-access the annotated image to confirm annotations persist.
        \end{enumerate}
    \end{itemize}
\end{enumerate}

\section*{FR14: Consolidated Annotation Report Test}

\begin{enumerate}
    \item \textbf{Test ID}: T-FR14
    \item \textbf{Control}: \textbf{Automated}
    \item \textbf{Initial State}: All required labeler annotations are complete for a service request.
    \item \textbf{Input}:
    \begin{itemize}
        \item System triggers consolidation process for annotations.
    \end{itemize}
    \item \textbf{Output}:
    \begin{itemize}
        \item \textbf{Expected Result}:
        \begin{itemize}
            \item Consolidated report is generated if label accuracy meets the predefined threshold.
            \item Report is stored and made accessible to the customer.
        \end{itemize}
    \end{itemize}
    \item \textbf{Test Case Derivation}:
    \begin{itemize}
        \item Verifies that the system consolidates annotations accurately, ensuring quality results for customers.
    \end{itemize}
    \item \textbf{How Test Will Be Performed}:
    \begin{itemize}
        \item \textbf{Automated Test Script Execution}:
        \begin{enumerate}
            \item \textbf{Step 1}: Simulate completion of all annotations for a service request.
            \item \textbf{Step 2}: Initiate the consolidation process (this may be automatic upon annotation completion).
            \item \textbf{Step 3}: Verify that the system calculates label accuracy and compares it against the threshold.
            \item \textbf{Step 4}: Confirm that if the accuracy meets or exceeds the threshold, the report is generated.
            \item \textbf{Step 5}: Check that the report contains consolidated data from all annotations.
            \item \textbf{Step 6}: Ensure the report is stored and accessible to the customer linked to the service request.
            \item \textbf{Step 7}: Validate that notifications are sent to the customer about the report availability.
        \end{enumerate}
        \item \textbf{Edge Case Testing}:
        \begin{enumerate}
            \item \textbf{Step 8}: Repeat the test with annotations that do not meet the accuracy threshold to verify that the report is not generated and appropriate actions are taken (e.g., re-annotation request).
        \end{enumerate}
    \end{itemize}
\end{enumerate}

\subsection{Tests for Nonfunctional Requirements}

\wss{The nonfunctional requirements for accuracy will likely just reference the
  appropriate functional tests from above.  The test cases should mention
  reporting the relative error for these tests.  Not all projects will
  necessarily have nonfunctional requirements related to accuracy.}

\wss{For some nonfunctional tests, you won't be setting a target threshold for
passing the test, but rather describing the experiment you will do to measure
the quality for different inputs.  For instance, you could measure speed versus
the problem size.  The output of the test isn't pass/fail, but rather a summary
table or graph.}

\wss{Tests related to usability could include conducting a usability test and
  survey.  The survey will be in the Appendix.}

\wss{Static tests, review, inspections, and walkthroughs, will not follow the
format for the tests given below.}

\wss{If you introduce static tests in your plan, you need to provide details.
How will they be done?  In cases like code (or document) walkthroughs, who will
be involved? Be specific.}

\subsubsection{Operational and Environmental}
		
\paragraph{Release Tests}

\begin{enumerate}

\item{Road Map Consistency: T-OE6\\}

Type: Manual, Static
					
Initial State: Application has a release road map that is publicly accessible.
					
Input/Condition: Team member conducts a review.
					
Output/Result: At least \hyperref[MIN_ON_TIME_MILESTONE]{MIN\_ON\_TIME\_MILESTONE}\% of the listed milestones have been met on time.
					
How test will be performed: The team member looks over the road map and cross references the completion date of milestones to the dates listed in the road map.

\item{Beta Testing: T-OE7\\}

Type: Dynamic, Exploratory
					
Initial State: Beta version of application is deployed and accessible
					
Input/Condition: At least \hyperref[BETA_TESTERS]{BETA\_TESTERS} beta testers are provided access to use the application.
					
Output/Result: Feedback on any bugs, navigation issues, or aesthetic problems is provided. Less than \hyperref[MAX_BUGS_FOUND]{MAX\_BUGS\_FOUND} bugs are found.
					
How test will be performed: Testers will be recruited and identified. They will be from fields of interest that include scientists, labelers, and domain experts. Then, the beta testing environment will be set up and the url will be distributed to the testers along with any other set up resources. Specific tasks are provided for testers to complete that focus on the annotation tools, sign up process and project creation. Feedback will be collected through direct comments from the tester.

\item{Regression Testing: T-OE8\\}

Type: Dynamic, Automated
					
Initial State: Application is deployed.
					
Input/Condition: Run regression test suite, consisting of unit tests.
					
Output/Result: All regression tests are passed.
					
How test will be performed: An automated script with regression tests will run when updates are made to the production build.

\end{enumerate}

\subsubsection{Maintainability and Support}
		
\paragraph{Maintenance Tests}

\begin{enumerate}

\item{Ease of Change: T-MS0\\}

Type: Manual, Static
					
Initial State: Application's source repository contains complete documentation.
					
Input/Condition: Competent software developer who has not previously worked on the app reviews documentation and attempts to perform tasks.
					
Output/Result: The developer can easily make a minor update to a specified part of the application.
					
How test will be performed: Give the developer time to read through the documentation. Give them a maintenance task, such as updating the size of the title font to 20px. Observe them and document how long it takes them and if they encountered any troubles.
\end{enumerate}

\subsubsection{Security}
		
\paragraph{Access Tests}

\begin{enumerate}

\item{Logged Out Permissions: T-SE0\\}

Type: Manual, Dynamic, White-box
					
Initial State: Application is deployed.
					
Input/Condition: Tester who is not signed in tries to access application paths for project creation and image labeling (Ex. /projects or /label).
					
Output/Result: The tester is denied access to these paths and is told to sign in.
					
How test will be performed: On the deployed application, the tester will visit all possible paths as a logged out user.

\item{Labeler Permissions: T-SE1\\}

Type: Manual, Dynamic, White-box
					
Initial State: Application is deployed.
					
Input/Condition: Tester who is signed in as a labeler tries to access application paths for project creation.
					
Output/Result: The tester is denied access to these paths. However, the tester has access to paths related to image labeling.
					
How test will be performed: On the deployed application, the tester will visit all possible paths as a labeler.

\item{Invalid Email Format: T-SE2\\}

Type: Automatic, Dynamic
					
Initial State: Front-end registration page is created and integrated with the database.
					
Input/Condition: Email with invalid format, such as an empty string or a string missing '@', is entered.
					
Output/Result: Application rejects email and tells the user that the email format is wrong.
					
How test will be performed: A unit test will be performed where the input is entered into the email section of the registration form.

\item{Duplicate Email: T-SE3\\}

Type: Automatic, Dynamic
					
Initial State: Front-end registration page is created and integrated with the database.
					
Input/Condition: Email that is already in database is entered.
					
Output/Result: Application rejects email and tells the user that the email is in use.
					
How test will be performed: A unit test will be performed where the input is entered into the email section of the registration form.

\item{Invalid Password Format: T-SE4\\}

Type: Automatic, Dynamic
					
Initial State: Front-end registration page is created and integrated with the database.
					
Input/Condition: Password with invalid format, such as an empty string or a string with no numbers, is entered.
					
Output/Result: Application rejects password and tells the user what requirements they have not met.
					
How test will be performed: A unit test will be performed where the input is entered into the password section of the registration form.

\item{System Error: T-SE5\\}

Type: Manual, Dynamic
					
Initial State: Application is deployed.
					
Input/Condition: Purposely invoke a system failure, and attempt to perform an action such as a label submission.
					
Output/Result: Application provides an error message on the user interface. The database has not changed in anyway.
					
How test will be performed: Go on to the application, start a labeling task, purposely disconnect from the internet, and try to submit a labeled image.

\end{enumerate}

\paragraph{Integrity Tests}

\begin{enumerate}

\item{Duplicate Entries: T-SE6\\}

Type: Manual, Dynamic
					
Initial State: Database is deployed.
					
Input/Condition: Duplicate database entry is inserted into the database.
					
Output/Result: Database has only one of the inputted entry and the duplicate has been removed.
					
How test will be performed: Attempt to insert the same entry twice into the database through the database UI.

\end{enumerate}

\paragraph{Privacy Tests}

\begin{enumerate}

\item{Encrypted User Data: T-SE7\\}

Type: Manual, Dynamic
					
Initial State: Application is deployed.
					
Input/Condition: Tester registers an account.
					
Output/Result: All sensitive user data that is stored in the database is encrypted.
					
How test will be performed: Tester will create a new account, then check the corresponding user entry in the database and see if the sensitive information is encrypted.

\item{Encrypted Payments: T-SE8\\}

Type: Manual, Dynamic
					
Initial State: Application is deployed.
					
Input/Condition: Tester enters sample payment details to pay for a labeling project that has been created.
					
Output/Result: These details are encrypted and can not be read through packet analyzers. The amount in the request can not be modified by an adversary.
					
How test will be performed: Tester will enter sample payment details, and submit their payment. Using a packet analyzer (such as Wireshark), packets from this request will be looked at to ensure all information is encrypted.

\end{enumerate}

\paragraph{Immunity Tests}

\begin{enumerate}

\item{SQL Injection: T-SE9\\}

Type: Manual, Dynamic
					
Initial State: Application is deployed.
					
Input/Condition: A malicious SQL statement is entered into a text field.
					
Output/Result: The system raises an error telling the user that it is invalid.
					
How test will be performed: Tester will enter a SQL statement such as "\{valid email\}'--" into an input such as the email input. This example has the potential to bypass a password check by commenting out the rest of the SQL query. The tester will check that when this statement is entered, the system gives feedback that it is invalid.

\end{enumerate}

\subsubsection{Cultural}
		
\paragraph{Language Tests}

\begin{enumerate}

\item{Support of Different Languages: T-CU0\\}

Type: Manual, Dynamic
					
Initial State: Application is deployed.
					
Input/Condition: Tester selects a language from a list of available languages.
					
Output/Result: All text on the website is translated and displayed in the selected language.
					
How test will be performed: Tester will check that the language selection list is accessible, and that the most popular languages are included. When a language is selected, the tester will check that the translation has been applied and there is no untranslated or gibberish text. This can be checked for each language.
\end{enumerate}

\subsubsection{Compliance}

\paragraph{Financial Tests}

\begin{enumerate}

\item{Compliant Payment Process: T-CO0\\}

Type: Manual, Static
					
Initial State: Application is deployed.
					
Input/Condition: Qualified Security Assessor (QSA) assesses the application.
					
Output/Result: They determine that it meets the PCI-DSS standard.
					
How test will be performed: A QSA will be found and contacted to perform an assessment. The QSA will be shown all parts of the application that deal will financial transactions and will be able to make a determination on if it meets the standard.
\end{enumerate}
		
\paragraph{Legal Tests}

\begin{enumerate}

\item{System Availability: T-CO1\\}

Type: Manual, Dynamic
					
Initial State: Application is deployed.
					
Input/Condition: Tester changes country they are accessing the application from using a tool such as a VPN.
					
Output/Result: Application is blocked in countries facing economic sanctions by the Government of Canada.
					
How test will be performed: A list of the countries facing economic sanctions by Canada will be compiled. Then, the tester will simulate that they are accessing the application from these countries, and ensure it is unreachable.

\item{Taxes: T-CO2\\}

Type: Manual, Dynamic
					
Initial State: Application is deployed.
					
Input/Condition: Tester redeems a cash balance.
					
Output/Result: If the cash balance exceeds a threshold, a tax form will be issued.
					
How test will be performed: Tester creates a test account with an account balance over the threshold. When they withdraw, they check that a tax form has been emailed to the email associated with the account. The tax form should reflect the withdrawal balance.

\item{Project Availability: T-CO3\\}

Type: Manual, Dynamic
					
Initial State: Application is deployed.
					
Input/Condition: Tester changes country they are accessing the application from using a tool such as a VPN.
					
Output/Result: Specific project is not shown.
					
How test will be performed: A project will be specified to only be distributed in a specific country. Then, the tester will simulate that they are accessing the application from other countries, and ensure the project does not show up.

\end{enumerate}

\subsubsection{User Documentation and Training}

\begin{enumerate}

\item{Helpfulness of User Aids: T-UDT0\\}

Type: Manual, Dynamic
					
Initial State: Application is deployed with all help features. Tutorials and user documentation have been created.
					
Input/Condition: Users attempt to complete a basic labeling task using only the platform’s built-in help resources (help system, quick start guide, tutorials and contextual tooltips).
					
Output/Result: At least \hyperref[MIN_USER_HELP_SATISFACTION]{MIN\_USER\_HELP\_SATISFACTION}\% of users who used a help feature found that feature helpful. With the assistance of the help tools, the user was able to perform the task within \hyperref[MAX_TASK_TIME]{MAX\_TASK\_TIME} minutes.
					
How test will be performed: The purpose of the platform will be explained to the users and the built-in help features will be shown. Then, the labeling task will be given to them. Time to complete task is observed and the help tools they use are recorded. Participants will then fill out a usability survey, which can be viewed in the appendix.

\item{Usefulness of Sandbox: T-UDT1\\}

Type: Automatic, Dynamic
					
Initial State: Application is deployed with all help features. Tutorials and user documentation has been created.
					
Input/Condition: A new user has accessed the platform.
					
Output/Result: At least \hyperref[MIN_PRACTICE_USAGE]{MIN\_PRACTICE\_USAGE}\% of new users utilize the practice environment, with self-assessment scores indicating an average improvement of \hyperref[IMPROVE_IN_ACC]{IMPROVE\_IN\_ACC}\% in labeling accuracy over their first three attempts.
					
How test will be performed: Practice environment utilization will be tracked by the application. Improvement in accuracy will also be tracked. If the metrics meet or succeed our thresholds, then we can conclude the sandbox is useful.
\end{enumerate}

...

\subsection{Traceability Between Test Cases and Requirements}

\setlength\LTleft{-3cm}
\tiny
\begin{longtable}{llllllllllllllll}
\multicolumn{1}{l|}{FR \#} & \multicolumn{15}{c}{Test Case} \\ \hline
\endfirsthead
%
\endhead
%
\multicolumn{1}{l|}{} & \multicolumn{1}{c}{T-FR0} & \multicolumn{1}{c}{T-FR1} & \multicolumn{1}{c}{T-FR2} & \multicolumn{1}{c}{T-FR3} & \multicolumn{1}{c}{T-FR4} & \multicolumn{1}{c}{T-FR5} & \multicolumn{1}{c}{T-FR6} & \multicolumn{1}{c}{T-FR7} & \multicolumn{1}{c}{T-FR8} & \multicolumn{1}{c}{T-FR9} & \multicolumn{1}{c}{T-FR10} & \multicolumn{1}{c}{T-FR11} & \multicolumn{1}{c}{T-FR12} & \multicolumn{1}{c}{T-FR13} & \multicolumn{1}{c}{T-FR14} \\
\multicolumn{1}{l|}{FR0} & \multicolumn{1}{c}{X} & \multicolumn{1}{c}{} & \multicolumn{1}{c}{} & \multicolumn{1}{c}{} & \multicolumn{1}{c}{} & \multicolumn{1}{c}{} & \multicolumn{1}{c}{} & \multicolumn{1}{c}{} & \multicolumn{1}{c}{} & \multicolumn{1}{c}{} &  & \multicolumn{1}{c}{} & \multicolumn{1}{c}{} & \multicolumn{1}{c}{} & \multicolumn{1}{c}{} \\
\multicolumn{1}{l|}{FR1} & \multicolumn{1}{c}{} & \multicolumn{1}{c}{X} & \multicolumn{1}{c}{} & \multicolumn{1}{c}{} & \multicolumn{1}{c}{} & \multicolumn{1}{c}{} & \multicolumn{1}{c}{} & \multicolumn{1}{c}{} & \multicolumn{1}{c}{} & \multicolumn{1}{c}{} &  & \multicolumn{1}{c}{} & \multicolumn{1}{c}{} & \multicolumn{1}{c}{} & \multicolumn{1}{c}{} \\
\multicolumn{1}{l|}{FR2} & \multicolumn{1}{c}{} & \multicolumn{1}{c}{} & \multicolumn{1}{c}{X} & \multicolumn{1}{c}{} & \multicolumn{1}{c}{} & \multicolumn{1}{c}{} & \multicolumn{1}{c}{} & \multicolumn{1}{c}{} & \multicolumn{1}{c}{} & \multicolumn{1}{c}{} &  & \multicolumn{1}{c}{} & \multicolumn{1}{c}{} & \multicolumn{1}{c}{} & \multicolumn{1}{c}{} \\
\multicolumn{1}{l|}{FR3} & \multicolumn{1}{c}{} & \multicolumn{1}{c}{} & \multicolumn{1}{c}{} & \multicolumn{1}{c}{X} & \multicolumn{1}{c}{} & \multicolumn{1}{c}{} & \multicolumn{1}{c}{} & \multicolumn{1}{c}{} & \multicolumn{1}{c}{} & \multicolumn{1}{c}{} &  & \multicolumn{1}{c}{} & \multicolumn{1}{c}{} & \multicolumn{1}{c}{} & \multicolumn{1}{c}{} \\
\multicolumn{1}{l|}{FR4} & \multicolumn{1}{c}{} & \multicolumn{1}{c}{} & \multicolumn{1}{c}{} & \multicolumn{1}{c}{} & \multicolumn{1}{c}{X} & \multicolumn{1}{c}{} & \multicolumn{1}{c}{} & \multicolumn{1}{c}{} & \multicolumn{1}{c}{} & \multicolumn{1}{c}{} &  & \multicolumn{1}{c}{} & \multicolumn{1}{c}{} & \multicolumn{1}{c}{} & \multicolumn{1}{c}{} \\
\multicolumn{1}{l|}{FR5} & \multicolumn{1}{c}{} & \multicolumn{1}{c}{} & \multicolumn{1}{c}{} & \multicolumn{1}{c}{} & \multicolumn{1}{c}{} & \multicolumn{1}{c}{X} & \multicolumn{1}{c}{} & \multicolumn{1}{c}{} & \multicolumn{1}{c}{} & \multicolumn{1}{c}{} &  & \multicolumn{1}{c}{} & \multicolumn{1}{c}{} & \multicolumn{1}{c}{} & \multicolumn{1}{c}{} \\
\multicolumn{1}{l|}{FR6} & \multicolumn{1}{c}{} & \multicolumn{1}{c}{} & \multicolumn{1}{c}{} & \multicolumn{1}{c}{} & \multicolumn{1}{c}{} & \multicolumn{1}{c}{} & \multicolumn{1}{c}{X} & \multicolumn{1}{c}{} & \multicolumn{1}{c}{} & \multicolumn{1}{c}{} &  & \multicolumn{1}{c}{} & \multicolumn{1}{c}{} & \multicolumn{1}{c}{} & \multicolumn{1}{c}{} \\
\multicolumn{1}{l|}{FR7} & \multicolumn{1}{c}{} & \multicolumn{1}{c}{} & \multicolumn{1}{c}{} & \multicolumn{1}{c}{} & \multicolumn{1}{c}{} & \multicolumn{1}{c}{} & \multicolumn{1}{c}{} & \multicolumn{1}{c}{X} & \multicolumn{1}{c}{} & \multicolumn{1}{c}{} &  & \multicolumn{1}{c}{} & \multicolumn{1}{c}{} & \multicolumn{1}{c}{} & \multicolumn{1}{c}{} \\
\multicolumn{1}{l|}{FR8} & \multicolumn{1}{c}{} & \multicolumn{1}{c}{} & \multicolumn{1}{c}{} & \multicolumn{1}{c}{} & \multicolumn{1}{c}{} & \multicolumn{1}{c}{} & \multicolumn{1}{c}{} & \multicolumn{1}{c}{} & \multicolumn{1}{c}{X} & \multicolumn{1}{c}{} &  & \multicolumn{1}{c}{} & \multicolumn{1}{c}{} & \multicolumn{1}{c}{} & \multicolumn{1}{c}{} \\
\multicolumn{1}{l|}{FR9} & \multicolumn{1}{c}{} & \multicolumn{1}{c}{} & \multicolumn{1}{c}{} & \multicolumn{1}{c}{} & \multicolumn{1}{c}{} & \multicolumn{1}{c}{} & \multicolumn{1}{c}{} & \multicolumn{1}{c}{} & \multicolumn{1}{c}{} & \multicolumn{1}{c}{X} &  & \multicolumn{1}{c}{} & \multicolumn{1}{c}{} & \multicolumn{1}{c}{} & \multicolumn{1}{c}{} \\
\multicolumn{1}{l|}{FR10} &  &  &  &  &  &  &  &  &  & \multicolumn{1}{c}{} & \multicolumn{1}{c}{X} & \multicolumn{1}{c}{} & \multicolumn{1}{c}{} & \multicolumn{1}{c}{} & \multicolumn{1}{c}{} \\
\multicolumn{1}{l|}{FR11} &  &  &  &  &  &  &  &  &  & \multicolumn{1}{c}{} &  & \multicolumn{1}{c}{X} & \multicolumn{1}{c}{} & \multicolumn{1}{c}{} & \multicolumn{1}{c}{} \\
\multicolumn{1}{l|}{FR12} &  &  &  &  &  &  &  &  &  & \multicolumn{1}{c}{} &  & \multicolumn{1}{c}{} & \multicolumn{1}{c}{X} & \multicolumn{1}{c}{} & \multicolumn{1}{c}{} \\
\multicolumn{1}{l|}{FR13} &  &  &  &  &  &  &  &  &  & \multicolumn{1}{c}{} &  & \multicolumn{1}{c}{} & \multicolumn{1}{c}{} & \multicolumn{1}{c}{X} & \multicolumn{1}{c}{} \\
\multicolumn{1}{l|}{FR14} &  &  &  &  &  &  &  &  &  & \multicolumn{1}{c}{} &  & \multicolumn{1}{c}{} & \multicolumn{1}{c}{} & \multicolumn{1}{c}{} & \multicolumn{1}{c}{X} \\
 &  &  &  &  &  &  &  &  &  &  &  &  &  &  &  \\
\multicolumn{1}{l|}{NFR \#} & \multicolumn{15}{c}{Test Case} \\ \hline
\multicolumn{1}{l|}{} & \multicolumn{1}{c}{T-LF0} & \multicolumn{1}{c}{T-LF1} & \multicolumn{1}{c}{T-LF2} & \multicolumn{1}{c}{T-UH0} & \multicolumn{1}{c}{T-UH1} & \multicolumn{1}{c}{T-UH2} & \multicolumn{1}{c}{T-UH3} & \multicolumn{1}{c}{T-UH4} & \multicolumn{1}{c}{T-UH5} & \multicolumn{1}{c}{T-UH6} & \multicolumn{1}{c}{T-UH7} & \multicolumn{1}{c}{T-UH8} & \multicolumn{1}{c}{T-UH9} & \multicolumn{1}{c}{T-PR0} & \multicolumn{1}{c}{T-PR1} \\
\multicolumn{1}{l|}{LF0} & \multicolumn{1}{c}{X} & \multicolumn{1}{c}{} & \multicolumn{1}{c}{} & \multicolumn{1}{c}{} & \multicolumn{1}{c}{} &  &  &  &  &  &  &  &  &  &  \\
\multicolumn{1}{l|}{LF1} & \multicolumn{1}{c}{} & \multicolumn{1}{c}{X} & \multicolumn{1}{c}{} & \multicolumn{1}{c}{} & \multicolumn{1}{c}{} &  &  &  &  &  &  &  &  &  &  \\
\multicolumn{1}{l|}{LF2} & \multicolumn{1}{c}{} & \multicolumn{1}{c}{} & \multicolumn{1}{c}{X} & \multicolumn{1}{c}{} & \multicolumn{1}{c}{} &  &  &  &  &  &  &  &  &  &  \\
\multicolumn{1}{l|}{UH0} & \multicolumn{1}{c}{} & \multicolumn{1}{c}{} & \multicolumn{1}{c}{} & \multicolumn{1}{c}{X} & \multicolumn{1}{c}{} &  &  &  &  &  &  &  &  &  &  \\
\multicolumn{1}{l|}{UH1} & \multicolumn{1}{c}{} & \multicolumn{1}{c}{} & \multicolumn{1}{c}{} & \multicolumn{1}{c}{} & \multicolumn{1}{c}{X} &  &  &  &  &  &  &  &  &  &  \\
\multicolumn{1}{l|}{UH2} & \multicolumn{1}{c}{} & \multicolumn{1}{c}{} & \multicolumn{1}{c}{} & \multicolumn{1}{c}{} & \multicolumn{1}{c}{} & \multicolumn{1}{c}{X} &  &  &  &  &  &  &  &  &  \\
\multicolumn{1}{l|}{UH3} & \multicolumn{1}{c}{} & \multicolumn{1}{c}{} & \multicolumn{1}{c}{} & \multicolumn{1}{c}{} & \multicolumn{1}{c}{} &  & \multicolumn{1}{c}{X} &  &  &  &  &  &  &  &  \\
\multicolumn{1}{l|}{UH4} & \multicolumn{1}{c}{} & \multicolumn{1}{c}{} & \multicolumn{1}{c}{} & \multicolumn{1}{c}{} & \multicolumn{1}{c}{} &  &  & \multicolumn{1}{c}{X} &  &  &  &  &  &  &  \\
\multicolumn{1}{l|}{UH5} &  &  &  &  &  &  &  &  & \multicolumn{1}{c}{X} & \multicolumn{1}{c}{} & \multicolumn{1}{c}{} & \multicolumn{1}{c}{} & \multicolumn{1}{c}{} & \multicolumn{1}{c}{} & \multicolumn{1}{c}{} \\
\multicolumn{1}{l|}{UH6} &  &  &  &  &  &  &  &  & \multicolumn{1}{c}{} & \multicolumn{1}{c}{X} & \multicolumn{1}{c}{} & \multicolumn{1}{c}{} & \multicolumn{1}{c}{} & \multicolumn{1}{c}{} & \multicolumn{1}{c}{} \\
\multicolumn{1}{l|}{UH7} &  &  &  &  &  &  &  &  & \multicolumn{1}{c}{} & \multicolumn{1}{c}{} & \multicolumn{1}{c}{X} & \multicolumn{1}{c}{} & \multicolumn{1}{c}{} & \multicolumn{1}{c}{} & \multicolumn{1}{c}{} \\
\multicolumn{1}{l|}{UH8} &  &  &  &  &  &  &  &  & \multicolumn{1}{c}{} & \multicolumn{1}{c}{} & \multicolumn{1}{c}{} & \multicolumn{1}{c}{X} & \multicolumn{1}{c}{} & \multicolumn{1}{c}{} & \multicolumn{1}{c}{} \\
\multicolumn{1}{l|}{UH9} &  &  &  &  &  &  &  &  & \multicolumn{1}{c}{} & \multicolumn{1}{c}{} & \multicolumn{1}{c}{} & \multicolumn{1}{c}{} & \multicolumn{1}{c}{X} & \multicolumn{1}{c}{} & \multicolumn{1}{c}{} \\
\multicolumn{1}{l|}{PR0} &  &  &  &  &  &  &  &  & \multicolumn{1}{c}{} & \multicolumn{1}{c}{} & \multicolumn{1}{c}{} & \multicolumn{1}{c}{} & \multicolumn{1}{c}{} & \multicolumn{1}{c}{X} & \multicolumn{1}{c}{} \\
\multicolumn{1}{l|}{PR1} &  &  &  &  &  &  &  &  & \multicolumn{1}{c}{} & \multicolumn{1}{c}{} & \multicolumn{1}{c}{} & \multicolumn{1}{c}{} & \multicolumn{1}{c}{} & \multicolumn{1}{c}{X} & \multicolumn{1}{c}{} \\
\multicolumn{1}{l|}{PR2} &  &  &  &  &  &  &  &  &  &  &  &  &  &  & \multicolumn{1}{c}{X} \\
 &  &  &  &  &  &  &  &  &  &  &  &  &  &  &  \\
\multicolumn{1}{l|}{NFR \#} & \multicolumn{15}{c}{Test Case} \\ \hline
\multicolumn{1}{l|}{} & \multicolumn{1}{c}{T-PR1} & \multicolumn{1}{c}{T-PR2} & \multicolumn{1}{c}{T-PR3} & \multicolumn{1}{c}{T-PR4} & \multicolumn{1}{c}{T-PR5} & \multicolumn{1}{c}{T-PR6} & \multicolumn{1}{c}{T-PR7} & \multicolumn{1}{c}{T-OE0} & \multicolumn{1}{c}{T-OE1} & \multicolumn{1}{c}{T-OE2} & \multicolumn{1}{c}{T-OE3} & \multicolumn{1}{c}{T-OE4} & \multicolumn{1}{c}{T-OE5} & \multicolumn{1}{c}{T-OE6} & \multicolumn{1}{c}{T-OE7} \\
\multicolumn{1}{l|}{PR3} & \multicolumn{1}{c}{} & \multicolumn{1}{c}{X} & \multicolumn{1}{c}{} & \multicolumn{1}{c}{} & \multicolumn{1}{c}{} &  &  &  &  &  &  &  &  &  &  \\
\multicolumn{1}{l|}{PR4} & \multicolumn{1}{c}{} & \multicolumn{1}{c}{} & \multicolumn{1}{c}{X} & \multicolumn{1}{c}{} & \multicolumn{1}{c}{} &  &  &  &  &  &  &  &  &  &  \\
\multicolumn{1}{l|}{PR5} & \multicolumn{1}{c}{} & \multicolumn{1}{c}{} & \multicolumn{1}{c}{} & \multicolumn{1}{c}{X} & \multicolumn{1}{c}{} &  &  &  &  &  &  &  &  &  &  \\
\multicolumn{1}{l|}{PR6} & \multicolumn{1}{c}{} & \multicolumn{1}{c}{} & \multicolumn{1}{c}{} & \multicolumn{1}{c}{} & \multicolumn{1}{c}{X} &  &  &  &  &  &  &  &  &  &  \\
\multicolumn{1}{l|}{PR7} & \multicolumn{1}{c}{X} & \multicolumn{1}{c}{} & \multicolumn{1}{c}{} & \multicolumn{1}{c}{} & \multicolumn{1}{c}{} &  &  &  &  &  &  &  &  &  &  \\
\multicolumn{1}{l|}{PR8} & \multicolumn{1}{c}{} & \multicolumn{1}{c}{} & \multicolumn{1}{c}{} & \multicolumn{1}{c}{} & \multicolumn{1}{c}{} & \multicolumn{1}{c}{X} &  &  &  &  &  &  &  &  &  \\
\multicolumn{1}{l|}{PR9} & \multicolumn{1}{c}{} & \multicolumn{1}{c}{} & \multicolumn{1}{c}{} & \multicolumn{1}{c}{} & \multicolumn{1}{c}{} &  & \multicolumn{1}{c}{X} &  &  &  &  &  &  &  &  \\
\multicolumn{1}{l|}{OE0} & \multicolumn{1}{c}{} & \multicolumn{1}{c}{} & \multicolumn{1}{c}{} & \multicolumn{1}{c}{} & \multicolumn{1}{c}{} &  &  & \multicolumn{1}{c}{X} &  &  &  &  &  &  &  \\
\multicolumn{1}{l|}{OE1} &  &  &  &  &  &  &  &  & \multicolumn{1}{c}{X} & \multicolumn{1}{c}{} & \multicolumn{1}{c}{} & \multicolumn{1}{c}{} & \multicolumn{1}{c}{} & \multicolumn{1}{c}{} & \multicolumn{1}{c}{} \\
\multicolumn{1}{l|}{OE2} &  &  &  &  &  &  &  &  & \multicolumn{1}{c}{} & \multicolumn{1}{c}{X} & \multicolumn{1}{c}{} & \multicolumn{1}{c}{} & \multicolumn{1}{c}{} & \multicolumn{1}{c}{} & \multicolumn{1}{c}{} \\
\multicolumn{1}{l|}{OE3} &  &  &  &  &  &  &  &  & \multicolumn{1}{c}{} & \multicolumn{1}{c}{} & \multicolumn{1}{c}{X} & \multicolumn{1}{c}{} & \multicolumn{1}{c}{} & \multicolumn{1}{c}{} & \multicolumn{1}{c}{} \\
\multicolumn{1}{l|}{OE4} &  &  &  &  &  &  &  &  & \multicolumn{1}{c}{} & \multicolumn{1}{c}{} & \multicolumn{1}{c}{} & \multicolumn{1}{c}{X} & \multicolumn{1}{c}{} & \multicolumn{1}{c}{} & \multicolumn{1}{c}{} \\
\multicolumn{1}{l|}{OE5} &  &  &  &  &  &  &  &  & \multicolumn{1}{c}{} & \multicolumn{1}{c}{} & \multicolumn{1}{c}{} & \multicolumn{1}{c}{X} & \multicolumn{1}{c}{} & \multicolumn{1}{c}{} & \multicolumn{1}{c}{} \\
\multicolumn{1}{l|}{OE6} &  &  &  &  &  &  &  &  & \multicolumn{1}{c}{} & \multicolumn{1}{c}{} & \multicolumn{1}{c}{} & \multicolumn{1}{c}{} & \multicolumn{1}{c}{X} & \multicolumn{1}{c}{} & \multicolumn{1}{c}{} \\
\multicolumn{1}{l|}{OE7} &  &  &  &  &  &  &  &  &  &  &  &  &  & \multicolumn{1}{c}{X} & \multicolumn{1}{c}{} \\
\multicolumn{1}{l|}{OE8} &  &  &  &  &  &  &  &  &  &  &  &  &  &  & \multicolumn{1}{c}{X} \\
 &  &  &  &  &  &  &  &  &  &  &  &  &  &  &  \\
\multicolumn{1}{l|}{NFR \#} & \multicolumn{15}{c}{Test Case} \\ \hline
\multicolumn{1}{l|}{} & \multicolumn{1}{c}{T-OE8} & \multicolumn{1}{c}{T-MS0} & \multicolumn{1}{c}{T-SE0} & \multicolumn{1}{c}{T-SE1} & \multicolumn{1}{c}{T-SE2} & \multicolumn{1}{c}{T-SE3} & \multicolumn{1}{c}{T-SE4} & \multicolumn{1}{c}{T-SE5} & \multicolumn{1}{c}{T-SE6} & \multicolumn{1}{c}{T-SE7} & \multicolumn{1}{c}{T-SE8} & \multicolumn{1}{c}{T-SE9} & \multicolumn{1}{c}{T-CU0} & \multicolumn{1}{c}{T-CO0} & \multicolumn{1}{c}{T-C01} \\
\multicolumn{1}{l|}{OE9} & \multicolumn{1}{c}{X} & \multicolumn{1}{c}{} & \multicolumn{1}{c}{} & \multicolumn{1}{c}{} & \multicolumn{1}{c}{} & \multicolumn{1}{c}{} & \multicolumn{1}{c}{} & \multicolumn{1}{c}{} & \multicolumn{1}{c}{} & \multicolumn{1}{c}{} & \multicolumn{1}{c}{} & \multicolumn{1}{c}{} & \multicolumn{1}{c}{} & \multicolumn{1}{c}{} & \multicolumn{1}{c}{} \\
\multicolumn{1}{l|}{MR0} & \multicolumn{1}{c}{} & \multicolumn{1}{c}{X} & \multicolumn{1}{c}{} & \multicolumn{1}{c}{} & \multicolumn{1}{c}{} & \multicolumn{1}{c}{} & \multicolumn{1}{c}{} & \multicolumn{1}{c}{} & \multicolumn{1}{c}{} & \multicolumn{1}{c}{} & \multicolumn{1}{c}{} & \multicolumn{1}{c}{} & \multicolumn{1}{c}{} & \multicolumn{1}{c}{} & \multicolumn{1}{c}{} \\
\multicolumn{1}{l|}{SE0} & \multicolumn{1}{c}{} & \multicolumn{1}{c}{} & \multicolumn{1}{c}{X} & \multicolumn{1}{c}{} & \multicolumn{1}{c}{} & \multicolumn{1}{c}{} & \multicolumn{1}{c}{} & \multicolumn{1}{c}{} & \multicolumn{1}{c}{} & \multicolumn{1}{c}{} & \multicolumn{1}{c}{} & \multicolumn{1}{c}{} & \multicolumn{1}{c}{} & \multicolumn{1}{c}{} & \multicolumn{1}{c}{} \\
\multicolumn{1}{l|}{SE1} & \multicolumn{1}{c}{} & \multicolumn{1}{c}{} & \multicolumn{1}{c}{} & \multicolumn{1}{c}{X} & \multicolumn{1}{c}{} & \multicolumn{1}{c}{} & \multicolumn{1}{c}{} & \multicolumn{1}{c}{} & \multicolumn{1}{c}{} & \multicolumn{1}{c}{} & \multicolumn{1}{c}{} & \multicolumn{1}{c}{} & \multicolumn{1}{c}{} & \multicolumn{1}{c}{} & \multicolumn{1}{c}{} \\
\multicolumn{1}{l|}{SE2} & \multicolumn{1}{c}{} & \multicolumn{1}{c}{} & \multicolumn{1}{c}{} & \multicolumn{1}{c}{} & \multicolumn{1}{c}{X} & \multicolumn{1}{c}{X} & \multicolumn{1}{c}{} & \multicolumn{1}{c}{} & \multicolumn{1}{c}{} & \multicolumn{1}{c}{} & \multicolumn{1}{c}{} & \multicolumn{1}{c}{} & \multicolumn{1}{c}{} & \multicolumn{1}{c}{} & \multicolumn{1}{c}{} \\
\multicolumn{1}{l|}{SE3} & \multicolumn{1}{c}{} & \multicolumn{1}{c}{} & \multicolumn{1}{c}{} & \multicolumn{1}{c}{} & \multicolumn{1}{c}{} & \multicolumn{1}{c}{} & \multicolumn{1}{c}{X} & \multicolumn{1}{c}{} & \multicolumn{1}{c}{} & \multicolumn{1}{c}{} & \multicolumn{1}{c}{} & \multicolumn{1}{c}{} & \multicolumn{1}{c}{} & \multicolumn{1}{c}{} & \multicolumn{1}{c}{} \\
\multicolumn{1}{l|}{SE4} & \multicolumn{1}{c}{} & \multicolumn{1}{c}{} & \multicolumn{1}{c}{} & \multicolumn{1}{c}{} & \multicolumn{1}{c}{} & \multicolumn{1}{c}{} & \multicolumn{1}{c}{} & \multicolumn{1}{c}{X} & \multicolumn{1}{c}{} & \multicolumn{1}{c}{} & \multicolumn{1}{c}{} & \multicolumn{1}{c}{} & \multicolumn{1}{c}{} & \multicolumn{1}{c}{} & \multicolumn{1}{c}{} \\
\multicolumn{1}{l|}{SE5} & \multicolumn{1}{c}{} & \multicolumn{1}{c}{} & \multicolumn{1}{c}{} & \multicolumn{1}{c}{} & \multicolumn{1}{c}{X} & \multicolumn{1}{c}{X} & \multicolumn{1}{c}{X} & \multicolumn{1}{c}{} & \multicolumn{1}{c}{} & \multicolumn{1}{c}{} & \multicolumn{1}{c}{} & \multicolumn{1}{c}{} & \multicolumn{1}{c}{} & \multicolumn{1}{c}{} & \multicolumn{1}{c}{} \\
\multicolumn{1}{l|}{SE6} &  &  &  &  &  &  &  & \multicolumn{1}{c}{X} & \multicolumn{1}{c}{} & \multicolumn{1}{c}{} & \multicolumn{1}{c}{} & \multicolumn{1}{c}{} & \multicolumn{1}{c}{} & \multicolumn{1}{c}{} & \multicolumn{1}{c}{} \\
\multicolumn{1}{l|}{SE7} &  &  &  &  &  &  &  &  & \multicolumn{1}{c}{X} & \multicolumn{1}{c}{} & \multicolumn{1}{c}{} & \multicolumn{1}{c}{} & \multicolumn{1}{c}{} & \multicolumn{1}{c}{} & \multicolumn{1}{c}{} \\
\multicolumn{1}{l|}{SE8} &  &  &  &  &  &  &  &  & \multicolumn{1}{c}{} & \multicolumn{1}{c}{X} & \multicolumn{1}{c}{} & \multicolumn{1}{c}{} & \multicolumn{1}{c}{} & \multicolumn{1}{c}{} & \multicolumn{1}{c}{} \\
\multicolumn{1}{l|}{SE9} &  &  &  &  &  &  &  &  & \multicolumn{1}{c}{} & \multicolumn{1}{c}{} & \multicolumn{1}{c}{X} & \multicolumn{1}{c}{} & \multicolumn{1}{c}{} & \multicolumn{1}{c}{X} & \multicolumn{1}{c}{} \\
\multicolumn{1}{l|}{SE10} &  &  &  &  &  &  &  &  & \multicolumn{1}{c}{} & \multicolumn{1}{c}{} & \multicolumn{1}{c}{} & \multicolumn{1}{c}{X} & \multicolumn{1}{c}{} & \multicolumn{1}{c}{} & \multicolumn{1}{c}{} \\
\multicolumn{1}{l|}{CU0} &  &  &  &  &  &  &  &  &  &  &  &  & \multicolumn{1}{c}{X} &  &  \\
\multicolumn{1}{l|}{CO0} &  &  &  &  &  &  &  &  &  &  &  &  &  &  & \multicolumn{1}{c}{X} \\
 &  &  &  &  &  &  &  &  &  &  &  &  &  &  &  \\
\multicolumn{1}{l|}{NFR \#} & \multicolumn{15}{c}{Test Case} \\ \hline
\multicolumn{1}{l|}{} & T-C02 & T-C03 & T-UDT0 & T-UDT1 &  &  &  &  &  &  &  &  &  &  &  \\
\multicolumn{1}{l|}{CO1} & \multicolumn{1}{c}{X} & \multicolumn{1}{c}{} & \multicolumn{1}{c}{} &  & \multicolumn{1}{c}{} &  &  &  &  &  &  &  &  &  &  \\
\multicolumn{1}{l|}{CO2} & \multicolumn{1}{c}{} & \multicolumn{1}{c}{X} & \multicolumn{1}{c}{} &  & \multicolumn{1}{c}{} &  &  &  &  &  &  &  &  &  &  \\
\multicolumn{1}{l|}{UD0} & \multicolumn{1}{c}{} & \multicolumn{1}{c}{} & \multicolumn{1}{c}{X} &  & \multicolumn{1}{c}{} &  &  &  &  &  &  &  &  &  &  \\
\multicolumn{1}{l|}{UD1} & \multicolumn{1}{c}{} & \multicolumn{1}{c}{} & \multicolumn{1}{c}{X} &  & \multicolumn{1}{c}{} &  &  &  &  &  &  &  &  &  &  \\
\multicolumn{1}{l|}{UD2} & \multicolumn{1}{c}{} & \multicolumn{1}{c}{} & \multicolumn{1}{c}{X} &  & \multicolumn{1}{c}{} &  &  &  &  &  &  &  &  &  &  \\
\multicolumn{1}{l|}{TR0} & \multicolumn{1}{c}{} & \multicolumn{1}{c}{} & \multicolumn{1}{c}{X} &  & \multicolumn{1}{c}{} &  &  &  &  &  &  &  &  &  &  \\
\multicolumn{1}{l|}{TR1} & \multicolumn{1}{c}{} & \multicolumn{1}{c}{} & \multicolumn{1}{c}{} & \multicolumn{1}{c}{X} & \multicolumn{1}{c}{} & \multicolumn{1}{c}{} &  &  &  &  &  &  &  &  &  \\
 &  &  &  &  &  &  &  &  &  &  &  &  &  &  &  \\
 & \multicolumn{1}{c}{} & \multicolumn{1}{c}{} & \multicolumn{1}{c}{} & \multicolumn{1}{c}{} & \multicolumn{1}{c}{} & \multicolumn{1}{c}{} & \multicolumn{1}{c}{} & \multicolumn{1}{c}{} & \multicolumn{1}{c}{} & \multicolumn{1}{c}{} &  & \multicolumn{1}{c}{} & \multicolumn{1}{c}{} & \multicolumn{1}{c}{} & \multicolumn{1}{c}{} \\
 &  &  &  &  &  &  &  &  &  & \multicolumn{1}{c}{} &  & \multicolumn{1}{c}{} & \multicolumn{1}{c}{} & \multicolumn{1}{c}{} & \multicolumn{1}{c}{} \\
 &  &  &  &  &  &  &  &  &  & \multicolumn{1}{c}{} &  & \multicolumn{1}{c}{} & \multicolumn{1}{c}{} & \multicolumn{1}{c}{} & \multicolumn{1}{c}{} \\
 &  &  &  &  &  &  &  &  &  & \multicolumn{1}{c}{} &  & \multicolumn{1}{c}{} & \multicolumn{1}{c}{} & \multicolumn{1}{c}{} & \multicolumn{1}{c}{} \\
 &  &  &  &  &  &  &  &  &  & \multicolumn{1}{c}{} &  & \multicolumn{1}{c}{} & \multicolumn{1}{c}{} & \multicolumn{1}{c}{} & \multicolumn{1}{c}{} \\
 &  &  &  &  &  &  &  &  &  & \multicolumn{1}{c}{} &  & \multicolumn{1}{c}{} & \multicolumn{1}{c}{} & \multicolumn{1}{c}{} & \multicolumn{1}{c}{} \\
 &  &  &  &  &  &  &  &  &  & \multicolumn{1}{c}{} &  & \multicolumn{1}{c}{} & \multicolumn{1}{c}{} & \multicolumn{1}{c}{} & \multicolumn{1}{c}{} \\
 &  &  &  &  &  &  &  &  &  &  &  &  &  &  &  \\
 &  &  &  &  &  &  &  &  &  &  &  &  &  &  &  \\
 &  &  &  &  &  &  &  &  &  &  &  &  &  &  &  \\
 &  &  &  &  &  &  &  &  &  &  &  &  &  &  &  \\
 &  &  &  &  &  &  &  &  &  &  &  &  &  &  &  \\
 &  &  &  &  &  &  &  &  &  &  &  &  &  &  &  \\
 &  &  &  &  &  &  &  &  &  &  &  &  &  &  &  \\
 &  &  &  &  &  &  &  &  &  &  &  &  &  &  &  \\
 &  &  &  &  &  &  &  &  &  &  &  &  &  &  &  \\
 &  &  &  &  &  &  &  &  &  &  &  &  &  &  &  \\
 &  &  &  &  &  &  &  &  &  &  &  &  &  &  &  \\
 &  &  &  &  &  &  &  &  &  &  &  &  &  &  &  \\
 &  &  &  &  &  &  &  &  &  &  &  &  &  &  &  \\
 &  &  &  &  &  &  &  &  &  &  &  &  &  &  &  \\
 &  &  &  &  &  &  &  &  &  &  &  &  &  &  &  \\
 &  &  &  &  &  &  &  &  &  &  &  &  &  &  &  \\
 &  &  &  &  &  &  &  &  &  &  &  &  &  &  &  \\
 &  &  &  &  &  &  &  &  &  &  &  &  &  &  &  \\
 &  &  &  &  &  &  &  &  &  &  &  &  &  &  &  \\
 &  &  &  &  &  &  &  &  &  &  &  &  &  &  &  \\
 &  &  &  &  &  &  &  &  &  &  &  &  &  &  &  \\
 &  &  &  &  &  &  &  &  &  &  &  &  &  &  &  \\
 &  &  &  &  &  &  &  &  &  &  &  &  &  &  &  \\
 &  &  &  &  &  &  &  &  &  &  &  &  &  &  &  \\
 &  &  &  &  &  &  &  &  &  &  &  &  &  &  &  \\
 &  &  &  &  &  &  &  &  &  &  &  &  &  &  &  \\
 &  &  &  &  &  &  &  &  &  &  &  &  &  &  &  \\
 &  &  &  &  &  &  &  &  &  &  &  &  &  &  &  \\
 &  &  &  &  &  &  &  &  &  &  &  &  &  &  &  \\
 &  &  &  &  &  &  &  &  &  &  &  &  &  &  &  \\
 &  &  &  &  &  &  &  &  &  &  &  &  &  &  &  \\
 &  &  &  &  &  &  &  &  &  &  &  &  &  &  &  \\
 &  &  &  &  &  &  &  &  &  &  &  &  &  &  & 
\end{longtable}
\normalsize

\section{Unit Test Description}

\wss{This section should not be filled in until after the MIS (detailed design
  document) has been completed.}

\wss{Reference your MIS (detailed design document) and explain your overall
philosophy for test case selection.}  

\wss{To save space and time, it may be an option to provide less detail in this section.  
For the unit tests you can potentially layout your testing strategy here.  That is, you 
can explain how tests will be selected for each module.  For instance, your test building 
approach could be test cases for each access program, including one test for normal behaviour 
and as many tests as needed for edge cases.  Rather than create the details of the input 
and output here, you could point to the unit testing code.  For this to work, you code 
needs to be well-documented, with meaningful names for all of the tests.}

\subsection{Unit Testing Scope}

\wss{What modules are outside of the scope.  If there are modules that are
  developed by someone else, then you would say here if you aren't planning on
  verifying them.  There may also be modules that are part of your software, but
  have a lower priority for verification than others.  If this is the case,
  explain your rationale for the ranking of module importance.}

\subsection{Tests for Functional Requirements}

\wss{Most of the verification will be through automated unit testing.  If
  appropriate specific modules can be verified by a non-testing based
  technique.  That can also be documented in this section.}

\subsubsection{Module 1}

\wss{Include a blurb here to explain why the subsections below cover the module.
  References to the MIS would be good.  You will want tests from a black box
  perspective and from a white box perspective.  Explain to the reader how the
  tests were selected.}

\begin{enumerate}

\item{test-id1\\}

Type: \wss{Functional, Dynamic, Manual, Automatic, Static etc. Most will
  be automatic}
					
Initial State: 
					
Input: 
					
Output: \wss{The expected result for the given inputs}

Test Case Derivation: \wss{Justify the expected value given in the Output field}

How test will be performed: 
					
\item{test-id2\\}

Type: \wss{Functional, Dynamic, Manual, Automatic, Static etc. Most will
  be automatic}
					
Initial State: 
					
Input: 
					
Output: \wss{The expected result for the given inputs}

Test Case Derivation: \wss{Justify the expected value given in the Output field}

How test will be performed: 

\item{...\\}
    
\end{enumerate}

\subsubsection{Module 2}

...

\subsection{Tests for Nonfunctional Requirements}

\wss{If there is a module that needs to be independently assessed for
  performance, those test cases can go here.  In some projects, planning for
  nonfunctional tests of units will not be that relevant.}

\wss{These tests may involve collecting performance data from previously
  mentioned functional tests.}

\subsubsection{Module ?}
		
\begin{enumerate}

\item{test-id1\\}

Type: \wss{Functional, Dynamic, Manual, Automatic, Static etc. Most will
  be automatic}
					
Initial State: 
					
Input/Condition: 
					
Output/Result: 
					
How test will be performed: 
					
\item{test-id2\\}

Type: Functional, Dynamic, Manual, Static etc.
					
Initial State: 
					
Input: 
					
Output: 
					
How test will be performed: 

\end{enumerate}

\subsubsection{Module ?}

...

\subsection{Traceability Between Test Cases and Modules}

\wss{Provide evidence that all of the modules have been considered.}
				
\bibliographystyle{plainnat}

\bibliography{../../refs/References}

\newpage

\section{Appendix}

This is where you can place additional information.

\subsection{Symbolic Parameters}

The definition of the test cases will call for SYMBOLIC\_CONSTANTS.
Their values are defined in this section for easy maintenance.

\begin{longtable}{|l|l|l|p{4cm}|}
  \hline
  \textbf{Parameter} & \textbf{Value} & \textbf{Unit} & \textbf{Description} \\ \hline
  \endfirsthead

  \hline
  \textbf{Parameter} & \textbf{Value} & \textbf{Unit} & \textbf{Description} \\ \hline
  \endhead

  \hline
  \endfoot

  \hline
  MIN\_ON\_TIME\_MILESTONE & 80 & \% & Minimum percent of milestones that have been met on time \label{MIN_ON_TIME_MILESTONE} \\ \hline
  BETA\_TESTERS & 50 & People & Number of beta testers \label{BETA_TESTERS} \\ \hline
  MAX\_BUGS\_FOUND & 10 & Bugs & Number of software bugs found \label{MAX_BUGS_FOUND} \\ \hline
  MIN\_USER\_HELP\_SATISFACTION & 80 & \% & Minimum percent of users satisfied with help feature \label{MIN_USER_HELP_SATISFACTION} \\ \hline
  MAX\_TASK\_TIME & 15 & Minutes & Maximum time it takes a user to complete a task \label{MAX_TASK_TIME} \\ \hline
  MIN\_PRACTICE\_USAGE & 80 & \% & Minimum percent of new users who have used the practice sandbox \label{MIN_PRACTICE_USAGE} \\ \hline
  IMPROVE\_IN\_ACC & 20 & \% & Improvement in accuracy of a user after practicing \label{IMPROVE_IN_ACC} \\ \hline
\end{longtable}

\subsection{Usability Survey Questions}
\begin{enumerate}
  \item Did you use the help system to aid in completing your task? Yes/No
  \item If you answered yes, please rate how useful it was in helping you accomplish your task: 1 (Not Useful) - 5 (Very Useful)
  \item Did you use the quick start guide to aid in completing your task? Yes/No
  \item If you answered yes, please rate the clarity and helpfulness of it: 1 (Not Helpful) - 5 (Very Helpful)
  \item Did you notice the tool-tips or pop-ups providing contextual help as you worked? Yes/No
  \item If you answered yes, please rate the clarity and usefulness of the in-app help indicators: 1 (Not Helpful) - 5 (Very Helpful)
  
\end{enumerate}

\newpage{}
\section*{Appendix --- Reflection}

\wss{This section is not required for CAS 741}

The information in this section will be used to evaluate the team members on the
graduate attribute of Lifelong Learning.

\input{../Reflection.tex}

\begin{enumerate}
  \item What went well while writing this deliverable? 
  \item What pain points did you experience during this deliverable, and how
    did you resolve them?
  \item What knowledge and skills will the team collectively need to acquire to
  successfully complete the verification and validation of your project?
  Examples of possible knowledge and skills include dynamic testing knowledge,
  static testing knowledge, specific tool usage, Valgrind etc.  You should look to
  identify at least one item for each team member.
  \item For each of the knowledge areas and skills identified in the previous
  question, what are at least two approaches to acquiring the knowledge or
  mastering the skill?  Of the identified approaches, which will each team
  member pursue, and why did they make this choice?
\end{enumerate}

\end{document}