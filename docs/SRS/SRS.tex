% THIS DOCUMENT IS FOLLOWS THE VOLERE TEMPLATE BY Suzanne Robertson and James Robertson
% ONLY THE SECTION HEADINGS ARE PROVIDED
%
% Initial draft from https://github.com/Dieblich/volere
%
% Risks are removed because they are covered by the Hazard Analysis
\documentclass[12pt]{article}

\usepackage{booktabs}
\usepackage{tabularx}
\usepackage{hyperref}
\usepackage{amsmath}
\hypersetup{
    bookmarks=true,         % show bookmarks bar?
      colorlinks=true,      % false: boxed links; true: colored links
    linkcolor=red,          % color of internal links (change box color with linkbordercolor)
    citecolor=green,        % color of links to bibliography
    filecolor=magenta,      % color of file links
    urlcolor=cyan           % color of external links
}

\newcommand{\lips}{\textit{Insert your content here.}}
\newcommand{\probP}{\text{I\kern-0.15em P}}

%% Comments

\usepackage{color}

\newif\ifcomments\commentstrue %displays comments
%\newif\ifcomments\commentsfalse %so that comments do not display

\ifcomments
\newcommand{\authornote}[3]{\textcolor{#1}{[#3 ---#2]}}
\newcommand{\todo}[1]{\textcolor{red}{[TODO: #1]}}
\else
\newcommand{\authornote}[3]{}
\newcommand{\todo}[1]{}
\fi

\newcommand{\wss}[1]{\authornote{blue}{SS}{#1}} 
\newcommand{\plt}[1]{\authornote{magenta}{TPLT}{#1}} %For explanation of the template
\newcommand{\an}[1]{\authornote{cyan}{Author}{#1}}

%% Common Parts

\newcommand{\progname}{Software Engineering} % PUT YOUR PROGRAM NAME HERE
\newcommand{\authname}{Team \#11, OKKM Insights
\\ Mathew Petronilho
\\ Oleg Glotov
\\ Kyle McMaster
\\ Kartik Chaudhari} % AUTHOR NAMES                  

\usepackage{hyperref}
    \hypersetup{colorlinks=true, linkcolor=blue, citecolor=blue, filecolor=blue,
                urlcolor=blue, unicode=false}
    \urlstyle{same}
                                


\begin{document}

\title{Software Requirements Specification for \progname: subtitle describing software} 
\author{\authname}
\date{\today}
	
\maketitle

~\newpage

\pagenumbering{roman}

\tableofcontents

~\newpage

\section*{Revision History}

\begin{tabularx}{\textwidth}{p{3cm}p{2cm}X}
\toprule {\textbf{Date}} & {\textbf{Version}} & {\textbf{Notes}}\\
\midrule
Date 1 & 1.0 & Notes\\
Date 2 & 1.1 & Notes\\
\bottomrule
\end{tabularx}

~\\

~\newpage
\section{Purpose of the Project}
\subsection{User Business}
\lips
\subsection{Goals of the Project}
\lips
\section{Stakeholders}
\subsection{Client}
\lips
\subsection{Customer}
\lips
\subsection{Other Stakeholders}
\lips
\subsection{Hands-On Users of the Project}
\lips
\subsection{Personas}
\lips
\subsection{Priorities Assigned to Users}
\lips
\subsection{User Participation}
\lips
\subsection{Maintenance Users and Service Technicians}
\lips

\section{Mandated Constraints}
\subsection{Solution Constraints}
\lips
\subsection{Implementation Environment of the Current System}
\lips
\subsection{Partner or Collaborative Applications}
\lips
\subsection{Off-the-Shelf Software}
\lips
\subsection{Anticipated Workplace Environment}
\lips
\subsection{Schedule Constraints}
\lips
\subsection{Budget Constraints}
\lips
\subsection{Enterprise Constraints}
\lips

\section{Naming Conventions and Terminology}
\subsection{Glossary of All Terms, Including Acronyms, Used by Stakeholders
involved in the Project}
\lips

\section{Relevant Facts And Assumptions}
\subsection{Relevant Facts}
\lips
\subsection{Business Rules}
\lips
\subsection{Assumptions}
\lips

\section{The Scope of the Work}
\subsection{The Current Situation}
\lips
\subsection{The Context of the Work}
\lips
\subsection{Work Partitioning}
\lips
\subsection{Specifying a Business Use Case (BUC)}
\lips

\section{Business Data Model and Data Dictionary}
\subsection{Business Data Model}
\lips
\subsection{Data Dictionary}
\lips

\section{The Scope of the Product}
\subsection{Product Boundary}
\lips
\subsection{Product Use Case Table}
\lips
\subsection{Individual Product Use Cases (PUC's)}
\lips

\section{Functional Requirements}
\subsection{Functional Requirements}
\lips

\section{Look and Feel Requirements}
\subsection{Appearance Requirements}
\lips
\subsection{Style Requirements}
\lips

\section{Usability and Humanity Requirements}
\subsection{Ease of Use Requirements}
\lips
\subsection{Personalization and Internationalization Requirements}
\lips
\subsection{Learning Requirements}
\lips
\subsection{Understandability and Politeness Requirements}
\lips
\subsection{Accessibility Requirements}
\lips

\section{Performance Requirements}
\subsection{Speed and Latency Requirements}
\subsubsection*{NFR-PR0}
\begin{itemize}
  \item \textbf{Description:} The system shall process new user requests within 15 minutes of a Customer completing the account creation process 90\% of the time, and within 48 hours in all cases. 
  \item \textbf{Rationale:} The user likely has urgent needs if they are signing up to access the system. They must have quick access to the services offered by the system, and authentication is the first step of this process.
  \item \textbf{Fit Criterion:}\\ Let $t_{\text{newCustomerAccount}}$ be the time it takes to process a new Customer account request, in hours.\\ \probP($t_{\text{newCustomerAccount}} < 0.25) \geq .90 \wedge \probP(t_{\text{newCustomerAccount}} < 48) = 1 $
\end{itemize}

\subsubsection*{NFR-PR1}
\begin{itemize}
  \item \textbf{Description:} The system shall process new user requests within 15 minutes of a Labeler  completing the account creation process 90\% of the time, and within 48 hours in all cases. 
  \item \textbf{Rationale:} To improve engagement from Labelers, there should be minimal delay in getting started.
  \item \textbf{Fit Criterion:}\\ Let $t_{\text{newLabelAccount}}$ be the time it takes to process a new Labeler account request, in hours.\\ \probP($t_{\text{newLabelAccount}} < 0.25) \geq 0.9 \wedge \probP(t_{\text{newLabelAccount}} < 48) = 1 $
\end{itemize}

\subsubsection*{NFR-PR2}
\begin{itemize}
  \item \textbf{Description:} The system shall return a complete report of results to the Customer within the negotiated amount of time, 90\% of the time, and within 48 additional hours in all cases.
  \item \textbf{Rationale:} The user is expecting to be able to act on the insights the system provides. To build trust and loyalty, the system must ensure that it is meeting the agreed upon timelines.
  \item \textbf{Fit Criterion:}\\ Let $t_{\text{serviceRequestTime}}$ be the time it takes to complete a service request, in hours.\\
  Let $t_{\text{serviceRequestTimeLimit}}$ be the negotiated time limit for completing a service request, in hours.\\ \probP($t_{\text{serviceRequestTime}} < t_{\text{serviceRequestTimeLimit}}) \geq 0.9\\ \wedge \probP(t_{\text{serviceRequestTime}} < t_{\text{serviceRequestTimeLimit}}+48) = 1 $
\end{itemize}

\subsubsection*{NFR-PR3}
\begin{itemize}
  \item \textbf{Description:} The system shall take no longer than 10 seconds to display the next image to be labeled to a labeler, if there is one available.
  \item \textbf{Rationale:} To reduce friction for labelers, the system must ensure there is no unnecessary delay in preparing the next job. For users exposed to modern media, attention spans can be expected to be less than 10 seconds \href{https://profiletree.com/attention-span-crisis-digital-age-statistics/#:~:text=Studies%20have%20proven%20that%20being,focus%20after%208%20mere%20seconds}{src}.
  \item \textbf{Fit Criterion:}\\ Let $t_{\text{imageServing}}$ be the time it takes to serve the next image, in seconds.\\
  Let $x_{\text{nextImage}}$ equal True if there is a next image available, and False otherwise.\\
  $x_{\text{nextImage}} \Rightarrow t_{\text{imageServing}} \leq 10$
\end{itemize}

\subsubsection*{NFR-PR4}
\begin{itemize}
  \item \textbf{Description:} The system shall deliver earned payouts to labelers within 7 business days of a request being made through the system.
  \item \textbf{Rationale:} The labelers are entitled to their earned income, and there must not be unnecessary delay. 7 days accounts for delays in the platform used to distribute payments.
  \item \textbf{Fit Criterion:}\\ Let $t_{\text{payoutDelay}}$ be the time it takes for a user to receive their payments after a request is received by the system, in days.\\
  $t_{\text{payoutDelay}} < 7$
\end{itemize}
\subsection{Safety-Critical Requirements}
N/A
\subsection{Precision or Accuracy Requirements}
\subsubsection*{NFR-PR5}
\begin{itemize}
  \item \textbf{Description:} The system shall report accurate labels 75\% of the time.
  \item \textbf{Rationale:} Average data label accuracy for competitors is greater than 75\% (\href{https://www.researchgate.net/publication/234774537_Data_quality_from_crowdsourcing_A_study_of_annotation_selection_criteria#:~:text=Depending%20on%20the%20number%20of,%5B13%5D%20%5B14%5D%20.}{link}). The system should at a minimum provide the same label accuracy.
  \item \textbf{Fit Criterion:}\\ Let $O$ be the set of objects to label.\\
  Let $C$ be the set of classes an object in $O$ can be.\\
  Let $L_{\text{True}}: O \rightarrow C $ be a function which maps objects in $O$ to their true classes in $C$.
  Let $L_{\text{Guess}}: O \rightarrow C $ be the funtion derived from the system which maps objects in $O$ to their assumed classes in $C$.
  $(\forall o \in O|: \probP(L_{\text{True}}(o) = L_{\text{Guess}}(o))\geq 0.75)$
\end{itemize}
\subsection{Robustness or Fault-Tolerance Requirements}
\subsubsection*{NFR-PR6}
\begin{itemize}
  \item \textbf{Description:} The system shall have 97\% uptime. 
  \item \textbf{Rationale:} It is crucial that in emergency response use cases, the system is able to accept and process requests with minimal delay.
  \item \textbf{Fit Criterion:}\\ Let $t_{\text{uptime}}$ be the uptime of the system.\\
  Let $t_{\text{downtime}}$ be the downtime of the system.\\
  $\frac{t_{\text{uptime}}}{t_{\text{uptime}} + t_{\text{downtime}}} > 0.97$
\end{itemize}
\subsection{Capacity Requirements}
\subsubsection*{NFR-PR7}
\begin{itemize}
  \item \textbf{Description:} The system shall have the capacity to support enough labelers to meet all service request deadlines.
  \item \textbf{Rationale:} This requirement is critical to satisfy NFR-PR2.
  \item \textbf{Fit Criterion:} See NFR-PR2.
\end{itemize}
\subsubsection*{NFR-PR8}
\begin{itemize}
  \item \textbf{Description:} The system shall have the capacity to store and process large image files.
  \item \textbf{Rationale:} This requirement is necessary to obtain information from satelite images, which can be several gigabites in size.
  \item \textbf{Fit Criterion:} The system will not crash or fail to store when given images files <50 Gb in size.
\end{itemize}
\subsection{Scalability or Extensibility Requirements}
\subsubsection*{NFR-PR9}
\begin{itemize}
  \item \textbf{Description:} The system shall be able to scale to meet the capacity specified in NFR-PR7.
  \item \textbf{Rationale:} This requirement is critical to satisfy NFR-PR7.
  \item \textbf{Fit Criterion:} See NFR-PR7.
\end{itemize}
\subsection{Longevity Requirements}
N/A

\section{Operational and Environmental Requirements}
\subsection{Expected Physical Environment}
\lips
\subsection{Wider Environment Requirements}
\lips
\subsection{Requirements for Interfacing with Adjacent Systems}
\lips
\subsection{Productization Requirements}
\lips
\subsection{Release Requirements}
\lips

\section{Maintainability and Support Requirements}
\subsection{Maintenance Requirements}
\lips
\subsection{Supportability Requirements}
\lips
\subsection{Adaptability Requirements}
\lips

\section{Security Requirements}
\subsection{Access Requirements}
\lips
\subsection{Integrity Requirements}
\lips
\subsection{Privacy Requirements}
\lips
\subsection{Audit Requirements}
\lips
\subsection{Immunity Requirements}
\lips

\section{Cultural Requirements}
\subsection{Cultural Requirements}
\lips

\section{Compliance Requirements}
\subsection{Legal Requirements}
\lips
\subsection{Standards Compliance Requirements}
\lips

\section{Open Issues}
\lips

\section{Off-the-Shelf Solutions}
\subsection{Ready-Made Products}
\lips
\subsection{Reusable Components}
\lips
\subsection{Products That Can Be Copied}
\lips

\section{New Problems}
\subsection{Effects on the Current Environment}
\lips
\subsection{Effects on the Installed Systems}
\lips
\subsection{Potential User Problems}
\lips
\subsection{Limitations in the Anticipated Implementation Environment That May
Inhibit the New Product}
\lips
\subsection{Follow-Up Problems}
\lips

\section{Tasks}
\subsection{Project Planning}
\lips
\subsection{Planning of the Development Phases}
\lips

\section{Migration to the New Product}
\subsection{Requirements for Migration to the New Product}
\lips
\subsection{Data That Has to be Modified or Translated for the New System}
\lips

\section{Costs}
\lips
\section{User Documentation and Training}
\subsection{User Documentation Requirements}
\lips
\subsection{Training Requirements}
\lips

\section{Waiting Room}
\lips

\section{Ideas for Solution}
\lips

\newpage{}
\section*{Appendix --- Reflection}

The information in this section will be used to evaluate the team members on the
graduate attribute of Lifelong Learning.  Please answer the following questions:

\begin{enumerate}
  \item What knowledge and skills will the team collectively need to acquire to
  successfully complete this capstone project?  Examples of possible knowledge
  to acquire include domain specific knowledge from the domain of your
  application, or software engineering knowledge, mechatronics knowledge or
  computer science knowledge.  Skills may be related to technology, or writing,
  or presentation, or team management, etc.  You should look to identify at
  least one item for each team member.
  \item For each of the knowledge areas and skills identified in the previous
  question, what are at least two approaches to acquiring the knowledge or
  mastering the skill?  Of the identified approaches, which will each team
  member pursue, and why did they make this choice?
\end{enumerate}

\end{document}
