% THIS DOCUMENT IS FOLLOWS THE VOLERE TEMPLATE BY Suzanne Robertson and James Robertson
% ONLY THE SECTION HEADINGS ARE PROVIDED
%
% Initial draft from https://github.com/Dieblich/volere
%
% Risks are removed because they are covered by the Hazard Analysis
\documentclass[12pt]{article}

\usepackage{booktabs}
\usepackage{tabularx}
\usepackage{hyperref}
\hypersetup{
    bookmarks=true,         % show bookmarks bar?
      colorlinks=true,      % false: boxed links; true: colored links
    linkcolor=red,          % color of internal links (change box color with linkbordercolor)
    citecolor=green,        % color of links to bibliography
    filecolor=magenta,      % color of file links
    urlcolor=cyan           % color of external links
}

\newcommand{\lips}{\textit{Insert your content here.}}

%% Comments

\usepackage{color}

\newif\ifcomments\commentstrue %displays comments
%\newif\ifcomments\commentsfalse %so that comments do not display

\ifcomments
\newcommand{\authornote}[3]{\textcolor{#1}{[#3 ---#2]}}
\newcommand{\todo}[1]{\textcolor{red}{[TODO: #1]}}
\else
\newcommand{\authornote}[3]{}
\newcommand{\todo}[1]{}
\fi

\newcommand{\wss}[1]{\authornote{blue}{SS}{#1}} 
\newcommand{\plt}[1]{\authornote{magenta}{TPLT}{#1}} %For explanation of the template
\newcommand{\an}[1]{\authornote{cyan}{Author}{#1}}

%% Common Parts

\newcommand{\progname}{Software Engineering} % PUT YOUR PROGRAM NAME HERE
\newcommand{\authname}{Team \#11, OKKM Insights
\\ Mathew Petronilho
\\ Oleg Glotov
\\ Kyle McMaster
\\ Kartik Chaudhari} % AUTHOR NAMES                  

\usepackage{hyperref}
    \hypersetup{colorlinks=true, linkcolor=blue, citecolor=blue, filecolor=blue,
                urlcolor=blue, unicode=false}
    \urlstyle{same}
                                


\begin{document}

\title{Software Requirements Specification for \progname: subtitle describing software} 
\author{\authname}
\date{\today}
	
\maketitle

~\newpage

\pagenumbering{roman}

\tableofcontents

~\newpage

\section*{Revision History}

\begin{tabularx}{\textwidth}{p{3cm}p{2cm}X}
\toprule {\textbf{Date}} & {\textbf{Version}} & {\textbf{Notes}}\\
\midrule
Date 1 & 1.0 & Notes\\
Date 2 & 1.1 & Notes\\
\bottomrule
\end{tabularx}

~\\

~\newpage
\section{Purpose of the Project}
\subsection{User Business}
\lips
\subsection{Goals of the Project}
\lips
\section{Stakeholders}
\subsection{Client}
\lips
\subsection{Customer}
\lips
\subsection{Other Stakeholders}
\lips
\subsection{Hands-On Users of the Project}
\lips
\subsection{Personas}
\lips
\subsection{Priorities Assigned to Users}
\lips
\subsection{User Participation}
\lips
\subsection{Maintenance Users and Service Technicians}
\lips

\section{Mandated Constraints}
\subsection{Solution Constraints}
\lips
\subsection{Implementation Environment of the Current System}
\lips
\subsection{Partner or Collaborative Applications}
\lips
\subsection{Off-the-Shelf Software}
\lips
\subsection{Anticipated Workplace Environment}
\lips
\subsection{Schedule Constraints}
\lips
\subsection{Budget Constraints}
\lips
\subsection{Enterprise Constraints}
\lips

\section{Naming Conventions and Terminology}
\subsection{Glossary of All Terms, Including Acronyms, Used by Stakeholders
involved in the Project}
\lips

\section*{5 Relevant Facts and Assumptions}

\subsection*{5.1 Relevant Facts}
\begin{itemize}[leftmargin=2cm]
    \item \textbf{Increasing Demand for Satellite Data}: There is a growing need for accurate satellite imagery across various industries, such as agriculture, disaster response, urban planning, and environmental monitoring, driven by advancements in technology and data analysis.
    \item \textbf{Advances in AI and Computer Vision}: The rapid development of AI and computer vision techniques has made it possible to analyze and extract meaningful insights from large-scale satellite imagery more efficiently than before.
    \item \textbf{High Cost of Manual Data Labeling}: Traditional data labeling is labor-intensive, costly, and requires expert knowledge in some cases, which slows down the process of dataset creation and model training.
    \item \textbf{Crowdsourcing as a Viable Solution}: Crowdsourcing has proven to be a successful method for data labeling in various applications (e.g., reCAPTCHA, image tagging), providing a scalable way to collect labeled data while offering opportunities for user engagement and compensation.
    \item \textbf{Growing Market for Geospatial Intelligence}: Governments, private enterprises, and NGOs are increasingly investing in geospatial intelligence to support decision-making processes, making high-quality, labeled datasets more valuable for predictive modeling and operational planning.
\end{itemize}

\subsection*{5.2 Business Rules}
\begin{itemize}[leftmargin=2cm]
    \item \textbf{Data Privacy and Confidentiality}: No confidential information is currently protected, but IP may need protection in the future as the project evolves.
    \item \textbf{Apache License 2.0 Usage}: The project will use the Apache License 2.0, allowing others to use and modify the software freely, with payment required if used commercially.
    \item \textbf{Stakeholder Communication Protocols}: There will be established communication protocols, including weekly team meetings, regular updates, and documentation of all key communications with stakeholders.
    \item \textbf{Workflow and Git Management}: A structured Git workflow is in place, ensuring code quality through branching strategies, approvals, and consistent naming conventions. CI/CD practices include static checks, linting, and testing before code is merged.
    \item \textbf{Payment and Compensation}: Compensation rules are designed to reward users for labeling data accurately and efficiently, with a clear system in place for managing payments for both users and clients.
\end{itemize}

\subsection*{5.3 Assumptions}
\begin{itemize}[leftmargin=2cm]
    \item \textbf{Accuracy and Reliability of Data Labeling}: It is assumed that the crowd-sourced labeling model will produce accurate datasets for training computer vision models, and there will be mechanisms to ensure data quality.
    \item \textbf{High User Engagement}: It is assumed that the system will attract and retain a sufficient number of users willing to label images in return for compensation, ensuring timely dataset creation.
    \item \textbf{Scalability of the Platform}: The web-based nature of the platform and its hosting on AWS/Azure will allow for easy scalability as user demand increases.
    \item \textbf{Effective AI Model Performance}: The assumption is that the AI models used for labeling and processing satellite images will perform well enough to offer insights within appropriate time frames, and that labeled data will be suitable for training accurate models.
    \item \textbf{Technology Stack Suitability}: It is assumed that the chosen technology stack (React, Node.js, Python, Flask, TensorFlow) will be adequate for both development and deployment of the platform, and that it will meet performance and security requirements.
\end{itemize}

\section{The Scope of the Work}
\subsection{The Current Situation}
\lips
\subsection{The Context of the Work}
\lips
\subsection{Work Partitioning}
\lips
\subsection{Specifying a Business Use Case (BUC)}
\lips

\section{Business Data Model and Data Dictionary}
\subsection{Business Data Model}
\lips
\subsection{Data Dictionary}
\lips

\section{The Scope of the Product}
\subsection{Product Boundary}
\lips
\subsection{Product Use Case Table}
\lips
\subsection{Individual Product Use Cases (PUC's)}
\lips

\section{Functional Requirements}
\subsection{Functional Requirements}
\lips

\section{Look and Feel Requirements}
\subsection{Appearance Requirements}
\lips
\subsection{Style Requirements}
\lips

\section*{11 Usability and Humanity Requirements}

\subsection*{11.1 Ease of Use Requirement}
\begin{itemize}[leftmargin=2cm]
    \item \textbf{Intuitive User Interface (UI)}: The platform will have a straightforward, user-friendly UI, allowing users to quickly understand how to label images, navigate datasets, and manage their accounts with minimal training.
    \item \textbf{Low Friction for Image Labeling}: Labelers should be able to start working on labeling tasks without the need for extensive setup or complex instructions, aiming for an onboarding process that takes under 10 minutes.
    \item \textbf{Responsive Design}: The platform should be responsive, making it usable on a wide range of devices, including laptops, desktops, and mobile phones, without compromising functionality or performance.
\end{itemize}

\subsection*{11.2 Personalization and Internationalization Requirements}
\begin{itemize}[leftmargin=2cm]
    \item \textbf{User Preferences}: Users will have the option to customize their experience, including setting preferences for notifications, task assignments, and data visualization. This will enhance their efficiency and satisfaction.
    \item \textbf{Language Support}: While the initial version may support English only, the platform should be designed to support future internationalization, allowing easy addition of multiple languages to accommodate users from different regions.
    \item \textbf{Custom Task Allocation}: Labelers can receive tasks based on their preferred categories (e.g., agriculture, disaster management), ensuring a personalized labeling experience and increased relevance to their interests.
\end{itemize}

\subsection*{11.3 Learning Requirements}
\begin{itemize}[leftmargin=2cm]
    \item \textbf{Guided Onboarding and Tutorials}: The platform will include brief, step-by-step tutorials to help new users quickly understand the labeling process and the features available.
    \item \textbf{Tooltips and Help Sections}: Throughout the platform, tooltips and easily accessible help sections will provide context-sensitive guidance to assist users without interrupting their workflow.
    \item \textbf{User Feedback for Continuous Improvement}: A mechanism will be in place to gather user feedback about the platform’s usability and learning curve. This feedback will be used to iteratively improve the system, especially in areas where users struggle.
\end{itemize}

\subsection*{11.4 Understandability and Politeness Requirements}
\begin{itemize}[leftmargin=2cm]
    \item \textbf{Clear Communication}: All instructions, notifications, and error messages will be clear, concise, and free of technical jargon to ensure that users of all skill levels can understand them.
    \item \textbf{Consistent Language and Tone}: The platform will use a friendly and professional tone across all interactions to build trust and encourage ongoing use. Terminology will be consistent throughout, avoiding confusion for new or returning users.
    \item \textbf{Respectful Notifications}: Notifications for completed tasks, payment updates, or errors will be designed to be informative without being intrusive. Users will have control over their notification settings, allowing them to manage how and when they receive updates.
\end{itemize}

\subsection*{11.5 Accessibility Requirements}
\begin{itemize}[leftmargin=2cm]
    \item \textbf{Speech Recognition Compatibility}: The platform will support speech recognition software, allowing users to interact with the system through voice commands as an alternative to manual input.
    \item \textbf{Time-Based Media Accessibility}: Any video or audio content provided on the platform will have captions, transcripts, or audio descriptions to ensure users with hearing or visual impairments can access the information.
    \item \textbf{Responsive Design for Accessibility}: The platform will adapt to assistive technology settings, such as high-contrast modes, screen magnifiers, and customizable accessibility settings, ensuring an optimal experience across all devices and user preferences.
\end{itemize}

\section{Performance Requirements}
\subsection{Speed and Latency Requirements}
\lips
\subsection{Safety-Critical Requirements}
\lips
\subsection{Precision or Accuracy Requirements}
\lips
\subsection{Robustness or Fault-Tolerance Requirements}
\lips
\subsection{Capacity Requirements}
\lips
\subsection{Scalability or Extensibility Requirements}
\lips
\subsection{Longevity Requirements}
\lips

\section{Operational and Environmental Requirements}
\subsection{Expected Physical Environment}
\lips
\subsection{Wider Environment Requirements}
\lips
\subsection{Requirements for Interfacing with Adjacent Systems}
\lips
\subsection{Productization Requirements}
\lips
\subsection{Release Requirements}
\lips

\section{Maintainability and Support Requirements}
\subsection{Maintenance Requirements}
\lips
\subsection{Supportability Requirements}
\lips
\subsection{Adaptability Requirements}
\lips

\section{Security Requirements}
\subsection{Access Requirements}
\lips
\subsection{Integrity Requirements}
\lips
\subsection{Privacy Requirements}
\lips
\subsection{Audit Requirements}
\lips
\subsection{Immunity Requirements}
\lips

\section{Cultural Requirements}
\subsection{Cultural Requirements}
\lips

\section{Compliance Requirements}
\subsection{Legal Requirements}
\lips
\subsection{Standards Compliance Requirements}
\lips

\section{Open Issues}
\lips

\section{Off-the-Shelf Solutions}
\subsection{Ready-Made Products}
\lips
\subsection{Reusable Components}
\lips
\subsection{Products That Can Be Copied}
\lips

\section{New Problems}
\subsection{Effects on the Current Environment}
\lips
\subsection{Effects on the Installed Systems}
\lips
\subsection{Potential User Problems}
\lips
\subsection{Limitations in the Anticipated Implementation Environment That May
Inhibit the New Product}
\lips
\subsection{Follow-Up Problems}
\lips

\section*{21 Tasks}

\subsection*{21.1 Project Planning}
\begin{itemize}[leftmargin=2cm]
    \item \textbf{Define Clear Milestones}: Establish milestones based on key project phases, such as requirements gathering, hazard analysis, proof of concept, backend development, and final project demonstration.
    \item \textbf{Task Assignment and Role Allocation}: Assign tasks to team members based on their strengths and expertise, with clear ownership over specific deliverables (e.g., backend development, frontend development, documentation, testing).
    \item \textbf{Use of Project Management Tools}: Utilize tools like Kanban boards in GitHub Projects to track progress, organize tasks into 'To Do,' 'In Progress,' and 'Completed' categories, and set internal deadlines to meet project milestones effectively.
    \item \textbf{Risk Management and Mitigation}: Identify potential project risks (e.g., data quality issues, development delays) and create mitigation strategies, including regular check-ins, backup plans, and prioritization of critical tasks.
\end{itemize}

\subsection*{21.2 Planning of the Development Phases}
\begin{itemize}[leftmargin=2cm]
    \item \textbf{Requirements Gathering (24-Sep-2024 to 08-Oct-2024)}: Engage in detailed discussions with stakeholders to gather comprehensive requirements for both frontend and backend components, ensuring all technical and business needs are documented.
    \item \textbf{Hazard Analysis (10-Oct-2024 to 23-Oct-2024)}: Identify risks related to data handling, server security, and other critical project components. Develop a hazard analysis document to mitigate potential threats, especially in handling satellite imagery.
    \item \textbf{Verification \& Validation (V\&V) Planning (24-Oct-2024 to 01-Nov-2024)}: Develop a detailed V\&V plan, including test plans and quality assurance measures. This will include setting up testing environments and criteria for ensuring code correctness and reliability.
    \item \textbf{Proof of Concept (01-Nov-2024 to 22-Nov-2024)}: Develop basic functionality for both backend and frontend, demonstrating image processing, API development, and the labeling workflow. Gather feedback to identify areas of improvement.
    \item \textbf{Backend Development (23-Nov-2024 to 05-Jan-2025)}: Focus on developing API endpoints, integrating satellite image data, and ensuring secure and efficient handling of large datasets. Aim to have a functional backend by the end of this phase.
    \item \textbf{Frontend Development (23-Nov-2024 to 05-Jan-2025)}: Design and implement UI/UX features, integrate backend APIs, and ensure an interactive user experience for both clients and labelers. Aim for a fully functional frontend that complements backend capabilities.
    \item \textbf{Design Document Revision (15-Dec-2024 to 15-Jan-2025)}: Revise the design document to reflect final system architecture, covering database schema, API structure, and AI model integration.
    \item \textbf{Mid-Project Demonstration (03-Feb-2025 to 14-Feb-2025)}: Prepare a functional prototype showcasing the integrated backend, frontend, and APIs. Demonstrate the labeling process, data flow, and key system functionalities.
    \item \textbf{Verification \& Validation (15-Feb-2025 to 07-Mar-2025)}: Test data processing pipelines, system reliability, and user interfaces to ensure accuracy and consistency. Run automated and manual tests for different use cases.
    \item \textbf{Final Project Demonstration \& Expo Preparation (24-Mar-2025 to 01-Apr-2025)}: Demonstrate the complete system, including all functionality, datasets, and models. Fine-tune UI and prepare for a polished presentation at the Expo.
    \item \textbf{Final Documentation (01-Apr-2025 to 15-Apr-2025)}: Compile comprehensive final reports covering the entire development process, technical documentation, user manuals, and deployment guides.
\end{itemize}


\section{Costs}
\lips

\section*{23 User Documentation and Training}

\subsection*{23.1 User Documentation Requirements}
\begin{itemize}[leftmargin=2cm]
    \item \textbf{Comprehensive User Manual}: Create a detailed user manual that covers all aspects of the platform, including how to access the system, navigate the UI, label satellite images, and manage user accounts. The manual should be easy to follow with step-by-step instructions and visual aids like screenshots.
    \item \textbf{Online Help System}: Develop an integrated online help system within the platform, accessible at any point during the user's journey. This should include a searchable knowledge base, FAQs, and how-to guides for common tasks (e.g., labeling images, uploading datasets, reviewing completed tasks).
    \item \textbf{API Documentation for Developers}: If external developers or advanced users need to interact with the backend, provide API documentation using a standardized format like OpenAPI. This documentation should detail endpoints, parameters, authentication, and example requests/responses.
    \item \textbf{Release Notes and Updates}: Ensure regular release notes are published for all major updates and feature releases. Include information about new features, bug fixes, and any changes to existing functionality.
    \item \textbf{Quick Start Guide}: Provide a condensed "Quick Start" guide that offers a brief overview of how to get started on the platform, aimed at getting users up and running quickly without going into extensive detail.
\end{itemize}

\subsection*{23.2 Training Requirement}
\begin{itemize}[leftmargin=2cm]
    \item \textbf{Interactive Onboarding Tutorial}: Implement an interactive onboarding process for new users, guiding them through the core functionalities (such as labeling images, setting preferences, and checking progress) with in-app tooltips and walk-through steps.
    \item \textbf{Video Tutorials and Webinars}: Create short video tutorials demonstrating different aspects of the platform (e.g., "How to label an image," "How to use the dashboard"). These videos should be easy to follow, with captions and explanations for both beginner and advanced users. Periodic live webinars for Q\&A and advanced training can also be organized.
    \item \textbf{Training for Specific User Groups}: Offer specialized training sessions for different user groups. For instance, labelers can have sessions on best practices for labeling, while clients can learn how to upload datasets and interpret results.
    \item \textbf{Self-Assessment and Practice Environment}: Provide a practice environment or sandbox within the platform where new users can try labeling without affecting the live dataset. Include self-assessment quizzes and exercises to help users gain confidence in using the platform.
\end{itemize}

\section{Waiting Room}
\lips

\section*{25 Ideas for Solution}

\begin{itemize}[leftmargin=2cm]
    \item \textbf{AI-Driven Labeling System}: Implement a machine learning model for preliminary image analysis to assist human labelers. This AI system can automatically pre-label sections of satellite images, allowing human labelers to validate or adjust these labels, significantly speeding up the process.
    
    \item \textbf{Gamification for Crowdsourced Labeling}: Introduce a gamification element to motivate and engage users who are labeling data. This can include earning points, badges, or rewards for completing labeling tasks accurately and efficiently, which can improve data quality and increase participation.
    
    \item \textbf{Intelligent Task Allocation}: Utilize an algorithm that matches labeling tasks with users based on their experience, preferences, and performance history. This ensures that difficult tasks are assigned to more experienced users while simpler tasks are made available to beginners.
    
    \item \textbf{Real-Time Collaboration and Consensus Model}: Allow multiple labelers to work on the same image in parallel, using a consensus model to determine the most accurate labels. Incorporate a review mechanism to resolve discrepancies and ensure that the final labeled data meets the required quality standards.
    
    \item \textbf{Automated Quality Assurance (QA) and Feedback}: Develop a QA system that automatically checks the quality of labeled data by comparing it to known benchmarks or running cross-validation with multiple labelers. Provide labelers with instant feedback to help improve their accuracy over time.
    
    \item \textbf{Modular Web-Based Platform Architecture}: Build a modular architecture for the platform, separating the frontend, backend, and AI components. This will allow easy updates, scalability, and the ability to add new features or services without disrupting existing functionality.
    
    \item \textbf{Flexible API for Client and User Integration}: Create an API that allows clients to easily upload satellite imagery, retrieve labeled data, and interact with the system programmatically. Also, provide endpoints for labelers to claim tasks, submit work, and check their progress or earnings.
    
    \item \textbf{Visualization Dashboard for Data Insights}: Provide clients with a dynamic dashboard that visualizes key insights derived from the labeled satellite imagery. This dashboard can display analytics, progress on labeling tasks, model accuracy, and any actionable insights relevant to the client’s industry.
    
    \item \textbf{Secure Payment and Compensation System}: Implement a secure and transparent payment system that allows clients to pay for labeled data and compensates labelers based on their contributions. Use automated transaction management to ensure timely and accurate payments to all users.
    
    \item \textbf{Cross-Platform Compatibility and Accessibility}: Design the platform to be fully accessible across devices (desktop, mobile, tablets) and compliant with accessibility standards (e.g., WCAG). This ensures that a wide range of users, including those with disabilities, can participate in labeling and use the platform effectively.
\end{itemize}

\newpage{}
\section*{Appendix --- Reflection}

The information in this section will be used to evaluate the team members on the
graduate attribute of Lifelong Learning.  Please answer the following questions:

\begin{enumerate}
  \item What knowledge and skills will the team collectively need to acquire to
  successfully complete this capstone project?  Examples of possible knowledge
  to acquire include domain specific knowledge from the domain of your
  application, or software engineering knowledge, mechatronics knowledge or
  computer science knowledge.  Skills may be related to technology, or writing,
  or presentation, or team management, etc.  You should look to identify at
  least one item for each team member.
  \item For each of the knowledge areas and skills identified in the previous
  question, what are at least two approaches to acquiring the knowledge or
  mastering the skill?  Of the identified approaches, which will each team
  member pursue, and why did they make this choice?
\end{enumerate}

\end{document}
