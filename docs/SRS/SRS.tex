% THIS DOCUMENT IS FOLLOWS THE VOLERE TEMPLATE BY Suzanne Robertson and James Robertson
% ONLY THE SECTION HEADINGS ARE PROVIDED
%
% Initial draft from https://github.com/Dieblich/volere
%
% Risks are removed because they are covered by the Hazard Analysis
\documentclass[12pt]{article}

\usepackage{booktabs}
\usepackage{tabularx}
\usepackage{hyperref}
\hypersetup{
    bookmarks=true,         % show bookmarks bar?
      colorlinks=true,      % false: boxed links; true: colored links
    linkcolor=red,          % color of internal links (change box color with linkbordercolor)
    citecolor=green,        % color of links to bibliography
    filecolor=magenta,      % color of file links
    urlcolor=cyan           % color of external links
}

\newcommand{\lips}{\textit{Insert your content here.}}

%% Comments

\usepackage{color}

\newif\ifcomments\commentstrue %displays comments
%\newif\ifcomments\commentsfalse %so that comments do not display

\ifcomments
\newcommand{\authornote}[3]{\textcolor{#1}{[#3 ---#2]}}
\newcommand{\todo}[1]{\textcolor{red}{[TODO: #1]}}
\else
\newcommand{\authornote}[3]{}
\newcommand{\todo}[1]{}
\fi

\newcommand{\wss}[1]{\authornote{blue}{SS}{#1}} 
\newcommand{\plt}[1]{\authornote{magenta}{TPLT}{#1}} %For explanation of the template
\newcommand{\an}[1]{\authornote{cyan}{Author}{#1}}

%% Common Parts

\newcommand{\progname}{Software Engineering} % PUT YOUR PROGRAM NAME HERE
\newcommand{\authname}{Team \#11, OKKM Insights
\\ Mathew Petronilho
\\ Oleg Glotov
\\ Kyle McMaster
\\ Kartik Chaudhari} % AUTHOR NAMES                  

\usepackage{hyperref}
    \hypersetup{colorlinks=true, linkcolor=blue, citecolor=blue, filecolor=blue,
                urlcolor=blue, unicode=false}
    \urlstyle{same}
                                


\begin{document}

\title{Software Requirements Specification for \progname: subtitle describing software} 
\author{\authname}
\date{\today}
	
\maketitle

~\newpage

\pagenumbering{roman}

\tableofcontents

~\newpage

\section*{Revision History}

\begin{tabularx}{\textwidth}{p{3cm}p{2cm}X}
\toprule {\textbf{Date}} & {\textbf{Version}} & {\textbf{Notes}}\\
\midrule
Date 1 & 1.0 & Notes\\
Date 2 & 1.1 & Notes\\
\bottomrule
\end{tabularx}

~\\

~\newpage
\section{Purpose of the Project}
\subsection{User Business}
\lips
\subsection{Goals of the Project}
\lips
\section{Stakeholders}
\subsection{Client}
\lips
\subsection{Customer}
\lips
\subsection{Other Stakeholders}
\lips
\subsection{Hands-On Users of the Project}
\lips
\subsection{Personas}
\lips
\subsection{Priorities Assigned to Users}
\lips
\subsection{User Participation}
\lips
\subsection{Maintenance Users and Service Technicians}
\lips

\section{Mandated Constraints}
\subsection{Solution Constraints}
\lips
\subsection{Implementation Environment of the Current System}
\lips
\subsection{Partner or Collaborative Applications}
\lips
\subsection{Off-the-Shelf Software}
\lips
\subsection{Anticipated Workplace Environment}
\lips
\subsection{Schedule Constraints}
\lips
\subsection{Budget Constraints}
\lips
\subsection{Enterprise Constraints}
\lips

\section{Naming Conventions and Terminology}
\subsection{Glossary of All Terms, Including Acronyms, Used by Stakeholders
involved in the Project}
\lips

\section*{5 Relevant Facts and Assumptions}

\subsection*{5.1 Relevant Facts}
\begin{itemize}[leftmargin=2cm]
    \item \textbf{Market Need for Labeled Satellite Data}: There is a significant gap in the availability of high-quality labeled satellite imagery tailored for various industries like disaster response, environmental monitoring, and defense.
    \item \textbf{Use of AI for Labeling}: The project employs an AI-powered crowdsourcing model to facilitate image labeling, which will then be used to build datasets for computer vision applications.
    \item \textbf{Crowdsourcing and Compensation Model}: Users on the platform will label images and receive compensation, helping to ensure data quality while incentivizing participation.
    \item \textbf{Platform Accessibility}: The platform will be web-based, accessible through any modern browser on major operating systems (Windows, Linux, macOS), enhancing its usability and portability.
    \item \textbf{Client Engagement}: The primary users include end clients who require labeled data and models, users who label data, and stakeholders like rescue services, financial data companies, and agricultural enterprises who benefit from the data insights.
\end{itemize}

\subsection*{5.2 Business Rules}
\begin{itemize}[leftmargin=2cm]
    \item \textbf{Data Privacy and Confidentiality}: No confidential information is currently protected, but IP may need protection in the future as the project evolves.
    \item \textbf{Apache License 2.0 Usage}: The project will use the Apache License 2.0, allowing others to use and modify the software freely, with payment required if used commercially.
    \item \textbf{Stakeholder Communication Protocols}: There will be established communication protocols, including weekly team meetings, regular updates, and documentation of all key communications with stakeholders.
    \item \textbf{Workflow and Git Management}: A structured Git workflow is in place, ensuring code quality through branching strategies, approvals, and consistent naming conventions. CI/CD practices include static checks, linting, and testing before code is merged.
    \item \textbf{Payment and Compensation}: Compensation rules are designed to reward users for labeling data accurately and efficiently, with a clear system in place for managing payments for both users and clients.
\end{itemize}

\subsection*{5.3 Assumptions}
\begin{itemize}[leftmargin=2cm]
    \item \textbf{Accuracy and Reliability of Data Labeling}: It is assumed that the crowd-sourced labeling model will produce accurate datasets for training computer vision models, and there will be mechanisms to ensure data quality.
    \item \textbf{High User Engagement}: It is assumed that the system will attract and retain a sufficient number of users willing to label images in return for compensation, ensuring timely dataset creation.
    \item \textbf{Scalability of the Platform}: The web-based nature of the platform and its hosting on AWS/Azure will allow for easy scalability as user demand increases.
    \item \textbf{Effective AI Model Performance}: The assumption is that the AI models used for labeling and processing satellite images will perform well enough to offer insights within appropriate time frames, and that labeled data will be suitable for training accurate models.
    \item \textbf{Technology Stack Suitability}: It is assumed that the chosen technology stack (React, Node.js, Python, Flask, TensorFlow) will be adequate for both development and deployment of the platform, and that it will meet performance and security requirements.
\end{itemize}

\section{The Scope of the Work}
\subsection{The Current Situation}
\lips
\subsection{The Context of the Work}
\lips
\subsection{Work Partitioning}
\lips
\subsection{Specifying a Business Use Case (BUC)}
\lips

\section{Business Data Model and Data Dictionary}
\subsection{Business Data Model}
\lips
\subsection{Data Dictionary}
\lips

\section{The Scope of the Product}
\subsection{Product Boundary}
\lips
\subsection{Product Use Case Table}
\lips
\subsection{Individual Product Use Cases (PUC's)}
\lips

\section{Functional Requirements}
\subsection{Functional Requirements}
\lips

\section{Look and Feel Requirements}
\subsection{Appearance Requirements}
\lips
\subsection{Style Requirements}
\lips

\section*{11 Usability and Humanity Requirements}

\subsection*{11.1 Ease of Use Requirement}
\begin{itemize}[leftmargin=2cm]
    \item \textbf{Intuitive User Interface (UI)}: The platform will have a straightforward, user-friendly UI, allowing users to quickly understand how to label images, navigate datasets, and manage their accounts with minimal training.
    \item \textbf{Low Friction for Image Labeling}: Labelers should be able to start working on labeling tasks without the need for extensive setup or complex instructions, aiming for an onboarding process that takes under 10 minutes.
    \item \textbf{Responsive Design}: The platform should be responsive, making it usable on a wide range of devices, including laptops, desktops, and mobile phones, without compromising functionality or performance.
\end{itemize}

\subsection*{11.2 Personalization and Internationalization Requirements}
\begin{itemize}[leftmargin=2cm]
    \item \textbf{User Preferences}: Users will have the option to customize their experience, including setting preferences for notifications, task assignments, and data visualization. This will enhance their efficiency and satisfaction.
    \item \textbf{Language Support}: While the initial version may support English only, the platform should be designed to support future internationalization, allowing easy addition of multiple languages to accommodate users from different regions.
    \item \textbf{Custom Task Allocation}: Labelers can receive tasks based on their preferred categories (e.g., agriculture, disaster management), ensuring a personalized labeling experience and increased relevance to their interests.
\end{itemize}

\subsection*{11.3 Learning Requirements}
\begin{itemize}[leftmargin=2cm]
    \item \textbf{Guided Onboarding and Tutorials}: The platform will include brief, step-by-step tutorials to help new users quickly understand the labeling process and the features available.
    \item \textbf{Tooltips and Help Sections}: Throughout the platform, tooltips and easily accessible help sections will provide context-sensitive guidance to assist users without interrupting their workflow.
    \item \textbf{User Feedback for Continuous Improvement}: A mechanism will be in place to gather user feedback about the platform’s usability and learning curve. This feedback will be used to iteratively improve the system, especially in areas where users struggle.
\end{itemize}

\subsection*{11.4 Understandability and Politeness Requirements}
\begin{itemize}[leftmargin=2cm]
    \item \textbf{Clear Communication}: All instructions, notifications, and error messages will be clear, concise, and free of technical jargon to ensure that users of all skill levels can understand them.
    \item \textbf{Consistent Language and Tone}: The platform will use a friendly and professional tone across all interactions to build trust and encourage ongoing use. Terminology will be consistent throughout, avoiding confusion for new or returning users.
    \item \textbf{Respectful Notifications}: Notifications for completed tasks, payment updates, or errors will be designed to be informative without being intrusive. Users will have control over their notification settings, allowing them to manage how and when they receive updates.
\end{itemize}

\subsection*{11.5 Accessibility Requirements}
\begin{itemize}[leftmargin=2cm]
    \item \textbf{Speech Recognition Compatibility}: The platform will support speech recognition software, allowing users to interact with the system through voice commands as an alternative to manual input.
    \item \textbf{Time-Based Media Accessibility}: Any video or audio content provided on the platform will have captions, transcripts, or audio descriptions to ensure users with hearing or visual impairments can access the information.
    \item \textbf{Responsive Design for Accessibility}: The platform will adapt to assistive technology settings, such as high-contrast modes, screen magnifiers, and customizable accessibility settings, ensuring an optimal experience across all devices and user preferences.
\end{itemize}

\section{Performance Requirements}
\subsection{Speed and Latency Requirements}
\lips
\subsection{Safety-Critical Requirements}
\lips
\subsection{Precision or Accuracy Requirements}
\lips
\subsection{Robustness or Fault-Tolerance Requirements}
\lips
\subsection{Capacity Requirements}
\lips
\subsection{Scalability or Extensibility Requirements}
\lips
\subsection{Longevity Requirements}
\lips

\section{Operational and Environmental Requirements}
\subsection{Expected Physical Environment}
\lips
\subsection{Wider Environment Requirements}
\lips
\subsection{Requirements for Interfacing with Adjacent Systems}
\lips
\subsection{Productization Requirements}
\lips
\subsection{Release Requirements}
\lips

\section{Maintainability and Support Requirements}
\subsection{Maintenance Requirements}
\lips
\subsection{Supportability Requirements}
\lips
\subsection{Adaptability Requirements}
\lips

\section{Security Requirements}
\subsection{Access Requirements}
\lips
\subsection{Integrity Requirements}
\lips
\subsection{Privacy Requirements}
\lips
\subsection{Audit Requirements}
\lips
\subsection{Immunity Requirements}
\lips

\section{Cultural Requirements}
\subsection{Cultural Requirements}
\lips

\section{Compliance Requirements}
\subsection{Legal Requirements}
\lips
\subsection{Standards Compliance Requirements}
\lips

\section{Open Issues}
\lips

\section{Off-the-Shelf Solutions}
\subsection{Ready-Made Products}
\lips
\subsection{Reusable Components}
\lips
\subsection{Products That Can Be Copied}
\lips

\section{New Problems}
\subsection{Effects on the Current Environment}
\lips
\subsection{Effects on the Installed Systems}
\lips
\subsection{Potential User Problems}
\lips
\subsection{Limitations in the Anticipated Implementation Environment That May
Inhibit the New Product}
\lips
\subsection{Follow-Up Problems}
\lips

\section{Tasks}
\subsection{Project Planning}
\lips
\subsection{Planning of the Development Phases}
\lips

\section{Costs}
\lips
\section{User Documentation and Training}
\subsection{User Documentation Requirements}
\lips
\subsection{Training Requirements}
\lips

\section{Waiting Room}
\lips

\section{Ideas for Solution}
\lips

\newpage{}
\section*{Appendix --- Reflection}

The information in this section will be used to evaluate the team members on the
graduate attribute of Lifelong Learning.  Please answer the following questions:

\begin{enumerate}
  \item What knowledge and skills will the team collectively need to acquire to
  successfully complete this capstone project?  Examples of possible knowledge
  to acquire include domain specific knowledge from the domain of your
  application, or software engineering knowledge, mechatronics knowledge or
  computer science knowledge.  Skills may be related to technology, or writing,
  or presentation, or team management, etc.  You should look to identify at
  least one item for each team member.
  \item For each of the knowledge areas and skills identified in the previous
  question, what are at least two approaches to acquiring the knowledge or
  mastering the skill?  Of the identified approaches, which will each team
  member pursue, and why did they make this choice?
\end{enumerate}

\end{document}
