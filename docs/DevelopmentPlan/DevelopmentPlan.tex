\documentclass{article}

\usepackage{booktabs}
\usepackage{tabularx}
\usepackage{hyperref}
\usepackage{multirow}
\usepackage{float}

\title{Development Plan\\\progname}

\author{\authname}

\date{}

%% Comments

\usepackage{color}

\newif\ifcomments\commentstrue %displays comments
%\newif\ifcomments\commentsfalse %so that comments do not display

\ifcomments
\newcommand{\authornote}[3]{\textcolor{#1}{[#3 ---#2]}}
\newcommand{\todo}[1]{\textcolor{red}{[TODO: #1]}}
\else
\newcommand{\authornote}[3]{}
\newcommand{\todo}[1]{}
\fi

\newcommand{\wss}[1]{\authornote{blue}{SS}{#1}} 
\newcommand{\plt}[1]{\authornote{magenta}{TPLT}{#1}} %For explanation of the template
\newcommand{\an}[1]{\authornote{cyan}{Author}{#1}}

%% Common Parts

\newcommand{\progname}{Software Engineering} % PUT YOUR PROGRAM NAME HERE
\newcommand{\authname}{Team \#11, OKKM Insights
\\ Mathew Petronilho
\\ Oleg Glotov
\\ Kyle McMaster
\\ Kartik Chaudhari} % AUTHOR NAMES                  

\usepackage{hyperref}
    \hypersetup{colorlinks=true, linkcolor=blue, citecolor=blue, filecolor=blue,
                urlcolor=blue, unicode=false}
    \urlstyle{same}
                                


\begin{document}

\maketitle

\begin{table}[hp]
\caption{Revision History} \label{TblRevisionHistory}
\begin{tabularx}{\textwidth}{llX}
\toprule
\textbf{Date} & \textbf{Developer(s)} & \textbf{Change}\\
\midrule
9/18/2024 & Mathew Petronilho & Added Member Roles and Coding Standards\\
9/23/2024 & Oleg Glotov & Added technical sections\\
Date2 & Name(s) & Description of changes\\
... & ... & ...\\
\bottomrule
\end{tabularx}
\end{table}

\newpage{}

\wss{Put your introductory blurb here.  Often the blurb is a brief roadmap of
what is contained in the report.}

\wss{Additional information on the development plan can be found in the
\href{https://gitlab.cas.mcmaster.ca/courses/capstone/-/blob/main/Lectures/L02b_POCAndDevPlan/POCAndDevPlan.pdf?ref_type=heads}
{lecture slides}.}

1.	Air Rescue Services: These organizations rely on satellite imagery to assist in critical missions like search and rescue. High-quality datasets can enhance their ability to assess disaster zones or monitor large areas efficiently, making response times quicker and saving lives. Satellite images tailored to disaster management, including flood zones or forest fires, are of particular importance to them. \\
\\

2.	Alternative Financial Data Companies: These companies use satellite data to analyze economic activities and trends. For example, satellite imagery of crop growth can be used to predict agricultural yields, or images of traffic patterns near malls can provide insights into retail performance. High-quality datasets enable these companies to develop more accurate financial models and market predictions. \\
\\

3.	Farmers and Agricultural Enterprises: Farmers benefit from satellite imagery for precision farming, monitoring crop health, soil conditions, and weather patterns. Access to customized datasets allows them to optimize planting schedules, monitor water usage, and make informed decisions about fertilizer application, improving yield and reducing costs. \\
\\

4.	Users: These are individuals or entities responsible for labeling the data on the platform. In return for their efforts, they receive compensation. Their primary role is to ensure that the datasets are correctly annotated according to specified requirements, which forms the basis of the models developed. Their work directly impacts the quality and usability of the final product. \\
\\

5.	End Clients/Customers: These stakeholders include governments, NGOs, private companies, and environmental organizations that pay for access to the labeled datasets and models. They rely on these datasets to make informed decisions in areas like environmental monitoring, urban planning, or defense-related tasks. Their satisfaction depends on the accuracy and reliability of both the data and the models provided. \\
\\


Other Stakeholders: Beyond the primary stakeholders, other key groups that benefit from high-quality satellite imagery datasets include defense agencies, which rely on tailored data for surveillance, intelligence, and threat detection to enhance national security. Environmental agencies use satellite data to monitor ecosystems, track deforestation, and respond to climate change. Similarly, urban planners leverage this data to manage land use, plan infrastructure development, and promote sustainable growth in cities. Additionally, disaster relief organizations depend on satellite imagery to assess damage in real-time and prioritize aid during crisis situations, making these datasets crucial for effective disaster response. Another important group includes the image labeling teams, who manually classify and annotate satellite images. Their work is crucial for building accurate datasets, and they benefit from improved tools and clearer guidelines to make the labeling process more efficient.
\end{document}



\section{Confidential Information?}

There is no confidential information to protect.

\section{IP to Protect}

Currently we chose not to protect our IP as we are aiming for a commercial product. Down the line we most likely will have to opt in for IP protection.

\section{Copyright License}

We opted to go with the Apache License 2.0. It allows anyone to use, modify, and distribute the software for free as long as they include the license and copyright notice. However, we are entitled to payments if it is used commercially.

https://github.com/OKKM-insights/OKKM.insights/blob/main/LICENSE

\section{Team Meeting Plan}
The team will meet weekly on Mondays from 11:30 until 13:20, in a study room booked by Kyle McMaster.
The schedule to meet with our supervisor will be determined when our supervisor is confirmed.\\\\
Weekly meetings will be chaired by Kyle, who will prepare an agenda to be sent in advance of team meetings. Team members can add additional agenda items as
needed. Team members are also able to request additional meetings if necessary. In this case, it is their responsibility to chair the meeting, 
organize a meeting location, and provide an agenda. Following all meetings, the meeting chair will prepare a list of action items.

\section{Team Communication Plan}

\wss{Issues on GitHub should be part of your communication plan.}

\section{Team Member Roles}
\begin{table}[H]
  \centering
  \begin{tabular}{|p{3cm}|p{6cm}|p{3cm}|}
    \hline
    \textbf{Member Name} & \multicolumn{2}{c|}{\textbf{Roles}} \\
    \hline
    Mathew Petronilho  & Document Manager, Front-End Development Expert & \multirow{4}{*}{\parbox{3cm}{Developer, Tester, PR Reviewer, Issue Creator, Meeting Participant, and Note Taker.}}\\
    \cline{1-2} 
    Oleg Glotov & Team Lead, Project Manager & \\
    \cline{1-2} 
    Kyle McMaster & Meeting Chair, Back-End Development \& AI Expert  & \\
    \cline{1-2}
    Kartik Chaudhari & Customer Relations Manager, Git Expert & \\
    \hline
  \end{tabular}
  \caption{Team Roles}
\end{table}

\begin{itemize}
	\item \textbf{All Team Members:} Every team member is responsible for developing code and creating tests for the code. Everyone is also responsible for reviewing open 
  pull requests and providing feedback if necessary. Additionally, all members will be tasked with creating issues using the appropriate templates, tracking issue status, 
  and updating issues assigned to them with relevant information. Each member is expected to attend meetings punctually and contribute ideas to discussions, while 
  maintaining respectful and concise communication. The role of meeting note taker will rotate among members in each meeting. The note taker should keep track of meeting attendance,
   issues discussed, decisions made, and action items. These notes should be well-maintained and easily accessible to all team members.
    \\ If we encounter challenges, we may consider switching roles to maintain progress and improve team performance. More specific roles can be assigned as the project evolves 
    and implementation details become clearer.
    \item \textbf{Mathew Petronilho:}
    Responsible for ensuring that all documents are formatted consistently, that all necessary components are included, and that there are no grammatical or spelling errors. 
    Also responsible for assisting team members with the front-end and taking the lead in implementing this component of the project.

    \item \textbf{Oleg Glotov:} Responsible for liaising with the supervisor, teaching assistant, and professor. 
    Coordinates project tasks among team members, organizes meetings, ensures equitable distribution of work, and monitors deadlines to ensure they are met.
    
    \item \textbf{Kyle McMaster:} Responsible for creating meeting agendas, guiding discussions, managing meeting time, and resolving conflicts. 
    Also responsible for assisting team members with the application's back-end logic, deploying services, and contributing expertise in machine learning and artificial intelligence 
    to integrate advanced data processing, and intelligent system functionalities.
    
    \item \textbf{Kartik Chaudhari:} Responsible for contacting potential customers and managing customer relationships. 
    Oversees the GitHub repository by organizing files and ensuring it is updated to reflect project progress. Provides guidance to team members on resolving issues related to Git.
    
\end{itemize}
\wss{You should identify the types of roles you anticipate, like notetaker,
leader, meeting chair, reviewer.  Assigning specific people to those roles is
not necessary at this stage.  In a student team the role of the individuals will
likely change throughout the year.}

\section{Workflow Plan}

\begin{itemize}
	\item How will you be using git, including branches, pull request, etc.?
	\item How will you be managing issues, including template issues, issue
	classification, etc.?
  \item Use of CI/CD
\end{itemize}

\subsection{Git Workflow}

\subsection{Issue Management}

\subsection{Usage of CI/CD}

\subsubsection{Tex Files}
To ensure the PDFs found in our repository are consistent with the tex files they are generated from, we have developed a CI workflow in GitHub actions.
 This workflow detects when a push has been made to the \textit{docs} folder in the \textit{main} branch. When this happens, the workflow automatically regenerates
 the relevant documents and pushes the new PDFs to the repository. This will ensure that our PDFs are always the most recent version.

\subsubsection{Linting \& Static Checks}
As discussed in sections 10 and 11, we expect we will use Python to develop the backend components of our software. To improve the clarity of our code, especially
when collaborating, we have decided to use type hinting. To enfore this requirement and reap further benefits of type hinting, we will use a static
type checker as part of our CI/CD workflow. This is possible with a package called 
 \href{https://mypy-lang.org/}{mypy}.\\\\

 For both the front and back ends of the project, we have selected coding standards as described in section 11. To ensure these standards are met, we will include
 a linting step before pull requests can be merged into the main branch.

\subsubsection{Testing}
We will use a suite of tests to ensure our code base continues to satisfy our requirements before integrating changes into the code base. These tests
will be run when a new pull request is created, before a code review is conducted.

\subsubsection{Deployment}
Depending on how we decided to host our server, we will also investigate if it would be beneficial to develop a continuous deployment workflow when changes are 
pushed to the main branch. This will be determined in the future when our requirements are more clear.
\section{Project Decomposition and Scheduling}

% \begin{itemize}
%   \item How will you be using GitHub projects?
%   \item Include a link to your GitHub project
% \end{itemize}

We will use a Kanban board to track tasks for each deliverable. First, work to be completed will be identified and added to the project backlog.
Then, team members will be assigned tasks. Once assigned, they will be moved to the `Assigned' section of the board. When actively being worked on, they will
move to the `inflight' section. After completion, and when a pull request is made, they will be moved to the `for review' board. Finally, when the review is
complete, they are moved to the archive. This organization will allow us to track how different components of the project are progressing, and identify
if a team member is overwhelmed before it becomes a catastrophic issue.

The project can be found at \\\url{https://github.com/orgs/OKKM-insights/projects/1/settings}

% \wss{How will the project be scheduled?  This is the big picture schedule, not
% details. You will need to reproduce information that is in the course outline
% for deadlines.}

\section{Proof of Concept Demonstration Plan}

What is the main risk, or risks, for the success of your project?  What will you
demonstrate during your proof of concept demonstration to convince yourself that
you will be able to overcome this risk?\\

Our project consists of two main components: the computer vision algorithm and the labeling tool. We intend to present both in our upcoming demo to illustrate the complete pipeline—from data labeling to model training and application on new data.\\

Computer vision demo:\\

We will simulate a specific business case that involves searching for a particular object within an image, using the "Where's Waldo?" book series as our dataset. In these books, readers are challenged to find the character Waldo, who is cleverly hidden among numerous other characters and objects. This scenario mirrors real-world challenges in satellite imagery analysis, where identifying specific targets within complex visuals is essential.\\

Demo Objectives:

\begin{itemize}
    \item Model Training: Train a computer vision model using the labeled data to recognize and locate Waldo.
    \item Out-of-Sample Testing: Apply the trained model to new, unlabeled images to assess its ability to find Waldo without prior hints.
\end{itemize}

This demonstration will showcase our ability to create a functional model capable of detecting specific objects in complex images.\\

Web application demo:\\

Our web application serves as the primary labeling tool within our platform. We aim to demonstrate a basic yet functional implementation that walks through all the steps involved in the labeling process.\\

Demo objectives (Frontend):

\begin{itemize}
    \item Client Upload Page: A dedicated interface where clients can upload a labeling "Job," consisting of images requiring annotation.
    \item Job Selection Screen: A portal where users (labelers) can view available jobs and select tasks they wish to work on.
    \item Labeling Interface: A user-friendly page where labelers can annotate images according to the job specifications.
\end{itemize}

Demo objectives (Backend):

\begin{itemize}
    \item Image Management: Accept and securely store incoming images from clients.
    \item Task Distribution: Break down images into manageable labeling tasks and assign them to users.
    \item Data Integration: Ensure the labeled data seamlessly feeds into the computer vision model for training.
\end{itemize}

By implementing these features, we aim to demonstrate how users can contribute to the labeling process and how clients can initiate jobs, demonstrating the full experience of our platform.\\

For both demos, we will utilize images from the "Where's Waldo?" book series. This choice offers several advantages:

\begin{itemize}
    \item Clarity: The task of finding Waldo is straightforward and easily understood by all audiences.
    \item Complexity: The crowded scenes provide a realistic challenge similar to locating objects in satellite imagery.
\end{itemize}

\section{Expected Technology}

% \wss{What programming language or languages do you expect to use?  What external
% libraries?  What frameworks?  What technologies.  Are there major components of
% the implementation that you expect you will implement, despite the existence of
% libraries that provide the required functionality.  For projects with machine
% learning, will you use pre-trained models, or be training your own model?  }

% \wss{The implementation decisions can, and likely will, change over the course
% of the project.  The initial documentation should be written in an abstract way;
% it should be agnostic of the implementation choices, unless the implementation
% choices are project constraints.  However, recording our initial thoughts on
% implementation helps understand the challenge level and feasibility of a
% project.  It may also help with early identification of areas where project
% members will need to augment their training.}

Topics to discuss include the following:

\begin{itemize}
\item Specific programming language
\item Specific libraries
\item Pre-trained models
\item Specific linter tool (if appropriate)
\item Specific unit testing framework
\item Investigation of code coverage measuring tools
\item Specific plans for Continuous Integration (CI), or an explanation that CI
  is not being done
\item Specific performance measuring tools (like Valgrind), if
  appropriate
\item Tools you will likely be using?
\end{itemize}

% \wss{git, GitHub and GitHub projects should be part of your technology.}

\section{Coding Standard}
Our back-end code, written in Python, will adhere to PEP 8 for code formatting and PEP 484 for type annotations. 
For our front-end code, written in JavaScript, we will follow the JavaScript Standard Style.
% \wss{What coding standard will you adopt?}

\newpage{}

\section*{Appendix --- Reflection}

% \wss{Not required for CAS 741}

% The purpose of reflection questions is to give you a chance to assess your own
learning and that of your group as a whole, and to find ways to improve in the
future. Reflection is an important part of the learning process.  Reflection is
also an essential component of a successful software development process.  

Reflections are most interesting and useful when they're honest, even if the
stories they tell are imperfect. You will be marked based on your depth of
thought and analysis, and not based on the content of the reflections
themselves. Thus, for full marks we encourage you to answer openly and honestly
and to avoid simply writing ``what you think the evaluator wants to hear.''

Please answer the following questions.  Some questions can be answered on the
team level, but where appropriate, each team member should write their own
response:


\begin{enumerate}
    \item Why is it important to create a development plan prior to starting the
    project?\\\\
    Starting with a development plan was useful for our team for two reasons. First, it helped us develop a clearer picture of what our project will look
    like. In order to make development plan decisions, we had to discuss the actual project itself. This gave us a chance to clear up misconceptions
    we were having, and discuss what our overall mission is. More importantly though, working on this document (and problem statement/goals)
    was a dry run for future deliverables. As we worked on this doc, we were able to put our plans of how we thought we would work together to the test. We found 
    that our git workflow needed a couple of refinements and clarifications. For example, we needed to create more descriptive pull request titles and descriptions
    so that reviewers know what they are looking to review before approving. Otherwise, it was very hard to get feedback on the \textit{Team Charter} section if the 
    pull review was called `8-expectations'.
  
    \item In your opinion, what are the advantages and disadvantages of using
    CI/CD?\\\\
    We are already feeling the benefits of CI/CD in our workflows. Kyle built a tex workflow to compile our latex documents every time there is a push to the
    main branch. This makes sure that the .tex file and its associated PDF never become out of sync. In the past, we have all experienced challenges working on
    text files in git, as it is very easy to upload a code change and much harder to upload a pdf (usually requires fixing a merge conflict). This means that
    we probably just would do one big compile and upload when the document is `finished'. The problem with that becomes when you are creating living documents
    and it never really gets finished and the PDF then becomes hard to trust as a source of truth.

    This workflow also exposed us to a disadvantage of CI/CD. It is very easy to spend a lot of time on it. Kyle spent close to 3 hours just getting the git workflow to work
    and that is not even a very complicated deployment. It is exciting to have such a hands on and interactive part of the code base, where changes are easy
    to see in real time. This makes it risky that we dedicate a lot of time to CI/CD, when it's not really necessary. We also intend to use CI/CD for regression testing.
    We have to be careful that we don't let the lovely green check mark showing that a PR passes all the tests, replace thorough code review. When people are pressed for time, it's
    easy to be tempted to skip reading the code if the tests pass. Although we will try to make a comprehensive test suite, there is always value to reading the code too.

    \item What disagreements did your group have in this deliverable, if any,
    and how did you resolve them?\\\\
    We only really had one disagreement for this deliverable and it had to do with our collaboration style on documentation. Some members of the team would prefer to 
    first write sections in a shared google doc, and then upload all of the changes at once when complete and reviewed. This would make it easier to contribute to 
    other sections, not bog us down with pull requests, and allow others to read the document in its entirety. The other members prefer using git issues and pull requests
    to improve the traceability of contributions to the project. We have had a few discussions regarding this and in general they have been productive. We conduct
    brainstorming and task allocation in a shared google doc, so that everyone has access to the task delineation. Before discussing the issue further as a group,
    we decided to confirm with Dr. Smith whether the first option is even permissible. Following this response, we will meet again to finalize our approach. For now, 
    we are using the git issues and pull requests to ensure traceability, as a small sacrifice to the ease of collaboration.
\end{enumerate}



\newpage{}

\section*{Appendix --- Team Charter}

% \wss{borrows from
% \href{https://engineering.up.edu/industry_partnerships/files/team-charter.pdf}
% {University of Portland Team Charter}}

\subsection*{External Goals}

As a team, we all see this project as an opportunity to demonstrate the extent of our skills and knowledge after several years of training.
As such, our main goal is to develop a project we can talk with excitement to future recruiters and for some of us, graduate school admissions committees.
We intend to work hard to make a project we are very proud of. We all expect this effort will translate into an A+ in the course, although that is somewhat
secondary. Our effort will not be limited by the expectations of the capstone course, and instead by the limits of our abilities.

\subsection*{Attendance}

\subsubsection*{Expectations}
\begin{itemize}
\item Team members are expected to attend every meeting in person. If attendance in person is not possible, they will notify the team at least 24 hours 
in advance so the team may discuss mkaing the meeting virtual or rescheduling.
\item If attendance is not possible in person or virtually, they will notify the team at least 24 hours in advance. Each team member will be permitted to have 2 absences per semester 
without penalty. Additional absences, or absences without proper warning, will have the following diciplinary action:

\begin{itemize}
  \item[] \textbf{1-2} Bring snacks to following meeting.
  \item[] \textbf{3-4} Bring snacks to following meeting \& a message will be sent to TA.
  \item[] \textbf{5+} Message is sent to course instructor. Meeting is scheduled with team to discuss absences. If meeting is not attended, student receives 2.5\% deduction in final course grade per offence.
\end{itemize}
\end{itemize}

\subsubsection*{Acceptable Excuse}

\begin{itemize}
\item Acceptable excuses are typically limited to exceptional circumstances that are urgent and could not have been foreseen. Examples of such events are medical emergencies of oneself or of family members, 
severe weather, or other risks for personal safety.
\item Unacceptable excuses can be reasonably foreseen or expected by the member of the group. Examples of unacceptable excuses are getting busy with other courses, missing the bus, or over sleeping.
\item If a team member is unable to meet a deadline due to other responsibilities, they are to alert the group with at least 7 days of notice. Each team member will
be permitted to request 1 reduction of work per semester, provided other team members are able to take over the task and 7 days of notice is provided.
\item Excuses are to be judged by group consensus on a case by case basis. If an excuse for absence is deemed unacceptable, penalties described above will be applied.
\item Missing a deadline with an unacceptable excuse will be met with an immediate penalty of a message sent to the TA describing the situation for the first offence. 
All other missed deadlines will result in a message sent to the instructor and a 2.5\% penalty on course grade.
\end{itemize}
% \wss{What constitutes an acceptable excuse for missing a meeting or a deadline?
% What types of excuses will not be considered acceptable?}

\subsubsection*{In Case of Emergency}
\begin{itemize}
  \item In the event of an emergency, team members must contact group as soon as reasonably possible if a meeting or deadline might be missed. Team members should avoid
  completing work at the last minute to lessen the effect of emergencies on the group.
  \end{itemize}
% \wss{What process will team members follow if they have an emergency and cannot
% attend a team meeting or complete their individual work promised for a team
% deliverable?}

\subsection*{Accountability and Teamwork}

\subsubsection*{Quality} 
Based on our expectations for the project, the team is required to meet a very high standard of quality.
Each addition to the project should be crafted with high attention to detail, and reviewed thoroughly by the creator.
We will expect that every pull request is adding code or text that is ready to be submitted to the best of our abilities.
To help each other ensure we meet this standard, we will require every addition to be reviewed by at least one other team member.
This reviewer will approach the review as if they are the last line of defense for the quality of our codebase.\\\\
Every team member will complete expected action items and review the provided agenda before each meeting. They will provide additional agenda items when necessary.
% \wss{What are your team's expectations regarding the quality
% of team members' preparation for team meetings and the quality of the
% deliverables that members bring to the team?}

\subsubsection*{Attitude}

\begin{itemize}
\item Team members will treat other with respect at all times
\item Team members will be friendly and collaborative
\item Team members will remember that we are all working towards a common goal, to develop a high quality product
\item Team members will not act hostily towards eachother. Offenders to this will face diciplinary action as outlined in the absence penalty structure
\item Team members will openly communicate
\item Team members will be open to team member ideas. If they do not agree, they are welcome to provide reasons as to why not. It is not permitted to decline an idea because it is `not the way we do things'
\item Team members will try to work together to make work load equitable
\item Team members will try to accommodate external responsibilities, but understand it might not always be possible around large deadlines
\item Team members will answer questions and provide additional information to team members who are interested in developing skills others are proficient in
\item \textbf{We will be kind}
\end{itemize}
% \wss{What are your team's expectations regarding team members' ideas,
% interactions with the team, cooperation, attitudes, and anything else regarding
% team member contributions?  Do you want to introduce a code of conduct?  Do you
% want a conflict resolution plan?  Can adopt existing codes of conduct.}

\subsubsection*{Stay on Track}

The team will work well in advance to assign work to team members. The team will schedule time in weekly meeting to have a `stand up' meeting, where they
discuss progress and blockers. This will ensure there are no surprises when a deadline is approaching. We will also define earlier internal deadlines to provide enough
time for someone to cover in the unlikely event that a team member is not able to meet a deadline.\\

We will track team member progress by the amount of deliverables they complete in each milestone. Typlically, one deliverable will correspond to one or two 
GitHub issues. We believe this metric will be more useful than lines of code or commits, both of which can be manipulated and do not always capture the 
amount of value contributed. If planned correctly, each issue should contain roughly the same amount of effort to complete (although this won't always be the case).\\

We will also track attendance at meetings and lecture. Attendance at meetings is very important and penalties for absence are discussed above. Attending lecture is
also important to the team, but less so. As long as we are able to find one person to attend, we will not worry much about lecture attendance. We will 
track lecture attendance so if we need to force someone to attend, we can choose the person with worst attendance.

% \wss{What methods will be used to keep the team on track? How will your team
% ensure that members contribute as expected to the team and that the team
% performs as expected? How will your team reward members who do well and manage
% members whose performance is below expectations?  What are the consequences for
% someone not contributing their fair share?}

% \wss{You may wish to use the project management metrics collected for the TA and
% instructor for this.}

% \wss{You can set target metrics for attendance, commits, etc.  What are the
% consequences if someone doesn't hit their targets?  Do they need to bring the
% coffee to the next team meeting?  Does the team need to make an appointment with
% their TA, or the instructor?  Are there incentives for reaching targets early?}

\subsubsection*{Team Building}

We already have a strong start with building rapport within the team. 
Everyone is very friendly towards each other and is aligned to the mission. We tend to sit together in lecture and are active in our group chats. 
After a large deliverable, the team has agreed to celebrate at a location of their choosing. Kyle might even host a party if there is interest within the team.

% \wss{How will you build team cohesion (fun time, group rituals, etc.)? }

\subsubsection*{Decision Making} 

In general, we will strive for team consensus. If it is not possible, opposing parties will prepare arguments for their proposal.
Other members of the team can then ask further questions, or announce their preference. To make a change to the current path, a majority is needed. Ties will
result in the status quo. If at any time a group member has \textit{new} arguments, they may bring the issue up again. It is important that team members are
poilte and rational, but also provide a realistic indication of how deeply they feel about their arguments. If someone could go either way on an issue, and 
we are stuck in a stalemate, learning that they do not have a deep stake might be enough to help the group move forward.\\\\
Regardless of how passionate team members are about certain issues, they will follow our code of conduct and continue to be kind.

% \wss{How will you make decisions in your group? Consensus?  Vote? How will you
% handle disagreements? }

\end{document}
