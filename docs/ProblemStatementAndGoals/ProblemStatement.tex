\documentclass{article}

\usepackage{tabularx}
\usepackage{booktabs}

\title{Problem Statement and Goals\\\progname}

\author{\authname}

\date{}

%% Comments

\usepackage{color}

\newif\ifcomments\commentstrue %displays comments
%\newif\ifcomments\commentsfalse %so that comments do not display

\ifcomments
\newcommand{\authornote}[3]{\textcolor{#1}{[#3 ---#2]}}
\newcommand{\todo}[1]{\textcolor{red}{[TODO: #1]}}
\else
\newcommand{\authornote}[3]{}
\newcommand{\todo}[1]{}
\fi

\newcommand{\wss}[1]{\authornote{blue}{SS}{#1}} 
\newcommand{\plt}[1]{\authornote{magenta}{TPLT}{#1}} %For explanation of the template
\newcommand{\an}[1]{\authornote{cyan}{Author}{#1}}

%% Common Parts

\newcommand{\progname}{Software Engineering} % PUT YOUR PROGRAM NAME HERE
\newcommand{\authname}{Team \#11, OKKM Insights
\\ Mathew Petronilho
\\ Oleg Glotov
\\ Kyle McMaster
\\ Kartik Chaudhari} % AUTHOR NAMES                  

\usepackage{hyperref}
    \hypersetup{colorlinks=true, linkcolor=blue, citecolor=blue, filecolor=blue,
                urlcolor=blue, unicode=false}
    \urlstyle{same}
                                


\begin{document}

\maketitle

\begin{table}[hp]
\caption{Revision History} \label{TblRevisionHistory}
\begin{tabularx}{\textwidth}{llX}
\toprule
\textbf{Date} & \textbf{Developer(s)} & \textbf{Change}\\
\midrule
9/24/2024 & Oleg, Kartik, Kyle, Mathew & First Revision\\

\bottomrule
\end{tabularx}
\end{table}

\section{Problem Statement}

% \wss{You should check your problem statement with the
% \href{https://github.com/smiths/capTemplate/blob/main/docs/Checklists/ProbState-Checklist.pdf}
% {problem statement checklist}.} 

% \wss{You can change the section headings, as long as you include the required
% information.}

\subsection{Problem}
There is currently a lack of high-quality, labeled satellite imagery datasets tailored for specific use cases. Many industries require specialized data for tasks like disaster 
response, environmental monitoring, urban planning, or defense, but building these datasets manually is time-consuming, costly, inefficient and may require expert data analysis. 
This hinders the development and deployment of accurate computer vision models for critical use cases across these various industries.

Our team at OKKM Insights aims to solve this problem by creating an online platform that accelerates this process and brings transparency to satellite imagery data analysis. 
Using an AI-powered crowd-sourcing model, our platform will allow users to label commercially available satellite images, helping us to build datasets. These datasets will then 
be used to train custom computer vision models for various use cases. At its core, the platform will offer a paid service for identifying objects within satellite images, 
and in turn, distribute the earnings to the users who contribute to the labeling effort.

\newpage{}
\subsection{Inputs and Outputs}

% \wss{Characterize the problem in terms of ``high level'' inputs and outputs.  
% Use abstraction so that you can avoid details.}
Inputs:

\begin{itemize}
    \item Satellite imagery: Raw, unlabeled (satellite) images that form the foundational data for analysis.
    \item Financial resources: Monetary resources for user compensations.
    \item Human effort: Time and expertise contributed by users who label the images.
\end{itemize}

Outputs:

\begin{itemize}
    \item Labeled datasets: high-quality, specialized datasets tailored to specific industry needs.
    \item Actionable insights: Industry specific information derived from the data as part of the labeling process
    \item Trained models: Trained CV models capable of automating or simplifying image analysis tasks for different use cases.
    \item User satisfaction: Users satisfaction stemming from platform use and monetary rewards.
    \item Revenue generation: Income derived from providing paid services to clients needing specialized datasets and models.
\end{itemize}

\section{Stakeholders}


\subsection{Customer}
End Clients/Customers: These stakeholders include governments, NGOs, private companies, and environmental organizations that pay for access to the labeled datasets and models. They rely on these datasets to make informed decisions in areas like environmental monitoring, urban planning, or defense-related tasks. Their satisfaction depends on the accuracy and reliability of both the data and the models provided.


\subsubsection{Users (Data Labelers)}
Data labelers are a core group of users responsible for annotating and classifying raw satellite imagery. Their role is fundamental to the project as they provide the labeled data required to train AI models. Data labelers are essential to the project's workflow, bridging the gap between raw data and actionable insights for stakeholders. Their contributions not only shape the quality of the AI models but also impact the effectiveness of the end clients decision-making processes.


\subsection{Other Stakeholders}
Beyond the primary stakeholders, other key groups that benefit from high-quality satellite imagery datasets include:
\begin{itemize}
    \item \textbf{Defense Agencies:} Rely on tailored data for surveillance, intelligence, and threat detection to enhance national security.
    \item \textbf{Environmental Agencies:} Use satellite data to monitor ecosystems, track deforestation, and respond to climate change.
    \item \textbf{Urban Planners:} Leverage data to manage land use, plan infrastructure development, and promote sustainable growth in cities.
    \item \textbf{Disaster Relief Organizations:} Depend on satellite imagery to assess damage in real-time and prioritize aid during crisis situations, making these datasets crucial for effective disaster response.
    \item \textbf{Image Labeling Teams:} Manually classify and annotate satellite images. Their work is crucial for building accurate datasets, and they benefit from improved tools and clearer guidelines to make the labeling process more efficient.
    \item \textbf{Alternative Financial Data Companies}: These companies use satellite data to analyze economic activities and trends. For example, satellite imagery of crop growth can be used to predict agricultural yields, or images of traffic patterns near malls can provide insights into retail performance. High-quality datasets enable these companies to develop more accurate financial models and market predictions.
    \item \textbf{Farmers and Agricultural Enterprises}: Farmers benefit from satellite imagery for precision farming, monitoring crop health, soil conditions, and weather patterns. Access to customized datasets allows them to optimize planting schedules, monitor water usage, and make informed decisions about fertilizer application, improving yield and reducing costs.
\end{itemize}

\subsection{Intended Audience}

This section outlines the primary, secondary, and tertiary audiences for this document, along with the scope of knowledge assumed and its relevance for onboarding and continued project development.

\subsection{Primary Audience}
\begin{itemize}
    \item \textbf{Development Team:} This document is intended for current or future developers and engineers working on the project. It assumes a technical background in areas such as crowdsourcing platforms, satellite image processing, and AI modeling. It provides a foundation for tracking design decisions and assumptions made by the original developers, who possess expertise in data processing, backend development, and labeling systems.
\end{itemize}

\subsection{Secondary Audience}
\begin{itemize}
    \item \textbf{Project Stakeholders:} Stakeholders, including clients, NGOs, or government agencies, may refer to this document to understand technical decisions, product constraints, and the alignment with business goals.
    \item \textbf{Platform Maintenance Technicians:} The document is relevant for ensuring ongoing maintenance and troubleshooting, offering a detailed understanding of system requirements and operations.
\end{itemize}

\subsection{Tertiary Audience}
\begin{itemize}
    \item \textbf{Future Researchers or Collaborators:} Academics or collaborators interested in extending the platform’s functionality can use this document as a foundation.
    \item \textbf{QA and Testing Teams:} The functional and non-functional requirements detailed in this document can be utilized for validation and verification processes.
\end{itemize}

\subsection{Scope of Knowledge Assumed}
\begin{itemize}
    \item \textbf{For Developers:} Familiarity with web-based systems, satellite data, and crowdsourcing concepts is assumed.
    \item \textbf{For Stakeholders:} A non-technical summary ensures usability and alignment with business objectives without requiring in-depth technical knowledge.
\end{itemize}

\subsection{Relevance for Onboarding}
This document serves as a foundational resource for onboarding new team members, providing insights into the project’s purpose, requirements, and future scalability. It should be updated regularly and used during project handovers or expansions to ensure continuity and adaptability.

\subsection{Access and Use Cases}
The intended audience is encouraged to use this document as a reference during:
\begin{itemize}
    \item Development and feature additions.
    \item Maintenance and troubleshooting.
    \item Stakeholder presentations and updates.
    \item Testing and quality assurance efforts.
\end{itemize}


\subsection{Hands-On Users of the Project}
Users: These are individuals or entities responsible for labeling the data on the platform. In return for their efforts, they receive compensation. Their primary role is to ensure that the datasets are correctly annotated according to specified requirements, which forms the basis of the models developed. Their work directly impacts the quality and usability of the final product.

\newpage

\subsection{Personas}
\begin{itemize}
    \item \textbf{Independent Data Annotators}
    \begin{itemize}
        \item \textbf{Description}: These are freelancers or part-time workers with some technical skills but little formal background in satellite imagery. They are responsible for labeling the satellite images provided on the platform.
        \item \textbf{Goals}: Complete labeling tasks quickly and accurately to maximize earnings and maintain a good reputation on the platform.
        \item \textbf{Challenges}: Limited domain knowledge about satellite data, difficulty in understanding complex labeling guidelines, and maintaining speed without sacrificing accuracy.
        \item \textbf{Needs}: Clear, user-friendly labeling tools, detailed guidelines, examples for training, and a feedback loop to improve their work quality.
    \end{itemize}
    \item \textbf{Local Environmental Monitors}
    \begin{itemize}
        \item \textbf{Description}: Local organizations or small environmental NGOs that use satellite data to monitor nearby ecosystems, track deforestation, or assess environmental degradation. They rely on affordable, easy-to-access datasets.
        \item \textbf{Goals}: Obtain accurate data on environmental changes in their local area to support conservation efforts or report on environmental impacts.
        \item \textbf{Challenges}: Lack of technical expertise in interpreting satellite data, a need for affordable solutions, and a dependence on pre-labeled data for quick analysis.
        \item \textbf{Needs}: Simplified tools for analyzing labeled satellite data, datasets that are easy to interpret without deep technical knowledge, and support for regional environmental concerns.
    \end{itemize}
    \item \textbf{Urban Development Analysts (Small Municipal Teams)}
    \begin{itemize}
        \item \textbf{Description}: Small urban planning teams in municipalities, often composed of only a few individuals, who use satellite imagery to plan local infrastructure projects or monitor land usage.
        \item \textbf{Goals}: Identify areas for potential development, assess land-use changes, and ensure that urban growth aligns with city plans.
        \item \textbf{Challenges}: Limited resources for purchasing datasets, difficulties in interpreting complex satellite imagery, and a need for user-friendly tools that allow for quick decision-making.
        \item \textbf{Needs}: Affordable, accurate data, easy-to-navigate analysis tools, and region-specific datasets that provide insights into urban growth and land-use patterns.
    \end{itemize}
    \item \textbf{Agricultural Cooperatives}
    \begin{itemize}
        \item \textbf{Description}: Small-scale farming groups or agricultural cooperatives that rely on satellite data to monitor crop health, soil conditions, and optimize resource usage.
        \item \textbf{Goals}: Improve crop yields, manage water and fertilizer usage efficiently, and identify early signs of crop stress.
        \item \textbf{Challenges}: Limited technical understanding of satellite data, difficulty in accessing real-time data, and a lack of granular, localized data for their specific farming regions.
        \item \textbf{Needs}: Simple dashboards that show actionable insights (e.g., crop health indicators), labeled data that is relevant to agricultural decision-making, and tools that provide early warning signs of potential issues.
    \end{itemize}
    \item \textbf{Disaster Response Volunteers (Local Level)}
    \begin{itemize}
        \item \textbf{Description}: Volunteers and small NGOs that assist in post-disaster assessments by using satellite imagery to evaluate damage to infrastructure, land, and homes.
        \item \textbf{Goals}: Rapidly assess the extent of damage and prioritize areas that need immediate relief and rebuilding efforts.
        \item \textbf{Challenges}: Limited resources, a lack of expertise in interpreting satellite data, and a dependence on accurate, pre-labeled datasets to understand the scope of the disaster.
        \item \textbf{Needs}: Simple tools that allow for quick damage assessments, access to accurate and recent satellite images, and region-specific data to support localized relief efforts.
    \end{itemize}
    \item \textbf{Platform Maintenance Technicians}
    \begin{itemize}
        \item \textbf{Description}: These users are technical support staff responsible for maintaining the functionality of the labeling platform and ensuring the labeling tools run smoothly for annotators.
        \item \textbf{Goals}: Minimize downtime, provide timely assistance to data labelers, and ensure that the labeling interface is bug-free.
        \item \textbf{Challenges}: Managing multiple technical issues at once, keeping the platform up-to-date without disrupting users, and ensuring a smooth user experience for data labelers.
        \item \textbf{Needs}: Access to detailed logs and system diagnostics, tools for quickly identifying and resolving platform issues, and efficient communication channels to address user concerns.
    \end{itemize}
\end{itemize}

\subsection{Priorities Assigned to Users}
The importance of stakeholders can be prioritized based on their reliance on accurate and reliable datasets:
\begin{itemize}
    \item \textbf{High Priority:} End Clients/Customers.
    \item \textbf{Medium Priority:} Air Rescue Services, Alternative Financial Data Companies.
    \item \textbf{Low Priority:} Urban Planners, Defense Agencies, Environmental Agencies.
\end{itemize}

\subsection{User Participation}
Users are integral to the platform as they label data in exchange for compensation. Their active participation is vital for ensuring that the datasets are annotated accurately and in accordance with the required specifications.

\subsection{Maintenance Users and Service Technicians}
These users are responsible for the upkeep of the platform, ensuring that the labeling tools and data processing pipelines are functioning smoothly. They also assist in troubleshooting any technical issues faced by the users.

\newpage


\section{Environment}

The primary environment for this project will be a web-based application, ensuring accessibility across a wide range of devices, including laptops, desktops, and potentially mobile devices. The web app will be accessible via any modern browser, making it cross-platform compatible and usable on Windows, Linux, and macOS.

The development and deployment will prioritize portability to maximize usability across different platforms. Key points regarding the environment include:

\begin{itemize}
    \item Web-Based: The software will be a web app, meaning it will run within web browsers, such as Google Chrome, Mozilla Firefox, and Safari. This will allow users to access the application regardless of their operating system, making it highly portable.

    \item Cross-Platform Compatibility: While primarily targeted for laptops and desktops, the web app will also be responsive to work on mobile devices. The development will aim to make the app usable across all major operating systems (Windows, Linux, macOS).

    \item Hosting and Backend: The backend services and APIs will be hosted on AWS/Azure, ensuring high scalability, performance, and availability. Users will not need to install any specific software or dependencies, as all services will be accessible through the web.

    \item Development Environment: The development will be done primarily on Linux for server-side services, but the tools and frameworks used (Node.js, Flask, Python) are cross-compatible, meaning that development and testing can also occur on Windows and macOS environments.

    \item Portability: Due to its web-based nature, the application will not be tied to any specific hardware and can be accessed from any device with internet connectivity, further enhancing its portability and ease of use.
\end{itemize}
% \wss{Hardware and software environment}

\section{Goals}
\subsection{High Data Accuracy}
\textbf{Description: }The system should have high classification accuracy for objects reported in the images. \\\\
\textbf{Rationale: }The core problem this system must solve is extracting useful information from the provided images.
 One key metric to determine the utility of the information found, is the classification accuracy of objects identified in the images. If the system is 
 not able to determine what is contained in an image, it will not be useful to stakeholders.
\subsection{Ease of use}
\textbf{Description: }The system should be very easy for stakeholders to use. There should be very low friction for users to classify images and objects found
within images, with minimal training. It should also be simple for users to upload images to be analyzed.\\\\
\textbf{Rationale: }To maximize the information gained from users who are contributing to classification efforts, the system must ensure it is simple for users to 
get started with, and continue using the system. This is necessary to build a large enough user base, which will make it more likely to get insights in an acceptable 
amount of time.
\subsection{Minimizing Cost to Analyze Images}
\textbf{Description: }The system should minimize the cost for users request insights from images. This could be implemented through intelligent algorithms for task delegation. \\\\
\textbf{Rationale: }Users of the system who upload images are interested in getting an appropriate return for their investment. If the cost to analyze is too high, the platform will not
retain a sufficiently large user base of purchasers.
\subsection{Results Returned Within Appropriate Timeframe}
\textbf{Description: }The system should ensure the time it takes to obtain information from images is within a specified limit, as determined by users who upload images. \\\\
\textbf{Rationale: }Purchasers will have some time limit they require the system to process images within. To ensure timing needs are met, the system should provide realistic timelines and stick to them.
\subsection{High System Reliability and Accessibility}
\textbf{Description: }The system should be useable remotely for purchasers and labellers, and have minimal downtime. \\\\
\textbf{Rationale: }The system should allow purchasers to upload images without being physically located where the system is hosted to ensure flexibility of use. The same should also be true for labellers, as they 
should be able to perform their tasks remotely. In both cases, the system should have low down time as to not introduce additional friction into the completion of tasks.

\section{Stretch Goals}
\subsection{Automatic Data Labelling}
\textbf{Description: }The system should be able to use our extensive data set to automatically label new images. \\\\
\textbf{Rationale: }The introduction of automatic data labelling will improve the speed and reduce the cost to label images. This will allow the system to provide more value to 
purchasers, and allow the labellers to focus on harder to label images.
\subsection{Multi-Source Integration}
\textbf{Description: }The system should combine additional geo-spatial datasets, such as weather or census data, to obtain additional information from satelite imagery. \\\\
\textbf{Rationale: }Additional data sources will improve the value of data collected for purchasers. This is especially true for those interested in predicting future trends from the 
data found in their satelite images.



\section{Challenge Level and Extras}
\subsection{Challenge Level}
We anticipate this project to be general. We will face some challenges due to our limited domain knowledge of satellite imagery and the complexity of the implementation.
However, we still believe we can create a viable solution through some additional training and resources.

\subsection{Extras}
\begin{itemize}
  \item Usability Testing: Conducted by allowing users to test the application interface and provide feedback to us through a questionnaire
  \item User Guide: Create a guide demonstrating how to use the product and its various features
\end{itemize}
% \wss{State your expected challenge level (advanced, general or basic).  The
% challenge can come through the required domain knowledge, the implementation or
% something else.  Usually the greater the novelty of a project the greater its
% challenge level.  You should include your rationale for the selected level.
% Approval of the level will be part of the discussion with the instructor for
% approving the project.  The challenge level, with the approval (or request) of
% the instructor, can be modified over the course of the term.}

% \wss{Teams may wish to include extras as either potential bonus grades, or to
% make up for a less advanced challenge level.  Potential extras include usability
% testing, code walkthroughs, user documentation, formal proof, GenderMag
% personas, Design Thinking, etc.  Normally the maximum number of extras will be
% two.  Approval of the extras will be part of the discussion with the instructor
% for approving the project.  The extras, with the approval (or request) of the
% instructor, can be modified over the course of the term.}

\newpage{}

\section*{Appendix --- Reflection}

% \wss{Not required for CAS 741}

The purpose of reflection questions is to give you a chance to assess your own
learning and that of your group as a whole, and to find ways to improve in the
future. Reflection is an important part of the learning process.  Reflection is
also an essential component of a successful software development process.  

Reflections are most interesting and useful when they're honest, even if the
stories they tell are imperfect. You will be marked based on your depth of
thought and analysis, and not based on the content of the reflections
themselves. Thus, for full marks we encourage you to answer openly and honestly
and to avoid simply writing ``what you think the evaluator wants to hear.''

Please answer the following questions.  Some questions can be answered on the
team level, but where appropriate, each team member should write their own
response:


\begin{enumerate}
  \item What went well while writing this deliverable? 

  This deliverable was relatively straightforward for our team to complete. We had decided on a project early in the process and made sure that all team members were genuinely interested in the proposal before moving forward. We regularly discussed the progress of the deliverable, both in our team meetings and following our capstone classes. The workload was distributed evenly, and team members cross-checked each other's work to ensure consistency in our understanding of the project's objectives. This collaborative approach helped us stay aligned and on track throughout the process.
  \item What pain points did you experience during this deliverable, and how
  did you resolve them?

  The primary challenge was establishing the project's workflow, specifically configuring the GitHub-LaTeX pipeline to automatically compile files upon upload. We addressed this by organizing a team-wide meeting to align everyone on the process and identify best practices. Additionally, we consulted with the professor, both in-person and through a discussion post on Teams, to ensure we were following the correct approach.
  \item How did you and your team adjust the scope of your goals to ensure
  they are suitable for a Capstone project (not overly ambitious but also of
  appropriate complexity for a senior design project)?

  Our project's flexible nature allows us to adjust its complexity by adding or removing features around the core idea. While the core concept itself is not entirely new, our focus is on applying it to a new domain, which will require research to determine the best approach. The core functionality will be supplemented with more advanced features and techniques to allow our project to be classified as “advanced”. However, should we encounter time constraints or technical challenges, these additional features can be omitted without compromising the project's core functionality.
\end{enumerate}  

\end{document}
