\documentclass{article}

\usepackage{tabularx}
\usepackage{booktabs}

\title{Problem Statement and Goals\\\progname}

\author{\authname}

\date{}

%% Comments

\usepackage{color}

\newif\ifcomments\commentstrue %displays comments
%\newif\ifcomments\commentsfalse %so that comments do not display

\ifcomments
\newcommand{\authornote}[3]{\textcolor{#1}{[#3 ---#2]}}
\newcommand{\todo}[1]{\textcolor{red}{[TODO: #1]}}
\else
\newcommand{\authornote}[3]{}
\newcommand{\todo}[1]{}
\fi

\newcommand{\wss}[1]{\authornote{blue}{SS}{#1}} 
\newcommand{\plt}[1]{\authornote{magenta}{TPLT}{#1}} %For explanation of the template
\newcommand{\an}[1]{\authornote{cyan}{Author}{#1}}

%% Common Parts

\newcommand{\progname}{Software Engineering} % PUT YOUR PROGRAM NAME HERE
\newcommand{\authname}{Team \#11, OKKM Insights
\\ Mathew Petronilho
\\ Oleg Glotov
\\ Kyle McMaster
\\ Kartik Chaudhari} % AUTHOR NAMES                  

\usepackage{hyperref}
    \hypersetup{colorlinks=true, linkcolor=blue, citecolor=blue, filecolor=blue,
                urlcolor=blue, unicode=false}
    \urlstyle{same}
                                


\begin{document}

\maketitle

\begin{table}[hp]
\caption{Revision History} \label{TblRevisionHistory}
\begin{tabularx}{\textwidth}{llX}
\toprule
\textbf{Date} & \textbf{Developer(s)} & \textbf{Change}\\
\midrule
9/18/2024 & Mathew Petronilho & Added Problem, Challenge level and Extras\\
9/23/2024 & Oleg Glotov & Added inputs output and Reflection sections\\

Date2 & Name(s) & Description of changes\\
... & ... & ...\\
\bottomrule
\end{tabularx}
\end{table}

\section{Problem Statement}

\wss{You should check your problem statement with the
\href{https://github.com/smiths/capTemplate/blob/main/docs/Checklists/ProbState-Checklist.pdf}
{problem statement checklist}.} 

\wss{You can change the section headings, as long as you include the required
information.}

\subsection{Problem}
There is currently a lack of high-quality, labeled satellite imagery datasets tailored for specific use cases. Many industries require specialized data for tasks like disaster 
response, environmental monitoring, urban planning, or defense, but building these datasets manually is time-consuming, costly, inefficient and may require expert data analysis. 
This hinders the development and deployment of accurate computer vision models for critical use cases across these various industries.

Our team at OKKM Insights aims to solve this problem by creating an online platform that accelerates this process and brings transparency to satellite imagery data analysis. 
Using an AI-powered crowd-sourcing model, our platform will allow users to label commercially available satellite images, helping us to build datasets. These datasets will then 
be used to train custom computer vision models for various use cases. At its core, the platform will offer a paid service for identifying objects within satellite images, 
and in turn, distribute the earnings to the users who contribute to the labeling effort.
\subsection{Inputs and Outputs}

Inputs:

\begin{itemize}
    \item Satellite imagery: Raw, unlabeled (satellite) images that form the foundational data for analysis.
    \item Financial resources: Monetary resources for user compensations.
    \item Human effort: Time and expertise contributed by users who label the images.
\end{itemize}

Outputs:

\begin{itemize}
    \item Labeled datasets: high-quality, specialized datasets tailored to specific industry needs.
    \item Actionable insights: Industry specific information derived from the data as part of the labeling process
    \item Trained models: Trained CV models capable of automating or simplifying image analysis tasks for different use cases.
    \item Customer satisfaction: Users satisfaction stemming from platform use and monetary rewards.
    \item Revenue generation: Income derived from providing paid services to clients needing specialized datasets and models.
\end{itemize}


\subsection{Stakeholders}

\subsection{Environment}

\wss{Hardware and software environment}

\section{Goals}
\subsection{High Data Accuracy}
\textbf{Description: }The system should have high classification accuracy for objects reported in the images. \\\\
\textbf{Rationale: }The core problem this system must solve is extracting useful information from the provided images.
 One key metric to determine the utility of the information found, is the classification accuracy of objects identified in the images. If the system is 
 not able to determine what is contained in an image, it will not be useful to stakeholders.
\subsection{Ease of use}
\textbf{Description: }The system should be very easy for stakeholders to use. There should be very low friction for users to classify images and objects found
within images, with minimal training. It should also be simple for users to upload images to be analyzed.\\\\
\textbf{Rationale: }To maximize the information gained from users who are contributing to classification efforts, the system must ensure it is simple for users to 
get started with, and continue using the system. This is necessary to build a large enough user base, which will make it more likely to get insights in an acceptable 
amount of time.
\subsection{Minimizing Cost to Analyze Images}
\textbf{Description: }The system should minimize the cost for users request insights from images. This could be implemented through intelligent algorithms for task delegation. \\\\
\textbf{Rationale: }Users of the system who upload images are interested in getting an appropriate return for their investment. If the cost to analyze is too high, the platform will not
retain a sufficiently large user base of purchasers.
\subsection{Results Returned Within Appropriate Timeframe}
\textbf{Description: }The system should ensure the time it takes to obtain information from images is within a specified limit, as determined by users who upload images. \\\\
\textbf{Rationale: }Purchasers will have some time limit they require the system to process images within. To ensure timing needs are met, the system should provide realistic timelines and stick to them.
\subsection{High System Reliability and Accessibility}
\textbf{Description: }The system should be useable remotely for purchasers and labellers, and have minimal downtime. \\\\
\textbf{Rationale: }The system should allow purchasers to upload images without being physically located where the system is hosted to ensure flexibility of use. The same should also be true for labellers, as they 
should be able to perform their tasks remotely. In both cases, the system should have low down time as to not introduce additional friction into the completion of tasks.

\section{Stretch Goals}
\subsection{Automatic Data Labelling}
\textbf{Description: }The system should be able to use our extensive data set to automatically label new images. \\\\
\textbf{Rationale: }The introduction of automatic data labelling will improve the speed and reduce the cost to label images. This will allow the system to provide more value to 
purchasers, and allow the labellers to focus on harder to label images.
\subsection{Multi-Source Integration}
\textbf{Description: }The system should combine additional geo-spatial datasets, such as weather or census data, to obtain additional information from satelite imagery. \\\\
\textbf{Rationale: }Additional data sources will improve the value of data collected for purchasers. This is especially true for those interested in predicting future trends from the 
data found in their satelite images.



\section{Challenge Level and Extras}
\subsection{Challenge Level}
We anticipate this project to be advanced due to our limited domain knowledge of satellite imagery and the complexity of the implementation. 
To begin, we will be developing a web application from scratch, which poses a challenge as most team members lack experience in front-end development. 
Additionally, we need to figure out how to seamlessly and automatically acquire paid satellite images for labeling from third-party providers upon customer request. 
We also need to consider how to break down and distribute image-labeling tasks. This may involve algorithms for splitting larger images into smaller pieces for analysis,
 identifying images with relevant objects, and determining which users are best suited to label specific images. Complicating matters further, our aim is for users to be able to label the 
 same images in parallel, which will require designing systems to manage simultaneous contributions and prevent conflicts or inconsistencies. We will also need to design a consensus algorithm to ensure accurate 
 labeling, likely incorporating a user accuracy system and a statistical model—both of which will require research to understand and implement effectively. 
 Once a dataset is validated, we will face additional challenges in the realm of computer vision models. Our team has minimal experience in this area, so selecting, tweaking, 
 training, and testing the appropriate model for optimal accuracy across diverse datasets will require significant effort. Moreover, we aim to automate the training of the model 
 once a labeled dataset is complete, which will add to the complexity. The application will also need to handle secure payments from customers and distribute payments to users 
 and third parties. Since we have no experience with online monetary transactions, this will involve additional research and effort to ensure security and reliability. 
 Finally, we must integrate all these components seamlessly and deploy the system in a way that ensures efficiency and an excellent user experience.

Overall, with the complexity of the implementation and our current knowledge gaps, we believe that the extra research and level of development will make this an advanced project.

\subsection{Extras}
\begin{itemize}
  \item Usability Testing: Conducted by allowing users to test the application interface and provide feedback to us through a questionnaire
  \item Demonstration Video: Create a video demonstrating how to use the product and its various features
  \item Formal Proof: Come up with a proof of convergence for labeled images to show that they have a certain level of consistency and accuracy
\end{itemize}
\wss{State your expected challenge level (advanced, general or basic).  The
challenge can come through the required domain knowledge, the implementation or
something else.  Usually the greater the novelty of a project the greater its
challenge level.  You should include your rationale for the selected level.
Approval of the level will be part of the discussion with the instructor for
approving the project.  The challenge level, with the approval (or request) of
the instructor, can be modified over the course of the term.}

\wss{Teams may wish to include extras as either potential bonus grades, or to
make up for a less advanced challenge level.  Potential extras include usability
testing, code walkthroughs, user documentation, formal proof, GenderMag
personas, Design Thinking, etc.  Normally the maximum number of extras will be
two.  Approval of the extras will be part of the discussion with the instructor
for approving the project.  The extras, with the approval (or request) of the
instructor, can be modified over the course of the term.}

\newpage{}

\section*{Appendix --- Reflection}

\wss{Not required for CAS 741}

The purpose of reflection questions is to give you a chance to assess your own
learning and that of your group as a whole, and to find ways to improve in the
future. Reflection is an important part of the learning process.  Reflection is
also an essential component of a successful software development process.  

Reflections are most interesting and useful when they're honest, even if the
stories they tell are imperfect. You will be marked based on your depth of
thought and analysis, and not based on the content of the reflections
themselves. Thus, for full marks we encourage you to answer openly and honestly
and to avoid simply writing ``what you think the evaluator wants to hear.''

Please answer the following questions.  Some questions can be answered on the
team level, but where appropriate, each team member should write their own
response:


\begin{enumerate}
    \item What went well while writing this deliverable? 
    
    This deliverable was relatively straightforward for our team to complete. We had decided on a project early in the process and made sure that all team members were genuinely interested in the proposal before moving forward. We regularly discussed the progress of the deliverable, both in our team meetings and following our capstone classes. The workload was distributed evenly, and team members cross-checked each other's work to ensure consistency in our understanding of the project's objectives. This collaborative approach helped us stay aligned and on track throughout the process.
    \item What pain points did you experience during this deliverable, and how
    did you resolve them?

    The primary challenge was establishing the project's workflow, specifically configuring the GitHub-LaTeX pipeline to automatically compile files upon upload. We addressed this by organizing a team-wide meeting to align everyone on the process and identify best practices. Additionally, we consulted with the professor, both in-person and through a discussion post on Teams, to ensure we were following the correct approach.
    \item How did you and your team adjust the scope of your goals to ensure
    they are suitable for a Capstone project (not overly ambitious but also of
    appropriate complexity for a senior design project)?
    
    Our project's flexible nature allows us to adjust its complexity by adding or removing features around the core idea. While the core concept itself is not entirely new, our focus is on applying it to a new domain, which will require research to determine the best approach. The core functionality will be supplemented with more advanced features and techniques to allow our project to be classified as “advanced”. However, should we encounter time constraints or technical challenges, these additional features can be omitted without compromising the project's core functionality.
\end{enumerate}  

\end{document}
