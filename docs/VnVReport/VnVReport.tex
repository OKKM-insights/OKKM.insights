\documentclass[12pt, titlepage]{article}

\usepackage{graphicx}
\usepackage{float}
\usepackage{booktabs}
\usepackage{tabularx}
\usepackage{hyperref}
\hypersetup{
    colorlinks,
    citecolor=black,
    filecolor=black,
    linkcolor=red,
    urlcolor=blue
}
\usepackage[round]{natbib}

\input{../Comments}
%% Common Parts

\newcommand{\progname}{Software Engineering} % PUT YOUR PROGRAM NAME HERE
\newcommand{\authname}{Team \#11, OKKM Insights
\\ Mathew Petronilho
\\ Oleg Glotov
\\ Kyle McMaster
\\ Kartik Chaudhari} % AUTHOR NAMES                  

\usepackage{hyperref}
    \hypersetup{colorlinks=true, linkcolor=blue, citecolor=blue, filecolor=blue,
                urlcolor=blue, unicode=false}
    \urlstyle{same}
                                


\begin{document}

\title{Verification and Validation Report: \progname} 
\author{\authname}
\date{\today}
	
\maketitle

\pagenumbering{roman}

\section{Revision History}

\begin{tabularx}{\textwidth}{p{3cm}p{2cm}X}
\toprule {\bf Date} & {\bf Version} & {\bf Notes}\\
\midrule
Date 1 & 1.0 & Notes\\
Date 2 & 1.1 & Notes\\
\bottomrule
\end{tabularx}

~\newpage

\section{Symbols, Abbreviations and Acronyms}

\renewcommand{\arraystretch}{1.2}
\begin{tabular}{l l} 
  \toprule		
  \textbf{symbol} & \textbf{description}\\
  \midrule 
  T & Test\\
  \bottomrule
\end{tabular}\\

\wss{symbols, abbreviations or acronyms -- you can reference the SRS tables if needed}

\newpage

\tableofcontents

\listoftables %if appropriate

\listoffigures %if appropriate

\newpage

\pagenumbering{arabic}

This document ...

\section{Functional Requirements Evaluation}

\section{Nonfunctional Requirements Evaluation}

\subsection{Usability}
		
\subsection{Performance}

\subsection{etc.}
	
\section{Comparison to Existing Implementation}	

This section will not be appropriate for every project.

\section{Unit Testing}

\section{Changes Due to Testing}

\wss{This section should highlight how feedback from the users and from 
the supervisor (when one exists) shaped the final product.  In particular 
the feedback from the Rev 0 demo to the supervisor (or to potential users) 
should be highlighted.}

\subsection{Changes to Front-end}
Our labeling tool was largely refactored to incorporate the feedback we received from our usability testing. We also considered some of the unit testing outcomes. These changes included:
\begin{itemize}
    \item Added clearer visual feedback to all the buttons present in the labeling tool. Also made the currently selected label type more obvious to the user.
    \item Changed the contextual pop ups to include more detailed descriptions and any short cuts associated with a button.
    \item Added more details and made steps more granular in the help walk-through of the labeling tool. These additional details should help the user in further understanding what they need to do.
    \item Changed the text of the main submission buttons so that it was clear what would happen when they were pressed. For example, submit was renamed to "submit labels".
    \item Fixed a bug where the tool would get stuck if the submit button was pushed when there was no labels made.
    \item The label button now stays selected after a label is created so the user can seamlessly label multiple objects of the same class without having to reselect it every time.
    \item A visual gif will be added to show the basic process of creating a label so that it is clear the labels are to be drawn on the image.
    \item Rather than have tools spread out, they have all been condensed into an easy access toolbar.
    \item New button was added to reset zoom, contrast, brightness and image position back to its initial state.
    \item Removed white space.
    \item Removed help button when the project was complete.
\end{itemize}

\section{Automated Testing}
		
\section{Trace to Requirements}
The traceability from tests to requirements can be seen in section 4.3 of the \href{https://github.com/OKKM-insights/OKKM.insights/blob/main/docs/VnVPlan/VnVPlan.pdf}{VnV Plan}.
		
\section{Trace to Modules}
The traceability from requirements to modules can be seen in section 8 of the \href{https://github.com/OKKM-insights/OKKM.insights/blob/main/docs/Design/SoftArchitecture/MG.pdf}{Module Guide}. The tests that cover a specific requirement also cover the modules associated with that requirement.	

\section{Code Coverage Metrics}
\subsection{Front-end Coverage}
The coverage results of the front-end unit testing can be seen in Figure~\ref{fig:FE_coverage}. Perfect coverage was not achieved, but we believe our unit tests supplemented with our manual and usability tests provide sufficient coverage of the code.
\begin{figure}[H]
    \centering
    \includegraphics[width=0.5\linewidth]{FE_coverage.png}
    \caption{Front-end Unit Testing Coverage Results}
    \label{fig:FE_coverage}
\end{figure}

\bibliographystyle{plainnat}
\bibliography{../../refs/References}

\newpage{}
\section*{Appendix --- Reflection}

The information in this section will be used to evaluate the team members on the
graduate attribute of Reflection.

\input{../Reflection.tex}

\begin{enumerate}
  \item What went well while writing this deliverable? 
  \item What pain points did you experience during this deliverable, and how
    did you resolve them?
  \item Which parts of this document stemmed from speaking to your client(s) or
  a proxy (e.g. your peers)? Which ones were not, and why?
  \item In what ways was the Verification and Validation (VnV) Plan different
  from the activities that were actually conducted for VnV?  If there were
  differences, what changes required the modification in the plan?  Why did
  these changes occur?  Would you be able to anticipate these changes in future
  projects?  If there weren't any differences, how was your team able to clearly
  predict a feasible amount of effort and the right tasks needed to build the
  evidence that demonstrates the required quality?  (It is expected that most
  teams will have had to deviate from their original VnV Plan.)
\end{enumerate}

\end{document}