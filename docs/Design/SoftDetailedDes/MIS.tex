\documentclass[12pt, titlepage]{article}

\usepackage{amsmath, mathtools}

\usepackage[round]{natbib}
\usepackage{amsfonts}
\usepackage{amssymb}
\usepackage{graphicx}
\usepackage{colortbl}
\usepackage{xr}
\usepackage{hyperref}
\usepackage{longtable}
\usepackage{xfrac}
\usepackage{tabularx}
\usepackage{float}
\usepackage{siunitx}
\usepackage{booktabs}
\usepackage{multirow}
\usepackage[section]{placeins}
\usepackage{caption}
\usepackage{fullpage}
\usepackage{amsfonts}

\hypersetup{
bookmarks=true,     % show bookmarks bar?
colorlinks=true,       % false: boxed links; true: colored links
linkcolor=red,          % color of internal links (change box color with linkbordercolor)
citecolor=blue,      % color of links to bibliography
filecolor=magenta,  % color of file links
urlcolor=cyan          % color of external links
}

\usepackage{array}

\externaldocument{../../SRS/SRS}

%% Comments

\usepackage{color}

\newif\ifcomments\commentstrue %displays comments
%\newif\ifcomments\commentsfalse %so that comments do not display

\ifcomments
\newcommand{\authornote}[3]{\textcolor{#1}{[#3 ---#2]}}
\newcommand{\todo}[1]{\textcolor{red}{[TODO: #1]}}
\else
\newcommand{\authornote}[3]{}
\newcommand{\todo}[1]{}
\fi

\newcommand{\wss}[1]{\authornote{blue}{SS}{#1}} 
\newcommand{\plt}[1]{\authornote{magenta}{TPLT}{#1}} %For explanation of the template
\newcommand{\an}[1]{\authornote{cyan}{Author}{#1}}

%% Common Parts

\newcommand{\progname}{Software Engineering} % PUT YOUR PROGRAM NAME HERE
\newcommand{\authname}{Team \#11, OKKM Insights
\\ Mathew Petronilho
\\ Oleg Glotov
\\ Kyle McMaster
\\ Kartik Chaudhari} % AUTHOR NAMES                  

\usepackage{hyperref}
    \hypersetup{colorlinks=true, linkcolor=blue, citecolor=blue, filecolor=blue,
                urlcolor=blue, unicode=false}
    \urlstyle{same}
                                


\begin{document}

\title{Module Interface Specification for \progname{}}

\author{\authname}

\date{\today}

\maketitle

\pagenumbering{roman}

\section{Revision History}

\begin{tabularx}{\textwidth}{p{3cm}p{2cm}X}
\toprule {\bf Date} & {\bf Version} & {\bf Notes}\\
\midrule
Date 1 & 1.0 & Notes\\
Date 2 & 1.1 & Notes\\
\bottomrule
\end{tabularx}

~\newpage

\section{Symbols, Abbreviations and Acronyms}

See SRS Documentation at \wss{give url}

\wss{Also add any additional symbols, abbreviations or acronyms}

\newpage

\tableofcontents

\newpage

\pagenumbering{arabic}

\section{Introduction}

The following document details the Module Interface Specifications for
\wss{Fill in your project name and description}

Complementary documents include the System Requirement Specifications
and Module Guide.  The full documentation and implementation can be
found at \url{...}.  \wss{provide the url for your repo}

\section{Notation}

\wss{You should describe your notation.  You can use what is below as
  a starting point.}

The structure of the MIS for modules comes from \citet{HoffmanAndStrooper1995},
with the addition that template modules have been adapted from
\cite{GhezziEtAl2003}.  The mathematical notation comes from Chapter 3 of
\citet{HoffmanAndStrooper1995}.  For instance, the symbol := is used for a
multiple assignment statement and conditional rules follow the form $(c_1
\Rightarrow r_1 | c_2 \Rightarrow r_2 | ... | c_n \Rightarrow r_n )$.

The following table summarizes the primitive data types used by \progname. 

\begin{center}
\renewcommand{\arraystretch}{1.2}
\noindent 
\begin{tabular}{l l p{7.5cm}} 
\toprule 
\textbf{Data Type} & \textbf{Notation} & \textbf{Description}\\ 
\midrule
character & char & a single symbol or digit\\
integer & $\mathbb{Z}$ & a number without a fractional component in (-$\infty$, $\infty$) \\
natural number & $\mathbb{N}$ & a number without a fractional component in [1, $\infty$) \\
real & $\mathbb{R}$ & any number in (-$\infty$, $\infty$)\\
\bottomrule
\end{tabular} 
\end{center}

\noindent
The specification of \progname \ uses some derived data types: sequences, strings, and
tuples. Sequences are lists filled with elements of the same data type. Strings
are sequences of characters. Tuples contain a list of values, potentially of
different types. In addition, \progname \ uses functions, which
are defined by the data types of their inputs and outputs. Local functions are
described by giving their type signature followed by their specification.

\section{Module Decomposition}

The following table is taken directly from the Module Guide document for this project.

\begin{table}[h!]
\centering
\begin{tabular}{p{0.3\textwidth} p{0.6\textwidth}}
\toprule
\textbf{Level 1} & \textbf{Level 2}\\
\midrule

{Hardware-Hiding} & ~ \\
\midrule

\multirow{7}{0.3\textwidth}{Behaviour-Hiding} & Input Parameters\\
& Output Format\\
& Output Verification\\
& Temperature ODEs\\
& Energy Equations\\ 
& Control Module\\
& Specification Parameters Module\\
\midrule

\multirow{3}{0.3\textwidth}{Software Decision} & {Sequence Data Structure}\\
& ODE Solver\\
& Plotting\\
\bottomrule

\end{tabular}
\caption{Module Hierarchy}
\label{TblMH}
\end{table}

\newpage
~\newpage

\section{MIS of \wss{Module Name}} \label{Module} \wss{Use labels for
  cross-referencing}

\wss{You can reference SRS labels, such as R\ref{R_Inputs}.}

\wss{It is also possible to use \LaTeX for hypperlinks to external documents.}

\subsection{Module}

\wss{Short name for the module}

\subsection{Uses}


\subsection{Syntax}

\subsubsection{Exported Constants}

\subsubsection{Exported Access Programs}

\begin{center}
\begin{tabular}{p{2cm} p{4cm} p{4cm} p{2cm}}
\hline
\textbf{Name} & \textbf{In} & \textbf{Out} & \textbf{Exceptions} \\
\hline
\wss{accessProg} & - & - & - \\
\hline
\end{tabular}
\end{center}

\subsection{Semantics}

\subsubsection{State Variables}

\wss{Not all modules will have state variables.  State variables give the module
  a memory.}

\subsubsection{Environment Variables}

\wss{This section is not necessary for all modules.  Its purpose is to capture
  when the module has external interaction with the environment, such as for a
  device driver, screen interface, keyboard, file, etc.}

\subsubsection{Assumptions}

\wss{Try to minimize assumptions and anticipate programmer errors via
  exceptions, but for practical purposes assumptions are sometimes appropriate.}

\subsubsection{Access Routine Semantics}

\noindent \wss{accessProg}():
\begin{itemize}
\item transition: \wss{if appropriate} 
\item output: \wss{if appropriate} 
\item exception: \wss{if appropriate} 
\end{itemize}

\wss{A module without environment variables or state variables is unlikely to
  have a state transition.  In this case a state transition can only occur if
  the module is changing the state of another module.}

\wss{Modules rarely have both a transition and an output.  In most cases you
  will have one or the other.}

\subsubsection{Local Functions}

\wss{As appropriate} \wss{These functions are for the purpose of specification.
  They are not necessarily something that is going to be implemented
  explicitly.  Even if they are implemented, they are not exported; they only
  have local scope.}


  \newpage
  ~\newpage
  
  \section{MIS of Label Server}\label{LabelServer}
  
  % \wss{You can reference SRS labels, such as R\ref{R_Inputs}.}
  
  % \wss{It is also possible to use \LaTeX for hypperlinks to external documents.}
  
  \subsection{Module}
  
  Label Server
  
  \subsection{Uses}
  
  Labeling Controller \ref{lc}\\
  Label \ref{label}\\ 
  Label Database Connector \ref{label database connector}
  
  \subsection{Syntax}


  
  \subsubsection{Exported Constants}
  None
  \subsubsection{Exported Access Programs}
  
  \begin{center}
  \begin{tabular}{p{2cm} p{4cm} p{4cm} p{2cm}}
  \hline
  \textbf{Name} & \textbf{In} & \textbf{Out} & \textbf{Exceptions} \\
  \hline
  acceptLabel & Label & - & ValueError, ConnectionError \\
  \hline
  \end{tabular}
  \end{center}
  
  \subsection{Semantics}
  
  \subsubsection{State Variables}
  
  None
  
  \subsubsection{Environment Variables}
  
  LabelDatabaseConnector
  
  \subsubsection{Assumptions}
  
  Label Objects are given to the label server in JSON format. Exceptions will be thrown based on failure to match this standard.
  
  \subsubsection{Access Routine Semantics}
  
  \noindent acceptLabel(object o):
  \begin{itemize}
  \item transition: Transition occurs in LabelDatabaseConnector
  \item output: Standard HTTP response codes
  \item exception: Let L be the set of valid Labels. Throw ValueError if $\neg (o \in \text{L})$\\
  Throw ConnectionError if ConnectionError is raised by LabelDatabaseConnector
  \end{itemize}
  
  % \wss{A module without environment variables or state variables is unlikely to
  %   have a state transition.  In this case a state transition can only occur if
  %   the module is changing the state of another module.}
  
  % \wss{Modules rarely have both a transition and an output.  In most cases you
  %   will have one or the other.}
  
  \subsubsection{Local Functions}

  JSONLabeltoLabel: converts a JSON object into a Label object. 
  
  % \wss{As appropriate} \wss{These functions are for the purpose of specification.
  %   They are not necessarily something that is going to be implemented
  %   explicitly.  Even if they are implemented, they are not exported; they only
  %   have local scope.}

\newpage

% --------------------------------------------------------------------------------------
% --/\/\/\/\/\/\/\/\/\/\/\/\/\/\/\/\/\/\/\/\/\/\/\/\/\/\/\/\/\/\/\/\/\/\/\/\/\/\/\/\/\--
% --------------------------------------------------------------------------------------

\section{MIS of Label Database Connector}\label{label database connector}
  
  % \wss{You can reference SRS labels, such as R\ref{R_Inputs}.}
  
  % \wss{It is also possible to use \LaTeX for hypperlinks to external documents.}
  
  \subsection{Module}
  
  Label Database Connector

  
  \subsection{Uses}
  
  Label Database \ref{label database}\\
  Label \ref{label}\\ 

  
  \subsection{Syntax}


  
  \subsubsection{Exported Constants}
  None
  \subsubsection{Exported Access Programs}
  
  \begin{center}
  \begin{tabular}{p{2cm} p{4cm} p{4cm} p{2cm}}
  \hline
  \textbf{Name} & \textbf{In} & \textbf{Out} & \textbf{Exceptions} \\
  \hline
  pushLabel & Label & - & ValueError, ConnectionError \\
  \hline
  makeDB Connection & Label & - & ConnectionError \\
  \hline
  getLabels & String & list[Label] & ValueError, ConnectionError \\
  \hline
  \end{tabular}
  \end{center}
  
  \subsection{Semantics}
  
  \subsubsection{State Variables}
  
  None
  
  \subsubsection{Environment Variables}
  
  None
  
  \subsubsection{Assumptions}
  
  
  \subsubsection{Access Routine Semantics}
  
  \noindent pushLabel(Label l):
  \begin{itemize}
  \item transition: Transition occurs in LabelDatabase
  \item output: 
  \item exception: Let L be the set of valid Labels. Throw ValueError if $\neg (l \in \text{L})$\\
  Throw ConnectionError if ConnectionError is raised by makeDBConnection
  \end{itemize}

  \noindent makeDBConnection():
  \begin{itemize}
  \item transition: 
  \item output: 
  \item exception: Throw ConnectionError if connection is not accepted by LabelDatabase
  \end{itemize}

  \noindent getLabels(String q):
  \begin{itemize}
  \item transition: 
  \item output: list of labels satisfying the provided query
  \item exception: Let Q be the set of valid Queries. Throw ValueError if $\neg (q \in \text{Q})$\\
  Throw ConnectionError if ConnectionError is raised by makeDBConnection

  \end{itemize}
  
  % \wss{A module without environment variables or state variables is unlikely to
  %   have a state transition.  In this case a state transition can only occur if
  %   the module is changing the state of another module.}
  
  % \wss{Modules rarely have both a transition and an output.  In most cases you
  %   will have one or the other.}
  
  \subsubsection{Local Functions}

  None
  
  % \wss{As appropriate} \wss{These functions are for the purpose of specification.
  %   They are not necessarily something that is going to be implemented
  %   explicitly.  Even if they are implemented, they are not exported; they only
  %   have local scope.}

\newpage

% --------------------------------------------------------------------------------------
% --/\/\/\/\/\/\/\/\/\/\/\/\/\/\/\/\/\/\/\/\/\/\/\/\/\/\/\/\/\/\/\/\/\/\/\/\/\/\/\/\/\--
% --------------------------------------------------------------------------------------

\section{MIS of Label Database }\label{label database}
  
  % \wss{You can reference SRS labels, such as R\ref{R_Inputs}.}
  
  % \wss{It is also possible to use \LaTeX for hypperlinks to external documents.}
  
  \subsection{Module}
  
  Label Database
  
  \subsection{Uses}
  
  None
  
  \subsection{Syntax}


  
  \subsubsection{Exported Constants}
  None
  \subsubsection{Exported Access Programs}
  
  \begin{center}
  \begin{tabular}{p{2cm} p{4cm} p{4cm} p{2cm}}
  \hline
  \textbf{Name} & \textbf{In} & \textbf{Out} & \textbf{Exceptions} \\
  \hline
  pushLabel & Label & - & ValueError \\
  \hline
  makeDB Connection & Label & - &  ConnectionError\\
  \hline
  getLabels & String & list[Label] & ValueError \\
  \hline
  \end{tabular}
  \end{center}
  
  \subsection{Semantics}
  
  \subsubsection{State Variables}
  
  labels: labels stored in the database
  users: list of authenticated users
  
  \subsubsection{Environment Variables}
  
  None
  
  \subsubsection{Assumptions}
  
  
  \subsubsection{Access Routine Semantics}
  
  \noindent pushLabel(Label l):
  \begin{itemize}
  \item transition: $\text{labels} := \text{labels} \cup l $
  \item output: 
  \item exception: Let L be the set of valid Labels. Throw ValueError if $\neg (l \in \text{L})$\\
  Throw ConnectionError if $\neg (\text{requestor} \in \text{users})$
  \end{itemize}

  \noindent makeDBConnection(credentials):
  \begin{itemize}
  \item transition: if credentials are valid, $\text{users} := \text{users} \cup credentials.user $
  \item output: 
  \item exception: Throw ConnectionError if credentials are not valid
  \end{itemize}

  \noindent getLabels(String q):
  \begin{itemize}
  \item transition: 
  \item output: list of labels satisfying the provided query
  \item exception: Let Q be the set of valid Queries. Throw ValueError if $\neg (q \in \text{Q})$\\
  Throw ConnectionError if $\neg (\text{requestor} \in \text{users})$

  \end{itemize}
  
  % \wss{A module without environment variables or state variables is unlikely to
  %   have a state transition.  In this case a state transition can only occur if
  %   the module is changing the state of another module.}
  
  % \wss{Modules rarely have both a transition and an output.  In most cases you
  %   will have one or the other.}
  
  \subsubsection{Local Functions}

  None
  
  % \wss{As appropriate} \wss{These functions are for the purpose of specification.
  %   They are not necessarily something that is going to be implemented
  %   explicitly.  Even if they are implemented, they are not exported; they only
  %   have local scope.}

\newpage

% --------------------------------------------------------------------------------------
% --/\/\/\/\/\/\/\/\/\/\/\/\/\/\/\/\/\/\/\/\/\/\/\/\/\/\/\/\/\/\/\/\/\/\/\/\/\/\/\/\/\--
% --------------------------------------------------------------------------------------

\section{MIS of ImageObject Database Connector}\label{ImageObject database connector}
  
  % \wss{You can reference SRS labels, such as R\ref{R_Inputs}.}
  
  % \wss{It is also possible to use \LaTeX for hypperlinks to external documents.}
  
  \subsection{Module}
  
  ImageObject Database Connector

  
  \subsection{Uses}
  
  ImageObject Database \ref{ImageObject database}\\
  ImageObject \ref{ImageObject}\\ 

  
  \subsection{Syntax}


  
  \subsubsection{Exported Constants}
  None
  \subsubsection{Exported Access Programs}
  
  \begin{center}
  \begin{tabular}{p{2cm} p{4cm} p{4cm} p{2cm}}
  \hline
  \textbf{Name} & \textbf{In} & \textbf{Out} & \textbf{Exceptions} \\
  \hline
  pushImageObject & ImageObject & - & ValueError, ConnectionError \\
  \hline
  makeDB Connection & ImageObject & - & ConnectionError \\
  \hline
  getImageObjects & String & list[ImageObject] & ValueError, ConnectionError \\
  \hline
  \end{tabular}
  \end{center}
  
  \subsection{Semantics}
  
  \subsubsection{State Variables}
  
  None
  
  \subsubsection{Environment Variables}
  
  None
  
  \subsubsection{Assumptions}
  
  
  \subsubsection{Access Routine Semantics}
  
  \noindent pushLabel(ImageObject l):
  \begin{itemize}
  \item transition: Transition occurs in ImageObjectDatabase
  \item output: 
  \item exception: Let L be the set of valid ImageObjects. Throw ValueError if $\neg (l \in \text{L})$\\
  Throw ConnectionError if ConnectionError is raised by makeDBConnection
  \end{itemize}

  \noindent makeDBConnection():
  \begin{itemize}
  \item transition: 
  \item output: 
  \item exception: Throw ConnectionError if connection is not accepted by ImageObjectDatabase
  \end{itemize}

  \noindent getLabels(String q):
  \begin{itemize}
  \item transition: 
  \item output: list of ImageObjects satisfying the provided query
  \item exception: Let Q be the set of valid Queries. Throw ValueError if $\neg (q \in \text{Q})$\\
  Throw ConnectionError if ConnectionError is raised by makeDBConnection

  \end{itemize}
  
  % \wss{A module without environment variables or state variables is unlikely to
  %   have a state transition.  In this case a state transition can only occur if
  %   the module is changing the state of another module.}
  
  % \wss{Modules rarely have both a transition and an output.  In most cases you
  %   will have one or the other.}
  
  \subsubsection{Local Functions}

  None
  
  % \wss{As appropriate} \wss{These functions are for the purpose of specification.
  %   They are not necessarily something that is going to be implemented
  %   explicitly.  Even if they are implemented, they are not exported; they only
  %   have local scope.}

\newpage

% --------------------------------------------------------------------------------------
% --/\/\/\/\/\/\/\/\/\/\/\/\/\/\/\/\/\/\/\/\/\/\/\/\/\/\/\/\/\/\/\/\/\/\/\/\/\/\/\/\/\--
% --------------------------------------------------------------------------------------

\section{MIS of ImageObject Database }\label{ImageObject database}
  
  % \wss{You can reference SRS labels, such as R\ref{R_Inputs}.}
  
  % \wss{It is also possible to use \LaTeX for hypperlinks to external documents.}
  
  \subsection{Module}
  
  ImageObject Database
  
  \subsection{Uses}
  
  None
  
  \subsection{Syntax}


  
  \subsubsection{Exported Constants}
  None
  \subsubsection{Exported Access Programs}
  
  \begin{center}
  \begin{tabular}{p{2cm} p{4cm} p{4cm} p{2cm}}
  \hline
  \textbf{Name} & \textbf{In} & \textbf{Out} & \textbf{Exceptions} \\
  \hline
  pushImageObject & ImageObject & - & ValueError \\
  \hline
  makeDB Connection & ImageObject & - &  ConnectionError\\
  \hline
  getImageObjects & String & list[ImageObject] & ValueError \\
  \hline
  \end{tabular}
  \end{center}
  
  \subsection{Semantics}
  
  \subsubsection{State Variables}
  
  ImageObjects: ImageObjects stored in the database
  users: list of authenticated users
  
  \subsubsection{Environment Variables}
  
  None
  
  \subsubsection{Assumptions}
  
  
  \subsubsection{Access Routine Semantics}
  
  \noindent pushLabel(ImageObject l):
  \begin{itemize}
  \item transition: $\text{labels} := \text{labels} \cup l $
  \item output: 
  \item exception: Let L be the set of valid ImageObjects. Throw ValueError if $\neg (l \in \text{L})$\\
  Throw ConnectionError if $\neg (\text{requestor} \in \text{users})$
  \end{itemize}

  \noindent makeDBConnection(credentials):
  \begin{itemize}
  \item transition: if credentials are valid, $\text{users} := \text{users} \cup credentials.user $
  \item output: 
  \item exception: Throw ConnectionError if credentials are not valid
  \end{itemize}

  \noindent getLabels(String q):
  \begin{itemize}
  \item transition: 
  \item output: list of ImageObjects satisfying the provided query
  \item exception: Let Q be the set of valid Queries. Throw ValueError if $\neg (q \in \text{Q})$\\
  Throw ConnectionError if $\neg (\text{requestor} \in \text{users})$

  \end{itemize}
  
  % \wss{A module without environment variables or state variables is unlikely to
  %   have a state transition.  In this case a state transition can only occur if
  %   the module is changing the state of another module.}
  
  % \wss{Modules rarely have both a transition and an output.  In most cases you
  %   will have one or the other.}
  
  \subsubsection{Local Functions}

  None
  
  % \wss{As appropriate} \wss{These functions are for the purpose of specification.
  %   They are not necessarily something that is going to be implemented
  %   explicitly.  Even if they are implemented, they are not exported; they only
  %   have local scope.}

\newpage

% --------------------------------------------------------------------------------------
% --/\/\/\/\/\/\/\/\/\/\/\/\/\/\/\/\/\/\/\/\/\/\/\/\/\/\/\/\/\/\/\/\/\/\/\/\/\/\/\/\/\--
% --------------------------------------------------------------------------------------

\section{MIS of ImageObject Database Connector}\label{ImageObject database connector}
  
  % \wss{You can reference SRS labels, such as R\ref{R_Inputs}.}
  
  % \wss{It is also possible to use \LaTeX for hypperlinks to external documents.}
  
  \subsection{Module}
  
  ImageObject Database Connector

  
  \subsection{Uses}
  
  ImageObject Database \ref{ImageObject database}\\
  ImageObject \ref{ImageObject}\\ 

  
  \subsection{Syntax}


  
  \subsubsection{Exported Constants}
  None
  \subsubsection{Exported Access Programs}
  
  \begin{center}
  \begin{tabular}{p{2cm} p{4cm} p{4cm} p{2cm}}
  \hline
  \textbf{Name} & \textbf{In} & \textbf{Out} & \textbf{Exceptions} \\
  \hline
  pushImageObject & ImageObject & - & ValueError, ConnectionError \\
  \hline
  makeDB Connection & ImageObject & - & ConnectionError \\
  \hline
  getImageObjects & String & list[ImageObject] & ValueError, ConnectionError \\
  \hline
  \end{tabular}
  \end{center}
  
  \subsection{Semantics}
  
  \subsubsection{State Variables}
  
  None
  
  \subsubsection{Environment Variables}
  
  None
  
  \subsubsection{Assumptions}
  
  
  \subsubsection{Access Routine Semantics}
  
  \noindent pushImageObject(ImageObject o):
  \begin{itemize}
  \item transition: Transition occurs in ImageObjectDatabase
  \item output: 
  \item exception: Let O be the set of valid ImageObjects. Throw ValueError if $\neg (o \in \text{O})$\\
  Throw ConnectionError if ConnectionError is raised by makeDBConnection
  \end{itemize}

  \noindent makeDBConnection():
  \begin{itemize}
  \item transition: 
  \item output: 
  \item exception: Throw ConnectionError if connection is not accepted by ImageObjectDatabase
  \end{itemize}

  \noindent getImageObjects(String q):
  \begin{itemize}
  \item transition: 
  \item output: list of ImageObjects satisfying the provided query
  \item exception: Let Q be the set of valid Queries. Throw ValueError if $\neg (q \in \text{Q})$\\
  Throw ConnectionError if ConnectionError is raised by makeDBConnection

  \end{itemize}
  
  % \wss{A module without environment variables or state variables is unlikely to
  %   have a state transition.  In this case a state transition can only occur if
  %   the module is changing the state of another module.}
  
  % \wss{Modules rarely have both a transition and an output.  In most cases you
  %   will have one or the other.}
  
  \subsubsection{Local Functions}

  None
  
  % \wss{As appropriate} \wss{These functions are for the purpose of specification.
  %   They are not necessarily something that is going to be implemented
  %   explicitly.  Even if they are implemented, they are not exported; they only
  %   have local scope.}

\newpage

% --------------------------------------------------------------------------------------
% --/\/\/\/\/\/\/\/\/\/\/\/\/\/\/\/\/\/\/\/\/\/\/\/\/\/\/\/\/\/\/\/\/\/\/\/\/\/\/\/\/\--
% --------------------------------------------------------------------------------------

\section{MIS of ImageObject Database }\label{ImageObject database}
  
  % \wss{You can reference SRS labels, such as R\ref{R_Inputs}.}
  
  % \wss{It is also possible to use \LaTeX for hypperlinks to external documents.}
  
  \subsection{Module}
  
  ImageObject Database
  
  \subsection{Uses}
  
  None
  
  \subsection{Syntax}


  
  \subsubsection{Exported Constants}
  None
  \subsubsection{Exported Access Programs}
  
  \begin{center}
  \begin{tabular}{p{2cm} p{4cm} p{4cm} p{2cm}}
  \hline
  \textbf{Name} & \textbf{In} & \textbf{Out} & \textbf{Exceptions} \\
  \hline
  pushImageObject & ImageObject & - & ValueError \\
  \hline
  makeDB Connection & ImageObject & - &  ConnectionError\\
  \hline
  getImageObjects & String & list[ImageObject] & ValueError \\
  \hline
  \end{tabular}
  \end{center}
  
  \subsection{Semantics}
  
  \subsubsection{State Variables}
  
  ImageObjects: ImageObjects stored in the database
  users: list of authenticated users
  
  \subsubsection{Environment Variables}
  
  None
  
  \subsubsection{Assumptions}
  
  
  \subsubsection{Access Routine Semantics}
  
  \noindent pushImageObject(ImageObject o):
  \begin{itemize}
  \item transition: $\text{labels} := \text{labels} \cup l $
  \item output: 
  \item exception: Let O be the set of valid ImageObjects. Throw ValueError if $\neg (o \in \text{O})$\\
  Throw ConnectionError if $\neg (\text{requestor} \in \text{users})$
  \end{itemize}

  \noindent makeDBConnection(credentials):
  \begin{itemize}
  \item transition: if credentials are valid, $\text{users} := \text{users} \cup credentials.user $
  \item output: 
  \item exception: Throw ConnectionError if credentials are not valid
  \end{itemize}

  \noindent getImageObjects(String q):
  \begin{itemize}
  \item transition: 
  \item output: list of ImageObjects satisfying the provided query
  \item exception: Let Q be the set of valid Queries. Throw ValueError if $\neg (q \in \text{Q})$\\
  Throw ConnectionError if $\neg (\text{requestor} \in \text{users})$

  \end{itemize}
  
  % \wss{A module without environment variables or state variables is unlikely to
  %   have a state transition.  In this case a state transition can only occur if
  %   the module is changing the state of another module.}
  
  % \wss{Modules rarely have both a transition and an output.  In most cases you
  %   will have one or the other.}
  
  \subsubsection{Local Functions}

  None
  
  % \wss{As appropriate} \wss{These functions are for the purpose of specification.
  %   They are not necessarily something that is going to be implemented
  %   explicitly.  Even if they are implemented, they are not exported; they only
  %   have local scope.}

\newpage


% --------------------------------------------------------------------------------------
% --/\/\/\/\/\/\/\/\/\/\/\/\/\/\/\/\/\/\/\/\/\/\/\/\/\/\/\/\/\/\/\/\/\/\/\/\/\/\/\/\/\--
% --------------------------------------------------------------------------------------

\section{MIS of Object Extraction Manager }\label{object extraction manager}
  
  % \wss{You can reference SRS labels, such as R\ref{R_Inputs}.}
  
  % \wss{It is also possible to use \LaTeX for hypperlinks to external documents.}
  
  \subsection{Module}
  
  Object Extraction Manager
  
  \subsection{Uses}
  
  ImageObject Database Connector \ref{ImageObject database connector}\\
  Label Database Connector \ref{label database connector}\\
  Labeller Database Connector \ref{labeller database connector}\\
  Image Prior Analyzer \ref{image prior analyzer}\\
  Label Confidence Service \ref{label confidence service}\\
  Object Extraction Service \ref{object extraction service}\\
  Labeller Expertise Calculator \ref{labeller expertise calculator}\\

  \subsection{Syntax}


  
  \subsubsection{Exported Constants}
  None
  \subsubsection{Exported Access Programs}
  
  \begin{center}
  \begin{tabular}{p{2cm} p{4cm} p{4cm} p{2cm}}
  \hline
  \textbf{Name} & \textbf{In} & \textbf{Out} & \textbf{Exceptions} \\
  \hline
  getObjects & projectID & - & ValueError \\
  
  \end{tabular}
  \end{center}
  
  \subsection{Semantics}
  
  \subsubsection{State Variables}
  
 None
  
  \subsubsection{Environment Variables}
  
  None
  
  \subsubsection{Assumptions}
  
  
  \subsubsection{Access Routine Semantics}
  
  \noindent getObjects(ProjectID p):
  \begin{itemize}
  \item transition: Updates ImageObject database with identified objects \& confidence\\
  and updates labeller expertise rating in labeller database
  \item output: 
  \item exception: Let P be the set of assigned ProjectIDs. Throw ValueError if $\neg (p \in \text{P})$\\
  \end{itemize}


  
  % \wss{A module without environment variables or state variables is unlikely to
  %   have a state transition.  In this case a state transition can only occur if
  %   the module is changing the state of another module.}
  
  % \wss{Modules rarely have both a transition and an output.  In most cases you
  %   will have one or the other.}
  
  \subsubsection{Local Functions}

  generate query:
   
  
  % \wss{As appropriate} \wss{These functions are for the purpose of specification.
  %   They are not necessarily something that is going to be implemented
  %   explicitly.  Even if they are implemented, they are not exported; they only
  %   have local scope.}

\newpage


% --------------------------------------------------------------------------------------
% --/\/\/\/\/\/\/\/\/\/\/\/\/\/\/\/\/\/\/\/\/\/\/\/\/\/\/\/\/\/\/\/\/\/\/\/\/\/\/\/\/\--
% --------------------------------------------------------------------------------------

\section{MIS of Label Confidence Service }\label{label confidence service}
  
  % \wss{You can reference SRS labels, such as R\ref{R_Inputs}.}
  
  % \wss{It is also possible to use \LaTeX for hypperlinks to external documents.}
  
  \subsection{Module}
  
  Label Confidence Service
  
  \subsection{Uses}
  
  None

  \subsection{Syntax}


  
  \subsubsection{Exported Constants}
  None
  \subsubsection{Exported Access Programs}
  
  \begin{center}
  \begin{tabular}{p{2cm} p{4cm} p{4cm} p{2cm}}
  \hline
  \textbf{Name} & \textbf{In} & \textbf{Out} & \textbf{Exceptions} \\
  \hline
  getConfidence & list[label], list[labeller], list[ImageObject] & list[list[float]]  & ValueError \\
  
  \end{tabular}
  \end{center}
  
  \subsection{Semantics}
  
  \subsubsection{State Variables}
  
 None
  
  \subsubsection{Environment Variables}
  
  None
  
  \subsubsection{Assumptions}
  
  
  \subsubsection{Access Routine Semantics}
  
  \noindent getConfidence(list[label] labels , list[labeller] labellers, list[ImageObject] imageobjects):
  \begin{itemize}
  \item transition: 
  \item output: return the confidence label of each extracted object
  \item exception: Let L be the set of valid Labels. Throw ValueError if ($\exists \text{label} \in \text{labels} |: \neg (\text{label} \in \text{L})$)\\
  Let X be the set of valid Labellers. Throw ValueError if ($\exists \text{labeller} \in \text{labellers} |: \neg (\text{labeller} \in \text{X})$)\\
  Let I be the set of valid ImageObjects. Throw ValueError if ($\exists \text{imageobject} \in \text{imageobjects} |: \neg (\text{imageobject} \in \text{I})$)\\
  \end{itemize}


  
  % \wss{A module without environment variables or state variables is unlikely to
  %   have a state transition.  In this case a state transition can only occur if
  %   the module is changing the state of another module.}
  
  % \wss{Modules rarely have both a transition and an output.  In most cases you
  %   will have one or the other.}
  
  \subsubsection{Local Functions}

   
  
  % \wss{As appropriate} \wss{These functions are for the purpose of specification.
  %   They are not necessarily something that is going to be implemented
  %   explicitly.  Even if they are implemented, they are not exported; they only
  %   have local scope.}

\newpage


% --------------------------------------------------------------------------------------
% --/\/\/\/\/\/\/\/\/\/\/\/\/\/\/\/\/\/\/\/\/\/\/\/\/\/\/\/\/\/\/\/\/\/\/\/\/\/\/\/\/\--
% --------------------------------------------------------------------------------------

\section{MIS of Object Extraction Service }\label{object extraction service}
  
  % \wss{You can reference SRS labels, such as R\ref{R_Inputs}.}
  
  % \wss{It is also possible to use \LaTeX for hypperlinks to external documents.}
  
  \subsection{Module}
  
  Object Extraction Service
  
  \subsection{Uses}
  
  None

  \subsection{Syntax}


  
  \subsubsection{Exported Constants}
  None
  \subsubsection{Exported Access Programs}
  
  \begin{center}
  \begin{tabular}{p{2cm} p{4cm} p{4cm} p{2cm}}
  \hline
  \textbf{Name} & \textbf{In} & \textbf{Out} & \textbf{Exceptions} \\
  \hline
  getObjects & list[label], list[labeller], list[ImageObject], list[list[float]]& list[ImageObject]  & ValueError \\
  
  \end{tabular}
  \end{center}
  
  \subsection{Semantics}
  
  \subsubsection{State Variables}
  
 None
  
  \subsubsection{Environment Variables}
  
  None
  
  \subsubsection{Assumptions}
  
  
  \subsubsection{Access Routine Semantics}
  
  \noindent getConfidence(list[label] labels, list[labeller] labellers, list[ImageObject] imageobjects, list[list[float]] confidence):
  \begin{itemize}
  \item transition: 
  \item output: returns a list of extracted image objects
  \item exception: Let L be the set of valid Labels. Throw ValueError if ($\exists \text{label} \in \text{labels} |: \neg (\text{label} \in \text{L})$)\\
  Let X be the set of valid Labellers. Throw ValueError if ($\exists \text{labeller} \in \text{labellers} |: \neg (\text{labeller} \in \text{X})$)\\
  Let I be the set of valid ImageObjects. Throw ValueError if ($\exists \text{imageobject} \in \text{imageobjects} |: \neg (\text{imageobject} \in \text{I})$)\\
  Throw ValueError if ($\exists i,j| x = \text{confidence[i][j]} : \neg (x \in \mathbb{R})$)\\
  \end{itemize}


  
  % \wss{A module without environment variables or state variables is unlikely to
  %   have a state transition.  In this case a state transition can only occur if
  %   the module is changing the state of another module.}
  
  % \wss{Modules rarely have both a transition and an output.  In most cases you
  %   will have one or the other.}
  
  \subsubsection{Local Functions}

   
  
  % \wss{As appropriate} \wss{These functions are for the purpose of specification.
  %   They are not necessarily something that is going to be implemented
  %   explicitly.  Even if they are implemented, they are not exported; they only
  %   have local scope.}

\newpage


% --------------------------------------------------------------------------------------
% --/\/\/\/\/\/\/\/\/\/\/\/\/\/\/\/\/\/\/\/\/\/\/\/\/\/\/\/\/\/\/\/\/\/\/\/\/\/\/\/\/\--
% --------------------------------------------------------------------------------------

\section{MIS of Label Confidence Service }\label{label confidence service}
  
  % \wss{You can reference SRS labels, such as R\ref{R_Inputs}.}
  
  % \wss{It is also possible to use \LaTeX for hypperlinks to external documents.}
  
  \subsection{Module}
  
  Label Confidence Service
  
  \subsection{Uses}
  
  None

  \subsection{Syntax}


  
  \subsubsection{Exported Constants}
  None
  \subsubsection{Exported Access Programs}
  
  \begin{center}
  \begin{tabular}{p{2cm} p{4cm} p{4cm} p{2cm}}
  \hline
  \textbf{Name} & \textbf{In} & \textbf{Out} & \textbf{Exceptions} \\
  \hline
  getObjects & list[label], list[labeller], list[ImageObject] & list[list[float]]  & ValueError \\
  
  \end{tabular}
  \end{center}
  
  \subsection{Semantics}
  
  \subsubsection{State Variables}
  
 None
  
  \subsubsection{Environment Variables}
  
  None
  
  \subsubsection{Assumptions}
  
  
  \subsubsection{Access Routine Semantics}
  
  \noindent getObjects(list[label] labels , list[labeller] labellers, list[ImageObject] imageobjects):
  \begin{itemize}
  \item transition: 
  \item output: return the confidence label of each extracted object
  \item exception: Let L be the set of valid Labels. Throw ValueError if ($\exists \text{label} \in \text{labels} |: \neg (\text{label} \in \text{L})$)\\
  Let X be the set of valid Labellers. Throw ValueError if ($\exists \text{labeller} \in \text{labellers} |: \neg (\text{labeller} \in \text{X})$)\\
  Let I be the set of valid ImageObjects. Throw ValueError if ($\exists \text{imageobject} \in \text{imageobjects} |: \neg (\text{imageobject} \in \text{I})$)\\
  \end{itemize}


  
  % \wss{A module without environment variables or state variables is unlikely to
  %   have a state transition.  In this case a state transition can only occur if
  %   the module is changing the state of another module.}
  
  % \wss{Modules rarely have both a transition and an output.  In most cases you
  %   will have one or the other.}
  
  \subsubsection{Local Functions}

   
  
  % \wss{As appropriate} \wss{These functions are for the purpose of specification.
  %   They are not necessarily something that is going to be implemented
  %   explicitly.  Even if they are implemented, they are not exported; they only
  %   have local scope.}

\newpage


% --------------------------------------------------------------------------------------
% --/\/\/\/\/\/\/\/\/\/\/\/\/\/\/\/\/\/\/\/\/\/\/\/\/\/\/\/\/\/\/\/\/\/\/\/\/\/\/\/\/\--
% --------------------------------------------------------------------------------------

\section{MIS of Image Prior Analyzer }\label{image prior analyzer}
  
  % \wss{You can reference SRS labels, such as R\ref{R_Inputs}.}
  
  % \wss{It is also possible to use \LaTeX for hypperlinks to external documents.}
  
  \subsection{Module}
  
  Image Prior Analyzer
  
  \subsection{Uses}
  
  None

  \subsection{Syntax}


  
  \subsubsection{Exported Constants}
  None
  \subsubsection{Exported Access Programs}
  
  \begin{center}
  \begin{tabular}{p{2cm} p{4cm} p{4cm} p{2cm}}
  \hline
  \textbf{Name} & \textbf{In} & \textbf{Out} & \textbf{Exceptions} \\
  \hline
  getPriors & list[image] & list[list[float]]  & ValueError \\
  
  \end{tabular}
  \end{center}
  
  \subsection{Semantics}
  
  \subsubsection{State Variables}
  
 None
  
  \subsubsection{Environment Variables}
  
  None
  
  \subsubsection{Assumptions}
  
  
  \subsubsection{Access Routine Semantics}
  
  \noindent getPriors(list[image] Images):
  \begin{itemize}
  \item transition: 
  \item output: returns a list of priors for each pixel in the given images
  \item exception: Let I be the set of valid Images. Throw ValueError if ($\exists \text{image} \in \text{images} |: \neg (\text{image} \in \text{I})$)\\
  \end{itemize}


  
  % \wss{A module without environment variables or state variables is unlikely to
  %   have a state transition.  In this case a state transition can only occur if
  %   the module is changing the state of another module.}
  
  % \wss{Modules rarely have both a transition and an output.  In most cases you
  %   will have one or the other.}
  
  \subsubsection{Local Functions}

   
  
  % \wss{As appropriate} \wss{These functions are for the purpose of specification.
  %   They are not necessarily something that is going to be implemented
  %   explicitly.  Even if they are implemented, they are not exported; they only
  %   have local scope.}

\newpage

% --------------------------------------------------------------------------------------
% --/\/\/\/\/\/\/\/\/\/\/\/\/\/\/\/\/\/\/\/\/\/\/\/\/\/\/\/\/\/\/\/\/\/\/\/\/\/\/\/\/\--
% --------------------------------------------------------------------------------------

\section{MIS of Labeller Expertise Calculator }\label{labeller expertise calculator}
  
  % \wss{You can reference SRS labels, such as R\ref{R_Inputs}.}
  
  % \wss{It is also possible to use \LaTeX for hypperlinks to external documents.}
  
  \subsection{Module}
  
  Labeller Expertise Calculator
  
  \subsection{Uses}
  
  None

  \subsection{Syntax}


  
  \subsubsection{Exported Constants}
  None
  \subsubsection{Exported Access Programs}
  
  \begin{center}
  \begin{tabular}{p{2cm} p{4cm} p{4cm} p{2cm}}
  \hline
  \textbf{Name} & \textbf{In} & \textbf{Out} & \textbf{Exceptions} \\
  \hline
  getExpertise & list[label], list[labeller], list[ImageObject], list[list[float]] & list[dict[string, tuple[float, float]]]  & ValueError \\
  
  \end{tabular}
  \end{center}
  
  \subsection{Semantics}
  
  \subsubsection{State Variables}
  
 None
  
  \subsubsection{Environment Variables}
  
  None
  
  \subsubsection{Assumptions}
  
  
  \subsubsection{Access Routine Semantics}
  
  \noindent getObjects(list[label] labels, list[labeller] labellers, list[ImageObject] imageobjects):
  \begin{itemize}
  \item transition: 
  \item output: return the weighed success rate for each class a labeler has contributed to
  \item exception: Let L be the set of valid Labels. Throw ValueError if ($\exists \text{label} \in \text{labels} |: \neg (\text{label} \in \text{L})$)\\
  Let X be the set of valid Labellers. Throw ValueError if ($\exists \text{labeller} \in \text{labellers} |: \neg (\text{labeller} \in \text{X})$)\\
  Let I be the set of valid ImageObjects. Throw ValueError if ($\exists \text{imageobject} \in \text{imageobjects} |: \neg (\text{imageobject} \in \text{I})$)\\
  Throw ValueError if ($\exists i,j| x = \text{confidence[i][j]} : \neg (x \in \mathbb{R})$)\\
  \end{itemize}


  
  % \wss{A module without environment variables or state variables is unlikely to
  %   have a state transition.  In this case a state transition can only occur if
  %   the module is changing the state of another module.}
  
  % \wss{Modules rarely have both a transition and an output.  In most cases you
  %   will have one or the other.}
  
  \subsubsection{Local Functions}

   
  
  % \wss{As appropriate} \wss{These functions are for the purpose of specification.
  %   They are not necessarily something that is going to be implemented
  %   explicitly.  Even if they are implemented, they are not exported; they only
  %   have local scope.}

\newpage

\bibliographystyle {plainnat}
\bibliography {../../../refs/References}

\newpage

\section{Appendix} \label{Appendix}

\wss{Extra information if required}

\newpage{}

\section*{Appendix --- Reflection}

\wss{Not required for CAS 741 projects}

The information in this section will be used to evaluate the team members on the
graduate attribute of Problem Analysis and Design.

The purpose of reflection questions is to give you a chance to assess your own
learning and that of your group as a whole, and to find ways to improve in the
future. Reflection is an important part of the learning process.  Reflection is
also an essential component of a successful software development process.  

Reflections are most interesting and useful when they're honest, even if the
stories they tell are imperfect. You will be marked based on your depth of
thought and analysis, and not based on the content of the reflections
themselves. Thus, for full marks we encourage you to answer openly and honestly
and to avoid simply writing ``what you think the evaluator wants to hear.''

Please answer the following questions.  Some questions can be answered on the
team level, but where appropriate, each team member should write their own
response:


\begin{enumerate}
  \item What went well while writing this deliverable? 
  \item What pain points did you experience during this deliverable, and how
    did you resolve them?
  \item Which of your design decisions stemmed from speaking to your client(s)
  or a proxy (e.g. your peers, stakeholders, potential users)? For those that
  were not, why, and where did they come from?
  \item While creating the design doc, what parts of your other documents (e.g.
  requirements, hazard analysis, etc), it any, needed to be changed, and why?
  \item What are the limitations of your solution?  Put another way, given
  unlimited resources, what could you do to make the project better? (LO\_ProbSolutions)
  \item Give a brief overview of other design solutions you considered.  What
  are the benefits and tradeoffs of those other designs compared with the chosen
  design?  From all the potential options, why did you select the documented design?
  (LO\_Explores)
\end{enumerate}


\end{document}